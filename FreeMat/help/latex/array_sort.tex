\section{SORT Sort}

\subsection{Usage}

Sorts an n-dimensional array along the specified dimensional.  The first
form sorts the array along the first non-singular dimension.
\begin{verbatim}
  B = sort(A)
\end{verbatim}
Alternately, the dimension along which to sort can be explicitly specified
\begin{verbatim}
  B = sort(A,dim)
\end{verbatim}
FreeMat does not support vector arguments for \verb|dim| - if you need \verb|A| to be
sorted along multiple dimensions (i.e., row first, then columns), make multiple
calls to \verb|sort|.  Also, the direction of the sort can be specified using the 
\verb|mode| argument
\begin{verbatim}
  B = sort(A,dim,mode)
\end{verbatim}
where \verb|mode = 'ascend'| means to sort the data in ascending order (the default),
and \verb|mode = 'descend'| means to sort the data into descending order.  

When two outputs are requested from \verb|sort|, the indexes are also returned.
Thus, for 
\begin{verbatim}
  [B,IX] = sort(A)
  [B,IX] = sort(A,dim)
  [B,IX] = sort(A,dim,mode)
\end{verbatim}
an array \verb|IX| of the same size as \verb|A|, where \verb|IX| records the indices of \verb|A|
(along the sorting dimension) corresponding to the output array \verb|B|. 

Two additional issues worth noting.  First, a cell array can be sorted if each 
cell contains a \verb|string|, in which case the strings are sorted by lexical order.
The second issue is that FreeMat uses the same method as MATLAB to sort complex
numbers.  In particular, a complex number \verb|a| is less than another complex
number \verb|b| if \verb|abs(a) < abs(b)|.  If the magnitudes are the same then we 
test the angle of \verb|a|, i.e. \verb|angle(a) < angle(b)|, where \verb|angle(a)| is the
phase of \verb|a| between \verb|-pi,pi|.
\subsection{Example}

Here are some examples of sorting on numerical arrays.
\begin{verbatim}
--> A = int32(10*rand(4,3))

A = 
 8 2 8 
 0 5 5 
 2 5 7 
 3 7 1 

--> [B,IX] = sort(A)
B = 
 0 2 1 
 2 5 5 
 3 5 7 
 8 7 8 

IX = 
 2 1 4 
 3 2 2 
 4 3 3 
 1 4 1 

--> [B,IX] = sort(A,2)
B = 
 2 8 8 
 0 5 5 
 2 5 7 
 1 3 7 

IX = 
 2 1 3 
 1 2 3 
 1 2 3 
 3 1 2 

--> [B,IX] = sort(A,1,'descend')
B = 
 8 7 8 
 3 5 7 
 2 5 5 
 0 2 1 

IX = 
 1 4 1 
 4 2 3 
 3 3 2 
 2 1 4 
\end{verbatim}
Here we sort a cell array of strings.
\begin{verbatim}
--> a = {'hello','abba','goodbye','jockey','cake'}

a = 
 [hello] [abba] [goodbye] [jockey] [cake] 

--> b = sort(a)

b = 
 [abba] [cake] [goodbye] [hello] [jockey] 
\end{verbatim}
