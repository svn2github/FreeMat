\section{BITCMP Bitwise Boolean Complement Operation}

\subsection{Usage}

 Usage
 
 Performs a bitwise binary complement operation on the argument and
 returns the result.  The syntax for its use is
\begin{verbatim}
    y = bitcmp(a)
\end{verbatim}
 where a is an unsigned integer arrays.  This version of the command
 uses as many bits as required by the type of a.  For example, if 
 a is an uint8 type, then the complement is formed using 8 bits.
 The second form of bitcmp allows you to specify the number of bits
 to use, 
\begin{verbatim}
    y = bitcmp(a,n)
\end{verbatim}
 in which case the complement is taken with respect to n bits, where n must be 
 less than the length of the integer type.

\subsection{Example}

\begin{verbatim}
--> bitcmp(uint16(2^14-2))

ans = 
 49153 

--> bitcmp(uint16(2^14-2),14)

ans = 
 1 
\end{verbatim}
