\section{INTERP2 2-D Interpolation}

\subsection{Usage}

Given a set of monotonically increasing \verb|x| coordinates and a 
corresponding set of \verb|y| values, performs simple linear 
interpolation to a new set of \verb|x| coordinates. The general syntax
for its usage is
\begin{verbatim}
   zi = interp2(z,xi,yi)
\end{verbatim}
where \verb|xi| and \verb|yi| are vectors of the same length. The output
vector \verb|zi| is the same size as the input vector \verb|xi|.
For each element of \verb|xi|, the values in \verb|zi| are linearly interpolated
by default. Interpolation method can be selected as:
\begin{verbatim}
   zi = interp2(z,xi,yi,method)
\end{verbatim}
Default interpolation method is \verb|'linear'|. Other methods are
 \verb|'nearest'|, and \verb|'cubic'|.
For values in \verb|xi, yi| that are outside the size of \verb|z|,
the default value returned is NaN.  To change this behavior,
you can specify the extrapolation value:
\begin{verbatim}
   zi = interp2(z,xi,yi,method,extrapval)
\end{verbatim}
The \verb|z| and \verb|xi,yi| vectors must be real, although complex types
are allowed for \verb|z|.
