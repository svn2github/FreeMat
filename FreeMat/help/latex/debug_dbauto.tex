\section{DBAUTO Control Dbauto Functionality}

\subsection{Usage}

The dbauto functionality in FreeMat allows you to debug your
FreeMat programs.  When \verb|dbauto| is \verb|on|, then any error
that occurs while the program is running causes FreeMat to 
stop execution at that point and return you to the command line
(just as if you had placed a \verb|keyboard| command there).  You can
then examine variables, modify them, and resume execution using
\verb|return|.  Alternately, you can exit out of all running routines
via a \verb|retall| statement.  Note that errors that occur inside of
\verb|try|/\verb|catch| blocks do not (by design) cause auto breakpoints.  The
\verb|dbauto| function toggles the dbauto state of FreeMat.  The
syntax for its use is
\begin{verbatim}
   dbauto(state)
\end{verbatim}
where \verb|state| is either
\begin{verbatim}
   dbauto('on')
\end{verbatim}
to activate dbauto, or
\begin{verbatim}
   dbauto('off')
\end{verbatim}
to deactivate dbauto.  Alternately, you can use FreeMat's string-syntax
equivalence and enter
\begin{verbatim}
   dbauto on
\end{verbatim}
or 
\begin{verbatim}
   dbauto off
\end{verbatim}
to turn dbauto on or off (respectively).  Entering \verb|dbauto| with no arguments
returns the current state (either 'on' or 'off').
