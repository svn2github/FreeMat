\section{DEAL Multiple Simultaneous Assignments}

\subsection{Usage}

When making a function call, it is possible to assign
multiple outputs in a single call, (see, e.g., \verb|max| for
an example).  The \verb|deal| call allows you to do the 
same thing with a simple assignment.  The syntax for its
use is
\begin{verbatim}
   [a,b,c,...] = deal(expr)
\end{verbatim}
where \verb|expr| is an expression with multiple values.  The simplest
example is where \verb|expr| is the dereference of a cell array, e.g.
\verb|expr <-- A{:}|.  In this case, the \verb|deal| call is equivalent
to
\begin{verbatim}
   a = A{1}; b = A{2}; C = A{3}; 
\end{verbatim}
Other expressions which are multivalued are structure arrays with
multiple entries (non-scalar), where field dereferencing has been
applied.
