\section{JITCONTROL Control the Just In Time Compiler}

\subsection{Usage}

The \verb|jitcontrol| functionality in FreeMat allows you to control
the use of the Just In Time (JIT) compiler.  Starting in FreeMat
version 4, the JIT compiler is enabled by default on all platforms
where it is successfully built.  The JIT compiler should significantly
improve the performance of loop intensive, scalar code.  As development
progresses, more and more functionality will be enabled under the JIT.
In the mean time (if you use the GUI version of FreeMat) you can use
the JIT chat window to get information on why your code was JIT compiled
(or not).
\begin{verbatim}
    do_jit_test.m
function test_val = do_jit_test(test_name)
printf('Running %s without JIT...',test_name);
jitcontrol off
tic; A1 = feval(test_name); nojit = toc;
jitcontrol on
% Run it twice to remove the compile time
tic; A2 = feval(test_name); wjit = toc;
tic; A2 = feval(test_name); wjit = toc;
printf('Speedup is %g...',nojit/wjit);
test_val = issame(A1,A2);
if (test_val)
  printf('PASS\n');
else
  printf('FAIL\n');
end
\end{verbatim}
\begin{verbatim}
    jit_test001.m
function A = jit_test001
  A = zeros(1,100000);
  for i=1:100000; A(i) = i; end;
\end{verbatim}
\begin{verbatim}
    jit_test002.m
function A = jit_test002
  A = zeros(512);
  for i=1:512;
    for j=1:512;
       A(i,j) = i-j;
    end
  end
\end{verbatim}
\begin{verbatim}
    jit_test003.m
function A = jit_test003
  A = zeros(512);
  for i=1:512
    for j=1:512
      k = i-j;
      if ((k < 5) && (k>-5))
         A(i,j) = k;
      end
    end
  end
\end{verbatim}
\begin{verbatim}
    jit_test004.m
function A = jit_test004
  A = zeros(512);
  for i=1:512
    for j=1:512
      k = i-j;
      if ((k < 5) && (k>-5))
         A(i,j) = k;
      else
         A(i,j) = i+j;
      end
    end
  end
\end{verbatim}
\begin{verbatim}
    jit_test005.m
function A = jit_test005
  A = zeros(1,1000);
  for i=1:1000
    for m=1:100
      if (i > 50)
        k = 55;
      elseif (i > 40)
        k = 45;
      elseif (i > 30)
        k = 35;
      else 
        k = 15;
      end
    end
    A(i) = k;
  end
\end{verbatim}
\begin{verbatim}
    jit_test006.m
function A = jit_test006
  A = zeros(1000);
  m = 0;
  for i=1:1000;
    for j=1:1000;
      A(i,j) = i-j;
      m = i;
    end
  end
  A = m;
\end{verbatim}
\begin{verbatim}
    jit_test007.m
function A = jit_test007
  B = ones(1,100000);
  j = 0;
  for i=1:100000
    j = j + B(1,i);
  end
  A = j;
\end{verbatim}
\begin{verbatim}
    jit_test008.m
function A = jit_test008
  A = zeros(1,10000);
  for i=1:10010;
    A(i) = i;
  end
\end{verbatim}
\begin{verbatim}
    jit_test009.m
function A = jit_test009
  B = ones(1,10000);
  A = 0;
  for k=1:10000
    A = A + B(k);
  end
\end{verbatim}
\begin{verbatim}
    jit_test010.m
function A = jit_test010
  B = 1:100000;
  A = zeros(1,100000);
  for i=1:100000
    A(i) = B(i);
  end
\end{verbatim}
\begin{verbatim}
    jit_test011.m
function A = jit_test011
  A = zeros(1,100000);
  B = 5;
  for i=1:100000
    C = B + i;
    A(i) = C;
  end
\end{verbatim}
\begin{verbatim}
    jit_test012.m
function A = jit_test012
  A = zeros(400);
  for i=1:500
    for j=1:500;
      A(i,j) = i-j;
    end
  end
\end{verbatim}
\begin{verbatim}
    jit_test013.m
function A = jit_test013
  A = zeros(512);
  for i=1:512
    for j=1:512
      A(i,j) = abs(i-j);
    end
  end
\end{verbatim}
\begin{verbatim}
    jit_test014.m
function A = jit_test014
  N = 500;
  A = zeros(N,N);
  for i=1:N;
    for j=1:N;
      A(j,i) = abs(i-j);
    end
  end
\end{verbatim}
\begin{verbatim}
    jit_test015.m
function A = jit_test015
  A = 0;
  for i=0:200000
    A = A + abs(sin(i/200000*pi));
  end
\end{verbatim}
\begin{verbatim}
    jit_test016.m
function A = jit_test016
  A = 0;
  for i=0:200000
    A = A + abs(cos(i/200000*pi));
  end
\end{verbatim}
\begin{verbatim}
    jit_test017.m
function A = jit_test017
  A = 0;
  for j=1:100
    B = j;
    for i=1:10000
      A = A + B;
    end
  end
\end{verbatim}
\begin{verbatim}
    jit_test018.m
function A = jit_test018
  A = 0;
  for i=1.2:4000.2
    A = A + i;
  end
\end{verbatim}
\begin{verbatim}
    jit_test019.m
function A = jit_test019
  A = 150000;
  C = 0;
  for b=1:A
    C = C + b;
  end
  A = C;
\end{verbatim}
\begin{verbatim}
    jit_test020.m
function a = jit_test020
  a = zeros(1,10000);
  for i=1:10000;
    a(i) = sec(i/10000);
  end
\end{verbatim}
\begin{verbatim}
    jit_test021.m
function A = jit_test021
  A = zeros(100,100);
  for i=1:(100*100)
    A(i) = i;
  end
\end{verbatim}
\begin{verbatim}
    jit_test022.m
function A = jit_test022
  A = 1:1000000;
  B = 0;
  for i=1:1000000;
    B = B + A(i);
  end
  A = B;
\end{verbatim}
\begin{verbatim}
    jit_test023.m
function A = jit_test023
  A = zeros(1,100000);
  for i=1:100000;
    A(i) = i/5.0;
  end
\end{verbatim}
\begin{verbatim}
    jit_test024.m
function A = jit_test024
  A = zeros(1,10000);
  for j=1:10
    for i=1:10000;
      A(i) = tjit_sum(A(i),i);
     end;
  end;

function y = tjit_sum(a,b)
  y = a + tjit_double(b);

function y = tjit_double(a)
  y = a*2;
\end{verbatim}
