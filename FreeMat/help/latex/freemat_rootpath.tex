\section{ROOTPATH Set FreeMat Root Path}

\subsection{Usage}

In order to function properly, FreeMat needs to know where to
find the \verb|toolbox| directory as well as the \verb|help| directory.
These directories are located on what is known as the \verb|root path|.
Normally, FreeMat should know where these directories are located.
However under some circumstances (usually when FreeMat is installed
into a non-default location), it may be necessary to indicate
a different root path location, or to specify a particular one.
Note that on the Mac OS platform, FreeMat is installed as a bundle,
and will use the toolbox that is installed in the bundle regardless of
the setting for \verb|rootpath|.
For Linux, FreeMat will typically use \verb|/usr/local/share/FreeMat-<Version>/|
for the root path.  Installations from source code will generally work,
but binary installations (e.g., from an \verb|RPM|) may need to have the
rootpath set.

The \verb|rootpath| function has two forms.  The first form takes no arguments
and returns the current root path
\begin{verbatim}
   rootpath
\end{verbatim}
The second form will set a rootpath directly from the command line
\begin{verbatim}
   rootpath(path)
\end{verbatim}
where \verb|path| is the full path to where the \verb|toolbox| and \verb|help| 
directories are located.  For example, \verb|rootpath('/usr/share/FreeMat-4.0')|.
The third form enables the GUI form 
\begin{verbatim}
   rootpath gui
\end{verbatim}
which activates a dialog box to pick a directory that is the root directory
of the FreeMat installation (e.g., where \verb|help| and \verb|toolbox| are located.
Changes to \verb|rootpath| are persistent (you do not need to run it every
time you start FreeMat).
