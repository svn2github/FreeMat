\section{SYMVAR Find Symbolic Variables in an Expression}

\subsection{Usage}

Finds the symbolic variables in an expression.  The syntax for its
use is 
\begin{verbatim}
  syms = symvar(expr)
\end{verbatim}
where \verb|expr| is a string containing an expression, such as
\verb|'x^2 + cos(t+alpha)'|.  The result is a cell array of strings
containing the non-function identifiers in the expression.  Because
they are usually not used as identifiers in expressions, the strings
 \verb|'pi','inf','nan','eps','i','j'| are ignored.
\subsection{Example}

Here are some simple examples:
\begin{verbatim}
--> symvar('x^2+sqrt(x)')  % sqrt is eliminated as a function

ans = 
 [x] 

--> symvar('pi+3')         % No identifiers here

ans = 
  Empty array 0 1
--> symvar('x + t*alpha')  % x, t and alpha

ans = 
 [alpha] 
 [t] 
 [x] 
\end{verbatim}
