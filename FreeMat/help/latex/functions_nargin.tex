\section{NARGIN Number of Input Arguments}

\subsection{Usage}

The \verb|nargin| function returns the number of arguments passed
to a function when it was called.  The general syntax for its
use is
\begin{verbatim}
  y = nargin
\end{verbatim}
FreeMat allows for
fewer arguments to be passed to a function than were declared,
and \verb|nargin|, along with \verb|isset| can be used to determine
exactly what subset of the arguments were defined.

You can also use \verb|nargin| on a function handle to return the
number of input arguments expected by the function
\begin{verbatim}
  y = nargin(fun)
\end{verbatim}
where \verb|fun| is the name of the function (e.g. \verb|'sin'|) or 
a function handle.
\subsection{Example}

Here is a function that is declared to take five 
arguments, and that simply prints the value of \verb|nargin|
each time it is called.
\begin{verbatim}
    nargintest.m
function nargintest(a1,a2,a3,a4,a5)
  printf('nargin = %d\n',nargin);
\end{verbatim}
\begin{verbatim}
--> nargintest(3);
nargin = 1
--> nargintest(3,'h');
nargin = 2
--> nargintest(3,'h',1.34);
nargin = 3
--> nargintest(3,'h',1.34,pi,e);
nargin = 5
--> nargin('sin')

ans = 
 1 

--> y = @sin

y = 
 @sin
--> nargin(y)

ans = 
 1 
\end{verbatim}
