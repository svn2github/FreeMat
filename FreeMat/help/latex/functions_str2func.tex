\section{STR2FUNC String to Function conversion}

\subsection{Usage}

The \verb|str2func| function converts a function name into a 
function pointer.  The syntax is 
\begin{verbatim}
    y = str2func('funcname')
\end{verbatim}
where \verb|funcname| is the name of the function. The return
variable \verb|y| is a function handle that points to the given
function.

An alternate syntax is used to construct an anonymous function
given an expression.  They syntax is
\begin{verbatim}
    y = str2func('anonymous def')
\end{verbatim}
where \verb|anonymous def| is an expression that defines an
anonymous function, for example \verb|'@(x) x.\^2'|.
\subsection{Example}

Here is a simple example of using \verb|str2func|.
\begin{verbatim}
--> sin(.5)              % Calling the function directly

ans = 
    0.4794 

--> y = str2func('sin')  % Convert it into a function handle

y = 
 @sin
--> y(.5)                % Calling 'sin' via the function handle

ans = 
    0.4794 
\end{verbatim}
Here we use \verb|str2func| to define an anonymous function
\begin{verbatim}
--> y = str2func('@(x) x.^2')

y = 
 @(x)   x.^2
--> y(2)

ans = 
 4 
\end{verbatim}
