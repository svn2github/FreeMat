\section{GLASSEMBLY Create a GL Assembly}

\subsection{Usage}

Define a GL Assembly.  A GL Assembly consists of one or more
GL Nodes or GL Assemblies that are placed relative to the 
coordinate system of the assembly.  For example, if we have
\verb|glnode| definitions for \verb|'bread'| and \verb|'cheese'|, then
a \verb|glassembly| of sandwich would consist of placements of
two \verb|'bread'| nodes with a \verb|'cheese'| node in between.
Furthermore, a \verb|'lunch'| assembly could consist of a \verb|'sandwich'|
a \verb|'chips'| and \verb|'soda'|.  Hopefully, you get the idea.  The
syntax for the \verb|glassembly| command is
\begin{verbatim}
   glassembly(name,part1,transform1,part2,transform2,...)
\end{verbatim}
where \verb|part1| is the name of the first part, and could be
either a \verb|glnode| or itself be another \verb|glassembly|.  
Here \verb|transform1| is the \verb|4 x 4 matrix| that transforms
the part into the local reference coordinate system.

WARNING!! Currently FreeMat does not detect or gracefully handle 
self-referential assemblies (i.e, if you try to make a \verb|sandwich| 
contain a \verb|sandwich|, which you can do by devious methods that I 
refuse to explain).  Do not do this!  You have been warned.
