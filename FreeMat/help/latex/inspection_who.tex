\section{WHO Describe Currently Defined Variables}

\subsection{Usage}

Reports information on either all variables in the current context
or on a specified set of variables.  For each variable, the \verb|who|
function indicates the size and type of the variable as well as 
if it is a global or persistent.  There are two formats for the 
function call.  The first is the explicit form, in which a list
of variables are provided:
\begin{verbatim}
  who a1 a2 ...
\end{verbatim}
In the second form
\begin{verbatim}
  who
\end{verbatim}
the \verb|who| function lists all variables defined in the current 
context (as well as global and persistent variables). Note that
there are two alternate forms for calling the \verb|who| function:
\begin{verbatim}
  who 'a1' 'a2' ...
\end{verbatim}
and
\begin{verbatim}
  who('a1','a2',...)
\end{verbatim}
\subsection{Example}

Here is an example of the general use of \verb|who|, which lists all of the variables defined.
\begin{verbatim}
--> c = [1,2,3];
--> f = 'hello';
--> p = randn(1,256);
--> who
  Variable Name       Type   Flags             Size
              c    double                    [1x3]
              f      char                    [1x5]
              p    double                    [1x256]
\end{verbatim}
In the second case, we examine only a specific variable:
\begin{verbatim}
--> who c
  Variable Name       Type   Flags             Size
              c    double                    [1x3]
--> who('c')
  Variable Name       Type   Flags             Size
              c    double                    [1x3]
\end{verbatim}
