\section{FORMAT Control the Format of Matrix Display}

\subsection{Usage}

FreeMat supports several modes for displaying matrices (either through the
\verb|disp| function or simply by entering expressions on the command line.  
There are several options for the format command.  The default mode is equivalent
to
\begin{verbatim}
   format short
\end{verbatim}
which generally displays matrices with 4 decimals, and scales matrices if the entries
have magnitudes larger than roughly \verb|1e2| or smaller than \verb|1e-2|.   For more 
information you can use 
\begin{verbatim}
   format long
\end{verbatim}
which displays roughly 7 decimals for \verb|float| and \verb|complex| arrays, and 14 decimals
for \verb|double| and \verb|dcomplex|.  You can also use
\begin{verbatim}
   format short e
\end{verbatim}
to get exponential format with 4 decimals.  Matrices are not scaled for exponential 
formats.  Similarly, you can use
\begin{verbatim}
   format long e
\end{verbatim}
which displays the same decimals as \verb|format long|, but in exponential format.
You can also use the \verb|format| command to retrieve the current format:
\begin{verbatim}
   s = format
\end{verbatim}
where \verb|s| is a string describing the current format.
\subsection{Example}

We start with the short format, and two matrices, one of double precision, and the
other of single precision.
\begin{verbatim}
--> format short
--> a = randn(4)

a = 
   -0.3756    0.0920    0.9516    1.8527 
    0.5078   -0.2088   -0.3120   -0.2380 
    0.5578    0.7695    0.0226    2.9326 
   -0.4420   -0.4871   -0.7582   -0.5059 

--> b = float(randn(4))

b = 
    0.2010    0.3416    0.1562   -0.5460 
    1.2842   -0.3808   -1.2720   -0.3398 
   -0.7660   -0.6251    2.4811    0.7956 
   -0.1727    0.8577    1.5701   -1.5048 
\end{verbatim}
Note that in the short format, these two matrices are displayed with the same format.
In \verb|long| format, however, they display differently
\begin{verbatim}
--> format long
--> a

ans = 
  -0.37559630424227   0.09196341864118   0.95155934364300   1.85265231634028 
   0.50776589164635  -0.20877480315311  -0.31198760445638  -0.23799081322695 
   0.55783547335483   0.76954243414671   0.02264031516947   2.93263318869123 
  -0.44202929771190  -0.48708606879623  -0.75822963661106  -0.50590405332950 

--> b

ans = 
   0.2010476   0.3415550   0.1561587  -0.5460028 
   1.2841575  -0.3808453  -1.2719837  -0.3397521 
  -0.7659672  -0.6251388   2.4811494   0.7956446 
  -0.1726678   0.8576548   1.5701485  -1.5048176 
\end{verbatim}
Note also that we we scale the contents of the matrices, FreeMat rescales the entries
with a scale premultiplier.
\begin{verbatim}
--> format short
--> a*1e4

ans = 

   1.0e+04 * 
   -0.3756    0.0920    0.9516    1.8527 
    0.5078   -0.2088   -0.3120   -0.2380 
    0.5578    0.7695    0.0226    2.9326 
   -0.4420   -0.4871   -0.7582   -0.5059 

--> a*1e-4

ans = 

   1.0e-04 * 
   -0.3756    0.0920    0.9516    1.8527 
    0.5078   -0.2088   -0.3120   -0.2380 
    0.5578    0.7695    0.0226    2.9326 
   -0.4420   -0.4871   -0.7582   -0.5059 

--> b*1e4

ans = 

   1.0e+04 * 
    0.2010    0.3416    0.1562   -0.5460 
    1.2842   -0.3808   -1.2720   -0.3398 
   -0.7660   -0.6251    2.4811    0.7956 
   -0.1727    0.8577    1.5701   -1.5048 

--> b*1e-4

ans = 

   1.0e-04 * 
    0.2010    0.3416    0.1562   -0.5460 
    1.2842   -0.3808   -1.2720   -0.3398 
   -0.7660   -0.6251    2.4811    0.7956 
   -0.1727    0.8577    1.5701   -1.5048 
\end{verbatim}
Next, we use the exponential formats:
\begin{verbatim}
--> format short e
--> a*1e4

ans = 
 -3.7560e+03  9.1963e+02  9.5156e+03  1.8527e+04 
  5.0777e+03 -2.0877e+03 -3.1199e+03 -2.3799e+03 
  5.5784e+03  7.6954e+03  2.2640e+02  2.9326e+04 
 -4.4203e+03 -4.8709e+03 -7.5823e+03 -5.0590e+03 

--> a*1e-4

ans = 
 -3.7560e-05  9.1963e-06  9.5156e-05  1.8527e-04 
  5.0777e-05 -2.0877e-05 -3.1199e-05 -2.3799e-05 
  5.5784e-05  7.6954e-05  2.2640e-06  2.9326e-04 
 -4.4203e-05 -4.8709e-05 -7.5823e-05 -5.0590e-05 

--> b*1e4

ans = 
  2.0105e+03  3.4156e+03  1.5616e+03 -5.4600e+03 
  1.2842e+04 -3.8085e+03 -1.2720e+04 -3.3975e+03 
 -7.6597e+03 -6.2514e+03  2.4811e+04  7.9564e+03 
 -1.7267e+03  8.5765e+03  1.5701e+04 -1.5048e+04 

--> b*1e-4

ans = 
  2.0105e-05  3.4155e-05  1.5616e-05 -5.4600e-05 
  1.2842e-04 -3.8085e-05 -1.2720e-04 -3.3975e-05 
 -7.6597e-05 -6.2514e-05  2.4811e-04  7.9564e-05 
 -1.7267e-05  8.5765e-05  1.5701e-04 -1.5048e-04 
\end{verbatim}
Finally, if we assign the \verb|format| function to a variable, we can retrieve the 
current format:
\begin{verbatim}
--> format short
--> t = format

t = 
short
\end{verbatim}
