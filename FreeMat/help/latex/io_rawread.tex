\section{RAWREAD Read N-dimensional Array From File}

\subsection{Usage}

The syntax for \verb|rawread| is
\begin{verbatim}
   function x = rawread(fname,size,precision,byteorder)
\end{verbatim}
where \verb|fname| is the name of the file to read from, 
and \verb|size| is an n-dimensional vector that stores the
size of the array in each dimension.  The argument \verb|precision|
is the type of the data to read in:
\begin{itemize}
\item  'uint8','uchar','unsigned char' for unsigned, 8-bit integers

\item  'int8','char','integer*1' for signed, 8-bit integers

\item  'uint16','unsigned short' for unsigned, 16-bit  integers

\item  'int16','short','integer*2' for  signed, 16-bit integers

\item  'uint32','unsigned int' for unsigned, 32-bit integers

\item  'int32','int','integer*4' for signed, 32-bit integers

\item  'uint64','unsigned int' for unsigned, 64-bit integers

\item  'int64','int','integer*8' for signed, 64-bit integers

\item  'single','float32','float','real*4' for 32-bit floating point

\item  'double','float64','real*8' for 64-bit floating point

\item  'complex','complex*8' for  64-bit complex floating point (32 bits 
         for the real and imaginary part).

\item  'dcomplex','complex*16' for 128-bit complex floating point (64
         bits for the real and imaginary part).

\end{itemize}
As a special feature, one of the size elements can be 'inf', 
in which case, the largest possible array is read in.
If \verb|byteorder| is left unspecified, the file is assumed to be
of the same byte-order as the machine \verb|FreeMat| is running on.
If you wish to force a particular byte order, specify the \verb|byteorder|
argument as
\begin{itemize}
\item  \verb|'le','ieee-le','little-endian','littleEndian','little'|

\item  \verb|'be','ieee-be','big-endian','bigEndian','big'|

\end{itemize}
