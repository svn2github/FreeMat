\section{RAWWRITE Write N-dimensional Array From File}

\subsection{Usage}

The syntax for \verb|rawwrite| is
\begin{verbatim}
   function rawwrite(fname,x,byteorder)
\end{verbatim}
where \verb|fname| is the name of the file to write to, and the
(numeric) array \verb|x| is writen to the file in its native
type (e.g. if \verb|x| is of type \verb|int16|, then it will be written
to the file as 16-bit signed integers.  If \verb|byteorder| is
left unspecified, the file is assumed to be
of the same byte-order as the machine \verb|FreeMat| is running on.
If you wish to force a particular byte order, specify the \verb|byteorder|
argument as
\begin{itemize}
\item  \verb|'le','ieee-le','little-endian','littleEndian','little'|

\item  \verb|'be','ieee-be','big-endian','bigEndian','big'|

\end{itemize}
