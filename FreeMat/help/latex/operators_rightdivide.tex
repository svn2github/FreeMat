\section{RIGHTDIVIDE Matrix Equation Solver/Divide Operator}

\subsection{Usage}

The divide operator \verb|/| is really a combination of three
operators, all of which have the same general syntax:
\begin{verbatim}
  Y = A / B
\end{verbatim}
where \verb|A| and \verb|B| are arrays of numerical type.  The result \verb|Y| depends
on which of the following three situations applies to the arguments
\verb|A| and \verb|B|:
\begin{enumerate}
  \item \verb|A| is a scalar, \verb|B| is an arbitrary \verb|n|-dimensional numerical array, in which case the output is the scalar \verb|A| divided into each element of \verb|B|.
  \item \verb|B| is a scalar, \verb|A| is an arbitrary \verb|n|-dimensional numerical array, in which case the output is each element of \verb|A| divided by the scalar \verb|B|.
  \item \verb|A,B| are matrices with the same number of columns, i.e., \verb|A| is of size \verb|K x M|, and \verb|B| is of size \verb|L x M|, in which case the output is of size \verb|K x L|.
\end{enumerate}
The output follows the standard type promotion rules, although in the first two cases, if \verb|A| and \verb|B| are integers, the output is an integer also, while in the third case if \verb|A| and \verb|B| are integers, the output is of type \verb|double|.

\subsection{Function Internals}

There are three formulae for the times operator.  For the first form
\[
Y(m_1,\ldots,m_d) = \frac{A}{B(m_1,\ldots,m_d)},
\]
and the second form
\[
Y(m_1,\ldots,m_d) = \frac{A(m_1,\ldots,m_d)}{B}.
\]
In the third form, the output is defined as:
\[
  Y = (B' \backslash A')'
\]
and is used in the equation \verb|Y B = A|.
\subsection{Examples}

The right-divide operator is much less frequently used than the left-divide operator, but the concepts are similar.  It can be used to find least-squares and minimum norm solutions.  It can also be used to solve systems of equations in much the same way.  Here's a simple example:
\begin{verbatim}
--> B = [1,1;0,1];
--> A = [4,5]

A = 
 4 5 

--> A/B

ans = 
 4 1 
\end{verbatim}
