\section{TRANSPOSE Matrix Transpose Operator}

\subsection{Usage}

Performs a transpose of the argument (a 2D matrix).  The syntax for its use is
\begin{verbatim}
  y = a.';
\end{verbatim}
where \verb|a| is a \verb|M x N| numerical matrix.  The output \verb|y| is a numerical matrix
of the same type of size \verb|N x M|.  This operator is the non-conjugating transpose,
which is different from the Hermitian operator \verb|'| (which conjugates complex values).
\subsection{Function Internals}

The transpose operator is defined simply as
\[
  y_{i,j} = a_{j,i}
\]
where \verb|y\_ij| is the element in the \verb|i|th row and \verb|j|th column of the output matrix \verb|y|.
\subsection{Examples}

A simple transpose example:
\begin{verbatim}
--> A = [1,2,0;4,1,-1]

A = 
  1  2  0 
  4  1 -1 

--> A.'

ans = 
  1  4 
  2  1 
  0 -1 
\end{verbatim}
Here, we use a complex matrix to demonstrate how the transpose does \emph{not} conjugate the entries.
\begin{verbatim}
--> A = [1+i,2-i]

A = 
   1.0000 +  1.0000i   2.0000 -  1.0000i 

--> A.'

ans = 
   1.0000 +  1.0000i 
   2.0000 -  1.0000i 
\end{verbatim}
