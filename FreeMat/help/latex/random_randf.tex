\section{RANDF Generate F-Distributed Random Variable}

\subsection{Usage}

Generates random variables with an F-distribution.  The general
syntax for its use is
\begin{verbatim}
   y = randf(n,m)
\end{verbatim}
where \verb|n| and \verb|m| are vectors of the number of degrees of freedom
in the numerator and denominator of the chi-square random variables
whose ratio defines the statistic.
\subsection{Function Internals}

The statistic \verb|F\_{n,m}| is defined as the ratio of two chi-square
random variables:
\[
  F_{n,m} = \frac{\chi_n^2/n}{\chi_m^2/m}
\]
The PDF is given by
\[
  f_{n,m} = \frac{m^{m/2}n^{n/2}x^{n/2-1}}{(m+nx)^{(n+m)/2}B(n/2,m/2)},
\]
where \verb|B(a,b)| is the beta function.
\subsection{Example}

Here we use \verb|randf| to generate some F-distributed random variables,
and then again using the \verb|randchi| function:
\begin{verbatim}
--> randf(5*ones(1,9),7)

ans = 
    0.5241    0.8414    0.4859    1.1266    0.4792    2.3743    2.9095    0.5825    0.4244 

--> randchi(5*ones(1,9))./randchi(7*ones(1,9))

ans = 
    0.3737    0.2363    1.5733    0.7003    1.1385    0.6337    0.4597    0.2691    0.5190 
\end{verbatim}
