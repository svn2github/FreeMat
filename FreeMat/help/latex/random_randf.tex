\section{RANDF Generate F-Distributed Random Variable}

\subsection{Usage}

Generates random variables with an F-distribution.  The general
syntax for its use is
\begin{verbatim}
   y = randf(n,m)
\end{verbatim}
where \verb|n| and \verb|m| are vectors of the number of degrees of freedom
in the numerator and denominator of the chi-square random variables
whose ratio defines the statistic.
\subsection{Function Internals}

The statistic \verb|F_{n,m}| is defined as the ratio of two chi-square
random variables:
\[
  F_{n,m} = \frac{\chi_n^2/n}{\chi_m^2/m}
\]
The PDF is given by
\[
  f_{n,m} = \frac{m^{m/2}n^{n/2}x^{n/2-1}}{(m+nx)^{(n+m)/2}B(n/2,m/2)},
\]
where \verb|B(a,b)| is the beta function.
\subsection{Example}

Here we use \verb|randf| to generate some F-distributed random variables,
and then again using the \verb|randchi| function:
\begin{verbatim}
--> randf(5*ones(1,9),7)

ans = 

 Columns 1 to 8

    1.1944    0.9069    0.7558    1.5029    0.0621    1.3860    1.8161    0.3755 

 Columns 9 to 9

    3.5794 

--> randchi(5*ones(1,9))./randchi(7*ones(1,9))

ans = 

 Columns 1 to 8

    1.3085    1.2693    1.0684    0.4377    1.1158    0.7171    0.4151    1.8022 

 Columns 9 to 9

    1.4606 
\end{verbatim}
