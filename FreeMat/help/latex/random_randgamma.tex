\section{RANDGAMMA Generate Gamma-Distributed Random Variable}

\subsection{Usage}

Generates random variables with a gamma distribution.  The general
syntax for its use is
\begin{verbatim}
   y = randgamma(a,r),
\end{verbatim}
where \verb|a| and \verb|r| are vectors describing the parameters of the
gamma distribution.  Roughly speaking, if \verb|a| is the mean time between
changes of a Poisson random process, and we wait for the \verb|r| change,
the resulting wait time is Gamma distributed with parameters \verb|a| 
and \verb|r|.
\subsection{Function Internals}

The Gamma distribution arises in Poisson random processes.  It represents
the waiting time to the occurance of the \verb|r|-th event in a process with
mean time \verb|a| between events.  The probability distribution of a Gamma
random variable is
\[
   P(x) = \frac{a^r x^{r-1} e^{-ax}}{\Gamma(r)}.
\]
Note also that for integer values of \verb|r| that a Gamma random variable
is effectively the sum of \verb|r| exponential random variables with parameter
\verb|a|.
\subsection{Example}

Here we use the \verb|randgamma| function to generate Gamma-distributed
random variables, and then generate them again using the \verb|randexp|
function.
\begin{verbatim}
--> randgamma(1,15*ones(1,9))

ans = 
   10.0227   12.4783   18.0388   21.7056   14.1249   15.9260   22.0177   15.9170   24.3781 

--> sum(randexp(ones(15,9)))

ans = 
   14.5031   12.8908   10.5201   16.9976    9.8463   12.7479   13.6879   21.7005   11.4172 
\end{verbatim}
