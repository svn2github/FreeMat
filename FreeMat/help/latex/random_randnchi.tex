\section{RANDNCHI Generate Noncentral Chi-Square Random Variable}

\subsection{Usage}

Generates a vector of non-central chi-square random variables
with the given number of degrees of freedom and the given
non-centrality parameters.  The general syntax for its use is
\begin{verbatim}
   y = randnchi(n,mu)
\end{verbatim}
where \verb|n| is an array containing the degrees of freedom for
each generated random variable (with each element of \verb|n| >= 1),
and \verb|mu| is the non-centrality shift (must be positive).
\subsection{Function Internals}

A non-central chi-square random variable is the sum of a chisquare
deviate with \verb|n-1| degrees of freedom plus the square of a normal
deviate with mean \verb|mu| and standard deviation 1.
\subsection{Examples}

Here is an example of a non-central chi-square random variable:
\begin{verbatim}
--> randnchi(5*ones(1,9),0.3)

ans = 

 Columns 1 to 8

    1.7097    6.6003   14.1463    2.0817    4.4984    5.3132    3.1775    3.5291 

 Columns 9 to 9

    5.5185 
\end{verbatim}
