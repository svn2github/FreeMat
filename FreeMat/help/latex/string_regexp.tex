\section{REGEXP Regular Expression Matching Function}

\subsection{Usage}

Matches regular expressions in the provided string.  This function is
complicated, and compatibility with MATLABs syntax is not perfect.  The
syntax for its use is
\begin{verbatim}
  regexp('str','expr')
\end{verbatim}
which returns a row vector containing the starting index of each substring
of \verb|str| that matches the regular expression described by \verb|expr|.  The
second form of \verb|regexp| returns six outputs in the following order:
\begin{verbatim}
  [start stop tokenExtents match tokens names] = regexp('str','expr')
\end{verbatim}
where the meaning of each of the outputs is defined below.
\begin{itemize}
\item  \verb|start| is a row vector containing the starting index of each 
substring that matches the regular expression.

\item  \verb|stop| is a row vector containing the ending index of each 
substring that matches the regular expression.

\item  \verb|tokenExtents| is a cell array containing the starting and ending
indices of each substring that matches the \verb|tokens| in the regular
expression.  A token is a captured part of the regular expression.
If the \verb|'once'| mode is used, then this output is a \verb|double| array.

\item  \verb|match| is a cell array containing the text for each substring
that matches the regular expression.  In \verb|'once'| mode, this is a 
string.

\item  \verb|tokens| is a cell array of cell arrays of strings that correspond
to the tokens in the regular expression.  In \verb|'once'| mode, this is a
cell array of strings.

\item  \verb|named| is a structure array containing the named tokens captured
in a regular expression. Each named token is assigned a field in the resulting
structure array, and each element of the array corresponds to a different
match.

\end{itemize}
If you want only some of the the outputs,  you can use the 
following variant of \verb|regexp|:
\begin{verbatim}
  [o1 o2 ...] = regexp('str','expr', 'p1', 'p2', ...)
\end{verbatim}
where \verb|p1| etc. are the names of the outputs (and the order we want
the outputs in).  As a final variant, you can supply some mode 
flags to \verb|regexp|
\begin{verbatim}
  [o1 o2 ...] = regexp('str','expr', p1, p2, ..., 'mode1', 'mode2')
\end{verbatim}
where acceptable \verb|mode| flags are:
\begin{itemize}
\item  \verb|'once'| - only the first match is returned.

\item  \verb|'matchcase'| - letter case must match (selected by default for \verb|regexp|)

\item  \verb|'ignorecase'| - letter case is ignored (selected by default for \verb|regexpi|)

\item  \verb|'dotall'| - the \verb|'.'| operator matches any character (default)

\item  \verb|'dotexceptnewline'| - the \verb|'.'| operator does not match the newline character

\item  \verb|'stringanchors'| - the \verb|\^| and \verb|$| operators match at the beginning and 
end (respectively) of a string.

\item  \verb|'lineanchors'| - the \verb|\^| and \verb|$| operators match at the beginning and
end (respectively) of a line.

\item  \verb|'literalspacing'| - the space characters and comment characters \verb|\#| are matched
as literals, just like any other ordinary character (default).

\item  \verb|'freespacing'| - all spaces and comments are ignored in the regular expression.
You must use '\ ' and '\#' to match spaces and comment characters, respectively.

\end{itemize}
Note the following behavior differences between MATLABs regexp and FreeMats:
\begin{itemize}
\item  If you have an old version of \verb|pcre| installed, then named tokens must use the
older \verb|<?P<name>| syntax, instead of the new \verb|<?<name>| syntax.  

\item  The \verb|pcre| library is pickier about named tokens and their appearance in 
expressions.  So, for example, the regexp from the MATLAB 
manual \verb|'(?<first>\\w+)\\s+(?<last>\\w+)|(?<last>\\w+),\\s+(?<first>\\w+)'|
does not work correctly (as of this writing) because the same named 
tokens appear multiple
times.  The workaround is to assign different names to each token, and then collapse
the results later.

\end{itemize}
\subsection{Example}

Some examples of using the \verb|regexp| function
\begin{verbatim}
--> [start,stop,tokenExtents,match,tokens,named] = regexp('quick down town zoo','(.)own')
start = 
  7 12 

stop = 
 10 15 

tokenExtents = 
 [1x2 double array] [1x2 double array] 

match = 
 [down] [town] 

tokens = 
 [1x1 cell array] [1x1 cell array] 

named = 
  []
\end{verbatim}
