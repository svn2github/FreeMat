\section{INT64 Convert to Signed 64-bit Integer}

\subsection{Usage}

Converts the argument to an signed 64-bit Integer.  The syntax
for its use is
\begin{verbatim}
   y = int64(x)
\end{verbatim}
where \verb|x| is an \verb|n|-dimensional numerical array.  Conversion
follows the saturation rules (e.g., if \verb|x| is outside the normal
range for a signed 64-bit integer of \verb|[-2\^63+1,2\^63-1]|, it is
truncated to that range).  Note that both \verb|NaN| and \verb|Inf| both map to 0.
\subsection{Example}

The following piece of code demonstrates several uses of \verb|int64|.  First, the routine uses
\begin{verbatim}
--> int64(100)

ans = 
 100 

--> int64(-100)

ans = 
 -100 
\end{verbatim}
In the next example, an integer outside the range  of the type is passed in.  
The result is truncated to the range of the data type.
\begin{verbatim}
--> int64(40e9)

ans = 
 40000000000 
\end{verbatim}
In the next example, a positive double precision argument is passed in.  The 
result is the signed integer that is closest to the argument.
\begin{verbatim}
--> int64(pi)

ans = 
 3 
\end{verbatim}
In the next example, a complex argument is passed in.  The result is the 
complex signed integer that is closest to the argument.
\begin{verbatim}
--> int64(5+2*i)

ans = 
   5.0000 +  2.0000i 
\end{verbatim}
In the next example, a string argument is passed in.  The string argument is 
converted into an integer array corresponding to the ASCII values of each character.
\begin{verbatim}
--> int64('helo')

ans = 
 104 101 108 111 
\end{verbatim}
In the last example, a cell-array is passed in.  For cell-arrays and structure 
arrays, the result is an error.
\begin{verbatim}
--> int64({4})
Error: Cannot perform type conversions with this type
\end{verbatim}
