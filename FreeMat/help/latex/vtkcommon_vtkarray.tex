\section{vtkArray}

\subsection{Usage}

 vtkArray is the root of a hierarchy of arrays that can be used to store data with
 any number of dimensions.  It provides an abstract interface for retrieving and
 setting array attributes that are independent of the type of values stored in the
 array - such as the number of dimensions, extents along each dimension, and number
 of values stored in the array.

 To get and set array values, the vtkTypedArray template class derives from vtkArray
 and provides type-specific methods for retrieval and update.

 Two concrete derivatives of vtkTypedArray are provided at the moment: vtkDenseArray
 and vtkSparseArray, which provide dense and sparse storage for arbitrary-dimension
 data, respectively.  Toolkit users can create their own concrete derivatives that
 implement alternative storage strategies, such as compressed-sparse-row, etc.  You
 could also create an array that provided read-only access to 'virtual' data, such
 as an array that returned a Fibonacci sequence, etc.


To create an instance of class vtkArray, simply
invoke its constructor as follows
\begin{verbatim}
  obj = vtkArray
\end{verbatim}
\subsection{Methods}

The class vtkArray has several methods that can be used.
  They are listed below.
Note that the documentation is translated automatically from the VTK sources,
and may not be completely intelligible.  When in doubt, consult the VTK website.
In the methods listed below, \verb|obj| is an instance of the vtkArray class.
\begin{itemize}
\item  \verb|string = obj.GetClassName ()|

\item  \verb|int = obj.IsA (string name)|

\item  \verb|vtkArray = obj.NewInstance ()|

\item  \verb|vtkArray = obj.SafeDownCast (vtkObject o)|

\item  \verb|bool = obj.IsDense ()| -  Returns true iff the underlying array storage is ''dense'', i.e. that GetSize() and
 GetNonNullSize() will always return the same value.  If not, the array is ''sparse''.

\item  \verb|obj.Resize (vtkIdType i)| -  Resizes the array to the given extents (number of dimensions and size of each dimension).
 Note that concrete implementations of vtkArray may place constraints on the the extents
 that they will store, so you cannot assume that GetExtents() will always return the same
 value passed to Resize().

 The contents of the array are undefined after calling Resize() - you should
 initialize its contents accordingly.  In particular, dimension-labels will be
 undefined, dense array values will be undefined, and sparse arrays will be
 empty.

\item  \verb|obj.Resize (vtkIdType i, vtkIdType j)| -  Resizes the array to the given extents (number of dimensions and size of each dimension).
 Note that concrete implementations of vtkArray may place constraints on the the extents
 that they will store, so you cannot assume that GetExtents() will always return the same
 value passed to Resize().

 The contents of the array are undefined after calling Resize() - you should
 initialize its contents accordingly.  In particular, dimension-labels will be
 undefined, dense array values will be undefined, and sparse arrays will be
 empty.

\item  \verb|obj.Resize (vtkIdType i, vtkIdType j, vtkIdType k)| -  Resizes the array to the given extents (number of dimensions and size of each dimension).
 Note that concrete implementations of vtkArray may place constraints on the the extents
 that they will store, so you cannot assume that GetExtents() will always return the same
 value passed to Resize().

 The contents of the array are undefined after calling Resize() - you should
 initialize its contents accordingly.  In particular, dimension-labels will be
 undefined, dense array values will be undefined, and sparse arrays will be
 empty.

\item  \verb|vtkIdType = obj.GetDimensions ()| -  Returns the number of dimensions stored in the array.  Note that this is the same as
 calling GetExtents().GetDimensions().

\item  \verb|vtkIdType = obj.GetSize ()| -  Returns the number of values stored in the array.  Note that this is the same as calling
 GetExtents().GetSize(), and represents the maximum number of values that could ever
 be stored using the current extents.  This is equal to the number of values stored in a
 dense array, but may be larger than the number of values stored in a sparse array.

\item  \verb|vtkIdType = obj.GetNonNullSize ()| -  Returns the number of non-null values stored in the array.  Note that this
 value will equal GetSize() for dense arrays, and will be less-than-or-equal
 to GetSize() for sparse arrays.

\item  \verb|obj.SetName (vtkStdString \&name)| -  Sets the array name.

\item  \verb|vtkStdString = obj.GetName ()| -  Returns the array name.

\item  \verb|obj.SetDimensionLabel (vtkIdType i, vtkStdString \&label)| -  Sets the label for the i-th array dimension.

\item  \verb|vtkStdString = obj.GetDimensionLabel (vtkIdType i)| -  Returns the label for the i-th array dimension.

\item  \verb|vtkArray = obj.DeepCopy ()| -  Returns a new array that is a deep copy of this array.

\end{itemize}
