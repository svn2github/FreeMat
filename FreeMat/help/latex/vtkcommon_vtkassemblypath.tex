\section{vtkAssemblyPath}

\subsection{Usage}

 vtkAssemblyPath represents an ordered list of assembly nodes that
 represent a fully evaluated assembly path. This class is used primarily
 for picking. Note that the use of this class is to add one or more
 assembly nodes to form the path. (An assembly node consists of an instance
 of vtkProp and vtkMatrix4x4, the matrix may be NULL.) As each node is
 added, the matrices are concatenated to create a final, evaluated matrix.

To create an instance of class vtkAssemblyPath, simply
invoke its constructor as follows
\begin{verbatim}
  obj = vtkAssemblyPath
\end{verbatim}
\subsection{Methods}

The class vtkAssemblyPath has several methods that can be used.
  They are listed below.
Note that the documentation is translated automatically from the VTK sources,
and may not be completely intelligible.  When in doubt, consult the VTK website.
In the methods listed below, \verb|obj| is an instance of the vtkAssemblyPath class.
\begin{itemize}
\item  \verb|string = obj.GetClassName ()|

\item  \verb|int = obj.IsA (string name)|

\item  \verb|vtkAssemblyPath = obj.NewInstance ()|

\item  \verb|vtkAssemblyPath = obj.SafeDownCast (vtkObject o)|

\item  \verb|obj.AddNode (vtkProp p, vtkMatrix4x4 m)| -  Convenience method adds a prop and matrix together,
 creating an assembly node transparently. The matrix
 pointer m may be NULL. Note: that matrix is the one,
 if any, associated with the prop. 

\item  \verb|vtkAssemblyNode = obj.GetNextNode ()| -  Get the next assembly node in the list. The node returned
 contains a pointer to a prop and a 4x4 matrix. The matrix
 is evaluated based on the preceding assembly hierarchy
 (i.e., the matrix is not necessarily as the same as the
 one that was added with AddNode() because of the 
 concatenation of matrices in the assembly hierarchy).

\item  \verb|vtkAssemblyNode = obj.GetFirstNode ()| -  Get the first assembly node in the list. See the comments for
 GetNextNode() regarding the contents of the returned node. (Note: This
 node corresponds to the vtkProp associated with the vtkRenderer.

\item  \verb|vtkAssemblyNode = obj.GetLastNode ()| -  Get the last assembly node in the list. See the comments
 for GetNextNode() regarding the contents of the returned node.

\item  \verb|obj.DeleteLastNode ()| -  Delete the last assembly node in the list. This is like
 a stack pop.

\item  \verb|obj.ShallowCopy (vtkAssemblyPath path)| -  Perform a shallow copy (reference counted) on the
 incoming path.

\item  \verb|long = obj.GetMTime ()| -  Override the standard GetMTime() to check for the modified times
 of the nodes in this path.

\end{itemize}
