\section{vtkBox}

\subsection{Usage}

 vtkBox computes the implicit function and/or gradient for a axis-aligned
 bounding box. (The superclasses transform can be used to modify this
 orientation.) Each side of the box is orthogonal to all other sides
 meeting along shared edges and all faces are orthogonal to the x-y-z
 coordinate axes.  (If you wish to orient this box differently, recall that
 the superclass vtkImplicitFunction supports a transformation matrix.)
 vtkCube is a concrete implementation of vtkImplicitFunction.
 

To create an instance of class vtkBox, simply
invoke its constructor as follows
\begin{verbatim}
  obj = vtkBox
\end{verbatim}
\subsection{Methods}

The class vtkBox has several methods that can be used.
  They are listed below.
Note that the documentation is translated automatically from the VTK sources,
and may not be completely intelligible.  When in doubt, consult the VTK website.
In the methods listed below, \verb|obj| is an instance of the vtkBox class.
\begin{itemize}
\item  \verb|string = obj.GetClassName ()|

\item  \verb|int = obj.IsA (string name)|

\item  \verb|vtkBox = obj.NewInstance ()|

\item  \verb|vtkBox = obj.SafeDownCast (vtkObject o)|

\item  \verb|double = obj.EvaluateFunction (double x[3])|

\item  \verb|double = obj.EvaluateFunction (double x, double y, double z)|

\item  \verb|obj.EvaluateGradient (double x[3], double n[3])|

\item  \verb|obj.SetXMin (double p[3])| -  Set / get the bounding box using various methods.

\item  \verb|obj.SetXMin (double x, double y, double z)| -  Set / get the bounding box using various methods.

\item  \verb|obj.GetXMin (double p[3])| -  Set / get the bounding box using various methods.

\item  \verb|obj.SetXMax (double p[3])|

\item  \verb|obj.SetXMax (double x, double y, double z)|

\item  \verb|obj.GetXMax (double p[3])|

\item  \verb|obj.SetBounds (double xMin, double xMax, double yMin, double yMax, double zMin, double zMax)|

\item  \verb|obj.SetBounds (double bounds[6])|

\item  \verb|obj.GetBounds (double bounds[6])|

\item  \verb|obj.AddBounds (double bounds[6])| -  A special method that allows union set operation on bounding boxes.
 Start with a SetBounds(). Subsequent AddBounds() methods are union set
 operations on the original bounds. Retrieve the final bounds with a 
 GetBounds() method.

\end{itemize}
