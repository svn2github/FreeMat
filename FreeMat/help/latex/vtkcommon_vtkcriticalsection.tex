\section{vtkCriticalSection}

\subsection{Usage}

 vtkCriticalSection allows the locking of variables which are accessed 
 through different threads.  This header file also defines 
 vtkSimpleCriticalSection which is not a subclass of vtkObject.
 The API is identical to that of vtkMutexLock, and the behavior is
 identical as well, except on Windows 9x/NT platforms. The only difference
 on these platforms is that vtkMutexLock is more flexible, in that
 it works across processes as well as across threads, but also costs
 more, in that it evokes a 600-cycle x86 ring transition. The 
 vtkCriticalSection provides a higher-performance equivalent (on 
 Windows) but won't work across processes. Since it is unclear how,
 in vtk, an object at the vtk level can be shared across processes
 in the first place, one should use vtkCriticalSection unless one has
 a very good reason to use vtkMutexLock. If higher-performance equivalents
 for non-Windows platforms (Irix, SunOS, etc) are discovered, they
 should replace the implementations in this class

To create an instance of class vtkCriticalSection, simply
invoke its constructor as follows
\begin{verbatim}
  obj = vtkCriticalSection
\end{verbatim}
\subsection{Methods}

The class vtkCriticalSection has several methods that can be used.
  They are listed below.
Note that the documentation is translated automatically from the VTK sources,
and may not be completely intelligible.  When in doubt, consult the VTK website.
In the methods listed below, \verb|obj| is an instance of the vtkCriticalSection class.
\begin{itemize}
\item  \verb|string = obj.GetClassName ()|

\item  \verb|int = obj.IsA (string name)|

\item  \verb|vtkCriticalSection = obj.NewInstance ()|

\item  \verb|vtkCriticalSection = obj.SafeDownCast (vtkObject o)|

\item  \verb|obj.Lock ()| -  Lock the vtkCriticalSection

\item  \verb|obj.Unlock ()| -  Unlock the vtkCriticalSection

\end{itemize}
