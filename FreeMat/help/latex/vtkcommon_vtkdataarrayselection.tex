\section{vtkDataArraySelection}

\subsection{Usage}

 vtkDataArraySelection can be used by vtkSource subclasses to store
 on/off settings for whether each vtkDataArray in its input should
 be passed in the source's output.  This is primarily intended to
 allow file readers to configure what data arrays are read from the
 file.

To create an instance of class vtkDataArraySelection, simply
invoke its constructor as follows
\begin{verbatim}
  obj = vtkDataArraySelection
\end{verbatim}
\subsection{Methods}

The class vtkDataArraySelection has several methods that can be used.
  They are listed below.
Note that the documentation is translated automatically from the VTK sources,
and may not be completely intelligible.  When in doubt, consult the VTK website.
In the methods listed below, \verb|obj| is an instance of the vtkDataArraySelection class.
\begin{itemize}
\item  \verb|string = obj.GetClassName ()|

\item  \verb|int = obj.IsA (string name)|

\item  \verb|vtkDataArraySelection = obj.NewInstance ()|

\item  \verb|vtkDataArraySelection = obj.SafeDownCast (vtkObject o)|

\item  \verb|obj.EnableArray (string name)| -  Enable the array with the given name.  Creates a new entry if
 none exists.

\item  \verb|obj.DisableArray (string name)| -  Disable the array with the given name.  Creates a new entry if
 none exists.

\item  \verb|int = obj.ArrayIsEnabled (string name)| -  Return whether the array with the given name is enabled.  If
 there is no entry, the array is assumed to be disabled.

\item  \verb|int = obj.ArrayExists (string name)| -  Return whether the array with the given name exists.

\item  \verb|obj.EnableAllArrays ()| -  Enable all arrays that currently have an entry.

\item  \verb|obj.DisableAllArrays ()| -  Disable all arrays that currently have an entry.

\item  \verb|int = obj.GetNumberOfArrays ()| -  Get the number of arrays that currently have an entry.

\item  \verb|int = obj.GetNumberOfArraysEnabled ()| -  Get the number of arrays that are enabled.

\item  \verb|string = obj.GetArrayName (int index)| -  Get the name of the array entry at the given index.

\item  \verb|int = obj.GetArrayIndex (string name)| -  Get an index of the array containing name within the enabled arrays

\item  \verb|int = obj.GetEnabledArrayIndex (string name)| -  Get the index of an array with the given name among those
 that are enabled.  Returns -1 if the array is not enabled.

\item  \verb|int = obj.GetArraySetting (string name)| -  Get whether the array at the given index is enabled.

\item  \verb|int = obj.GetArraySetting (int index)| -  Get whether the array at the given index is enabled.

\item  \verb|obj.RemoveAllArrays ()| -  Remove all array entries.

\item  \verb|obj.CopySelections (vtkDataArraySelection selections)| -  Copy the selections from the given vtkDataArraySelection instance.

\end{itemize}
