\section{vtkDirectory}

\subsection{Usage}

 vtkDirectory provides a portable way of finding the names of the files
 in a system directory.  It also provides methods of manipulating directories.

To create an instance of class vtkDirectory, simply
invoke its constructor as follows
\begin{verbatim}
  obj = vtkDirectory
\end{verbatim}
\subsection{Methods}

The class vtkDirectory has several methods that can be used.
  They are listed below.
Note that the documentation is translated automatically from the VTK sources,
and may not be completely intelligible.  When in doubt, consult the VTK website.
In the methods listed below, \verb|obj| is an instance of the vtkDirectory class.
\begin{itemize}
\item  \verb|string = obj.GetClassName ()| -  Return the class name as a string.

\item  \verb|int = obj.IsA (string name)| -  Return the class name as a string.

\item  \verb|vtkDirectory = obj.NewInstance ()| -  Return the class name as a string.

\item  \verb|vtkDirectory = obj.SafeDownCast (vtkObject o)| -  Return the class name as a string.

\item  \verb|int = obj.Open (string dir)| -  Open the specified directory and load the names of the files
 in that directory. 0 is returned if the directory can not be 
 opened, 1 if it is opened.   

\item  \verb|vtkIdType = obj.GetNumberOfFiles ()| -  Return the number of files in the current directory.

\item  \verb|string = obj.GetFile (vtkIdType index)| -  Return the file at the given index, the indexing is 0 based

\item  \verb|int = obj.FileIsDirectory (string name)| -  Return true if the file is a directory.  If the file is not an
 absolute path, it is assumed to be relative to the opened
 directory. If no directory has been opened, it is assumed to
 be relative to the current working directory.

\item  \verb|vtkStringArray = obj.GetFiles ()| -  Get an array that contains all the file names.

\end{itemize}
