\section{vtkEdgeTable}

\subsection{Usage}

 vtkEdgeTable is a general object for keeping track of lists of edges. An
 edge is defined by the pair of point id's (p1,p2). Methods are available
 to insert edges, check if edges exist, and traverse the list of edges.
 Also, it's possible to associate attribute information with each edge.
 The attribute information may take the form of vtkIdType id's, void*
 pointers, or points. To store attributes, make sure that
 InitEdgeInsertion() is invoked with the storeAttributes flag set properly.
 If points are inserted, use the methods InitPointInsertion() and 
 InsertUniquePoint().

To create an instance of class vtkEdgeTable, simply
invoke its constructor as follows
\begin{verbatim}
  obj = vtkEdgeTable
\end{verbatim}
\subsection{Methods}

The class vtkEdgeTable has several methods that can be used.
  They are listed below.
Note that the documentation is translated automatically from the VTK sources,
and may not be completely intelligible.  When in doubt, consult the VTK website.
In the methods listed below, \verb|obj| is an instance of the vtkEdgeTable class.
\begin{itemize}
\item  \verb|string = obj.GetClassName ()|

\item  \verb|int = obj.IsA (string name)|

\item  \verb|vtkEdgeTable = obj.NewInstance ()|

\item  \verb|vtkEdgeTable = obj.SafeDownCast (vtkObject o)|

\item  \verb|obj.Initialize ()| -  Free memory and return to the initially instantiated state.

\item  \verb|int = obj.InitEdgeInsertion (vtkIdType numPoints, int storeAttributes)| -  Initialize the edge insertion process. Provide an estimate of the number
 of points in a dataset (the maximum range value of p1 or p2).  The
 storeAttributes variable controls whether attributes are to be stored
 with the edge, and what type of attributes. If storeAttributes==1, then
 attributes of vtkIdType can be stored. If storeAttributes==2, then
 attributes of type void* can be stored. In either case, additional
 memory will be required by the data structure to store attribute data
 per each edge.  This method is used in conjunction with one of the three
 InsertEdge() methods described below (don't mix the InsertEdge()
 methods---make sure that the one used is consistent with the
 storeAttributes flag set in InitEdgeInsertion()).

\item  \verb|vtkIdType = obj.InsertEdge (vtkIdType p1, vtkIdType p2)| -  Insert the edge (p1,p2) into the table. It is the user's
 responsibility to check if the edge has already been inserted
 (use IsEdge()). If the storeAttributes flag in InitEdgeInsertion()
 has been set, then the method returns a unique integer id (i.e.,
 the edge id) that can be used to set and get edge
 attributes. Otherwise, the method will return 1. Do not mix this
 method with the InsertEdge() method that follows.

\item  \verb|obj.InsertEdge (vtkIdType p1, vtkIdType p2, vtkIdType attributeId)| -  Insert the edge (p1,p2) into the table with the attribute id
 specified (make sure the attributeId >= 0). Note that the
 attributeId is ignored if the storeAttributes variable was set to
 0 in the InitEdgeInsertion() method. It is the user's
 responsibility to check if the edge has already been inserted
 (use IsEdge()). Do not mix this method with the other two
 InsertEdge() methods.

\item  \verb|vtkIdType = obj.IsEdge (vtkIdType p1, vtkIdType p2)| -  Return an integer id for the edge, or an attribute id of the edge
 (p1,p2) if the edge has been previously defined (it depends upon
 which version of InsertEdge() is being used); otherwise -1. The
 unique integer id can be used to set and retrieve attributes to
 the edge.

\item  \verb|int = obj.InitPointInsertion (vtkPoints newPts, vtkIdType estSize)| -  Initialize the point insertion process. The newPts is an object
 representing point coordinates into which incremental insertion methods
 place their data. The points are associated with the edge.

\item  \verb|vtkIdType = obj.GetNumberOfEdges ()| -  Return the number of edges that have been inserted thus far.

\item  \verb|obj.InitTraversal ()| -  Intialize traversal of edges in table.

\item  \verb|obj.Reset ()| -  Reset the object and prepare for reinsertion of edges. Does not delete
 memory like the Initialize() method.

\end{itemize}
