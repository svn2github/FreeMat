\section{vtkExtentSplitter}

\subsection{Usage}

 vtkExtentSplitter splits each input extent into non-overlapping
 sub-extents that are completely contained within other ''source
 extents''.  A source extent corresponds to some resource providing
 an extent.  Each source extent has an integer identifier, integer
 priority, and an extent.  The input extents are split into
 sub-extents according to priority, availability, and amount of
 overlap of the source extents.  This can be used by parallel data
 readers to read as few piece files as possible.

To create an instance of class vtkExtentSplitter, simply
invoke its constructor as follows
\begin{verbatim}
  obj = vtkExtentSplitter
\end{verbatim}
\subsection{Methods}

The class vtkExtentSplitter has several methods that can be used.
  They are listed below.
Note that the documentation is translated automatically from the VTK sources,
and may not be completely intelligible.  When in doubt, consult the VTK website.
In the methods listed below, \verb|obj| is an instance of the vtkExtentSplitter class.
\begin{itemize}
\item  \verb|string = obj.GetClassName ()|

\item  \verb|int = obj.IsA (string name)|

\item  \verb|vtkExtentSplitter = obj.NewInstance ()|

\item  \verb|vtkExtentSplitter = obj.SafeDownCast (vtkObject o)|

\item  \verb|obj.AddExtentSource (int id, int priority, int x0, int x1, int y0, int y1, int z0, int z1)| -  Add/Remove a source providing the given extent.  Sources with
 higher priority numbers are favored.  Source id numbers and
 priorities must be non-negative.

\item  \verb|obj.AddExtentSource (int id, int priority, int extent)| -  Add/Remove a source providing the given extent.  Sources with
 higher priority numbers are favored.  Source id numbers and
 priorities must be non-negative.

\item  \verb|obj.RemoveExtentSource (int id)| -  Add/Remove a source providing the given extent.  Sources with
 higher priority numbers are favored.  Source id numbers and
 priorities must be non-negative.

\item  \verb|obj.RemoveAllExtentSources ()| -  Add/Remove a source providing the given extent.  Sources with
 higher priority numbers are favored.  Source id numbers and
 priorities must be non-negative.

\item  \verb|obj.AddExtent (int x0, int x1, int y0, int y1, int z0, int z1)| -  Add an extent to the queue of extents to be split among the
 available sources.

\item  \verb|obj.AddExtent (int extent)| -  Add an extent to the queue of extents to be split among the
 available sources.

\item  \verb|int = obj.ComputeSubExtents ()| -  Split the extents currently in the queue among the available
 sources.  The queue is empty when this returns.  Returns 1 if all
 extents could be read.  Returns 0 if any portion of any extent
 was not available through any source.

\item  \verb|int = obj.GetNumberOfSubExtents ()| -  Get the number of sub-extents into which the original set of
 extents have been split across the available sources.  Valid
 after a call to ComputeSubExtents.

\item  \verb|int = obj.GetSubExtent (int index)| -  Get the sub-extent associated with the given index.  Use
 GetSubExtentSource to get the id of the source from which this
 sub-extent should be read.  Valid after a call to
 ComputeSubExtents.

\item  \verb|obj.GetSubExtent (int index, int extent)| -  Get the sub-extent associated with the given index.  Use
 GetSubExtentSource to get the id of the source from which this
 sub-extent should be read.  Valid after a call to
 ComputeSubExtents.

\item  \verb|int = obj.GetSubExtentSource (int index)| -  Get the id of the source from which the sub-extent associated
 with the given index should be read.  Returns -1 if no source
 provides the sub-extent.

\item  \verb|int = obj.GetPointMode ()| -  Get/Set whether ''point mode'' is on.  In point mode, sub-extents
 are generated to ensure every point in the update request is
 read, but not necessarily every cell.  This can be used when
 point data are stored in a planar slice per piece with no cell
 data.  The default is OFF.

\item  \verb|obj.SetPointMode (int )| -  Get/Set whether ''point mode'' is on.  In point mode, sub-extents
 are generated to ensure every point in the update request is
 read, but not necessarily every cell.  This can be used when
 point data are stored in a planar slice per piece with no cell
 data.  The default is OFF.

\item  \verb|obj.PointModeOn ()| -  Get/Set whether ''point mode'' is on.  In point mode, sub-extents
 are generated to ensure every point in the update request is
 read, but not necessarily every cell.  This can be used when
 point data are stored in a planar slice per piece with no cell
 data.  The default is OFF.

\item  \verb|obj.PointModeOff ()| -  Get/Set whether ''point mode'' is on.  In point mode, sub-extents
 are generated to ensure every point in the update request is
 read, but not necessarily every cell.  This can be used when
 point data are stored in a planar slice per piece with no cell
 data.  The default is OFF.

\end{itemize}
