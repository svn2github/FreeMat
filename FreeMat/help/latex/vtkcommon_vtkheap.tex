\section{vtkHeap}

\subsection{Usage}

 This class is a replacement for malloc/free and new/delete for software
 that has inherent memory leak or performance problems. For example,
 external software such as the PLY library (vtkPLY) and VRML importer
 (vtkVRMLImporter) are often written with lots of malloc() calls but
 without the corresponding free() invocations. The class
 vtkOrderedTriangulator may create and delete millions of new/delete calls.
 This class allows the overloading of the C++ new operator (or other memory
 allocation requests) by using the method AllocateMemory(). Memory is
 deleted with an invocation of CleanAll() (which deletes ALL memory; any
 given memory allocation cannot be deleted). Note: a block size can be used
 to control the size of each memory allocation. Requests for memory are
 fulfilled from the block until the block runs out, then a new block is
 created.


To create an instance of class vtkHeap, simply
invoke its constructor as follows
\begin{verbatim}
  obj = vtkHeap
\end{verbatim}
\subsection{Methods}

The class vtkHeap has several methods that can be used.
  They are listed below.
Note that the documentation is translated automatically from the VTK sources,
and may not be completely intelligible.  When in doubt, consult the VTK website.
In the methods listed below, \verb|obj| is an instance of the vtkHeap class.
\begin{itemize}
\item  \verb|string = obj.GetClassName ()|

\item  \verb|int = obj.IsA (string name)|

\item  \verb|vtkHeap = obj.NewInstance ()|

\item  \verb|vtkHeap = obj.SafeDownCast (vtkObject o)|

\item  \verb|int = obj.GetNumberOfBlocks ()| -  Get the number of allocations thus far.

\item  \verb|int = obj.GetNumberOfAllocations ()| -  Get the number of allocations thus far.

\item  \verb|obj.Reset ()| -  This methods resets the current allocation location
 back to the beginning of the heap. This allows
 reuse of previously allocated memory which may be
 beneficial to performance in many cases.

\item  \verb|string = obj.StringDup (string str)| -  Convenience method performs string duplication.

\end{itemize}
