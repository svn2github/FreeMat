\section{vtkImplicitFunction}

\subsection{Usage}

 vtkImplicitFunction specifies an abstract interface for implicit
 functions. Implicit functions are real valued functions defined in 3D
 space, w = F(x,y,z). Two primitive operations are required: the ability to
 evaluate the function, and the function gradient at a given point. The 
 implicit function divides space into three regions: on the surface
 (F(x,y,z)=w), outside of the surface (F(x,y,z)>c), and inside the
 surface (F(x,y,z)<c). (When c is zero, positive values are outside,
 negative values are inside, and zero is on the surface. Note also
 that the function gradient points from inside to outside.)

 Implicit functions are very powerful. It is possible to represent almost
 any type of geometry with the level sets w = const, especially if you use 
 boolean combinations of implicit functions (see vtkImplicitBoolean).

 vtkImplicitFunction provides a mechanism to transform the implicit
 function(s) via a vtkAbstractTransform.  This capability can be used to 
 translate, orient, scale, or warp implicit functions.  For example, 
 a sphere implicit function can be transformed into an oriented ellipse. 

To create an instance of class vtkImplicitFunction, simply
invoke its constructor as follows
\begin{verbatim}
  obj = vtkImplicitFunction
\end{verbatim}
\subsection{Methods}

The class vtkImplicitFunction has several methods that can be used.
  They are listed below.
Note that the documentation is translated automatically from the VTK sources,
and may not be completely intelligible.  When in doubt, consult the VTK website.
In the methods listed below, \verb|obj| is an instance of the vtkImplicitFunction class.
\begin{itemize}
\item  \verb|string = obj.GetClassName ()|

\item  \verb|int = obj.IsA (string name)|

\item  \verb|vtkImplicitFunction = obj.NewInstance ()|

\item  \verb|vtkImplicitFunction = obj.SafeDownCast (vtkObject o)|

\item  \verb|long = obj.GetMTime ()| -  Overload standard modified time function. If Transform is modified,
 then this object is modified as well.

\item  \verb|double = obj.FunctionValue (double x[3])| -  Evaluate function at position x-y-z and return value. Point x[3] is
 transformed through transform (if provided).

\item  \verb|double = obj.FunctionValue (double x, double y, double z)| -  Evaluate function at position x-y-z and return value. Point x[3] is
 transformed through transform (if provided).

\item  \verb|obj.FunctionGradient (double x[3], double g[3])| -  Evaluate function gradient at position x-y-z and pass back vector. Point
 x[3] is transformed through transform (if provided).

\item  \verb|double = obj.FunctionGradient (double x[3])| -  Evaluate function gradient at position x-y-z and pass back vector. Point
 x[3] is transformed through transform (if provided).

\item  \verb|double = obj.FunctionGradient (double x, double y, double z)| -  Evaluate function gradient at position x-y-z and pass back vector. Point
 x[3] is transformed through transform (if provided).

\item  \verb|obj.SetTransform (vtkAbstractTransform )| -  Set/Get a transformation to apply to input points before
 executing the implicit function.

\item  \verb|obj.SetTransform (double elements[16])| -  Set/Get a transformation to apply to input points before
 executing the implicit function.

\item  \verb|vtkAbstractTransform = obj.GetTransform ()| -  Set/Get a transformation to apply to input points before
 executing the implicit function.

\item  \verb|double = obj.EvaluateFunction (double x[3])| -  Evaluate function at position x-y-z and return value.  You should
 generally not call this method directly, you should use 
 FunctionValue() instead.  This method must be implemented by 
 any derived class. 

\item  \verb|double = obj.EvaluateFunction (double x, double y, double z)| -  Evaluate function at position x-y-z and return value.  You should
 generally not call this method directly, you should use 
 FunctionValue() instead.  This method must be implemented by 
 any derived class. 

\item  \verb|obj.EvaluateGradient (double x[3], double g[3])| -  Evaluate function gradient at position x-y-z and pass back vector. 
 You should generally not call this method directly, you should use 
 FunctionGradient() instead.  This method must be implemented by 
 any derived class. 

\end{itemize}
