\section{vtkMinimalStandardRandomSequence}

\subsection{Usage}

 vtkMinimalStandardRandomSequence is a sequence of statistically independent
 pseudo random numbers uniformly distributed  between 0.0 and 1.0.

 The sequence is generated by a prime modulus multiplicative linear
 congruential generator (PMMLCG) or ''Lehmer generator'' with multiplier 16807
 and prime modulus 2\^(31)-1. The authors calls it
 ''minimal standard random number generator''

 ref: ''Random Number Generators: Good Ones are Hard to Find,''
 by Stephen K. Park and Keith W. Miller in Communications of the ACM,
 31, 10 (Oct. 1988) pp. 1192-1201.
 Code is at page 1195, ''Integer version 2''

 Correctness test is described in first column, page 1195:
 A seed of 1 at step 1 should give a seed of 1043618065 at step 10001.

To create an instance of class vtkMinimalStandardRandomSequence, simply
invoke its constructor as follows
\begin{verbatim}
  obj = vtkMinimalStandardRandomSequence
\end{verbatim}
\subsection{Methods}

The class vtkMinimalStandardRandomSequence has several methods that can be used.
  They are listed below.
Note that the documentation is translated automatically from the VTK sources,
and may not be completely intelligible.  When in doubt, consult the VTK website.
In the methods listed below, \verb|obj| is an instance of the vtkMinimalStandardRandomSequence class.
\begin{itemize}
\item  \verb|string = obj.GetClassName ()|

\item  \verb|int = obj.IsA (string name)|

\item  \verb|vtkMinimalStandardRandomSequence = obj.NewInstance ()|

\item  \verb|vtkMinimalStandardRandomSequence = obj.SafeDownCast (vtkObject o)|

\item  \verb|obj.SetSeed (int value)| -  Set the seed of the random sequence.
 The following pre-condition is stated page 1197, second column:
 valid\_seed: value>=1 \&\& value<=2147483646
 2147483646=(2\^31)-2
 This method does not have this criterium as a pre-condition (ie it will
 not fail if an incorrect seed value is passed) but the value is silently
 changed to fit in the valid range [1,2147483646].
 2147483646 is added to a null or negative value.
 2147483647 is changed to be 1 (ie 2147483646 is substracted).
 Implementation note: it also performs 3 calls to Next() to avoid the
 bad property that the first random number is proportional to the seed
 value.

\item  \verb|obj.SetSeedOnly (int value)| -  Set the seed of the random sequence. There is no extra internal
 ajustment. Only useful for writing correctness test.
 The following pre-condition is stated page 1197, second column
 2147483646=(2\^31)-2
 This method does not have this criterium as a pre-condition (ie it will
 not fail if an incorrect seed value is passed) but the value is silently
 changed to fit in the valid range [1,2147483646].
 2147483646 is added to a null or negative value.
 2147483647 is changed to be 1 (ie 2147483646 is substracted).

\item  \verb|int = obj.GetSeed ()| -  Get the seed of the random sequence.
 Only useful for writing correctness test.

\item  \verb|double = obj.GetValue ()| -  Current value
 

\item  \verb|obj.Next ()| -  Move to the next number in the random sequence.

\item  \verb|double = obj.GetRangeValue (double rangeMin, double rangeMax)| -  Convenient method to return a value in a specific range from the
 range [0,1. There is an initial implementation that can be overridden
 by a subclass.
 There is no pre-condition on the range:
 - it can be in increasing order: rangeMin<rangeMax
 - it can be empty: rangeMin=rangeMax
 - it can be in decreasing order: rangeMin>rangeMax
 
 (rangeMin<=rangeMax \&\& result>=rangeMin \&\& result<=rangeMax)
 || (rangeMax<=rangeMin \&\& result>=rangeMax \&\& result<=rangeMin)

\end{itemize}
