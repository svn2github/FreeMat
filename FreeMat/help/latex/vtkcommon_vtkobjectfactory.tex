\section{vtkObjectFactory}

\subsection{Usage}

 vtkObjectFactory is used to create vtk objects.   The base class
 vtkObjectFactory contains a static method CreateInstance which is used
 to create vtk objects from the list of registered vtkObjectFactory 
 sub-classes.   The first time CreateInstance is called, all dll's or shared
 libraries in the environment variable VTK\_AUTOLOAD\_PATH are loaded into
 the current process.   The C functions vtkLoad, vtkGetFactoryCompilerUsed, 
 and vtkGetFactoryVersion are called on each dll.  To implement these
 functions in a shared library or dll, use the macro:
 VTK\_FACTORY\_INTERFACE\_IMPLEMENT.
 VTK\_AUTOLOAD\_PATH is an environment variable 
 containing a colon separated (semi-colon on win32) list of paths.

 The vtkObjectFactory can be use to override the creation of any object
 in VTK with a sub-class of that object.  The factories can be registered
 either at run time with the VTK\_AUTOLOAD\_PATH, or at compile time
 with the vtkObjectFactory::RegisterFactory method.


To create an instance of class vtkObjectFactory, simply
invoke its constructor as follows
\begin{verbatim}
  obj = vtkObjectFactory
\end{verbatim}
\subsection{Methods}

The class vtkObjectFactory has several methods that can be used.
  They are listed below.
Note that the documentation is translated automatically from the VTK sources,
and may not be completely intelligible.  When in doubt, consult the VTK website.
In the methods listed below, \verb|obj| is an instance of the vtkObjectFactory class.
\begin{itemize}
\item  \verb|string = obj.GetClassName ()|

\item  \verb|int = obj.IsA (string name)|

\item  \verb|vtkObjectFactory = obj.NewInstance ()|

\item  \verb|vtkObjectFactory = obj.SafeDownCast (vtkObject o)|

\item  \verb|string = obj.GetVTKSourceVersion ()| -  All sub-classes of vtkObjectFactory should must return the version of 
 VTK they were built with.  This should be implemented with the macro
 VTK\_SOURCE\_VERSION and NOT a call to vtkVersion::GetVTKSourceVersion.
 As the version needs to be compiled into the file as a string constant.
 This is critical to determine possible incompatible dynamic factory loads.

\item  \verb|string = obj.GetDescription ()| -  Return a descriptive string describing the factory.

\item  \verb|int = obj.GetNumberOfOverrides ()| -  Return number of overrides this factory can create.

\item  \verb|string = obj.GetClassOverrideName (int index)| -  Return the name of a class override at the given index.

\item  \verb|string = obj.GetClassOverrideWithName (int index)| -  Return the name of the class that will override the class
 at the given index

\item  \verb|int = obj.GetEnableFlag (int index)| -  Return the enable flag for the class at the given index.

\item  \verb|string = obj.GetOverrideDescription (int index)| -  Return the description for a the class override at the given 
 index.

\item  \verb|obj.SetEnableFlag (int flag, string className, string subclassName)| -  Set and Get the Enable flag for the specific override of className.
 if subclassName is null, then it is ignored.

\item  \verb|int = obj.GetEnableFlag (string className, string subclassName)| -  Set and Get the Enable flag for the specific override of className.
 if subclassName is null, then it is ignored.

\item  \verb|int = obj.HasOverride (string className)| -  Return 1 if this factory overrides the given class name, 0 otherwise.

\item  \verb|int = obj.HasOverride (string className, string subclassName)| -  Return 1 if this factory overrides the given class name, 0 otherwise.

\item  \verb|obj.Disable (string className)| -  Set all enable flags for the given class to 0.  This will
 mean that the factory will stop producing class with the given
 name.

\item  \verb|string = obj.GetLibraryPath ()| -  This returns the path to a dynamically loaded factory.

\end{itemize}
