\section{vtkProperty2D}

\subsection{Usage}

 vtkProperty2D contains properties used to render two dimensional images
 and annotations.

To create an instance of class vtkProperty2D, simply
invoke its constructor as follows
\begin{verbatim}
  obj = vtkProperty2D
\end{verbatim}
\subsection{Methods}

The class vtkProperty2D has several methods that can be used.
  They are listed below.
Note that the documentation is translated automatically from the VTK sources,
and may not be completely intelligible.  When in doubt, consult the VTK website.
In the methods listed below, \verb|obj| is an instance of the vtkProperty2D class.
\begin{itemize}
\item  \verb|string = obj.GetClassName ()|

\item  \verb|int = obj.IsA (string name)|

\item  \verb|vtkProperty2D = obj.NewInstance ()|

\item  \verb|vtkProperty2D = obj.SafeDownCast (vtkObject o)|

\item  \verb|obj.DeepCopy (vtkProperty2D p)| -  Assign one property to another. 

\item  \verb|obj.SetColor (double , double , double )| -  Set/Get the RGB color of this property.

\item  \verb|obj.SetColor (double  a[3])| -  Set/Get the RGB color of this property.

\item  \verb|double = obj. GetColor ()| -  Set/Get the RGB color of this property.

\item  \verb|double = obj.GetOpacity ()| -  Set/Get the Opacity of this property.

\item  \verb|obj.SetOpacity (double )| -  Set/Get the Opacity of this property.

\item  \verb|obj.SetPointSize (float )| -  Set/Get the diameter of a Point. The size is expressed in screen units.
 This is only implemented for OpenGL. The default is 1.0.

\item  \verb|float = obj.GetPointSizeMinValue ()| -  Set/Get the diameter of a Point. The size is expressed in screen units.
 This is only implemented for OpenGL. The default is 1.0.

\item  \verb|float = obj.GetPointSizeMaxValue ()| -  Set/Get the diameter of a Point. The size is expressed in screen units.
 This is only implemented for OpenGL. The default is 1.0.

\item  \verb|float = obj.GetPointSize ()| -  Set/Get the diameter of a Point. The size is expressed in screen units.
 This is only implemented for OpenGL. The default is 1.0.

\item  \verb|obj.SetLineWidth (float )| -  Set/Get the width of a Line. The width is expressed in screen units.
 This is only implemented for OpenGL. The default is 1.0.

\item  \verb|float = obj.GetLineWidthMinValue ()| -  Set/Get the width of a Line. The width is expressed in screen units.
 This is only implemented for OpenGL. The default is 1.0.

\item  \verb|float = obj.GetLineWidthMaxValue ()| -  Set/Get the width of a Line. The width is expressed in screen units.
 This is only implemented for OpenGL. The default is 1.0.

\item  \verb|float = obj.GetLineWidth ()| -  Set/Get the width of a Line. The width is expressed in screen units.
 This is only implemented for OpenGL. The default is 1.0.

\item  \verb|obj.SetLineStipplePattern (int )| -  Set/Get the stippling pattern of a Line, as a 16-bit binary pattern 
 (1 = pixel on, 0 = pixel off).
 This is only implemented for OpenGL. The default is 0xFFFF.

\item  \verb|int = obj.GetLineStipplePattern ()| -  Set/Get the stippling pattern of a Line, as a 16-bit binary pattern 
 (1 = pixel on, 0 = pixel off).
 This is only implemented for OpenGL. The default is 0xFFFF.

\item  \verb|obj.SetLineStippleRepeatFactor (int )| -  Set/Get the stippling repeat factor of a Line, which specifies how
 many times each bit in the pattern is to be repeated.
 This is only implemented for OpenGL. The default is 1.

\item  \verb|int = obj.GetLineStippleRepeatFactorMinValue ()| -  Set/Get the stippling repeat factor of a Line, which specifies how
 many times each bit in the pattern is to be repeated.
 This is only implemented for OpenGL. The default is 1.

\item  \verb|int = obj.GetLineStippleRepeatFactorMaxValue ()| -  Set/Get the stippling repeat factor of a Line, which specifies how
 many times each bit in the pattern is to be repeated.
 This is only implemented for OpenGL. The default is 1.

\item  \verb|int = obj.GetLineStippleRepeatFactor ()| -  Set/Get the stippling repeat factor of a Line, which specifies how
 many times each bit in the pattern is to be repeated.
 This is only implemented for OpenGL. The default is 1.

\item  \verb|obj.SetDisplayLocation (int )| -  The DisplayLocation is either background or foreground.
 If it is background, then this 2D actor will be drawn
 behind all 3D props or foreground 2D actors. If it is
 background, then this 2D actor will be drawn in front of
 all 3D props and background 2D actors. Within 2D actors
 of the same DisplayLocation type, order is determined by
 the order in which the 2D actors were added to the viewport.

\item  \verb|int = obj.GetDisplayLocationMinValue ()| -  The DisplayLocation is either background or foreground.
 If it is background, then this 2D actor will be drawn
 behind all 3D props or foreground 2D actors. If it is
 background, then this 2D actor will be drawn in front of
 all 3D props and background 2D actors. Within 2D actors
 of the same DisplayLocation type, order is determined by
 the order in which the 2D actors were added to the viewport.

\item  \verb|int = obj.GetDisplayLocationMaxValue ()| -  The DisplayLocation is either background or foreground.
 If it is background, then this 2D actor will be drawn
 behind all 3D props or foreground 2D actors. If it is
 background, then this 2D actor will be drawn in front of
 all 3D props and background 2D actors. Within 2D actors
 of the same DisplayLocation type, order is determined by
 the order in which the 2D actors were added to the viewport.

\item  \verb|int = obj.GetDisplayLocation ()| -  The DisplayLocation is either background or foreground.
 If it is background, then this 2D actor will be drawn
 behind all 3D props or foreground 2D actors. If it is
 background, then this 2D actor will be drawn in front of
 all 3D props and background 2D actors. Within 2D actors
 of the same DisplayLocation type, order is determined by
 the order in which the 2D actors were added to the viewport.

\item  \verb|obj.SetDisplayLocationToBackground ()| -  The DisplayLocation is either background or foreground.
 If it is background, then this 2D actor will be drawn
 behind all 3D props or foreground 2D actors. If it is
 background, then this 2D actor will be drawn in front of
 all 3D props and background 2D actors. Within 2D actors
 of the same DisplayLocation type, order is determined by
 the order in which the 2D actors were added to the viewport.

\item  \verb|obj.SetDisplayLocationToForeground ()| -  The DisplayLocation is either background or foreground.
 If it is background, then this 2D actor will be drawn
 behind all 3D props or foreground 2D actors. If it is
 background, then this 2D actor will be drawn in front of
 all 3D props and background 2D actors. Within 2D actors
 of the same DisplayLocation type, order is determined by
 the order in which the 2D actors were added to the viewport.

\end{itemize}
