\section{vtkVariantArray}

\subsection{Usage}


 .SECTION Thanks
 Thanks to Patricia Crossno, Ken Moreland, Andrew Wilson and Brian Wylie from
 Sandia National Laboratories for their help in developing this class.

To create an instance of class vtkVariantArray, simply
invoke its constructor as follows
\begin{verbatim}
  obj = vtkVariantArray
\end{verbatim}
\subsection{Methods}

The class vtkVariantArray has several methods that can be used.
  They are listed below.
Note that the documentation is translated automatically from the VTK sources,
and may not be completely intelligible.  When in doubt, consult the VTK website.
In the methods listed below, \verb|obj| is an instance of the vtkVariantArray class.
\begin{itemize}
\item  \verb|string = obj.GetClassName ()|

\item  \verb|int = obj.IsA (string name)|

\item  \verb|vtkVariantArray = obj.NewInstance ()|

\item  \verb|vtkVariantArray = obj.SafeDownCast (vtkObject o)|

\item  \verb|int = obj.Allocate (vtkIdType sz, vtkIdType ext)| -  Allocate memory for this array. Delete old storage only if necessary.
 Note that ext is no longer used.

\item  \verb|obj.Initialize ()| -  Release storage and reset array to initial state.

\item  \verb|int = obj.GetDataType ()| -  Return the underlying data type. An integer indicating data type is 
 returned as specified in vtkSetGet.h.

\item  \verb|int = obj.GetDataTypeSize ()| -  Return the size of the underlying data type.  For a bit, 1 is
 returned.  For string 0 is returned. Arrays with variable length
 components return 0.

\item  \verb|int = obj.GetElementComponentSize ()| -  Return the size, in bytes, of the lowest-level element of an
 array.  For vtkDataArray and subclasses this is the size of the
 data type.  For vtkStringArray, this is
 sizeof(vtkStdString::value\_type), which winds up being
 sizeof(char).  

\item  \verb|obj.SetNumberOfTuples (vtkIdType number)| -  Set the number of tuples (a component group) in the array. Note that 
 this may allocate space depending on the number of components.

\item  \verb|obj.SetTuple (vtkIdType i, vtkIdType j, vtkAbstractArray source)| -  Set the tuple at the ith location using the jth tuple in the source array.
 This method assumes that the two arrays have the same type
 and structure. Note that range checking and memory allocation is not 
 performed; use in conjunction with SetNumberOfTuples() to allocate space.

\item  \verb|obj.InsertTuple (vtkIdType i, vtkIdType j, vtkAbstractArray source)| -  Insert the jth tuple in the source array, at ith location in this array. 
 Note that memory allocation is performed as necessary to hold the data.

\item  \verb|vtkIdType = obj.InsertNextTuple (vtkIdType j, vtkAbstractArray source)| -  Insert the jth tuple in the source array, at the end in this array. 
 Note that memory allocation is performed as necessary to hold the data.
 Returns the location at which the data was inserted.

\item  \verb|obj.DeepCopy (vtkAbstractArray da)| -  Deep copy of data. Implementation left to subclasses, which
 should support as many type conversions as possible given the
 data type.

\item  \verb|obj.InterpolateTuple (vtkIdType i, vtkIdList ptIndices, vtkAbstractArray source, double weights)| -  Set the ith tuple in this array as the interpolated tuple value,
 given the ptIndices in the source array and associated 
 interpolation weights.
 This method assumes that the two arrays are of the same type
 and strcuture.

\item  \verb|obj.InterpolateTuple (vtkIdType i, vtkIdType id1, vtkAbstractArray source1, vtkIdType id2, vtkAbstractArray source2, double t)|

\item  \verb|obj.Squeeze ()| -  Resize object to just fit data requirement. Reclaims extra memory.

\item  \verb|int = obj.Resize (vtkIdType numTuples)| -  Resize the array while conserving the data.  Returns 1 if
 resizing succeeded and 0 otherwise.

\item  \verb|long = obj.GetActualMemorySize ()| -  Return the memory in kilobytes consumed by this data array. Used to
 support streaming and reading/writing data. The value returned is
 guaranteed to be greater than or equal to the memory required to
 actually represent the data represented by this object. The 
 information returned is valid only after the pipeline has 
 been updated.

\item  \verb|int = obj.IsNumeric ()| -  Since each item can be of a different type, we say that a variant array is not numeric.

\item  \verb|vtkArrayIterator = obj.NewIterator ()| -  Subclasses must override this method and provide the right 
 kind of templated vtkArrayIteratorTemplate.

\item  \verb|obj.SetNumberOfValues (vtkIdType number)| -  Specify the number of values for this object to hold. Does an
 allocation as well as setting the MaxId ivar. Used in conjunction with
 SetValue() method for fast insertion.

\item  \verb|vtkIdType = obj.GetNumberOfValues ()| -  Tell the array explicitly that the data has changed.
 This is only necessary to call when you modify the array contents
 without using the array's API (i.e. you retrieve a pointer to the
 data and modify the array contents).  You need to call this so that
 the fast lookup will know to rebuild itself.  Otherwise, the lookup
 functions will give incorrect results.

\item  \verb|obj.DataChanged ()| -  Tell the array explicitly that the data has changed.
 This is only necessary to call when you modify the array contents
 without using the array's API (i.e. you retrieve a pointer to the
 data and modify the array contents).  You need to call this so that
 the fast lookup will know to rebuild itself.  Otherwise, the lookup
 functions will give incorrect results.

\item  \verb|obj.DataElementChanged (vtkIdType id)| -  Tell the array explicitly that a single data element has
 changed. Like DataChanged(), then is only necessary when you
 modify the array contents without using the array's API. 

\item  \verb|obj.ClearLookup ()| -  Delete the associated fast lookup data structure on this array,
 if it exists.  The lookup will be rebuilt on the next call to a lookup
 function.

\item  \verb|~vtkVariantArray = obj.()| -  This destructor is public to work around a bug in version 1.36.0 of
 the Boost.Serialization library.

\end{itemize}
