\section{vtkWindow}

\subsection{Usage}

 vtkWindow is an abstract object to specify the behavior of a
 rendering window.  It contains vtkViewports.

To create an instance of class vtkWindow, simply
invoke its constructor as follows
\begin{verbatim}
  obj = vtkWindow
\end{verbatim}
\subsection{Methods}

The class vtkWindow has several methods that can be used.
  They are listed below.
Note that the documentation is translated automatically from the VTK sources,
and may not be completely intelligible.  When in doubt, consult the VTK website.
In the methods listed below, \verb|obj| is an instance of the vtkWindow class.
\begin{itemize}
\item  \verb|string = obj.GetClassName ()|

\item  \verb|int = obj.IsA (string name)|

\item  \verb|vtkWindow = obj.NewInstance ()|

\item  \verb|vtkWindow = obj.SafeDownCast (vtkObject o)|

\item  \verb|obj.SetWindowInfo (string )| -  These are window system independent methods that are used
 to help interface vtkWindow to native windowing systems.

\item  \verb|obj.SetParentInfo (string )| -  These are window system independent methods that are used
 to help interface vtkWindow to native windowing systems.

\item  \verb|int = obj.GetPosition ()| -  Set/Get the position in screen coordinates of the rendering window.

\item  \verb|obj.SetPosition (int , int )| -  Set/Get the position in screen coordinates of the rendering window.

\item  \verb|obj.SetPosition (int a[2])| -  Set/Get the position in screen coordinates of the rendering window.

\item  \verb|int = obj.GetSize ()| -  Set/Get the size of the window in screen coordinates in pixels.

\item  \verb|obj.SetSize (int , int )| -  Set/Get the size of the window in screen coordinates in pixels.

\item  \verb|obj.SetSize (int a[2])| -  Set/Get the size of the window in screen coordinates in pixels.

\item  \verb|obj.SetMapped (int )| -  Keep track of whether the rendering window has been mapped to screen.

\item  \verb|int = obj.GetMapped ()| -  Keep track of whether the rendering window has been mapped to screen.

\item  \verb|obj.MappedOn ()| -  Keep track of whether the rendering window has been mapped to screen.

\item  \verb|obj.MappedOff ()| -  Keep track of whether the rendering window has been mapped to screen.

\item  \verb|obj.SetErase (int )| -  Turn on/off erasing the screen between images. This allows multiple 
 exposure sequences if turned on. You will need to turn double 
 buffering off or make use of the SwapBuffers methods to prevent
 you from swapping buffers between exposures.

\item  \verb|int = obj.GetErase ()| -  Turn on/off erasing the screen between images. This allows multiple 
 exposure sequences if turned on. You will need to turn double 
 buffering off or make use of the SwapBuffers methods to prevent
 you from swapping buffers between exposures.

\item  \verb|obj.EraseOn ()| -  Turn on/off erasing the screen between images. This allows multiple 
 exposure sequences if turned on. You will need to turn double 
 buffering off or make use of the SwapBuffers methods to prevent
 you from swapping buffers between exposures.

\item  \verb|obj.EraseOff ()| -  Turn on/off erasing the screen between images. This allows multiple 
 exposure sequences if turned on. You will need to turn double 
 buffering off or make use of the SwapBuffers methods to prevent
 you from swapping buffers between exposures.

\item  \verb|obj.SetDoubleBuffer (int )| -  Keep track of whether double buffering is on or off

\item  \verb|int = obj.GetDoubleBuffer ()| -  Keep track of whether double buffering is on or off

\item  \verb|obj.DoubleBufferOn ()| -  Keep track of whether double buffering is on or off

\item  \verb|obj.DoubleBufferOff ()| -  Keep track of whether double buffering is on or off

\item  \verb|string = obj.GetWindowName ()| -  Get name of rendering window

\item  \verb|obj.SetWindowName (string )| -  Get name of rendering window

\item  \verb|obj.Render ()| -  Ask each viewport owned by this Window to render its image and 
 synchronize this process.

\item  \verb|int = obj.GetPixelData (int x, int y, int x2, int y2, int front, vtkUnsignedCharArray data)| -  Get the pixel data of an image, transmitted as RGBRGBRGB. The
 front argument indicates if the front buffer should be used or the back 
 buffer. It is the caller's responsibility to delete the resulting 
 array. It is very important to realize that the memory in this array
 is organized from the bottom of the window to the top. The origin
 of the screen is in the lower left corner. The y axis increases as
 you go up the screen. So the storage of pixels is from left to right
 and from bottom to top.
 (x,y) is any corner of the rectangle. (x2,y2) is its opposite corner on
 the diagonal.

\item  \verb|int = obj.GetDPI ()| -  Return a best estimate to the dots per inch of the display
 device being rendered (or printed).

\item  \verb|obj.SetDPI (int )| -  Return a best estimate to the dots per inch of the display
 device being rendered (or printed).

\item  \verb|int = obj.GetDPIMinValue ()| -  Return a best estimate to the dots per inch of the display
 device being rendered (or printed).

\item  \verb|int = obj.GetDPIMaxValue ()| -  Return a best estimate to the dots per inch of the display
 device being rendered (or printed).

\item  \verb|obj.SetOffScreenRendering (int )| -  Create a window in memory instead of on the screen. This may not be
 supported for every type of window and on some windows you may need to
 invoke this prior to the first render.

\item  \verb|int = obj.GetOffScreenRendering ()| -  Create a window in memory instead of on the screen. This may not be
 supported for every type of window and on some windows you may need to
 invoke this prior to the first render.

\item  \verb|obj.OffScreenRenderingOn ()| -  Create a window in memory instead of on the screen. This may not be
 supported for every type of window and on some windows you may need to
 invoke this prior to the first render.

\item  \verb|obj.OffScreenRenderingOff ()| -  Create a window in memory instead of on the screen. This may not be
 supported for every type of window and on some windows you may need to
 invoke this prior to the first render.

\item  \verb|obj.MakeCurrent ()| -  Make the window current. May be overridden in subclasses to do
 for example a glXMakeCurrent or a wglMakeCurrent.

\item  \verb|obj.SetTileScale (int , int )| -  These methods are used by vtkWindowToImageFilter to tell a VTK window
 to simulate a larger window by tiling. For 3D geometry these methods
 have no impact. It is just in handling annotation that this information
 must be available to the mappers and the coordinate calculations.

\item  \verb|obj.SetTileScale (int  a[2])| -  These methods are used by vtkWindowToImageFilter to tell a VTK window
 to simulate a larger window by tiling. For 3D geometry these methods
 have no impact. It is just in handling annotation that this information
 must be available to the mappers and the coordinate calculations.

\item  \verb|int = obj. GetTileScale ()| -  These methods are used by vtkWindowToImageFilter to tell a VTK window
 to simulate a larger window by tiling. For 3D geometry these methods
 have no impact. It is just in handling annotation that this information
 must be available to the mappers and the coordinate calculations.

\item  \verb|obj.SetTileScale (int s)| -  These methods are used by vtkWindowToImageFilter to tell a VTK window
 to simulate a larger window by tiling. For 3D geometry these methods
 have no impact. It is just in handling annotation that this information
 must be available to the mappers and the coordinate calculations.

\item  \verb|obj.SetTileViewport (double , double , double , double )| -  These methods are used by vtkWindowToImageFilter to tell a VTK window
 to simulate a larger window by tiling. For 3D geometry these methods
 have no impact. It is just in handling annotation that this information
 must be available to the mappers and the coordinate calculations.

\item  \verb|obj.SetTileViewport (double  a[4])| -  These methods are used by vtkWindowToImageFilter to tell a VTK window
 to simulate a larger window by tiling. For 3D geometry these methods
 have no impact. It is just in handling annotation that this information
 must be available to the mappers and the coordinate calculations.

\item  \verb|double = obj. GetTileViewport ()| -  These methods are used by vtkWindowToImageFilter to tell a VTK window
 to simulate a larger window by tiling. For 3D geometry these methods
 have no impact. It is just in handling annotation that this information
 must be available to the mappers and the coordinate calculations.

\end{itemize}
