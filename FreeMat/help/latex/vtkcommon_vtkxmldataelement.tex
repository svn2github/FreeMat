\section{vtkXMLDataElement}

\subsection{Usage}

 vtkXMLDataElement is used by vtkXMLDataParser to represent an XML
 element.  It provides methods to access the element's attributes
 and nested elements in a convenient manner.  This allows easy
 traversal of an input XML file by vtkXMLReader and its subclasses.

To create an instance of class vtkXMLDataElement, simply
invoke its constructor as follows
\begin{verbatim}
  obj = vtkXMLDataElement
\end{verbatim}
\subsection{Methods}

The class vtkXMLDataElement has several methods that can be used.
  They are listed below.
Note that the documentation is translated automatically from the VTK sources,
and may not be completely intelligible.  When in doubt, consult the VTK website.
In the methods listed below, \verb|obj| is an instance of the vtkXMLDataElement class.
\begin{itemize}
\item  \verb|string = obj.GetClassName ()|

\item  \verb|int = obj.IsA (string name)|

\item  \verb|vtkXMLDataElement = obj.NewInstance ()|

\item  \verb|vtkXMLDataElement = obj.SafeDownCast (vtkObject o)|

\item  \verb|string = obj.GetName ()| -  Set/Get the name of the element.  This is its XML tag.

\item  \verb|obj.SetName (string \_arg)| -  Set/Get the name of the element.  This is its XML tag.

\item  \verb|string = obj.GetId ()| -  Set/Get the value of the id attribute of the element, if any.

\item  \verb|obj.SetId (string )| -  Set/Get the value of the id attribute of the element, if any.

\item  \verb|string = obj.GetAttribute (string name)| -  Get the attribute with the given name.  If it doesn't exist,
 returns 0.

\item  \verb|obj.SetAttribute (string name, string value)| -  Set the attribute with the given name and value. If it doesn't exist,
 adds it.

\item  \verb|obj.SetCharacterData (string c, int length)| -  Set/Get the character data between XML start/end tags.

\item  \verb|string = obj.GetCharacterData ()| -  Set/Get the character data between XML start/end tags.

\item  \verb|obj.SetIntAttribute (string name, int value)| -  Set the attribute with the given name.
 We can not use the same GetScalarAttribute() construct since
 the compiler will not be able to resolve between
 SetAttribute(..., int) and SetAttribute(..., unsigned long).

\item  \verb|obj.SetFloatAttribute (string name, float value)| -  Set the attribute with the given name.
 We can not use the same GetScalarAttribute() construct since
 the compiler will not be able to resolve between
 SetAttribute(..., int) and SetAttribute(..., unsigned long).

\item  \verb|obj.SetDoubleAttribute (string name, double value)| -  Set the attribute with the given name.
 We can not use the same GetScalarAttribute() construct since
 the compiler will not be able to resolve between
 SetAttribute(..., int) and SetAttribute(..., unsigned long).

\item  \verb|obj.SetUnsignedLongAttribute (string name, long value)| -  Set the attribute with the given name.
 We can not use the same GetScalarAttribute() construct since
 the compiler will not be able to resolve between
 SetAttribute(..., int) and SetAttribute(..., unsigned long).

\item  \verb|int = obj.GetVectorAttribute (string name, int length, int value)| -  Get the attribute with the given name and converted to a scalar
 value.  Returns length of vector read.

\item  \verb|int = obj.GetVectorAttribute (string name, int length, float value)| -  Get the attribute with the given name and converted to a scalar
 value.  Returns length of vector read.

\item  \verb|int = obj.GetVectorAttribute (string name, int length, double value)| -  Get the attribute with the given name and converted to a scalar
 value.  Returns length of vector read.

\item  \verb|int = obj.GetVectorAttribute (string name, int length, long value)| -  Get the attribute with the given name and converted to a scalar
 value.  Returns length of vector read.

\item  \verb|obj.SetVectorAttribute (string name, int length, int value)| -  Set the attribute with the given name.

\item  \verb|obj.SetVectorAttribute (string name, int length, float value)| -  Set the attribute with the given name.

\item  \verb|obj.SetVectorAttribute (string name, int length, double value)| -  Set the attribute with the given name.

\item  \verb|obj.SetVectorAttribute (string name, int length, long value)| -  Set the attribute with the given name.

\item  \verb|int = obj.GetNumberOfAttributes ()| -  Get the number of attributes.

\item  \verb|string = obj.GetAttributeName (int idx)| -  Get the n-th attribute name.
 Returns 0 if there is no such attribute.

\item  \verb|string = obj.GetAttributeValue (int idx)| -  Get the n-th attribute value.
 Returns 0 if there is no such attribute.

\item  \verb|obj.RemoveAttribute (string name)| -  Remove one or all attributes.

\item  \verb|obj.RemoveAllAttributes ()| -  Remove one or all attributes.

\item  \verb|vtkXMLDataElement = obj.GetParent ()| -  Set/Get the parent of this element.

\item  \verb|obj.SetParent (vtkXMLDataElement parent)| -  Set/Get the parent of this element.

\item  \verb|vtkXMLDataElement = obj.GetRoot ()| -  Get root of the XML tree this element is part of.

\item  \verb|int = obj.GetNumberOfNestedElements ()| -  Get the number of elements nested in this one.

\item  \verb|vtkXMLDataElement = obj.GetNestedElement (int index)| -  Get the element nested in this one at the given index.

\item  \verb|obj.AddNestedElement (vtkXMLDataElement element)| -  Add nested element

\item  \verb|obj.RemoveNestedElement (vtkXMLDataElement )| -  Remove nested element.

\item  \verb|obj.RemoveAllNestedElements ()| -  Remove all nested elements.

\item  \verb|vtkXMLDataElement = obj.FindNestedElement (string id)| -  Find the first nested element with the given id, given name, or given
 name and id.
 WARNING: the search is only performed on the children, not 
 the grand-children.

\item  \verb|vtkXMLDataElement = obj.FindNestedElementWithName (string name)| -  Find the first nested element with the given id, given name, or given
 name and id.
 WARNING: the search is only performed on the children, not 
 the grand-children.

\item  \verb|vtkXMLDataElement = obj.FindNestedElementWithNameAndId (string name, string id)| -  Find the first nested element with the given id, given name, or given
 name and id.
 WARNING: the search is only performed on the children, not 
 the grand-children.

\item  \verb|vtkXMLDataElement = obj.FindNestedElementWithNameAndAttribute (string name, string att\_name, string att\_value)| -  Find the first nested element with the given id, given name, or given
 name and id.
 WARNING: the search is only performed on the children, not 
 the grand-children.

\item  \verb|vtkXMLDataElement = obj.LookupElementWithName (string name)| -  Find the first nested element with given name.
 WARNING: the search is performed on the whole XML tree.

\item  \verb|vtkXMLDataElement = obj.LookupElement (string id)| -  Lookup the element with the given id, starting at this scope.

\item  \verb|long = obj.GetXMLByteIndex ()| -  Set/Get the offset from the beginning of the XML document to this element.

\item  \verb|obj.SetXMLByteIndex (long )| -  Set/Get the offset from the beginning of the XML document to this element.

\item  \verb|int = obj.IsEqualTo (vtkXMLDataElement elem)| -  Check if the instance has the same name, attributes, character data
 and nested elements contents than the given element (this method is
 applied recursively on the nested elements, and they must be stored in
 the same order).
 Warning: Id, Parent, XMLByteIndex are ignored.

\item  \verb|obj.DeepCopy (vtkXMLDataElement elem)| -  Copy this element from another of the same type (elem), recursively.
 Old attributes and nested elements are removed, new ones are created
 given the contents of 'elem'.
 Warning: Parent is ignored.

\item  \verb|obj.SetAttributeEncoding (int )| -  Get/Set the internal character encoding of the attributes.
 Default type is VTK\_ENCODING\_UTF\_8.
 Note that a vtkXMLDataParser has its own AttributesEncoding ivar. If 
 this ivar is set to something other than VTK\_ENCODING\_NONE, it will be
 used to set the attribute encoding of each vtkXMLDataElement 
 created by this vtkXMLDataParser.

\item  \verb|int = obj.GetAttributeEncodingMinValue ()| -  Get/Set the internal character encoding of the attributes.
 Default type is VTK\_ENCODING\_UTF\_8.
 Note that a vtkXMLDataParser has its own AttributesEncoding ivar. If 
 this ivar is set to something other than VTK\_ENCODING\_NONE, it will be
 used to set the attribute encoding of each vtkXMLDataElement 
 created by this vtkXMLDataParser.

\item  \verb|int = obj.GetAttributeEncodingMaxValue ()| -  Get/Set the internal character encoding of the attributes.
 Default type is VTK\_ENCODING\_UTF\_8.
 Note that a vtkXMLDataParser has its own AttributesEncoding ivar. If 
 this ivar is set to something other than VTK\_ENCODING\_NONE, it will be
 used to set the attribute encoding of each vtkXMLDataElement 
 created by this vtkXMLDataParser.

\item  \verb|int = obj.GetAttributeEncoding ()| -  Get/Set the internal character encoding of the attributes.
 Default type is VTK\_ENCODING\_UTF\_8.
 Note that a vtkXMLDataParser has its own AttributesEncoding ivar. If 
 this ivar is set to something other than VTK\_ENCODING\_NONE, it will be
 used to set the attribute encoding of each vtkXMLDataElement 
 created by this vtkXMLDataParser.

\item  \verb|obj.PrintXML (string fname)| -  Prints element tree as XML.

\item  \verb|int = obj.GetCharacterDataWidth ()| -  Get/Set the width (in number of fields) that character
 data (that between open and closing tags ie. <X> ... </X>)
 is printed. If the width is less than one the tag's character
 data is printed all on one line. If it is greater than one
 the character data is streamed insterting line feeds every
 width number of fields. See PrintXML.

\item  \verb|obj.SetCharacterDataWidth (int )| -  Get/Set the width (in number of fields) that character
 data (that between open and closing tags ie. <X> ... </X>)
 is printed. If the width is less than one the tag's character
 data is printed all on one line. If it is greater than one
 the character data is streamed insterting line feeds every
 width number of fields. See PrintXML.

\end{itemize}
