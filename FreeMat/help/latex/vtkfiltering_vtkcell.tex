\section{vtkCell}

\subsection{Usage}

 vtkCell is an abstract class that specifies the interfaces for data cells.
 Data cells are simple topological elements like points, lines, polygons, 
 and tetrahedra of which visualization datasets are composed. In some 
 cases visualization datasets may explicitly represent cells (e.g., 
 vtkPolyData, vtkUnstructuredGrid), and in some cases, the datasets are 
 implicitly composed of cells (e.g., vtkStructuredPoints).


To create an instance of class vtkCell, simply
invoke its constructor as follows
\begin{verbatim}
  obj = vtkCell
\end{verbatim}
\subsection{Methods}

The class vtkCell has several methods that can be used.
  They are listed below.
Note that the documentation is translated automatically from the VTK sources,
and may not be completely intelligible.  When in doubt, consult the VTK website.
In the methods listed below, \verb|obj| is an instance of the vtkCell class.
\begin{itemize}
\item  \verb|string = obj.GetClassName ()|

\item  \verb|int = obj.IsA (string name)|

\item  \verb|vtkCell = obj.NewInstance ()|

\item  \verb|vtkCell = obj.SafeDownCast (vtkObject o)|

\item  \verb|obj.ShallowCopy (vtkCell c)| -  Copy this cell by reference counting the internal data structures. 
 This is safe if you want a ''read-only'' copy. If you modify the cell
 you might wish to use DeepCopy().

\item  \verb|obj.DeepCopy (vtkCell c)| -  Copy this cell by completely copying internal data structures. This is
 slower but safer than ShallowCopy().

\item  \verb|int = obj.GetCellType ()| -  Return the type of cell.

\item  \verb|int = obj.GetCellDimension ()| -  Return the topological dimensional of the cell (0,1,2, or 3).

\item  \verb|int = obj.IsLinear ()| -  Some cells require initialization prior to access. For example, they
 may have to triangulate themselves or set up internal data structures.

\item  \verb|int = obj.RequiresInitialization ()| -  Some cells require initialization prior to access. For example, they
 may have to triangulate themselves or set up internal data structures.

\item  \verb|obj.Initialize ()| -  Explicit cells require additional representational information
 beyond the usual cell type and connectivity list information.
 Most cells in VTK are implicit cells.

\item  \verb|int = obj.IsExplicitCell ()| -  Get the point coordinates for the cell.

\item  \verb|vtkPoints = obj.GetPoints ()| -  Return the number of points in the cell.

\item  \verb|vtkIdType = obj.GetNumberOfPoints ()| -  Return the number of edges in the cell.

\item  \verb|int = obj.GetNumberOfEdges ()| -  Return the number of edges in the cell.

\item  \verb|int = obj.GetNumberOfFaces ()| -  Return the number of faces in the cell.

\item  \verb|vtkIdList = obj.GetPointIds ()| -  For cell point i, return the actual point id.

\item  \verb|vtkIdType = obj.GetPointId (int ptId)| -  Return the edge cell from the edgeId of the cell.

\item  \verb|vtkCell = obj.GetEdge (int edgeId)| -  Return the edge cell from the edgeId of the cell.

\item  \verb|vtkCell = obj.GetFace (int faceId)| -  Return the face cell from the faceId of the cell.

\item  \verb|int = obj.CellBoundary (int subId, double pcoords[3], vtkIdList pts)| -  Given parametric coordinates of a point, return the closest cell
 boundary, and whether the point is inside or outside of the cell. The
 cell boundary is defined by a list of points (pts) that specify a face
 (3D cell), edge (2D cell), or vertex (1D cell). If the return value of
 the method is != 0, then the point is inside the cell.

\item  \verb|obj.Contour (double value, vtkDataArray cellScalars, vtkIncrementalPointLocator locator, vtkCellArray verts, vtkCellArray lines, vtkCellArray polys, vtkPointData inPd, vtkPointData outPd, vtkCellData inCd, vtkIdType cellId, vtkCellData outCd)| -  Generate contouring primitives. The scalar list cellScalars are
 scalar values at each cell point. The point locator is essentially a 
 points list that merges points as they are inserted (i.e., prevents 
 duplicates). Contouring primitives can be vertices, lines, or
 polygons. It is possible to interpolate point data along the edge
 by providing input and output point data - if outPd is NULL, then
 no interpolation is performed. Also, if the output cell data is
 non-NULL, the cell data from the contoured cell is passed to the
 generated contouring primitives. (Note: the CopyAllocate() method
 must be invoked on both the output cell and point data. The 
 cellId refers to the cell from which the cell data is copied.)

\item  \verb|obj.Clip (double value, vtkDataArray cellScalars, vtkIncrementalPointLocator locator, vtkCellArray connectivity, vtkPointData inPd, vtkPointData outPd, vtkCellData inCd, vtkIdType cellId, vtkCellData outCd, int insideOut)| -  Cut (or clip) the cell based on the input cellScalars and the
 specified value. The output of the clip operation will be one or
 more cells of the same topological dimension as the original cell. 
 The flag insideOut controls what part of the cell is considered inside - 
 normally cell points whose scalar value is greater than ''value'' are
 considered inside. If insideOut is on, this is reversed. Also, if the 
 output cell data is non-NULL, the cell data from the clipped cell is 
 passed to the generated contouring primitives. (Note: the CopyAllocate() 
 method must be invoked on both the output cell and point data. The
 cellId refers to the cell from which the cell data is copied.)

\item  \verb|int = obj.Triangulate (int index, vtkIdList ptIds, vtkPoints pts)| -  Generate simplices of proper dimension. If cell is 3D, tetrahedron are 
 generated; if 2D triangles; if 1D lines; if 0D points. The form of the
 output is a sequence of points, each n+1 points (where n is topological 
 cell dimension) defining a simplex. The index is a parameter that controls
 which triangulation to use (if more than one is possible). If numerical
 degeneracy encountered, 0 is returned, otherwise 1 is returned.
 This method does not insert new points: all the points that define the
 simplices are the points that define the cell.

\item  \verb|obj.Derivatives (int subId, double pcoords[3], double values, int dim, double derivs)| -  Compute derivatives given cell subId and parametric coordinates. The
 values array is a series of data value(s) at the cell points. There is a
 one-to-one correspondence between cell point and data value(s). Dim is
 the number of data values per cell point. Derivs are derivatives in the
 x-y-z coordinate directions for each data value. Thus, if computing
 derivatives for a scalar function in a hexahedron, dim=1, 8 values are
 supplied, and 3 deriv values are returned (i.e., derivatives in x-y-z
 directions). On the other hand, if computing derivatives of velocity
 (vx,vy,vz) dim=3, 24 values are supplied ((vx,vy,vz)1, (vx,vy,vz)2,
 ....()8), and 9 deriv values are returned
 ((d(vx)/dx),(d(vx)/dy),(d(vx)/dz), (d(vy)/dx),(d(vy)/dy), (d(vy)/dz),
 (d(vz)/dx),(d(vz)/dy),(d(vz)/dz)).

\item  \verb|obj.GetBounds (double bounds[6])| -  Compute cell bounding box (xmin,xmax,ymin,ymax,zmin,zmax). Copy result
 into user provided array.

\item  \verb|double = obj.GetBounds ()| -  Compute cell bounding box (xmin,xmax,ymin,ymax,zmin,zmax). Return pointer
 to array of six double values.

\item  \verb|double = obj.GetLength2 ()| -  Compute Length squared of cell (i.e., bounding box diagonal squared).

\item  \verb|int = obj.GetParametricCenter (double pcoords[3])| -  Return center of the cell in parametric coordinates.  Note that the
 parametric center is not always located at (0.5,0.5,0.5). The return
 value is the subId that the center is in (if a composite cell). If you
 want the center in x-y-z space, invoke the EvaluateLocation() method.

\item  \verb|double = obj.GetParametricDistance (double pcoords[3])| -  Return the distance of the parametric coordinate provided to the
 cell. If inside the cell, a distance of zero is returned. This is
 used during picking to get the correct cell picked. (The tolerance
 will occasionally allow cells to be picked who are not really
 intersected ''inside'' the cell.)

\item  \verb|int = obj.IsPrimaryCell ()| -  Return a contiguous array of parametric coordinates of the points
 defining this cell. In other words, (px,py,pz, px,py,pz, etc..)  The
 coordinates are ordered consistent with the definition of the point
 ordering for the cell. This method returns a non-NULL pointer when
 the cell is a primary type (i.e., IsPrimaryCell() is true). Note that
 3D parametric coordinates are returned no matter what the topological
 dimension of the cell.

\item  \verb|obj.InterpolateFunctions (double pcoords[3], double weights[3])| -  Compute the interpolation functions/derivatives
 (aka shape functions/derivatives)
 No-ops at this level. Typically overridden in subclasses.

\item  \verb|obj.InterpolateDerivs (double pcoords[3], double derivs[3])|

\end{itemize}
