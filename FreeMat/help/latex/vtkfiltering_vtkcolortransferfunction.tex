\section{vtkColorTransferFunction}

\subsection{Usage}

 vtkColorTransferFunction is a color mapping in RGB or HSV space that
 uses piecewise hermite functions to allow interpolation that can be
 piecewise constant, piecewise linear, or somewhere in-between 
 (a modified piecewise hermite function that squishes the function
 according to a sharpness parameter). The function also allows for
 the specification of the midpoint (the place where the function
 reaches the average of the two bounding nodes) as a normalize distance
 between nodes.
 See the description of class vtkPiecewiseFunction for an explanation of
 midpoint and sharpness.

To create an instance of class vtkColorTransferFunction, simply
invoke its constructor as follows
\begin{verbatim}
  obj = vtkColorTransferFunction
\end{verbatim}
\subsection{Methods}

The class vtkColorTransferFunction has several methods that can be used.
  They are listed below.
Note that the documentation is translated automatically from the VTK sources,
and may not be completely intelligible.  When in doubt, consult the VTK website.
In the methods listed below, \verb|obj| is an instance of the vtkColorTransferFunction class.
\begin{itemize}
\item  \verb|string = obj.GetClassName ()|

\item  \verb|int = obj.IsA (string name)|

\item  \verb|vtkColorTransferFunction = obj.NewInstance ()|

\item  \verb|vtkColorTransferFunction = obj.SafeDownCast (vtkObject o)|

\item  \verb|obj.DeepCopy (vtkColorTransferFunction f)|

\item  \verb|obj.ShallowCopy (vtkColorTransferFunction f)|

\item  \verb|int = obj.GetSize ()| -  How many points are there defining this function?

\item  \verb|int = obj.AddRGBPoint (double x, double r, double g, double b)| -  Add/Remove a point to/from the function defined in RGB or HSV
 Return the index of the point (0 based), or -1 on error.
 See the description of class vtkPiecewiseFunction for an explanation of
 midpoint and sharpness.

\item  \verb|int = obj.AddRGBPoint (double x, double r, double g, double b, double midpoint, double sharpness)| -  Add/Remove a point to/from the function defined in RGB or HSV
 Return the index of the point (0 based), or -1 on error.
 See the description of class vtkPiecewiseFunction for an explanation of
 midpoint and sharpness.

\item  \verb|int = obj.AddHSVPoint (double x, double h, double s, double v)| -  Add/Remove a point to/from the function defined in RGB or HSV
 Return the index of the point (0 based), or -1 on error.
 See the description of class vtkPiecewiseFunction for an explanation of
 midpoint and sharpness.

\item  \verb|int = obj.AddHSVPoint (double x, double h, double s, double v, double midpoint, double sharpness)| -  Add/Remove a point to/from the function defined in RGB or HSV
 Return the index of the point (0 based), or -1 on error.
 See the description of class vtkPiecewiseFunction for an explanation of
 midpoint and sharpness.

\item  \verb|int = obj.RemovePoint (double x)| -  Add/Remove a point to/from the function defined in RGB or HSV
 Return the index of the point (0 based), or -1 on error.
 See the description of class vtkPiecewiseFunction for an explanation of
 midpoint and sharpness.

\item  \verb|obj.AddRGBSegment (double x1, double r1, double g1, double b1, double x2, double r2, double g2, double b2)| -  Add two points to the function and remove all the points 
 between them

\item  \verb|obj.AddHSVSegment (double x1, double h1, double s1, double v1, double x2, double h2, double s2, double v2)| -  Add two points to the function and remove all the points 
 between them

\item  \verb|obj.RemoveAllPoints ()| -  Remove all points

\item  \verb|double = obj.GetColor (double x)| -  Returns an RGB color for the specified scalar value 

\item  \verb|obj.GetColor (double x, double rgb[3])| -  Returns an RGB color for the specified scalar value 

\item  \verb|double = obj.GetRedValue (double x)| -  Get the color components individually.

\item  \verb|double = obj.GetGreenValue (double x)| -  Get the color components individually.

\item  \verb|double = obj.GetBlueValue (double x)| -  Get the color components individually.

\item  \verb|int = obj.GetNodeValue (int index, double val[6])| -  For the node specified by index, set/get the
 location (X), R, G, and B values, midpoint, and 
 sharpness values at the node.

\item  \verb|int = obj.SetNodeValue (int index, double val[6])| -  For the node specified by index, set/get the
 location (X), R, G, and B values, midpoint, and 
 sharpness values at the node.

\item  \verb|double = obj. GetRange ()| -  Returns min and max position of all function points.

\item  \verb|int = obj.AdjustRange (double range[2])| -  Remove all points out of the new range, and make sure there is a point
 at each end of that range.
 Return 1 on success, 0 otherwise.

\item  \verb|obj.GetTable (double x1, double x2, int n, double table)| -  Fills in a table of n function values between x1 and x2

\item  \verb|obj.GetTable (double x1, double x2, int n, float table)| -  Fills in a table of n function values between x1 and x2

\item  \verb|obj.BuildFunctionFromTable (double x1, double x2, int size, double table)| -  Construct a color transfer function from a table. Function range is
 is set to [x1, x2], each function size is set to size, and function 
 points are regularly spaced between x1 and x2. Parameter ''table'' is 
 assumed to be a block of memory of size [3*size]

\item  \verb|obj.SetClamping (int )| -  Sets and gets the clamping value for this transfer function.

\item  \verb|int = obj.GetClampingMinValue ()| -  Sets and gets the clamping value for this transfer function.

\item  \verb|int = obj.GetClampingMaxValue ()| -  Sets and gets the clamping value for this transfer function.

\item  \verb|int = obj.GetClamping ()| -  Sets and gets the clamping value for this transfer function.

\item  \verb|obj.ClampingOn ()| -  Sets and gets the clamping value for this transfer function.

\item  \verb|obj.ClampingOff ()| -  Sets and gets the clamping value for this transfer function.

\item  \verb|obj.SetColorSpace (int )| -  Set/Get the color space used for interpolation: RGB, HSV, CIELAB, or
 Diverging.  In HSV mode, if HSVWrap is on, it will take the shortest path
 in Hue (going back through 0 if that is the shortest way around the hue
 circle) whereas if HSVWrap is off it will not go through 0 (in order the
 match the current functionality of vtkLookupTable).  Diverging is a special
 mode where colors will pass through white when interpolating between two
 saturated colors.

\item  \verb|int = obj.GetColorSpaceMinValue ()| -  Set/Get the color space used for interpolation: RGB, HSV, CIELAB, or
 Diverging.  In HSV mode, if HSVWrap is on, it will take the shortest path
 in Hue (going back through 0 if that is the shortest way around the hue
 circle) whereas if HSVWrap is off it will not go through 0 (in order the
 match the current functionality of vtkLookupTable).  Diverging is a special
 mode where colors will pass through white when interpolating between two
 saturated colors.

\item  \verb|int = obj.GetColorSpaceMaxValue ()| -  Set/Get the color space used for interpolation: RGB, HSV, CIELAB, or
 Diverging.  In HSV mode, if HSVWrap is on, it will take the shortest path
 in Hue (going back through 0 if that is the shortest way around the hue
 circle) whereas if HSVWrap is off it will not go through 0 (in order the
 match the current functionality of vtkLookupTable).  Diverging is a special
 mode where colors will pass through white when interpolating between two
 saturated colors.

\item  \verb|obj.SetColorSpaceToRGB ()| -  Set/Get the color space used for interpolation: RGB, HSV, CIELAB, or
 Diverging.  In HSV mode, if HSVWrap is on, it will take the shortest path
 in Hue (going back through 0 if that is the shortest way around the hue
 circle) whereas if HSVWrap is off it will not go through 0 (in order the
 match the current functionality of vtkLookupTable).  Diverging is a special
 mode where colors will pass through white when interpolating between two
 saturated colors.

\item  \verb|obj.SetColorSpaceToHSV ()| -  Set/Get the color space used for interpolation: RGB, HSV, CIELAB, or
 Diverging.  In HSV mode, if HSVWrap is on, it will take the shortest path
 in Hue (going back through 0 if that is the shortest way around the hue
 circle) whereas if HSVWrap is off it will not go through 0 (in order the
 match the current functionality of vtkLookupTable).  Diverging is a special
 mode where colors will pass through white when interpolating between two
 saturated colors.

\item  \verb|obj.SetColorSpaceToLab ()| -  Set/Get the color space used for interpolation: RGB, HSV, CIELAB, or
 Diverging.  In HSV mode, if HSVWrap is on, it will take the shortest path
 in Hue (going back through 0 if that is the shortest way around the hue
 circle) whereas if HSVWrap is off it will not go through 0 (in order the
 match the current functionality of vtkLookupTable).  Diverging is a special
 mode where colors will pass through white when interpolating between two
 saturated colors.

\item  \verb|obj.SetColorSpaceToDiverging ()| -  Set/Get the color space used for interpolation: RGB, HSV, CIELAB, or
 Diverging.  In HSV mode, if HSVWrap is on, it will take the shortest path
 in Hue (going back through 0 if that is the shortest way around the hue
 circle) whereas if HSVWrap is off it will not go through 0 (in order the
 match the current functionality of vtkLookupTable).  Diverging is a special
 mode where colors will pass through white when interpolating between two
 saturated colors.

\item  \verb|int = obj.GetColorSpace ()| -  Set/Get the color space used for interpolation: RGB, HSV, CIELAB, or
 Diverging.  In HSV mode, if HSVWrap is on, it will take the shortest path
 in Hue (going back through 0 if that is the shortest way around the hue
 circle) whereas if HSVWrap is off it will not go through 0 (in order the
 match the current functionality of vtkLookupTable).  Diverging is a special
 mode where colors will pass through white when interpolating between two
 saturated colors.

\item  \verb|obj.SetHSVWrap (int )| -  Set/Get the color space used for interpolation: RGB, HSV, CIELAB, or
 Diverging.  In HSV mode, if HSVWrap is on, it will take the shortest path
 in Hue (going back through 0 if that is the shortest way around the hue
 circle) whereas if HSVWrap is off it will not go through 0 (in order the
 match the current functionality of vtkLookupTable).  Diverging is a special
 mode where colors will pass through white when interpolating between two
 saturated colors.

\item  \verb|int = obj.GetHSVWrap ()| -  Set/Get the color space used for interpolation: RGB, HSV, CIELAB, or
 Diverging.  In HSV mode, if HSVWrap is on, it will take the shortest path
 in Hue (going back through 0 if that is the shortest way around the hue
 circle) whereas if HSVWrap is off it will not go through 0 (in order the
 match the current functionality of vtkLookupTable).  Diverging is a special
 mode where colors will pass through white when interpolating between two
 saturated colors.

\item  \verb|obj.HSVWrapOn ()| -  Set/Get the color space used for interpolation: RGB, HSV, CIELAB, or
 Diverging.  In HSV mode, if HSVWrap is on, it will take the shortest path
 in Hue (going back through 0 if that is the shortest way around the hue
 circle) whereas if HSVWrap is off it will not go through 0 (in order the
 match the current functionality of vtkLookupTable).  Diverging is a special
 mode where colors will pass through white when interpolating between two
 saturated colors.

\item  \verb|obj.HSVWrapOff ()| -  Set/Get the color space used for interpolation: RGB, HSV, CIELAB, or
 Diverging.  In HSV mode, if HSVWrap is on, it will take the shortest path
 in Hue (going back through 0 if that is the shortest way around the hue
 circle) whereas if HSVWrap is off it will not go through 0 (in order the
 match the current functionality of vtkLookupTable).  Diverging is a special
 mode where colors will pass through white when interpolating between two
 saturated colors.

\item  \verb|obj.SetScale (int )| -  Set the type of scale to use, linear or logarithmic.  The default
 is linear.  If the scale is logarithmic, and the range contains
 zero, the color mapping will be linear.

\item  \verb|obj.SetScaleToLinear ()| -  Set the type of scale to use, linear or logarithmic.  The default
 is linear.  If the scale is logarithmic, and the range contains
 zero, the color mapping will be linear.

\item  \verb|obj.SetScaleToLog10 ()| -  Set the type of scale to use, linear or logarithmic.  The default
 is linear.  If the scale is logarithmic, and the range contains
 zero, the color mapping will be linear.

\item  \verb|int = obj.GetScale ()| -  Set the type of scale to use, linear or logarithmic.  The default
 is linear.  If the scale is logarithmic, and the range contains
 zero, the color mapping will be linear.

\item  \verb|obj.FillFromDataPointer (int , double )| -  Returns a list of all nodes
 Fills from a pointer to data stored in a similar list of nodes.

\item  \verb|obj.SetAllowDuplicateScalars (int )| -  Toggle whether to allow duplicate scalar values in the color transfer
 function (off by default).

\item  \verb|int = obj.GetAllowDuplicateScalars ()| -  Toggle whether to allow duplicate scalar values in the color transfer
 function (off by default).

\item  \verb|obj.AllowDuplicateScalarsOn ()| -  Toggle whether to allow duplicate scalar values in the color transfer
 function (off by default).

\item  \verb|obj.AllowDuplicateScalarsOff ()| -  Toggle whether to allow duplicate scalar values in the color transfer
 function (off by default).

\end{itemize}
