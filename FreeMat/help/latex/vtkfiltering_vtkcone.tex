\section{vtkCone}

\subsection{Usage}

 vtkCone computes the implicit function and function gradient for a cone.
 vtkCone is a concrete implementation of vtkImplicitFunction. The cone vertex
 is located at the origin with axis of rotation coincident with x-axis. (Use
 the superclass' vtkImplicitFunction transformation matrix if necessary to 
 reposition.) The angle specifies the angle between the axis of rotation 
 and the side of the cone.

To create an instance of class vtkCone, simply
invoke its constructor as follows
\begin{verbatim}
  obj = vtkCone
\end{verbatim}
\subsection{Methods}

The class vtkCone has several methods that can be used.
  They are listed below.
Note that the documentation is translated automatically from the VTK sources,
and may not be completely intelligible.  When in doubt, consult the VTK website.
In the methods listed below, \verb|obj| is an instance of the vtkCone class.
\begin{itemize}
\item  \verb|string = obj.GetClassName ()|

\item  \verb|int = obj.IsA (string name)|

\item  \verb|vtkCone = obj.NewInstance ()|

\item  \verb|vtkCone = obj.SafeDownCast (vtkObject o)|

\item  \verb|double = obj.EvaluateFunction (double x[3])|

\item  \verb|double = obj.EvaluateFunction (double x, double y, double z)|

\item  \verb|obj.EvaluateGradient (double x[3], double g[3])|

\item  \verb|obj.SetAngle (double )| -  Set/Get the cone angle (expressed in degrees).

\item  \verb|double = obj.GetAngleMinValue ()| -  Set/Get the cone angle (expressed in degrees).

\item  \verb|double = obj.GetAngleMaxValue ()| -  Set/Get the cone angle (expressed in degrees).

\item  \verb|double = obj.GetAngle ()| -  Set/Get the cone angle (expressed in degrees).

\end{itemize}
