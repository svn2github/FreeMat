\section{vtkDistributedGraphHelper}

\subsection{Usage}

 A distributed graph helper can be attached to an empty vtkGraph
 object to turn the vtkGraph into a distributed graph, whose
 vertices and edges are distributed across several different
 processors. vtkDistributedGraphHelper is an abstract class. Use a
 subclass of vtkDistributedGraphHelper, such as
 vtkPBGLDistributedGraphHelper, to build distributed graphs.

 The distributed graph helper provides facilities used by vtkGraph
 to communicate with other processors that store other parts of the
 same distributed graph. The only user-level functionality provided
 by vtkDistributedGraphHelper involves this communication among
 processors and the ability to map between ''distributed'' vertex and
 edge IDs and their component parts (processor and local index). For
 example, the Synchronize() method provides a barrier that allows
 all processors to catch up to the same point in the code before any
 processor can leave that Synchronize() call. For example, one would
 call Synchronize() after adding many edges to a distributed graph,
 so that all processors can handle the addition of inter-processor
 edges and continue, after the Synchronize() call, with a consistent
 view of the distributed graph. For more information about
 manipulating (distributed) graphs, see the vtkGraph documentation.


To create an instance of class vtkDistributedGraphHelper, simply
invoke its constructor as follows
\begin{verbatim}
  obj = vtkDistributedGraphHelper
\end{verbatim}
\subsection{Methods}

The class vtkDistributedGraphHelper has several methods that can be used.
  They are listed below.
Note that the documentation is translated automatically from the VTK sources,
and may not be completely intelligible.  When in doubt, consult the VTK website.
In the methods listed below, \verb|obj| is an instance of the vtkDistributedGraphHelper class.
\begin{itemize}
\item  \verb|string = obj.GetClassName ()|

\item  \verb|int = obj.IsA (string name)|

\item  \verb|vtkDistributedGraphHelper = obj.NewInstance ()|

\item  \verb|vtkDistributedGraphHelper = obj.SafeDownCast (vtkObject o)|

\item  \verb|vtkIdType = obj.GetVertexOwner (vtkIdType v) const| -  Returns owner of vertex v, by extracting top ceil(log2 P) bits of v.

\item  \verb|vtkIdType = obj.GetVertexIndex (vtkIdType v) const| -  Returns local index of vertex v, by masking off top ceil(log2 P) bits of v.

\item  \verb|vtkIdType = obj.GetEdgeOwner (vtkIdType e\_id) const| -  Returns owner of edge with ID e\_id, by extracting top ceil(log2 P) bits of e\_id.

\item  \verb|vtkIdType = obj.GetEdgeIndex (vtkIdType e\_id) const| -  Returns local index of edge with ID e\_id, by masking off top ceil(log2 P)
 bits of e\_id.

\item  \verb|vtkIdType = obj.MakeDistributedId (int owner, vtkIdType local)| -  Builds a distributed ID consisting of the given owner and the local ID.

\item  \verb|obj.Synchronize ()| -  Synchronizes all of the processors involved in this distributed
 graph, so that all processors have a consistent view of the
 distributed graph for the computation that follows. This routine
 should be invoked after adding new edges into the distributed
 graph, so that other processors will see those edges (or their
 corresponding back-edges).

\item  \verb|vtkDistributedGraphHelper = obj.Clone ()| -  Clones the distributed graph helper, returning another
 distributed graph helper of the same kind that can be used in
 another vtkGraph.

\end{itemize}
