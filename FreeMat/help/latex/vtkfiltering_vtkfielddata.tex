\section{vtkFieldData}

\subsection{Usage}

 vtkFieldData represents and manipulates fields of data. The model of a field
 is a m x n matrix of data values, where m is the number of tuples, and n
 is the number of components. (A tuple is a row of n components in the
 matrix.) The field is assumed to be composed of a set of one or more data
 arrays, where the data in the arrays are of different types (e.g., int,
 double, char, etc.), and there may be variable numbers of components in
 each array. Note that each data array is assumed to be ''m'' in length
 (i.e., number of tuples), which typically corresponds to the number of
 points or cells in a dataset. Also, each data array must have a
 character-string name. (This is used to manipulate data.)

 There are two ways of manipulating and interfacing to fields. You can do
 it generically by manipulating components/tuples via a double-type data
 exchange, or you can do it by grabbing the arrays and manipulating them
 directly. The former is simpler but performs type conversion, which is bad
 if your data has non-castable types like (void) pointers, or you lose
 information as a result of the cast. The, more efficient method means
 managing each array in the field.  Using this method you can create
 faster, more efficient algorithms that do not lose information.

To create an instance of class vtkFieldData, simply
invoke its constructor as follows
\begin{verbatim}
  obj = vtkFieldData
\end{verbatim}
\subsection{Methods}

The class vtkFieldData has several methods that can be used.
  They are listed below.
Note that the documentation is translated automatically from the VTK sources,
and may not be completely intelligible.  When in doubt, consult the VTK website.
In the methods listed below, \verb|obj| is an instance of the vtkFieldData class.
\begin{itemize}
\item  \verb|string = obj.GetClassName ()|

\item  \verb|int = obj.IsA (string name)|

\item  \verb|vtkFieldData = obj.NewInstance ()|

\item  \verb|vtkFieldData = obj.SafeDownCast (vtkObject o)|

\item  \verb|obj.Initialize ()| -  Release all data but do not delete object.
 Also, clear the copy flags.

\item  \verb|int = obj.Allocate (vtkIdType sz, vtkIdType ext)| -  Allocate data for each array.
 Note that ext is no longer used.

\item  \verb|obj.CopyStructure (vtkFieldData )| -  Copy data array structure from a given field.  The same arrays
 will exist with the same types, but will contain nothing in the
 copy.

\item  \verb|obj.AllocateArrays (int num)| -  AllocateOfArrays actually sets the number of
 vtkAbstractArray pointers in the vtkFieldData object, not the
 number of used pointers (arrays). Adding more arrays will
 cause the object to dynamically adjust the number of pointers
 if it needs to extend. Although AllocateArrays can
 be used if the number of arrays which will be added is
 known, it can be omitted with a small computation cost.

\item  \verb|int = obj.GetNumberOfArrays ()| -  Add an array to the array list. If an array with the same name
 already exists - then the added array will replace it.

\item  \verb|int = obj.AddArray (vtkAbstractArray array)| -  Add an array to the array list. If an array with the same name
 already exists - then the added array will replace it.

\item  \verb|obj.RemoveArray (string name)| -  Return the ith array in the field. A NULL is returned if the
 index i is out of range. A NULL is returned if the array at the given 
 index is not a vtkDataArray.

\item  \verb|vtkDataArray = obj.GetArray (int i)| -  Return the ith array in the field. A NULL is returned if the
 index i is out of range. A NULL is returned if the array at the given 
 index is not a vtkDataArray.

\item  \verb|vtkDataArray = obj.GetArray (string arrayName)| -  Returns the ith array in the field. Unlike GetArray(), this method returns
 a vtkAbstractArray. A NULL is returned only if the index i is 
 out of range.

\item  \verb|vtkAbstractArray = obj.GetAbstractArray (int i)| -  Returns the ith array in the field. Unlike GetArray(), this method returns
 a vtkAbstractArray. A NULL is returned only if the index i is 
 out of range.

\item  \verb|vtkAbstractArray = obj.GetAbstractArray (string arrayName)| -  Return 1 if an array with the given name could be found. 0 otherwise.

\item  \verb|int = obj.HasArray (string name)| -  Get the name of ith array.
 Note that this is equivalent to:
 GetAbstractArray(i)->GetName() if ith array pointer is not NULL

\item  \verb|string = obj.GetArrayName (int i)| -  Pass entire arrays of input data through to output. Obey the ''copy''
 flags.

\item  \verb|obj.PassData (vtkFieldData fd)| -  Pass entire arrays of input data through to output. Obey the ''copy''
 flags.

\item  \verb|obj.CopyFieldOn (string name)| -  Turn on/off the copying of the field specified by name.
 During the copying/passing, the following rules are followed for each
 array:
 1. If the copy flag for an array is set (on or off), it is applied
    This overrides rule 2.
 2. If CopyAllOn is set, copy the array.
    If CopyAllOff is set, do not copy the array

\item  \verb|obj.CopyFieldOff (string name)| -  Turn on copying of all data.
 During the copying/passing, the following rules are followed for each
 array:
 1. If the copy flag for an array is set (on or off), it is applied
    This overrides rule 2.
 2. If CopyAllOn is set, copy the array.
    If CopyAllOff is set, do not copy the array

\item  \verb|obj.CopyAllOn (int unused)| -  Turn on copying of all data.
 During the copying/passing, the following rules are followed for each
 array:
 1. If the copy flag for an array is set (on or off), it is applied
    This overrides rule 2.
 2. If CopyAllOn is set, copy the array.
    If CopyAllOff is set, do not copy the array

\item  \verb|obj.CopyAllOff (int unused)| -  Turn off copying of all data.
 During the copying/passing, the following rules are followed for each
 array:
 1. If the copy flag for an array is set (on or off), it is applied
    This overrides rule 2.
 2. If CopyAllOn is set, copy the array.
    If CopyAllOff is set, do not copy the array

\item  \verb|obj.DeepCopy (vtkFieldData da)| -  Copy a field by creating new data arrays (i.e., duplicate storage).

\item  \verb|obj.ShallowCopy (vtkFieldData da)| -  Copy a field by reference counting the data arrays.

\item  \verb|obj.Squeeze ()| -  Squeezes each data array in the field (Squeeze() reclaims unused memory.)

\item  \verb|obj.Reset ()| -  Resets each data array in the field (Reset() does not release memory but
 it makes the arrays look like they are empty.)

\item  \verb|long = obj.GetActualMemorySize ()| -  Return the memory in kilobytes consumed by this field data. Used to
 support streaming and reading/writing data. The value returned is
 guaranteed to be greater than or equal to the memory required to
 actually represent the data represented by this object.

\item  \verb|long = obj.GetMTime ()| -  Check object's components for modified times.

\item  \verb|obj.GetField (vtkIdList ptId, vtkFieldData f)| -  Get a field from a list of ids. Supplied field f should have same
 types and number of data arrays as this one (i.e., like
 CopyStructure() creates).  This method should not be used if the
 instance is from a subclass of vtkFieldData (vtkPointData or
 vtkCellData).  This is because in those cases, the attribute data
 is stored with the other fields and will cause the method to
 behave in an unexpected way.

\item  \verb|int = obj.GetNumberOfComponents ()| -  Get the number of components in the field. This is determined by adding
 up the components in each non-NULL array.
 This method should not be used if the instance is from a
 subclass of vtkFieldData (vtkPointData or vtkCellData).
 This is because in those cases, the attribute data is 
 stored with the other fields and will cause the method
 to behave in an unexpected way.

\item  \verb|vtkIdType = obj.GetNumberOfTuples ()| -  Get the number of tuples in the field. Note: some fields have arrays with
 different numbers of tuples; this method returns the number of tuples in
 the first array. Mixed-length arrays may have to be treated specially.
 This method should not be used if the instance is from a
 subclass of vtkFieldData (vtkPointData or vtkCellData).
 This is because in those cases, the attribute data is 
 stored with the other fields and will cause the method
 to behave in an unexpected way.

\item  \verb|obj.SetNumberOfTuples (vtkIdType number)| -  Set the number of tuples for each data array in the field.
 This method should not be used if the instance is from a
 subclass of vtkFieldData (vtkPointData or vtkCellData).
 This is because in those cases, the attribute data is 
 stored with the other fields and will cause the method
 to behave in an unexpected way.

\item  \verb|obj.SetTuple (vtkIdType i, vtkIdType j, vtkFieldData source)| -  Set the jth tuple in source field data at the ith location. 
 Set operations mean that no range checking is performed, so 
 they're faster.

\item  \verb|obj.InsertTuple (vtkIdType i, vtkIdType j, vtkFieldData source)| -  Insert the jth tuple in source field data at the ith location. 
 Range checking is performed and memory allocates as necessary.

\item  \verb|vtkIdType = obj.InsertNextTuple (vtkIdType j, vtkFieldData source)| -  Insert the jth tuple in source field data  at the end of the 
 tuple matrix. Range checking is performed and memory is allocated 
 as necessary.

\item  \verb|obj.GetTuple (vtkIdType i, double tuple)| -  Copy the ith tuple value into a user provided tuple array. Make
 sure that you've allocated enough space for the copy.
 @deprecated as of VTK 5.2. Using this method for FieldData
 having arrays that are not subclasses of vtkDataArray may
 yield unexpected results.

\item  \verb|obj.SetTuple (vtkIdType i, double tuple)| -  Set the tuple value at the ith location. Set operations
 mean that no range checking is performed, so they're faster.
 @deprecated as of VTK 5.2. Using this method for FieldData
 having arrays that are not subclasses of vtkDataArray may
 yield unexpected results.

\item  \verb|obj.InsertTuple (vtkIdType i, double tuple)| -  Insert the tuple value at the ith location. Range checking is
 performed and memory allocates as necessary.
 @deprecated as of VTK 5.2. Using this method for FieldData
 having arrays that are not subclasses of vtkDataArray may
 yield unexpected results.

\item  \verb|vtkIdType = obj.InsertNextTuple (double tuple)| -  Insert the tuple value at the end of the tuple matrix. Range
 checking is performed and memory is allocated as necessary.
 @deprecated as of VTK 5.2. Using this method for FieldData
 having arrays that are not subclasses of vtkDataArray may
 yield unexpected results.

\item  \verb|double = obj.GetComponent (vtkIdType i, int j)| -  Get the component value at the ith tuple (or row) and jth component (or
 column).
 @deprecated as of VTK 5.2. Using this method for FieldData
 having arrays that are not subclasses of vtkDataArray may
 yield unexpected results.

\item  \verb|obj.SetComponent (vtkIdType i, int j, double c)| -  Set the component value at the ith tuple (or row) and jth component (or
 column).  Range checking is not performed, so set the object up properly
 before invoking.
 @deprecated as of VTK 5.2. Using this method for FieldData
 having arrays that are not subclasses of vtkDataArray may
 yield unexpected results.

\item  \verb|obj.InsertComponent (vtkIdType i, int j, double c)| -  Insert the component value at the ith tuple (or row) and jth component
 (or column).  Range checking is performed and memory allocated as
 necessary o hold data.
 @deprecated as of VTK 5.2. Using this method for FieldData
 having arrays that are not subclasses of vtkDataArray may
 yield unexpected results.

\end{itemize}
