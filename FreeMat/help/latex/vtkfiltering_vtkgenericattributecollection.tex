\section{vtkGenericAttributeCollection}

\subsection{Usage}

 vtkGenericAttributeCollection is a class that collects attributes 
 (represented by vtkGenericAttribute).

To create an instance of class vtkGenericAttributeCollection, simply
invoke its constructor as follows
\begin{verbatim}
  obj = vtkGenericAttributeCollection
\end{verbatim}
\subsection{Methods}

The class vtkGenericAttributeCollection has several methods that can be used.
  They are listed below.
Note that the documentation is translated automatically from the VTK sources,
and may not be completely intelligible.  When in doubt, consult the VTK website.
In the methods listed below, \verb|obj| is an instance of the vtkGenericAttributeCollection class.
\begin{itemize}
\item  \verb|string = obj.GetClassName ()| -  Standard type definition and print methods for a VTK class.

\item  \verb|int = obj.IsA (string name)| -  Standard type definition and print methods for a VTK class.

\item  \verb|vtkGenericAttributeCollection = obj.NewInstance ()| -  Standard type definition and print methods for a VTK class.

\item  \verb|vtkGenericAttributeCollection = obj.SafeDownCast (vtkObject o)| -  Standard type definition and print methods for a VTK class.

\item  \verb|int = obj.GetNumberOfAttributes ()| -  Return the number of attributes (e.g., instances of vtkGenericAttribute) 
 in the collection.
 

\item  \verb|int = obj.GetNumberOfComponents ()| -  Return the number of components. This is the sum of all components
 found in all attributes.
 

\item  \verb|int = obj.GetNumberOfPointCenteredComponents ()| -  Return the number of components. This is the sum of all components
 found in all point centered attributes.
 

\item  \verb|int = obj.GetMaxNumberOfComponents ()| -  Maximum number of components encountered among all attributes.
 
 

\item  \verb|long = obj.GetActualMemorySize ()| -  Actual size of the data in kilobytes; only valid after the pipeline has
 updated. It is guaranteed to be greater than or equal to the memory
 required to represent the data.

\item  \verb|int = obj.IsEmpty ()| -  Indicate whether the collection contains any attributes.
 

\item  \verb|vtkGenericAttribute = obj.GetAttribute (int i)| -  Return a pointer to the ith instance of vtkGenericAttribute.
 
 
 

\item  \verb|int = obj.FindAttribute (string name)| -  Return the index of the attribute named `name'. Return the non-negative
 index if found. Return -1 otherwise.
 
 

\item  \verb|int = obj.GetAttributeIndex (int i)| -  Return the index of the first component of attribute `i' in an array of
 format attrib0comp0 attrib0comp1 ... attrib4comp0 ...
 
 

\item  \verb|obj.InsertNextAttribute (vtkGenericAttribute a)| -  Add the attribute `a' to the end of the collection.
 
 
 

\item  \verb|obj.InsertAttribute (int i, vtkGenericAttribute a)| -  Replace the attribute at index `i' by `a'.
 
 
 
 
 

\item  \verb|obj.RemoveAttribute (int i)| -  Remove the attribute at `i'.
 
 
 

\item  \verb|obj.Reset ()| -  Remove all attributes.
 

\item  \verb|obj.DeepCopy (vtkGenericAttributeCollection other)| -  Copy, without reference counting, the other attribute array.
 
 
 

\item  \verb|obj.ShallowCopy (vtkGenericAttributeCollection other)| -  Copy, via reference counting, the other attribute array.
 
 
 

\item  \verb|long = obj.GetMTime ()| -  vtkAttributeCollection is a composite object and needs to check each
 member of its collection for modified time.

\item  \verb|int = obj.GetActiveAttribute ()| -  Index of the attribute to be processed (not necessarily scalar).
 
 

\item  \verb|int = obj.GetActiveComponent ()| -  Component of the active attribute to be processed. -1 means module.
 
 
       result<GetAttribute(GetActiveAttribute())->GetNumberOfComponents()

\item  \verb|obj.SetActiveAttribute (int attribute, int component)| -  Set the scalar attribute to be processed. -1 means module.
 
 
 
              component<GetAttribute(attribute)->GetNumberOfComponents()
 
               GetActiveComponent()==component

\item  \verb|int = obj.GetNumberOfAttributesToInterpolate ()| -  Number of attributes to interpolate.
 
 

\item  \verb|int = obj.HasAttribute (int size, int attributes, int attribute)|

\item  \verb|obj.SetAttributesToInterpolate (int size, int attributes)| -  Set the attributes to interpolate.
 
 
 
 
                     !HasAttributes(size,attributes,GetActiveAttribute())
 
               (GetAttributesToInterpolate()==attributes)

\item  \verb|obj.SetAttributesToInterpolateToAll ()| -  Set the attributes to interpolate.
 
 
 
 
                     !HasAttributes(size,attributes,GetActiveAttribute())
 
               (GetAttributesToInterpolate()==attributes)

\end{itemize}
