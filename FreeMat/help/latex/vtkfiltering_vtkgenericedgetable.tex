\section{vtkGenericEdgeTable}

\subsection{Usage}

 vtkGenericEdgeTable is used to indicate the existance of and hold
 information about edges. Similar to vtkEdgeTable, this class is
 more sophisticated in that it uses reference counting to keep track
 of when information about an edge should be deleted.

 vtkGenericEdgeTable is a helper class used in the adaptor framework.  It
 is used during the tessellation process to hold information about the
 error metric on each edge. This avoids recomputing the error metric each
 time the same edge is visited.

To create an instance of class vtkGenericEdgeTable, simply
invoke its constructor as follows
\begin{verbatim}
  obj = vtkGenericEdgeTable
\end{verbatim}
\subsection{Methods}

The class vtkGenericEdgeTable has several methods that can be used.
  They are listed below.
Note that the documentation is translated automatically from the VTK sources,
and may not be completely intelligible.  When in doubt, consult the VTK website.
In the methods listed below, \verb|obj| is an instance of the vtkGenericEdgeTable class.
\begin{itemize}
\item  \verb|string = obj.GetClassName ()| -  Standard VTK type and print macros.

\item  \verb|int = obj.IsA (string name)| -  Standard VTK type and print macros.

\item  \verb|vtkGenericEdgeTable = obj.NewInstance ()| -  Standard VTK type and print macros.

\item  \verb|vtkGenericEdgeTable = obj.SafeDownCast (vtkObject o)| -  Standard VTK type and print macros.

\item  \verb|obj.InsertEdge (vtkIdType e1, vtkIdType e2, vtkIdType cellId, int ref)| -  Insert an edge but do not split it.

\item  \verb|int = obj.RemoveEdge (vtkIdType e1, vtkIdType e2)| -  Method to remove an edge from the table. The method returns the
 current reference count.

\item  \verb|int = obj.IncrementEdgeReferenceCount (vtkIdType e1, vtkIdType e2, vtkIdType cellId)| -  Method that increments the referencecount and returns it.

\item  \verb|int = obj.CheckEdgeReferenceCount (vtkIdType e1, vtkIdType e2)| -  Return the edge reference count.

\item  \verb|obj.Initialize (vtkIdType start)| -  To specify the starting point id. It will initialize LastPointId
 This is very sensitive the start point should be cautiously chosen

\item  \verb|int = obj.GetNumberOfComponents ()| -  Return the total number of components for the point-centered attributes.
 

\item  \verb|obj.SetNumberOfComponents (int count)| -  Set the total number of components for the point-centered attributes.
 

\item  \verb|int = obj.CheckPoint (vtkIdType ptId)| -  Check if a point is already in the point table.

\item  \verb|int = obj.CheckPoint (vtkIdType ptId, double point[3], double scalar)| -  Check for the existence of a point and return its coordinate value.
 

\item  \verb|obj.InsertPoint (vtkIdType ptId, double point[3])| -  Insert point associated with an edge.

\item  \verb|obj.InsertPointAndScalar (vtkIdType ptId, double pt[3], double s)| -  Insert point associated with an edge.
 re: sizeof(s)==GetNumberOfComponents()

\item  \verb|obj.RemovePoint (vtkIdType ptId)| -  Remove a point from the point table.

\item  \verb|obj.IncrementPointReferenceCount (vtkIdType ptId)| -  Increment the reference count for the indicated point.

\item  \verb|obj.DumpTable ()| -  For debugging purposes. It is particularly useful to dump the table
 and check that nothing is left after a complete iteration. LoadFactor
 should ideally be very low to be able to have a constant time access

\item  \verb|obj.LoadFactor ()| -  For debugging purposes. It is particularly useful to dump the table
 and check that nothing is left after a complete iteration. LoadFactor
 should ideally be very low to be able to have a constant time access

\end{itemize}
