\section{vtkGenericInterpolatedVelocityField}

\subsection{Usage}

 vtkGenericInterpolatedVelocityField acts as a continuous velocity field
 by performing cell interpolation on the underlying vtkDataSet.
 This is a concrete sub-class of vtkFunctionSet with 
 NumberOfIndependentVariables = 4 (x,y,z,t) and 
 NumberOfFunctions = 3 (u,v,w). Normally, every time an evaluation
 is performed, the cell which contains the point (x,y,z) has to
 be found by calling FindCell. This is a computationally expansive 
 operation. In certain cases, the cell search can be avoided or shortened 
 by providing a guess for the cell iterator. For example, in streamline
 integration, the next evaluation is usually in the same or a neighbour
 cell. For this reason, vtkGenericInterpolatedVelocityField stores the last
 cell iterator. If caching is turned on, it uses this iterator as the
 starting point.

To create an instance of class vtkGenericInterpolatedVelocityField, simply
invoke its constructor as follows
\begin{verbatim}
  obj = vtkGenericInterpolatedVelocityField
\end{verbatim}
\subsection{Methods}

The class vtkGenericInterpolatedVelocityField has several methods that can be used.
  They are listed below.
Note that the documentation is translated automatically from the VTK sources,
and may not be completely intelligible.  When in doubt, consult the VTK website.
In the methods listed below, \verb|obj| is an instance of the vtkGenericInterpolatedVelocityField class.
\begin{itemize}
\item  \verb|string = obj.GetClassName ()|

\item  \verb|int = obj.IsA (string name)|

\item  \verb|vtkGenericInterpolatedVelocityField = obj.NewInstance ()|

\item  \verb|vtkGenericInterpolatedVelocityField = obj.SafeDownCast (vtkObject o)|

\item  \verb|int = obj.FunctionValues (double x, double f)| -  Evaluate the velocity field, f, at (x, y, z, t).
 For now, t is ignored.

\item  \verb|obj.AddDataSet (vtkGenericDataSet dataset)| -  Add a dataset used for the implicit function evaluation.
 If more than one dataset is added, the evaluation point is
 searched in all until a match is found. THIS FUNCTION
 DOES NOT CHANGE THE REFERENCE COUNT OF dataset FOR THREAD
 SAFETY REASONS.

\item  \verb|obj.ClearLastCell ()| -  Set the last cell id to -1 so that the next search does not
 start from the previous cell

\item  \verb|vtkGenericAdaptorCell = obj.GetLastCell ()| -  Return the cell cached from last evaluation.

\item  \verb|int = obj.GetLastLocalCoordinates (double pcoords[3])| -  Returns the interpolation weights cached from last evaluation
 if the cached cell is valid (returns 1). Otherwise, it does not
 change w and returns 0.

\item  \verb|int = obj.GetCaching ()| -  Turn caching on/off.

\item  \verb|obj.SetCaching (int )| -  Turn caching on/off.

\item  \verb|obj.CachingOn ()| -  Turn caching on/off.

\item  \verb|obj.CachingOff ()| -  Turn caching on/off.

\item  \verb|int = obj.GetCacheHit ()| -  Caching statistics.

\item  \verb|int = obj.GetCacheMiss ()| -  Caching statistics.

\item  \verb|string = obj.GetVectorsSelection ()| -  If you want to work with an arbitrary vector array, then set its name 
 here. By default this in NULL and the filter will use the active vector 
 array.

\item  \verb|obj.SelectVectors (string fieldName)| -  Returns the last dataset that was visited. Can be used
 as a first guess as to where the next point will be as
 well as to avoid searching through all datasets to get
 more information about the point.

\item  \verb|vtkGenericDataSet = obj.GetLastDataSet ()| -  Returns the last dataset that was visited. Can be used
 as a first guess as to where the next point will be as
 well as to avoid searching through all datasets to get
 more information about the point.

\item  \verb|obj.CopyParameters (vtkGenericInterpolatedVelocityField from)| -  Copy the user set parameters from source. This copies
 the Caching parameters. Sub-classes can add more after
 chaining.

\end{itemize}
