\section{vtkHexagonalPrism}

\subsection{Usage}

 vtkHexagonalPrism is a concrete implementation of vtkCell to represent a
 linear 3D prism with hexagonal base. Such prism is defined by the twelve points 
 (0-12) where (0,1,2,3,4,5) is the base of the prism which, using the right
 hand rule, forms a hexagon whose normal points is in the direction of the
 opposite face (6,7,8,9,10,11).

To create an instance of class vtkHexagonalPrism, simply
invoke its constructor as follows
\begin{verbatim}
  obj = vtkHexagonalPrism
\end{verbatim}
\subsection{Methods}

The class vtkHexagonalPrism has several methods that can be used.
  They are listed below.
Note that the documentation is translated automatically from the VTK sources,
and may not be completely intelligible.  When in doubt, consult the VTK website.
In the methods listed below, \verb|obj| is an instance of the vtkHexagonalPrism class.
\begin{itemize}
\item  \verb|string = obj.GetClassName ()|

\item  \verb|int = obj.IsA (string name)|

\item  \verb|vtkHexagonalPrism = obj.NewInstance ()|

\item  \verb|vtkHexagonalPrism = obj.SafeDownCast (vtkObject o)|

\item  \verb|int = obj.GetCellType ()| -  See the vtkCell API for descriptions of these methods.

\item  \verb|int = obj.GetCellDimension ()| -  See the vtkCell API for descriptions of these methods.

\item  \verb|int = obj.GetNumberOfEdges ()| -  See the vtkCell API for descriptions of these methods.

\item  \verb|int = obj.GetNumberOfFaces ()| -  See the vtkCell API for descriptions of these methods.

\item  \verb|vtkCell = obj.GetEdge (int edgeId)| -  See the vtkCell API for descriptions of these methods.

\item  \verb|vtkCell = obj.GetFace (int faceId)| -  See the vtkCell API for descriptions of these methods.

\item  \verb|int = obj.CellBoundary (int subId, double pcoords[3], vtkIdList pts)| -  See the vtkCell API for descriptions of these methods.

\item  \verb|int = obj.Triangulate (int index, vtkIdList ptIds, vtkPoints pts)|

\item  \verb|obj.Derivatives (int subId, double pcoords[3], double values, int dim, double derivs)|

\item  \verb|int = obj.GetParametricCenter (double pcoords[3])| -  Return the center of the wedge in parametric coordinates.

\item  \verb|obj.InterpolateFunctions (double pcoords[3], double weights[12])| -  Compute the interpolation functions/derivatives
 (aka shape functions/derivatives)

\item  \verb|obj.InterpolateDerivs (double pcoords[3], double derivs[36])| -  Return the ids of the vertices defining edge/face (`edgeId`/`faceId').
 Ids are related to the cell, not to the dataset.

\end{itemize}
