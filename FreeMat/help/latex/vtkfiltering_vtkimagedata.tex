\section{vtkImageData}

\subsection{Usage}

 vtkImageData is a data object that is a concrete implementation of
 vtkDataSet. vtkImageData represents a geometric structure that is
 a topological and geometrical regular array of points. Examples include
 volumes (voxel data) and pixmaps.

To create an instance of class vtkImageData, simply
invoke its constructor as follows
\begin{verbatim}
  obj = vtkImageData
\end{verbatim}
\subsection{Methods}

The class vtkImageData has several methods that can be used.
  They are listed below.
Note that the documentation is translated automatically from the VTK sources,
and may not be completely intelligible.  When in doubt, consult the VTK website.
In the methods listed below, \verb|obj| is an instance of the vtkImageData class.
\begin{itemize}
\item  \verb|string = obj.GetClassName ()|

\item  \verb|int = obj.IsA (string name)|

\item  \verb|vtkImageData = obj.NewInstance ()|

\item  \verb|vtkImageData = obj.SafeDownCast (vtkObject o)|

\item  \verb|obj.CopyStructure (vtkDataSet ds)| -  Copy the geometric and topological structure of an input image data
 object.

\item  \verb|int = obj.GetDataObjectType ()| -  Return what type of dataset this is.

\item  \verb|vtkIdType = obj.GetNumberOfCells ()| -  Standard vtkDataSet API methods. See vtkDataSet for more information.

\item  \verb|vtkIdType = obj.GetNumberOfPoints ()| -  Standard vtkDataSet API methods. See vtkDataSet for more information.

\item  \verb|double = obj.GetPoint (vtkIdType ptId)| -  Standard vtkDataSet API methods. See vtkDataSet for more information.

\item  \verb|obj.GetPoint (vtkIdType id, double x[3])| -  Standard vtkDataSet API methods. See vtkDataSet for more information.

\item  \verb|vtkCell = obj.GetCell (vtkIdType cellId)| -  Standard vtkDataSet API methods. See vtkDataSet for more information.

\item  \verb|obj.GetCell (vtkIdType cellId, vtkGenericCell cell)| -  Standard vtkDataSet API methods. See vtkDataSet for more information.

\item  \verb|obj.GetCellBounds (vtkIdType cellId, double bounds[6])| -  Standard vtkDataSet API methods. See vtkDataSet for more information.

\item  \verb|vtkIdType = obj.FindPoint (double x, double y, double z)| -  Standard vtkDataSet API methods. See vtkDataSet for more information.

\item  \verb|vtkIdType = obj.FindPoint (double x[3])| -  Standard vtkDataSet API methods. See vtkDataSet for more information.

\item  \verb|int = obj.GetCellType (vtkIdType cellId)| -  Standard vtkDataSet API methods. See vtkDataSet for more information.

\item  \verb|obj.GetCellPoints (vtkIdType cellId, vtkIdList ptIds)| -  Standard vtkDataSet API methods. See vtkDataSet for more information.

\item  \verb|obj.GetPointCells (vtkIdType ptId, vtkIdList cellIds)| -  Standard vtkDataSet API methods. See vtkDataSet for more information.

\item  \verb|obj.ComputeBounds ()| -  Standard vtkDataSet API methods. See vtkDataSet for more information.

\item  \verb|int = obj.GetMaxCellSize ()| -  Standard vtkDataSet API methods. See vtkDataSet for more information.

\item  \verb|obj.Initialize ()| -  Restore data object to initial state,

\item  \verb|obj.SetDimensions (int i, int j, int k)| -  Pass your way. This is for backward compatibility only.
 Use SetExtent() instead.
 Same as SetExtent(0, i-1, 0, j-1, 0, k-1)

\item  \verb|obj.SetDimensions (int dims[3])| -  Pass your way. This is for backward compatibility only.
 Use SetExtent() instead.
 Same as SetExtent(0, dims[0]-1, 0, dims[1]-1, 0, dims[2]-1)

\item  \verb|int = obj.GetDimensions ()| -  Get dimensions of this structured points dataset.
 It is the number of points on each axis.
 Dimensions are computed from Extents during this call.

\item  \verb|obj.GetDimensions (int dims[3])| -  Get dimensions of this structured points dataset.
 It is the number of points on each axis.
 Dimensions are computed from Extents during this call.

\item  \verb|int = obj.ComputeStructuredCoordinates (double x[3], int ijk[3], double pcoords[3])| -  Convenience function computes the structured coordinates for a point x[3].
 The voxel is specified by the array ijk[3], and the parametric coordinates
 in the cell are specified with pcoords[3]. The function returns a 0 if the
 point x is outside of the volume, and a 1 if inside the volume.

\item  \verb|obj.GetVoxelGradient (int i, int j, int k, vtkDataArray s, vtkDataArray g)| -  Given structured coordinates (i,j,k) for a voxel cell, compute the eight
 gradient values for the voxel corners. The order in which the gradient
 vectors are arranged corresponds to the ordering of the voxel points.
 Gradient vector is computed by central differences (except on edges of
 volume where forward difference is used). The scalars s are the scalars
 from which the gradient is to be computed. This method will treat
 only 3D structured point datasets (i.e., volumes).

\item  \verb|obj.GetPointGradient (int i, int j, int k, vtkDataArray s, double g[3])| -  Given structured coordinates (i,j,k) for a point in a structured point
 dataset, compute the gradient vector from the scalar data at that point.
 The scalars s are the scalars from which the gradient is to be computed.
 This method will treat structured point datasets of any dimension.

\item  \verb|int = obj.GetDataDimension ()| -  Return the dimensionality of the data.

\item  \verb|vtkIdType = obj.ComputePointId (int ijk[3])| -  Given a location in structured coordinates (i-j-k), return the point id.

\item  \verb|vtkIdType = obj.ComputeCellId (int ijk[3])| -  Given a location in structured coordinates (i-j-k), return the cell id.

\item  \verb|obj.SetAxisUpdateExtent (int axis, int min, int max)| -  Set / Get the extent on just one axis

\item  \verb|obj.UpdateInformation ()| -  Override to copy information from pipeline information to data
 information for backward compatibility.  See
 vtkDataObject::UpdateInformation for details.

\item  \verb|obj.SetExtent (int extent[6])| -  Set/Get the extent. On each axis, the extent is defined by the index
 of the first point and the index of the last point.  The extent should
 be set before the ''Scalars'' are set or allocated.  The Extent is
 stored in the order (X, Y, Z).
 The dataset extent does not have to start at (0,0,0). (0,0,0) is just the
 extent of the origin.
 The first point (the one with Id=0) is at extent
 (Extent[0],Extent[2],Extent[4]). As for any dataset, a data array on point
 data starts at Id=0.

\item  \verb|obj.SetExtent (int x1, int x2, int y1, int y2, int z1, int z2)| -  Set/Get the extent. On each axis, the extent is defined by the index
 of the first point and the index of the last point.  The extent should
 be set before the ''Scalars'' are set or allocated.  The Extent is
 stored in the order (X, Y, Z).
 The dataset extent does not have to start at (0,0,0). (0,0,0) is just the
 extent of the origin.
 The first point (the one with Id=0) is at extent
 (Extent[0],Extent[2],Extent[4]). As for any dataset, a data array on point
 data starts at Id=0.

\item  \verb|int = obj. GetExtent ()| -  Set/Get the extent. On each axis, the extent is defined by the index
 of the first point and the index of the last point.  The extent should
 be set before the ''Scalars'' are set or allocated.  The Extent is
 stored in the order (X, Y, Z).
 The dataset extent does not have to start at (0,0,0). (0,0,0) is just the
 extent of the origin.
 The first point (the one with Id=0) is at extent
 (Extent[0],Extent[2],Extent[4]). As for any dataset, a data array on point
 data starts at Id=0.

\item  \verb|long = obj.GetEstimatedMemorySize ()| -  Get the estimated size of this data object itself. Should be called
 after UpdateInformation() and PropagateUpdateExtent() have both been
 called. This estimate should be fairly accurate since this is structured
 data.

\item  \verb|double = obj.GetScalarTypeMin ()| -  These returns the minimum and maximum values the ScalarType can hold
 without overflowing.

\item  \verb|double = obj.GetScalarTypeMax ()| -  These returns the minimum and maximum values the ScalarType can hold
 without overflowing.

\item  \verb|int = obj.GetScalarSize ()| -  Set the size of the scalar type in bytes.

\item  \verb|vtkIdType = obj.GetIncrements ()| -  Different ways to get the increments for moving around the data.
 GetIncrements() calls ComputeIncrements() to ensure the increments are
 up to date.

\item  \verb|obj.GetIncrements (vtkIdType inc[3])| -  Different ways to get the increments for moving around the data.
 GetIncrements() calls ComputeIncrements() to ensure the increments are
 up to date.

\item  \verb|float = obj.GetScalarComponentAsFloat (int x, int y, int z, int component)| -  For access to data from tcl

\item  \verb|obj.SetScalarComponentFromFloat (int x, int y, int z, int component, float v)| -  For access to data from tcl

\item  \verb|double = obj.GetScalarComponentAsDouble (int x, int y, int z, int component)| -  For access to data from tcl

\item  \verb|obj.SetScalarComponentFromDouble (int x, int y, int z, int component, double v)| -  For access to data from tcl

\item  \verb|obj.AllocateScalars ()| -  Allocate the vtkScalars object associated with this object.

\item  \verb|obj.CopyAndCastFrom (vtkImageData inData, int extent[6])| -  This method is passed a input and output region, and executes the filter
 algorithm to fill the output from the input.
 It just executes a switch statement to call the correct function for
 the regions data types.

\item  \verb|obj.CopyAndCastFrom (vtkImageData inData, int x0, int x1, int y0, int y1, int z0, int z1)| -  Reallocates and copies to set the Extent to the UpdateExtent.
 This is used internally when the exact extent is requested,
 and the source generated more than the update extent.

\item  \verb|obj.Crop ()| -  Reallocates and copies to set the Extent to the UpdateExtent.
 This is used internally when the exact extent is requested,
 and the source generated more than the update extent.

\item  \verb|long = obj.GetActualMemorySize ()| -  Return the actual size of the data in kilobytes. This number
 is valid only after the pipeline has updated. The memory size
 returned is guaranteed to be greater than or equal to the
 memory required to represent the data (e.g., extra space in
 arrays, etc. are not included in the return value). THIS METHOD
 IS THREAD SAFE.

\item  \verb|obj.SetSpacing (double , double , double )| -  Set the spacing (width,height,length) of the cubical cells that
 compose the data set.

\item  \verb|obj.SetSpacing (double  a[3])| -  Set the spacing (width,height,length) of the cubical cells that
 compose the data set.

\item  \verb|double = obj. GetSpacing ()| -  Set the spacing (width,height,length) of the cubical cells that
 compose the data set.

\item  \verb|obj.SetOrigin (double , double , double )| -  Set/Get the origin of the dataset. The origin is the position in world
 coordinates of the point of extent (0,0,0). This point does not have to be
 part of the dataset, in other words, the dataset extent does not have to
 start at (0,0,0) and the origin can be outside of the dataset bounding
 box.
 The origin plus spacing determine the position in space of the points.

\item  \verb|obj.SetOrigin (double  a[3])| -  Set/Get the origin of the dataset. The origin is the position in world
 coordinates of the point of extent (0,0,0). This point does not have to be
 part of the dataset, in other words, the dataset extent does not have to
 start at (0,0,0) and the origin can be outside of the dataset bounding
 box.
 The origin plus spacing determine the position in space of the points.

\item  \verb|double = obj. GetOrigin ()| -  Set/Get the origin of the dataset. The origin is the position in world
 coordinates of the point of extent (0,0,0). This point does not have to be
 part of the dataset, in other words, the dataset extent does not have to
 start at (0,0,0) and the origin can be outside of the dataset bounding
 box.
 The origin plus spacing determine the position in space of the points.

\item  \verb|obj.SetScalarTypeToFloat ()| -  Set/Get the data scalar type (i.e VTK\_DOUBLE). Note that these methods
 are setting and getting the pipeline scalar type. i.e. they are setting
 the type that the image data will be once it has executed. Until the
 REQUEST\_DATA pass the actual scalars may be of some other type. This is
 for backwards compatibility

\item  \verb|obj.SetScalarTypeToDouble ()| -  Set/Get the data scalar type (i.e VTK\_DOUBLE). Note that these methods
 are setting and getting the pipeline scalar type. i.e. they are setting
 the type that the image data will be once it has executed. Until the
 REQUEST\_DATA pass the actual scalars may be of some other type. This is
 for backwards compatibility

\item  \verb|obj.SetScalarTypeToInt ()| -  Set/Get the data scalar type (i.e VTK\_DOUBLE). Note that these methods
 are setting and getting the pipeline scalar type. i.e. they are setting
 the type that the image data will be once it has executed. Until the
 REQUEST\_DATA pass the actual scalars may be of some other type. This is
 for backwards compatibility

\item  \verb|obj.SetScalarTypeToUnsignedInt ()| -  Set/Get the data scalar type (i.e VTK\_DOUBLE). Note that these methods
 are setting and getting the pipeline scalar type. i.e. they are setting
 the type that the image data will be once it has executed. Until the
 REQUEST\_DATA pass the actual scalars may be of some other type. This is
 for backwards compatibility

\item  \verb|obj.SetScalarTypeToLong ()| -  Set/Get the data scalar type (i.e VTK\_DOUBLE). Note that these methods
 are setting and getting the pipeline scalar type. i.e. they are setting
 the type that the image data will be once it has executed. Until the
 REQUEST\_DATA pass the actual scalars may be of some other type. This is
 for backwards compatibility

\item  \verb|obj.SetScalarTypeToUnsignedLong ()| -  Set/Get the data scalar type (i.e VTK\_DOUBLE). Note that these methods
 are setting and getting the pipeline scalar type. i.e. they are setting
 the type that the image data will be once it has executed. Until the
 REQUEST\_DATA pass the actual scalars may be of some other type. This is
 for backwards compatibility

\item  \verb|obj.SetScalarTypeToShort ()| -  Set/Get the data scalar type (i.e VTK\_DOUBLE). Note that these methods
 are setting and getting the pipeline scalar type. i.e. they are setting
 the type that the image data will be once it has executed. Until the
 REQUEST\_DATA pass the actual scalars may be of some other type. This is
 for backwards compatibility

\item  \verb|obj.SetScalarTypeToUnsignedShort ()| -  Set/Get the data scalar type (i.e VTK\_DOUBLE). Note that these methods
 are setting and getting the pipeline scalar type. i.e. they are setting
 the type that the image data will be once it has executed. Until the
 REQUEST\_DATA pass the actual scalars may be of some other type. This is
 for backwards compatibility

\item  \verb|obj.SetScalarTypeToUnsignedChar ()| -  Set/Get the data scalar type (i.e VTK\_DOUBLE). Note that these methods
 are setting and getting the pipeline scalar type. i.e. they are setting
 the type that the image data will be once it has executed. Until the
 REQUEST\_DATA pass the actual scalars may be of some other type. This is
 for backwards compatibility

\item  \verb|obj.SetScalarTypeToSignedChar ()| -  Set/Get the data scalar type (i.e VTK\_DOUBLE). Note that these methods
 are setting and getting the pipeline scalar type. i.e. they are setting
 the type that the image data will be once it has executed. Until the
 REQUEST\_DATA pass the actual scalars may be of some other type. This is
 for backwards compatibility

\item  \verb|obj.SetScalarTypeToChar ()| -  Set/Get the data scalar type (i.e VTK\_DOUBLE). Note that these methods
 are setting and getting the pipeline scalar type. i.e. they are setting
 the type that the image data will be once it has executed. Until the
 REQUEST\_DATA pass the actual scalars may be of some other type. This is
 for backwards compatibility

\item  \verb|obj.SetScalarType (int )| -  Set/Get the data scalar type (i.e VTK\_DOUBLE). Note that these methods
 are setting and getting the pipeline scalar type. i.e. they are setting
 the type that the image data will be once it has executed. Until the
 REQUEST\_DATA pass the actual scalars may be of some other type. This is
 for backwards compatibility

\item  \verb|int = obj.GetScalarType ()| -  Set/Get the data scalar type (i.e VTK\_DOUBLE). Note that these methods
 are setting and getting the pipeline scalar type. i.e. they are setting
 the type that the image data will be once it has executed. Until the
 REQUEST\_DATA pass the actual scalars may be of some other type. This is
 for backwards compatibility

\item  \verb|string = obj.GetScalarTypeAsString ()| -  Set/Get the data scalar type (i.e VTK\_DOUBLE). Note that these methods
 are setting and getting the pipeline scalar type. i.e. they are setting
 the type that the image data will be once it has executed. Until the
 REQUEST\_DATA pass the actual scalars may be of some other type. This is
 for backwards compatibility

\item  \verb|obj.SetNumberOfScalarComponents (int n)| -  Set/Get the number of scalar components for points. As with the
 SetScalarType method this is setting pipeline info.

\item  \verb|int = obj.GetNumberOfScalarComponents ()| -  Set/Get the number of scalar components for points. As with the
 SetScalarType method this is setting pipeline info.

\item  \verb|obj.CopyTypeSpecificInformation (vtkDataObject image)|

\item  \verb|obj.CopyInformationToPipeline (vtkInformation request, vtkInformation input, vtkInformation output, int forceCopy)| -  Override these to handle origin, spacing, scalar type, and scalar
 number of components.  See vtkDataObject for details.

\item  \verb|obj.CopyInformationFromPipeline (vtkInformation request)| -  Override these to handle origin, spacing, scalar type, and scalar
 number of components.  See vtkDataObject for details.

\item  \verb|obj.PrepareForNewData ()| -  make the output data ready for new data to be inserted. For most
 objects we just call Initialize. But for image data we leave the old
 data in case the memory can be reused.

\item  \verb|obj.ShallowCopy (vtkDataObject src)| -  Shallow and Deep copy.

\item  \verb|obj.DeepCopy (vtkDataObject src)| -  Shallow and Deep copy.

\item  \verb|obj.ComputeInternalExtent (int intExt, int tgtExt, int bnds)| -  Given how many pixel are required on a side for bounrary conditions (in
 bnds), the target extent to traverse, compute the internal extent (the
 extent for this ImageData that does nto suffer from any boundary
 conditions) and place it in intExt

\item  \verb|int = obj.GetExtentType ()| -  The extent type is a 3D extent

\end{itemize}
