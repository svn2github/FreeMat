\section{vtkImplicitWindowFunction}

\subsection{Usage}

 vtkImplicitWindowFunction is used to modify the output of another
 implicit function to lie within a specified ''window'', or function
 range. This can be used to add ''thickness'' to cutting or clipping
 functions. 

 This class works as follows. First, it evaluates the function value of the 
 user-specified implicit function. Then, based on the window range specified,
 it maps the function value into the window values specified. 


To create an instance of class vtkImplicitWindowFunction, simply
invoke its constructor as follows
\begin{verbatim}
  obj = vtkImplicitWindowFunction
\end{verbatim}
\subsection{Methods}

The class vtkImplicitWindowFunction has several methods that can be used.
  They are listed below.
Note that the documentation is translated automatically from the VTK sources,
and may not be completely intelligible.  When in doubt, consult the VTK website.
In the methods listed below, \verb|obj| is an instance of the vtkImplicitWindowFunction class.
\begin{itemize}
\item  \verb|string = obj.GetClassName ()|

\item  \verb|int = obj.IsA (string name)|

\item  \verb|vtkImplicitWindowFunction = obj.NewInstance ()|

\item  \verb|vtkImplicitWindowFunction = obj.SafeDownCast (vtkObject o)|

\item  \verb|double = obj.EvaluateFunction (double x[3])|

\item  \verb|double = obj.EvaluateFunction (double x, double y, double z)|

\item  \verb|obj.EvaluateGradient (double x[3], double n[3])|

\item  \verb|obj.SetImplicitFunction (vtkImplicitFunction )| -  Specify an implicit function to operate on.

\item  \verb|vtkImplicitFunction = obj.GetImplicitFunction ()| -  Specify an implicit function to operate on.

\item  \verb|obj.SetWindowRange (double , double )| -  Specify the range of function values which are considered to lie within
 the window. WindowRange[0] is assumed to be less than WindowRange[1].

\item  \verb|obj.SetWindowRange (double  a[2])| -  Specify the range of function values which are considered to lie within
 the window. WindowRange[0] is assumed to be less than WindowRange[1].

\item  \verb|double = obj. GetWindowRange ()| -  Specify the range of function values which are considered to lie within
 the window. WindowRange[0] is assumed to be less than WindowRange[1].

\item  \verb|obj.SetWindowValues (double , double )| -  Specify the range of output values that the window range is mapped
 into. This is effectively a scaling and shifting of the original
 function values.

\item  \verb|obj.SetWindowValues (double  a[2])| -  Specify the range of output values that the window range is mapped
 into. This is effectively a scaling and shifting of the original
 function values.

\item  \verb|double = obj. GetWindowValues ()| -  Specify the range of output values that the window range is mapped
 into. This is effectively a scaling and shifting of the original
 function values.

\item  \verb|long = obj.GetMTime ()| -  Override modified time retrieval because of object dependencies.

\item  \verb|obj.Register (vtkObjectBase o)| -  Participate in garbage collection.

\item  \verb|obj.UnRegister (vtkObjectBase o)| -  Participate in garbage collection.

\end{itemize}
