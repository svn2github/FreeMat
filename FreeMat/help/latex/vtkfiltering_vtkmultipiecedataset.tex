\section{vtkMultiPieceDataSet}

\subsection{Usage}

 A vtkMultiPieceDataSet dataset groups multiple data pieces together. 
 For example, say that a simulation broke a volume into 16 piece so that 
 each piece can be processed with 1 process in parallel. We want to load 
 this volume in a visualization cluster of 4 nodes. Each node will get 4 
 pieces, not necessarily forming a whole rectangular piece. In this case, 
 it is not possible to append the 4 pieces together into a vtkImageData. 
 In this case, these 4 pieces can be collected together using a 
 vtkMultiPieceDataSet. 
 Note that vtkMultiPieceDataSet is intended to be included in other composite
 datasets eg. vtkMultiBlockDataSet, vtkHierarchicalBoxDataSet. Hence the lack
 of algorithms producting vtkMultiPieceDataSet.

To create an instance of class vtkMultiPieceDataSet, simply
invoke its constructor as follows
\begin{verbatim}
  obj = vtkMultiPieceDataSet
\end{verbatim}
\subsection{Methods}

The class vtkMultiPieceDataSet has several methods that can be used.
  They are listed below.
Note that the documentation is translated automatically from the VTK sources,
and may not be completely intelligible.  When in doubt, consult the VTK website.
In the methods listed below, \verb|obj| is an instance of the vtkMultiPieceDataSet class.
\begin{itemize}
\item  \verb|string = obj.GetClassName ()|

\item  \verb|int = obj.IsA (string name)|

\item  \verb|vtkMultiPieceDataSet = obj.NewInstance ()|

\item  \verb|vtkMultiPieceDataSet = obj.SafeDownCast (vtkObject o)|

\item  \verb|int = obj.GetDataObjectType ()| -  Set the number of pieces. This will cause allocation if the new number of
 pieces is greater than the current size. All new pieces are initialized to
 null.

\item  \verb|obj.SetNumberOfPieces (int numpieces)| -  Set the number of pieces. This will cause allocation if the new number of
 pieces is greater than the current size. All new pieces are initialized to
 null.

\item  \verb|int = obj.GetNumberOfPieces ()| -  Returns the number of pieces.

\item  \verb|vtkDataSet = obj.GetPiece (int pieceno)| -  Returns the piece at the given index. 

\item  \verb|vtkDataObject = obj.GetPieceAsDataObject (int pieceno)| -  Returns the piece at the given index. 

\item  \verb|obj.SetPiece (int pieceno, vtkDataObject piece)| -  Sets the data object as the given piece. The total number of pieces will 
 be resized to fit the requested piece no.

\item  \verb|int = obj.HasMetaData (int piece)| -  Returns the meta-data for the piece. If none is already present, a new
 vtkInformation object will be allocated. Use HasMetaData to avoid
 allocating vtkInformation objects.

\item  \verb|vtkInformation = obj.GetMetaData (int pieceno)| -  Unhiding superclass method.

\item  \verb|vtkInformation = obj.GetMetaData (vtkCompositeDataIterator iter)| -  Unhiding superclass method.

\item  \verb|int = obj.HasMetaData (vtkCompositeDataIterator iter)|

\end{itemize}
