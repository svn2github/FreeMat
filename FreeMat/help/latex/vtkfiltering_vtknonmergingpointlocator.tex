\section{vtkNonMergingPointLocator}

\subsection{Usage}

  As a special sub-class of vtkPointLocator, vtkNonMergingPointLocator is
  intended for direct / check-free insertion of points into a vtkPoints 
  object. In other words, any given point is always directly inserted. 
  The name emphasizes the difference between this class and its sibling 
  class vtkMergePoints in that the latter class performs check-based zero 
  tolerance point insertion (or to 'merge' exactly duplicate / coincident
  points) by exploiting the uniform bin mechanism employed by the parent 
  class vtkPointLocator. vtkPointLocator allows for generic (zero and non-
  zero) tolerance point insertion as well as point location.


To create an instance of class vtkNonMergingPointLocator, simply
invoke its constructor as follows
\begin{verbatim}
  obj = vtkNonMergingPointLocator
\end{verbatim}
\subsection{Methods}

The class vtkNonMergingPointLocator has several methods that can be used.
  They are listed below.
Note that the documentation is translated automatically from the VTK sources,
and may not be completely intelligible.  When in doubt, consult the VTK website.
In the methods listed below, \verb|obj| is an instance of the vtkNonMergingPointLocator class.
\begin{itemize}
\item  \verb|string = obj.GetClassName ()|

\item  \verb|int = obj.IsA (string name)|

\item  \verb|vtkNonMergingPointLocator = obj.NewInstance ()|

\item  \verb|vtkNonMergingPointLocator = obj.SafeDownCast (vtkObject o)|

\end{itemize}
