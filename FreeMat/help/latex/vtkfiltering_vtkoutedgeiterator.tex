\section{vtkOutEdgeIterator}

\subsection{Usage}

 vtkOutEdgeIterator iterates through all edges whose source is a particular
 vertex. Instantiate this class directly and call Initialize() to traverse
 the vertex of a graph. Alternately, use GetInEdges() on the graph to
 initialize the iterator. it->Next() returns a vtkOutEdgeType structure,
 which contains Id, the edge's id, and Target, the edge's target vertex.


To create an instance of class vtkOutEdgeIterator, simply
invoke its constructor as follows
\begin{verbatim}
  obj = vtkOutEdgeIterator
\end{verbatim}
\subsection{Methods}

The class vtkOutEdgeIterator has several methods that can be used.
  They are listed below.
Note that the documentation is translated automatically from the VTK sources,
and may not be completely intelligible.  When in doubt, consult the VTK website.
In the methods listed below, \verb|obj| is an instance of the vtkOutEdgeIterator class.
\begin{itemize}
\item  \verb|string = obj.GetClassName ()|

\item  \verb|int = obj.IsA (string name)|

\item  \verb|vtkOutEdgeIterator = obj.NewInstance ()|

\item  \verb|vtkOutEdgeIterator = obj.SafeDownCast (vtkObject o)|

\item  \verb|obj.Initialize (vtkGraph g, vtkIdType v)| -  Initialize the iterator with a graph and vertex.

\item  \verb|vtkGraph = obj.GetGraph ()| -  Get the graph and vertex associated with this iterator.

\item  \verb|vtkIdType = obj.GetVertex ()| -  Get the graph and vertex associated with this iterator.

\item  \verb|vtkGraphEdge = obj.NextGraphEdge ()| -  Just like Next(), but
 returns heavy-weight vtkGraphEdge object instead of
 the vtkEdgeType struct, for use with wrappers.
 The graph edge is owned by this iterator, and changes
 after each call to NextGraphEdge().

\item  \verb|bool = obj.HasNext ()|

\end{itemize}
