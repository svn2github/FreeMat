\section{vtkParametricSpline}

\subsection{Usage}

 vtkParametricSpline is a parametric function for 1D interpolating splines.
 vtkParametricSpline maps the single parameter u into a 3D point (x,y,z)
 using three instances of interpolating splines.  This family of 1D splines
 is quaranteed to be parameterized in the interval [0,1].  Attempting to
 evaluate outside this interval will cause the parameter u to be clamped in
 the range [0,1].

 When constructed, this class creates instances of vtkCardinalSpline for
 each of the x-y-z coordinates. The user may choose to replace these with
 their own instances of subclasses of vtkSpline.


To create an instance of class vtkParametricSpline, simply
invoke its constructor as follows
\begin{verbatim}
  obj = vtkParametricSpline
\end{verbatim}
\subsection{Methods}

The class vtkParametricSpline has several methods that can be used.
  They are listed below.
Note that the documentation is translated automatically from the VTK sources,
and may not be completely intelligible.  When in doubt, consult the VTK website.
In the methods listed below, \verb|obj| is an instance of the vtkParametricSpline class.
\begin{itemize}
\item  \verb|string = obj.GetClassName ()|

\item  \verb|int = obj.IsA (string name)|

\item  \verb|vtkParametricSpline = obj.NewInstance ()|

\item  \verb|vtkParametricSpline = obj.SafeDownCast (vtkObject o)|

\item  \verb|int = obj.GetDimension ()| -  Evaluate the spline at parametric coordinate u[0] returning
 the point coordinate Pt[3].

\item  \verb|obj.Evaluate (double u[3], double Pt[3], double Du[9])| -  Evaluate the spline at parametric coordinate u[0] returning
 the point coordinate Pt[3].

\item  \verb|double = obj.EvaluateScalar (double u[3], double Pt[3], double Du[9])| -  Evaluate a scalar value at parametric coordinate u[0] and Pt[3].
 The scalar value is just the parameter u[0].

\item  \verb|obj.SetXSpline (vtkSpline )| -  By default, this class is constructed with three instances of
 vtkCardinalSpline (for each of the x-y-z coordinate axes). The user may
 choose to create and assign their own instances of vtkSpline.

\item  \verb|obj.SetYSpline (vtkSpline )| -  By default, this class is constructed with three instances of
 vtkCardinalSpline (for each of the x-y-z coordinate axes). The user may
 choose to create and assign their own instances of vtkSpline.

\item  \verb|obj.SetZSpline (vtkSpline )| -  By default, this class is constructed with three instances of
 vtkCardinalSpline (for each of the x-y-z coordinate axes). The user may
 choose to create and assign their own instances of vtkSpline.

\item  \verb|vtkSpline = obj.GetXSpline ()| -  By default, this class is constructed with three instances of
 vtkCardinalSpline (for each of the x-y-z coordinate axes). The user may
 choose to create and assign their own instances of vtkSpline.

\item  \verb|vtkSpline = obj.GetYSpline ()| -  By default, this class is constructed with three instances of
 vtkCardinalSpline (for each of the x-y-z coordinate axes). The user may
 choose to create and assign their own instances of vtkSpline.

\item  \verb|vtkSpline = obj.GetZSpline ()| -  By default, this class is constructed with three instances of
 vtkCardinalSpline (for each of the x-y-z coordinate axes). The user may
 choose to create and assign their own instances of vtkSpline.

\item  \verb|obj.SetPoints (vtkPoints )| -  Specify the list of points defining the spline. Do this by 
 specifying a vtkPoints array containing the points. Note that
 the order of the points in vtkPoints is the order that the
 splines will be fit.

\item  \verb|vtkPoints = obj.GetPoints ()| -  Specify the list of points defining the spline. Do this by 
 specifying a vtkPoints array containing the points. Note that
 the order of the points in vtkPoints is the order that the
 splines will be fit.

\item  \verb|obj.SetNumberOfPoints (vtkIdType numPts)| -  Another API to set the points. Set the number of points and then set the
 individual point coordinates.

\item  \verb|obj.SetPoint (vtkIdType index, double x, double y, double z)| -  Another API to set the points. Set the number of points and then set the
 individual point coordinates.

\item  \verb|obj.SetClosed (int )| -  Control whether the spline is open or closed. A closed spline forms
 a continuous loop: the first and last points are the same, and
 derivatives are continuous.

\item  \verb|int = obj.GetClosed ()| -  Control whether the spline is open or closed. A closed spline forms
 a continuous loop: the first and last points are the same, and
 derivatives are continuous.

\item  \verb|obj.ClosedOn ()| -  Control whether the spline is open or closed. A closed spline forms
 a continuous loop: the first and last points are the same, and
 derivatives are continuous.

\item  \verb|obj.ClosedOff ()| -  Control whether the spline is open or closed. A closed spline forms
 a continuous loop: the first and last points are the same, and
 derivatives are continuous.

\item  \verb|obj.SetParameterizeByLength (int )| -  Control whether the spline is parameterized by length or by point index.
 Default is by length.

\item  \verb|int = obj.GetParameterizeByLength ()| -  Control whether the spline is parameterized by length or by point index.
 Default is by length.

\item  \verb|obj.ParameterizeByLengthOn ()| -  Control whether the spline is parameterized by length or by point index.
 Default is by length.

\item  \verb|obj.ParameterizeByLengthOff ()| -  Control whether the spline is parameterized by length or by point index.
 Default is by length.

\item  \verb|obj.SetLeftConstraint (int )| -  Set the type of constraint of the left(right) end points. Four
 constraints are available:

 0: the first derivative at left(right) most point is determined
 from the line defined from the first(last) two points.

 1: the first derivative at left(right) most point is set to
 Left(Right)Value.
 
 2: the second derivative at left(right) most point is set to
 Left(Right)Value.
 
 3: the second derivative at left(right)most points is Left(Right)Value
 times second derivative at first interior point.

\item  \verb|int = obj.GetLeftConstraintMinValue ()| -  Set the type of constraint of the left(right) end points. Four
 constraints are available:

 0: the first derivative at left(right) most point is determined
 from the line defined from the first(last) two points.

 1: the first derivative at left(right) most point is set to
 Left(Right)Value.
 
 2: the second derivative at left(right) most point is set to
 Left(Right)Value.
 
 3: the second derivative at left(right)most points is Left(Right)Value
 times second derivative at first interior point.

\item  \verb|int = obj.GetLeftConstraintMaxValue ()| -  Set the type of constraint of the left(right) end points. Four
 constraints are available:

 0: the first derivative at left(right) most point is determined
 from the line defined from the first(last) two points.

 1: the first derivative at left(right) most point is set to
 Left(Right)Value.
 
 2: the second derivative at left(right) most point is set to
 Left(Right)Value.
 
 3: the second derivative at left(right)most points is Left(Right)Value
 times second derivative at first interior point.

\item  \verb|int = obj.GetLeftConstraint ()| -  Set the type of constraint of the left(right) end points. Four
 constraints are available:

 0: the first derivative at left(right) most point is determined
 from the line defined from the first(last) two points.

 1: the first derivative at left(right) most point is set to
 Left(Right)Value.
 
 2: the second derivative at left(right) most point is set to
 Left(Right)Value.
 
 3: the second derivative at left(right)most points is Left(Right)Value
 times second derivative at first interior point.

\item  \verb|obj.SetRightConstraint (int )| -  Set the type of constraint of the left(right) end points. Four
 constraints are available:

 0: the first derivative at left(right) most point is determined
 from the line defined from the first(last) two points.

 1: the first derivative at left(right) most point is set to
 Left(Right)Value.
 
 2: the second derivative at left(right) most point is set to
 Left(Right)Value.
 
 3: the second derivative at left(right)most points is Left(Right)Value
 times second derivative at first interior point.

\item  \verb|int = obj.GetRightConstraintMinValue ()| -  Set the type of constraint of the left(right) end points. Four
 constraints are available:

 0: the first derivative at left(right) most point is determined
 from the line defined from the first(last) two points.

 1: the first derivative at left(right) most point is set to
 Left(Right)Value.
 
 2: the second derivative at left(right) most point is set to
 Left(Right)Value.
 
 3: the second derivative at left(right)most points is Left(Right)Value
 times second derivative at first interior point.

\item  \verb|int = obj.GetRightConstraintMaxValue ()| -  Set the type of constraint of the left(right) end points. Four
 constraints are available:

 0: the first derivative at left(right) most point is determined
 from the line defined from the first(last) two points.

 1: the first derivative at left(right) most point is set to
 Left(Right)Value.
 
 2: the second derivative at left(right) most point is set to
 Left(Right)Value.
 
 3: the second derivative at left(right)most points is Left(Right)Value
 times second derivative at first interior point.

\item  \verb|int = obj.GetRightConstraint ()| -  Set the type of constraint of the left(right) end points. Four
 constraints are available:

 0: the first derivative at left(right) most point is determined
 from the line defined from the first(last) two points.

 1: the first derivative at left(right) most point is set to
 Left(Right)Value.
 
 2: the second derivative at left(right) most point is set to
 Left(Right)Value.
 
 3: the second derivative at left(right)most points is Left(Right)Value
 times second derivative at first interior point.

\item  \verb|obj.SetLeftValue (double )| -  The values of the derivative on the left and right sides. The value
 is used only if the left(right) constraint is type 1-3.

\item  \verb|double = obj.GetLeftValue ()| -  The values of the derivative on the left and right sides. The value
 is used only if the left(right) constraint is type 1-3.

\item  \verb|obj.SetRightValue (double )| -  The values of the derivative on the left and right sides. The value
 is used only if the left(right) constraint is type 1-3.

\item  \verb|double = obj.GetRightValue ()| -  The values of the derivative on the left and right sides. The value
 is used only if the left(right) constraint is type 1-3.

\end{itemize}
