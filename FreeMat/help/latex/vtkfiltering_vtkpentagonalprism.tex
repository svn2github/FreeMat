\section{vtkPentagonalPrism}

\subsection{Usage}

 vtkPentagonalPrism is a concrete implementation of vtkCell to represent a
 linear 3D prism with pentagonal base. Such prism is defined by the ten points (0-9)
 where (0,1,2,3,4) is the base of the prism which, using the right hand
 rule, forms a pentagon whose normal points is in the direction of the
 opposite face (5,6,7,8,9).

To create an instance of class vtkPentagonalPrism, simply
invoke its constructor as follows
\begin{verbatim}
  obj = vtkPentagonalPrism
\end{verbatim}
\subsection{Methods}

The class vtkPentagonalPrism has several methods that can be used.
  They are listed below.
Note that the documentation is translated automatically from the VTK sources,
and may not be completely intelligible.  When in doubt, consult the VTK website.
In the methods listed below, \verb|obj| is an instance of the vtkPentagonalPrism class.
\begin{itemize}
\item  \verb|string = obj.GetClassName ()|

\item  \verb|int = obj.IsA (string name)|

\item  \verb|vtkPentagonalPrism = obj.NewInstance ()|

\item  \verb|vtkPentagonalPrism = obj.SafeDownCast (vtkObject o)|

\item  \verb|int = obj.GetCellType ()| -  See the vtkCell3D API for descriptions of these methods.

\item  \verb|int = obj.GetCellDimension ()| -  See the vtkCell3D API for descriptions of these methods.

\item  \verb|int = obj.GetNumberOfEdges ()| -  See the vtkCell3D API for descriptions of these methods.

\item  \verb|int = obj.GetNumberOfFaces ()| -  See the vtkCell3D API for descriptions of these methods.

\item  \verb|vtkCell = obj.GetEdge (int edgeId)| -  See the vtkCell3D API for descriptions of these methods.

\item  \verb|vtkCell = obj.GetFace (int faceId)| -  See the vtkCell3D API for descriptions of these methods.

\item  \verb|int = obj.CellBoundary (int subId, double pcoords[3], vtkIdList pts)| -  See the vtkCell3D API for descriptions of these methods.

\item  \verb|int = obj.Triangulate (int index, vtkIdList ptIds, vtkPoints pts)|

\item  \verb|obj.Derivatives (int subId, double pcoords[3], double values, int dim, double derivs)|

\item  \verb|int = obj.GetParametricCenter (double pcoords[3])| -  Return the center of the wedge in parametric coordinates.

\item  \verb|obj.InterpolateFunctions (double pcoords[3], double weights[10])| -  Compute the interpolation functions/derivatives
 (aka shape functions/derivatives)

\item  \verb|obj.InterpolateDerivs (double pcoords[3], double derivs[30])| -  Return the ids of the vertices defining edge/face (`edgeId`/`faceId').
 Ids are related to the cell, not to the dataset.

\end{itemize}
