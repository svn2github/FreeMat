\section{vtkQuadraticEdge}

\subsection{Usage}

 vtkQuadraticEdge is a concrete implementation of vtkNonLinearCell to
 represent a one-dimensional, 3-nodes, isoparametric parabolic line. The
 interpolation is the standard finite element, quadratic isoparametric
 shape function. The cell includes a mid-edge node. The ordering of the
 three points defining the cell is point ids (0,1,2) where id \#2 is the
 midedge node.

To create an instance of class vtkQuadraticEdge, simply
invoke its constructor as follows
\begin{verbatim}
  obj = vtkQuadraticEdge
\end{verbatim}
\subsection{Methods}

The class vtkQuadraticEdge has several methods that can be used.
  They are listed below.
Note that the documentation is translated automatically from the VTK sources,
and may not be completely intelligible.  When in doubt, consult the VTK website.
In the methods listed below, \verb|obj| is an instance of the vtkQuadraticEdge class.
\begin{itemize}
\item  \verb|string = obj.GetClassName ()|

\item  \verb|int = obj.IsA (string name)|

\item  \verb|vtkQuadraticEdge = obj.NewInstance ()|

\item  \verb|vtkQuadraticEdge = obj.SafeDownCast (vtkObject o)|

\item  \verb|int = obj.GetCellType ()| -  Implement the vtkCell API. See the vtkCell API for descriptions
 of these methods.

\item  \verb|int = obj.GetCellDimension ()| -  Implement the vtkCell API. See the vtkCell API for descriptions
 of these methods.

\item  \verb|int = obj.GetNumberOfEdges ()| -  Implement the vtkCell API. See the vtkCell API for descriptions
 of these methods.

\item  \verb|int = obj.GetNumberOfFaces ()| -  Implement the vtkCell API. See the vtkCell API for descriptions
 of these methods.

\item  \verb|vtkCell = obj.GetEdge (int )| -  Implement the vtkCell API. See the vtkCell API for descriptions
 of these methods.

\item  \verb|vtkCell = obj.GetFace (int )|

\item  \verb|int = obj.CellBoundary (int subId, double pcoords[3], vtkIdList pts)|

\item  \verb|obj.Contour (double value, vtkDataArray cellScalars, vtkIncrementalPointLocator locator, vtkCellArray verts, vtkCellArray lines, vtkCellArray polys, vtkPointData inPd, vtkPointData outPd, vtkCellData inCd, vtkIdType cellId, vtkCellData outCd)|

\item  \verb|int = obj.Triangulate (int index, vtkIdList ptIds, vtkPoints pts)|

\item  \verb|obj.Derivatives (int subId, double pcoords[3], double values, int dim, double derivs)|

\item  \verb|obj.Clip (double value, vtkDataArray cellScalars, vtkIncrementalPointLocator locator, vtkCellArray lines, vtkPointData inPd, vtkPointData outPd, vtkCellData inCd, vtkIdType cellId, vtkCellData outCd, int insideOut)| -  Clip this edge using scalar value provided. Like contouring, except
 that it cuts the edge to produce linear line segments.

\item  \verb|int = obj.GetParametricCenter (double pcoords[3])| -  Return the center of the quadratic tetra in parametric coordinates.

\item  \verb|obj.InterpolateFunctions (double pcoords[3], double weights[3])| -  Compute the interpolation functions/derivatives
 (aka shape functions/derivatives)

\item  \verb|obj.InterpolateDerivs (double pcoords[3], double derivs[3])|

\end{itemize}
