\section{vtkQuadraticLinearWedge}

\subsection{Usage}

 vtkQuadraticLinearWedge is a concrete implementation of vtkNonLinearCell to
 represent a three-dimensional, 12-node isoparametric linear quadratic
 wedge. The interpolation is the standard finite element, quadratic
 isoparametric shape function in xy - layer and the linear functions in z - direction. 
 The cell includes mid-edge node in the triangle edges. The
 ordering of the 12 points defining the cell is point ids (0-5,6-12)
 where point ids 0-5 are the six corner vertices of the wedge; followed by
 six midedge nodes (6-12). Note that these midedge nodes correspond lie
 on the edges defined by (0,1), (1,2), (2,0), (3,4), (4,5), (5,3).
 The Edges (0,3), (1,4), (2,5) dont have midedge nodes.


To create an instance of class vtkQuadraticLinearWedge, simply
invoke its constructor as follows
\begin{verbatim}
  obj = vtkQuadraticLinearWedge
\end{verbatim}
\subsection{Methods}

The class vtkQuadraticLinearWedge has several methods that can be used.
  They are listed below.
Note that the documentation is translated automatically from the VTK sources,
and may not be completely intelligible.  When in doubt, consult the VTK website.
In the methods listed below, \verb|obj| is an instance of the vtkQuadraticLinearWedge class.
\begin{itemize}
\item  \verb|string = obj.GetClassName ()|

\item  \verb|int = obj.IsA (string name)|

\item  \verb|vtkQuadraticLinearWedge = obj.NewInstance ()|

\item  \verb|vtkQuadraticLinearWedge = obj.SafeDownCast (vtkObject o)|

\item  \verb|int = obj.GetCellType ()| -  Implement the vtkCell API. See the vtkCell API for descriptions
 of these methods.

\item  \verb|int = obj.GetCellDimension ()| -  Implement the vtkCell API. See the vtkCell API for descriptions
 of these methods.

\item  \verb|int = obj.GetNumberOfEdges ()| -  Implement the vtkCell API. See the vtkCell API for descriptions
 of these methods.

\item  \verb|int = obj.GetNumberOfFaces ()| -  Implement the vtkCell API. See the vtkCell API for descriptions
 of these methods.

\item  \verb|vtkCell = obj.GetEdge (int edgeId)| -  Implement the vtkCell API. See the vtkCell API for descriptions
 of these methods.

\item  \verb|vtkCell = obj.GetFace (int faceId)| -  Implement the vtkCell API. See the vtkCell API for descriptions
 of these methods.

\item  \verb|int = obj.CellBoundary (int subId, double pcoords[3], vtkIdList pts)|

\item  \verb|obj.Contour (double value, vtkDataArray cellScalars, vtkIncrementalPointLocator locator, vtkCellArray verts, vtkCellArray lines, vtkCellArray polys, vtkPointData inPd, vtkPointData outPd, vtkCellData inCd, vtkIdType cellId, vtkCellData outCd)| -  The quadratic linear wege is splitted into 4 linear wedges,
 each of them is contoured by a provided scalar value

\item  \verb|int = obj.Triangulate (int index, vtkIdList ptIds, vtkPoints pts)| -  The quadratic linear wege is splitted into 4 linear wedges,
 each of them is contoured by a provided scalar value

\item  \verb|obj.Derivatives (int subId, double pcoords[3], double values, int dim, double derivs)| -  The quadratic linear wege is splitted into 4 linear wedges,
 each of them is contoured by a provided scalar value

\item  \verb|obj.Clip (double value, vtkDataArray cellScalars, vtkIncrementalPointLocator locator, vtkCellArray tetras, vtkPointData inPd, vtkPointData outPd, vtkCellData inCd, vtkIdType cellId, vtkCellData outCd, int insideOut)| -  Clip this quadratic linear wedge using scalar value provided. Like
 contouring, except that it cuts the hex to produce linear
 tetrahedron.

\item  \verb|int = obj.GetParametricCenter (double pcoords[3])| -  Return the center of the quadratic linear wedge in parametric coordinates.

\item  \verb|obj.InterpolateFunctions (double pcoords[3], double weights[15])| -  Compute the interpolation functions/derivatives
 (aka shape functions/derivatives)

\item  \verb|obj.InterpolateDerivs (double pcoords[3], double derivs[45])| -  Return the ids of the vertices defining edge/face (`edgeId`/`faceId').
 Ids are related to the cell, not to the dataset.

\end{itemize}
