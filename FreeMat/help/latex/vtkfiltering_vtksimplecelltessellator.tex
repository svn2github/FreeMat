\section{vtkSimpleCellTessellator}

\subsection{Usage}

 vtkSimpleCellTessellator is a helper class to perform adaptive tessellation
 of particular cell topologies. The major purpose for this class is to
 transform higher-order cell types (e.g., higher-order finite elements)
 into linear cells that can then be easily visualized by VTK. This class
 works in conjunction with the vtkGenericDataSet and vtkGenericAdaptorCell
 classes.

 This algorithm is based on edge subdivision. An error metric along each
 edge is evaluated, and if the error is greater than some tolerance, the
 edge is subdivided (as well as all connected 2D and 3D cells). The process
 repeats until the error metric is satisfied. Since the algorithm is based
 on edge subdivision it inherently avoid T-junctions.

 A significant issue addressed by this algorithm is to insure face
 compatibility across neigboring cells. That is, diagonals due to face
 triangulation must match to insure that the mesh is compatible. The
 algorithm employs a precomputed table to accelerate the tessellation
 process. The table was generated with the help of vtkOrderedTriangulator
 the basic idea is that the choice of diagonal is made only by considering the
 relative value of the point ids.


To create an instance of class vtkSimpleCellTessellator, simply
invoke its constructor as follows
\begin{verbatim}
  obj = vtkSimpleCellTessellator
\end{verbatim}
\subsection{Methods}

The class vtkSimpleCellTessellator has several methods that can be used.
  They are listed below.
Note that the documentation is translated automatically from the VTK sources,
and may not be completely intelligible.  When in doubt, consult the VTK website.
In the methods listed below, \verb|obj| is an instance of the vtkSimpleCellTessellator class.
\begin{itemize}
\item  \verb|string = obj.GetClassName ()|

\item  \verb|int = obj.IsA (string name)|

\item  \verb|vtkSimpleCellTessellator = obj.NewInstance ()|

\item  \verb|vtkSimpleCellTessellator = obj.SafeDownCast (vtkObject o)|

\item  \verb|vtkGenericAdaptorCell = obj.GetGenericCell ()| -  Get the higher order cell in order to access the evaluation function.

\item  \verb|obj.TessellateFace (vtkGenericAdaptorCell cell, vtkGenericAttributeCollection att, vtkIdType index, vtkDoubleArray points, vtkCellArray cellArray, vtkPointData internalPd)| -  Tessellate a face of a 3D `cell'. The face is specified by the
 index value.
 The result is a set of smaller linear triangles in `cellArray' with
 `points' and point data `internalPd'.
 
 
 
 
 
 
 

\item  \verb|obj.Tessellate (vtkGenericAdaptorCell cell, vtkGenericAttributeCollection att, vtkDoubleArray points, vtkCellArray cellArray, vtkPointData internalPd)| -  Tessellate a 3D `cell'. The result is a set of smaller linear
 tetrahedra in `cellArray' with `points' and point data `internalPd'.
 
 
 
 
 
 

\item  \verb|obj.Triangulate (vtkGenericAdaptorCell cell, vtkGenericAttributeCollection att, vtkDoubleArray points, vtkCellArray cellArray, vtkPointData internalPd)| -  Triangulate a 2D `cell'. The result is a set of smaller linear triangles
 in `cellArray' with `points' and point data `internalPd'.
 
 
 
 
 
 

\item  \verb|obj.Reset ()| -  Reset the output for repeated use of this class.

\item  \verb|obj.Initialize (vtkGenericDataSet ds)| -  Initialize the tessellator with a data set `ds'.

\item  \verb|int = obj.GetFixedSubdivisions ()| -  Return the number of fixed subdivisions. It is used to prevent from
 infinite loop in degenerated cases. For order 3 or higher, if the
 inflection point is exactly on the mid-point, error metric will not
 detect that a subdivision is required. 0 means no fixed subdivision:
 there will be only adaptive subdivisions.

 The algorithm first performs `GetFixedSubdivisions' non adaptive
 subdivisions followed by at most `GetMaxAdaptiveSubdivisions' adaptive
 subdivisions. Hence, there are at most `GetMaxSubdivisionLevel'
 subdivisions.
 

\item  \verb|int = obj.GetMaxSubdivisionLevel ()| -  Return the maximum level of subdivision. It is used to prevent from
 infinite loop in degenerated cases. For order 3 or higher, if the
 inflection point is exactly on the mid-point, error metric will not
 detect that a subdivision is required. 0 means no subdivision,
 neither fixed nor adaptive.
 

\item  \verb|int = obj.GetMaxAdaptiveSubdivisions ()| -  Return the maximum number of adaptive subdivisions.
 

\item  \verb|obj.SetFixedSubdivisions (int level)| -  Set the number of fixed subdivisions. See GetFixedSubdivisions() for
 more explanations.
 
 

\item  \verb|obj.SetMaxSubdivisionLevel (int level)| -  Set the maximum level of subdivision. See GetMaxSubdivisionLevel() for
 more explanations.
 
 

\item  \verb|obj.SetSubdivisionLevels (int fixed, int maxLevel)| -  Set both the number of fixed subdivisions and the maximum level of
 subdivisions. See GetFixedSubdivisions(), GetMaxSubdivisionLevel() and
 GetMaxAdaptiveSubdivisions() for more explanations.
 
 
 
 

\end{itemize}
