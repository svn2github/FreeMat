\section{vtkSource}

\subsection{Usage}

 vtkSource is an abstract object that specifies behavior and interface
 of source objects. Source objects are objects that begin visualization
 pipeline. Sources include readers (read data from file or communications
 port) and procedural sources (generate data programmatically). vtkSource 
 objects are also objects that generate output data. In this sense
 vtkSource is used as a superclass to vtkFilter.

 Concrete subclasses of vtkSource must define Update() and Execute() 
 methods. The public method Update() invokes network execution and will
 bring the network up-to-date. The protected Execute() method actually
 does the work of data creation/generation. The difference between the two
 methods is that Update() implements input consistency checks and modified
 time comparisons and then invokes the Execute() which is an implementation 
 of a particular algorithm.

 An important feature of subclasses of vtkSource is that it is possible 
 to control the memory-management model (i.e., retain output versus delete
 output data). If enabled the ReleaseDataFlag enables the deletion of the
 output data once the downstream process object finishes processing the
 data (please see text).

To create an instance of class vtkSource, simply
invoke its constructor as follows
\begin{verbatim}
  obj = vtkSource
\end{verbatim}
\subsection{Methods}

The class vtkSource has several methods that can be used.
  They are listed below.
Note that the documentation is translated automatically from the VTK sources,
and may not be completely intelligible.  When in doubt, consult the VTK website.
In the methods listed below, \verb|obj| is an instance of the vtkSource class.
\begin{itemize}
\item  \verb|string = obj.GetClassName ()|

\item  \verb|int = obj.IsA (string name)|

\item  \verb|vtkSource = obj.NewInstance ()|

\item  \verb|vtkSource = obj.SafeDownCast (vtkObject o)|

\item  \verb|obj.Update ()| -  Bring object up-to-date before execution. Update() checks modified
 time against last execution time, and re-executes object if necessary.

\item  \verb|obj.UpdateWholeExtent ()| -  Like update, but make sure the update extent is the whole extent in
 the output.

\item  \verb|obj.UpdateInformation ()| -  Updates any global information about the data 
 (like spacing for images)

\item  \verb|obj.PropagateUpdateExtent (vtkDataObject output)| -  WARNING: INTERNAL METHOD - NOT FOR GENERAL USE. 
 THIS METHOD IS PART OF THE PIPELINE UPDATE FUNCTIONALITY.
 The update extent for this object is propagated up the pipeline.
 This propagation may early terminate based on the PipelineMTime.

\item  \verb|obj.TriggerAsynchronousUpdate ()| -  WARNING: INTERNAL METHOD - NOT FOR GENERAL USE. 
 THIS METHOD IS PART OF THE PIPELINE UPDATE FUNCTIONALITY.
 Propagate back up the pipeline for ports and trigger the update on the
 other side of the port to allow for asynchronous parallel processing in
 the pipeline.
 This propagation may early terminate based on the PipelineMTime.

\item  \verb|obj.UpdateData (vtkDataObject output)| -  WARNING: INTERNAL METHOD - NOT FOR GENERAL USE. 
 THIS METHOD IS PART OF THE PIPELINE UPDATE FUNCTIONALITY.
 Propagate the update back up the pipeline, and perform the actual 
 work of updating on the way down. When the propagate arrives at a
 port, block and wait for the asynchronous update to finish on the
 other side.
 This propagation may early terminate based on the PipelineMTime.

\item  \verb|obj.ComputeInputUpdateExtents (vtkDataObject output)| -  What is the input update extent that is required to produce the
 desired output? By default, the whole input is always required but
 this is overridden in many subclasses. 

\item  \verb|obj.SetReleaseDataFlag (int )| -  Turn on/off flag to control whether this object's data is released
 after being used by a source.

\item  \verb|int = obj.GetReleaseDataFlag ()| -  Turn on/off flag to control whether this object's data is released
 after being used by a source.

\item  \verb|obj.ReleaseDataFlagOn ()| -  Turn on/off flag to control whether this object's data is released
 after being used by a source.

\item  \verb|obj.ReleaseDataFlagOff ()| -  Turn on/off flag to control whether this object's data is released
 after being used by a source.

\item  \verb|int = obj.GetNumberOfOutputs ()| -  Return an array with all the inputs of this process object.
 This is useful for tracing back in the pipeline to construct
 graphs etc.

\item  \verb|obj.UnRegisterAllOutputs (void )| -  Release/disconnect all outputs of this source. This is intended to be
 called prior to Delete() if the user is concerned about outputs holding
 on to the filter/source.

\item  \verb|int = obj.GetOutputIndex (vtkDataObject out)| -  Return what index output the passed in output is, return -1 if it
 does not match any of the outputs

\item  \verb|obj.SetExecutive (vtkExecutive executive)| -  Set this algorithm's executive.  This algorithm is removed from
 any executive to which it has previously been assigned and then
 assigned to the given executive.

\end{itemize}
