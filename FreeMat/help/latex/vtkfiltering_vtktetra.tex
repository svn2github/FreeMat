\section{vtkTetra}

\subsection{Usage}

 vtkTetra is a concrete implementation of vtkCell to represent a 3D
 tetrahedron. vtkTetra uses the standard isoparametric shape functions
 for a linear tetrahedron. The tetrahedron is defined by the four points
 (0-3); where (0,1,2) is the base of the tetrahedron which, using the
 right hand rule, forms a triangle whose normal points in the direction
 of the fourth point.

To create an instance of class vtkTetra, simply
invoke its constructor as follows
\begin{verbatim}
  obj = vtkTetra
\end{verbatim}
\subsection{Methods}

The class vtkTetra has several methods that can be used.
  They are listed below.
Note that the documentation is translated automatically from the VTK sources,
and may not be completely intelligible.  When in doubt, consult the VTK website.
In the methods listed below, \verb|obj| is an instance of the vtkTetra class.
\begin{itemize}
\item  \verb|string = obj.GetClassName ()|

\item  \verb|int = obj.IsA (string name)|

\item  \verb|vtkTetra = obj.NewInstance ()|

\item  \verb|vtkTetra = obj.SafeDownCast (vtkObject o)|

\item  \verb|int = obj.GetCellType ()| -  See the vtkCell API for descriptions of these methods.

\item  \verb|int = obj.GetNumberOfEdges ()| -  See the vtkCell API for descriptions of these methods.

\item  \verb|int = obj.GetNumberOfFaces ()| -  See the vtkCell API for descriptions of these methods.

\item  \verb|vtkCell = obj.GetEdge (int edgeId)| -  See the vtkCell API for descriptions of these methods.

\item  \verb|vtkCell = obj.GetFace (int faceId)| -  See the vtkCell API for descriptions of these methods.

\item  \verb|obj.Contour (double value, vtkDataArray cellScalars, vtkIncrementalPointLocator locator, vtkCellArray verts, vtkCellArray lines, vtkCellArray polys, vtkPointData inPd, vtkPointData outPd, vtkCellData inCd, vtkIdType cellId, vtkCellData outCd)| -  See the vtkCell API for descriptions of these methods.

\item  \verb|obj.Clip (double value, vtkDataArray cellScalars, vtkIncrementalPointLocator locator, vtkCellArray connectivity, vtkPointData inPd, vtkPointData outPd, vtkCellData inCd, vtkIdType cellId, vtkCellData outCd, int insideOut)| -  See the vtkCell API for descriptions of these methods.

\item  \verb|int = obj.Triangulate (int index, vtkIdList ptIds, vtkPoints pts)| -  See the vtkCell API for descriptions of these methods.

\item  \verb|obj.Derivatives (int subId, double pcoords[3], double values, int dim, double derivs)| -  See the vtkCell API for descriptions of these methods.

\item  \verb|int = obj.CellBoundary (int subId, double pcoords[3], vtkIdList pts)| -  Returns the set of points that are on the boundary of the tetrahedron that
 are closest parametrically to the point specified. This may include faces,
 edges, or vertices.

\item  \verb|int = obj.GetParametricCenter (double pcoords[3])| -  Return the center of the tetrahedron in parametric coordinates.

\item  \verb|double = obj.GetParametricDistance (double pcoords[3])| -  Return the distance of the parametric coordinate provided to the
 cell. If inside the cell, a distance of zero is returned.

\item  \verb|obj.InterpolateFunctions (double pcoords[3], double weights[4])| -  Compute the interpolation functions/derivatives
 (aka shape functions/derivatives)

\item  \verb|obj.InterpolateDerivs (double pcoords[3], double derivs[12])| -  Return the ids of the vertices defining edge/face (`edgeId`/`faceId').
 Ids are related to the cell, not to the dataset.

\end{itemize}
