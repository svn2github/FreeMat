\section{vtkTrivialProducer}

\subsection{Usage}

 vtkTrivialProducer allows stand-alone data objects to be connected
 as inputs in a pipeline.  All data objects that are connected to a
 pipeline involving vtkAlgorithm must have a producer.  This trivial
 producer allows data objects that are hand-constructed in a program
 without another vtk producer to be connected.

To create an instance of class vtkTrivialProducer, simply
invoke its constructor as follows
\begin{verbatim}
  obj = vtkTrivialProducer
\end{verbatim}
\subsection{Methods}

The class vtkTrivialProducer has several methods that can be used.
  They are listed below.
Note that the documentation is translated automatically from the VTK sources,
and may not be completely intelligible.  When in doubt, consult the VTK website.
In the methods listed below, \verb|obj| is an instance of the vtkTrivialProducer class.
\begin{itemize}
\item  \verb|string = obj.GetClassName ()|

\item  \verb|int = obj.IsA (string name)|

\item  \verb|vtkTrivialProducer = obj.NewInstance ()|

\item  \verb|vtkTrivialProducer = obj.SafeDownCast (vtkObject o)|

\item  \verb|obj.SetOutput (vtkDataObject output)| -  Set the data object that is ''produced'' by this producer.  It is
 never really modified.

\item  \verb|long = obj.GetMTime ()| -  The modified time of this producer is the newer of this object or
 the assigned output.

\end{itemize}
