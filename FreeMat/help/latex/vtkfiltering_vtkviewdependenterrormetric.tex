\section{vtkViewDependentErrorMetric}

\subsection{Usage}

 It is a concrete error metric, based on a geometric criterium in 
 the screen space: the variation of the projected edge from a projected 
 straight line


To create an instance of class vtkViewDependentErrorMetric, simply
invoke its constructor as follows
\begin{verbatim}
  obj = vtkViewDependentErrorMetric
\end{verbatim}
\subsection{Methods}

The class vtkViewDependentErrorMetric has several methods that can be used.
  They are listed below.
Note that the documentation is translated automatically from the VTK sources,
and may not be completely intelligible.  When in doubt, consult the VTK website.
In the methods listed below, \verb|obj| is an instance of the vtkViewDependentErrorMetric class.
\begin{itemize}
\item  \verb|string = obj.GetClassName ()| -  Standard VTK type and error macros.

\item  \verb|int = obj.IsA (string name)| -  Standard VTK type and error macros.

\item  \verb|vtkViewDependentErrorMetric = obj.NewInstance ()| -  Standard VTK type and error macros.

\item  \verb|vtkViewDependentErrorMetric = obj.SafeDownCast (vtkObject o)| -  Standard VTK type and error macros.

\item  \verb|double = obj.GetPixelTolerance ()| -  Return the squared screen-based geometric accurary measured in pixels.
 An accuracy less or equal to 0.25 (0.5\^2) ensures that the screen-space
 interpolation of a mid-point matchs exactly with the projection of the
 mid-point (a value less than 1 but greater than 0.25 is not enough,
 because of 8-neighbors). Maybe it is useful for lower accuracy in case of
 anti-aliasing?
 

\item  \verb|obj.SetPixelTolerance (double value)| -  Set the squared screen-based geometric accuracy measured in pixels.
 Subdivision will be required if the square distance between the projection
 of the real point and the straight line passing through the projection
 of the vertices of the edge is greater than `value'.
 For instance, 0.25 will give better result than 1.
 

\item  \verb|vtkViewport = obj.GetViewport ()| -  Set/Get the renderer with `renderer' on which the error metric 
 is based. The error metric use the active camera of the renderer.

\item  \verb|obj.SetViewport (vtkViewport viewport)| -  Set/Get the renderer with `renderer' on which the error metric 
 is based. The error metric use the active camera of the renderer.

\item  \verb|int = obj.RequiresEdgeSubdivision (double leftPoint, double midPoint, double rightPoint, double alpha)| -  Does the edge need to be subdivided according to the distance between
 the line passing through its endpoints in screen space and the projection
 of its mid point?
 The edge is defined by its `leftPoint' and its `rightPoint'.
 `leftPoint', `midPoint' and `rightPoint' have to be initialized before
 calling RequiresEdgeSubdivision().
 Their format is global coordinates, parametric coordinates and
 point centered attributes: xyx rst abc de...
 `alpha' is the normalized abscissa of the midpoint along the edge.
 (close to 0 means close to the left point, close to 1 means close to the
 right point)
 
 
 
 
 
          =GetAttributeCollection()->GetNumberOfPointCenteredComponents()+6

\item  \verb|double = obj.GetError (double leftPoint, double midPoint, double rightPoint, double alpha)| -  Return the error at the mid-point. The type of error depends on the state
 of the concrete error metric. For instance, it can return an absolute
 or relative error metric.
 See RequiresEdgeSubdivision() for a description of the arguments.
 
 
 
 
 
          =GetAttributeCollection()->GetNumberOfPointCenteredComponents()+6
 

\end{itemize}
