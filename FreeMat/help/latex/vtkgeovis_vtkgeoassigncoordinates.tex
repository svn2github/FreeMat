\section{vtkGeoAssignCoordinates}

\subsection{Usage}

 Givem latitude and longitude arrays, 
 take the values in those arrays and convert them to x,y,z world coordinates.
 Uses a spherical model of the earth to do the conversion.
 The position is in meters relative to the center of the earth.

 If a transform is given, use the transform to convert latitude and longitude
 to the world coordinate.

To create an instance of class vtkGeoAssignCoordinates, simply
invoke its constructor as follows
\begin{verbatim}
  obj = vtkGeoAssignCoordinates
\end{verbatim}
\subsection{Methods}

The class vtkGeoAssignCoordinates has several methods that can be used.
  They are listed below.
Note that the documentation is translated automatically from the VTK sources,
and may not be completely intelligible.  When in doubt, consult the VTK website.
In the methods listed below, \verb|obj| is an instance of the vtkGeoAssignCoordinates class.
\begin{itemize}
\item  \verb|string = obj.GetClassName ()|

\item  \verb|int = obj.IsA (string name)|

\item  \verb|vtkGeoAssignCoordinates = obj.NewInstance ()|

\item  \verb|vtkGeoAssignCoordinates = obj.SafeDownCast (vtkObject o)|

\item  \verb|obj.SetLongitudeArrayName (string )| -  Set the longitude coordinate array name. 

\item  \verb|string = obj.GetLongitudeArrayName ()| -  Set the longitude coordinate array name. 

\item  \verb|obj.SetLatitudeArrayName (string )| -  Set the latitude coordinate array name. 

\item  \verb|string = obj.GetLatitudeArrayName ()| -  Set the latitude coordinate array name. 

\item  \verb|obj.SetGlobeRadius (double )| -  The base radius to use in GLOBAL mode.
 Default is the earth's radius.

\item  \verb|double = obj.GetGlobeRadius ()| -  The base radius to use in GLOBAL mode.
 Default is the earth's radius.

\item  \verb|obj.SetTransform (vtkAbstractTransform trans)| -  The transform to use to convert coordinates of the form
 (lat, long, 0) to (x, y z). If this is NULL (the default),
 use GlobeRadius to perform a spherical embedding.

\item  \verb|vtkAbstractTransform = obj.GetTransform ()| -  The transform to use to convert coordinates of the form
 (lat, long, 0) to (x, y z). If this is NULL (the default),
 use GlobeRadius to perform a spherical embedding.

\item  \verb|obj.SetCoordinatesInArrays (bool )| -  If on, uses LatitudeArrayName and LongitudeArrayName to
 move values in data arrays into the points of the data set.
 Turn off if the lattitude and longitude are already in
 the points.

\item  \verb|bool = obj.GetCoordinatesInArrays ()| -  If on, uses LatitudeArrayName and LongitudeArrayName to
 move values in data arrays into the points of the data set.
 Turn off if the lattitude and longitude are already in
 the points.

\item  \verb|obj.CoordinatesInArraysOn ()| -  If on, uses LatitudeArrayName and LongitudeArrayName to
 move values in data arrays into the points of the data set.
 Turn off if the lattitude and longitude are already in
 the points.

\item  \verb|obj.CoordinatesInArraysOff ()| -  If on, uses LatitudeArrayName and LongitudeArrayName to
 move values in data arrays into the points of the data set.
 Turn off if the lattitude and longitude are already in
 the points.

\end{itemize}
