\section{vtkGeoTerrainNode}

\subsection{Usage}


To create an instance of class vtkGeoTerrainNode, simply
invoke its constructor as follows
\begin{verbatim}
  obj = vtkGeoTerrainNode
\end{verbatim}
\subsection{Methods}

The class vtkGeoTerrainNode has several methods that can be used.
  They are listed below.
Note that the documentation is translated automatically from the VTK sources,
and may not be completely intelligible.  When in doubt, consult the VTK website.
In the methods listed below, \verb|obj| is an instance of the vtkGeoTerrainNode class.
\begin{itemize}
\item  \verb|string = obj.GetClassName ()|

\item  \verb|int = obj.IsA (string name)|

\item  \verb|vtkGeoTerrainNode = obj.NewInstance ()|

\item  \verb|vtkGeoTerrainNode = obj.SafeDownCast (vtkObject o)|

\item  \verb|vtkGeoTerrainNode = obj.GetChild (int idx)|

\item  \verb|vtkGeoTerrainNode = obj.GetParent ()|

\item  \verb|double = obj.GetAltitude (double longitude, double latitude)|

\item  \verb|vtkPolyData = obj.GetModel ()| -  Get the terrrain model.  The user has to copy the terrain
 into this object.

\item  \verb|obj.SetModel (vtkPolyData model)| -  Get the terrrain model.  The user has to copy the terrain
 into this object.

\item  \verb|obj.UpdateBoundingSphere ()| -  Bounding sphere is precomputed for faster updates of terrain.

\item  \verb|double = obj.GetBoundingSphereRadius ()| -  Bounding sphere is precomputed for faster updates of terrain.

\item  \verb|double = obj. GetBoundingSphereCenter ()| -  Bounding sphere is precomputed for faster updates of terrain.

\item  \verb|double = obj. GetCornerNormal00 ()|

\item  \verb|double = obj. GetCornerNormal01 ()|

\item  \verb|double = obj. GetCornerNormal10 ()|

\item  \verb|double = obj. GetCornerNormal11 ()|

\item  \verb|double = obj. GetProjectionBounds ()| -  For 2D projections, store the bounds of the node in projected space
 to quickly determine if a node is offscreen.

\item  \verb|obj.SetProjectionBounds (double , double , double , double )| -  For 2D projections, store the bounds of the node in projected space
 to quickly determine if a node is offscreen.

\item  \verb|obj.SetProjectionBounds (double  a[4])| -  For 2D projections, store the bounds of the node in projected space
 to quickly determine if a node is offscreen.

\item  \verb|int = obj.GetGraticuleLevel ()| -  For 2D projections, store the granularity of the graticule in this node.

\item  \verb|obj.SetGraticuleLevel (int )| -  For 2D projections, store the granularity of the graticule in this node.

\item  \verb|double = obj.GetError ()| -  For 2D projections, store the maximum deviation of line segment centers
 from the actual projection value.

\item  \verb|obj.SetError (double )| -  For 2D projections, store the maximum deviation of line segment centers
 from the actual projection value.

\item  \verb|float = obj.GetCoverage ()| -  For 2D projections, store the maximum deviation of line segment centers
 from the actual projection value.

\item  \verb|obj.SetCoverage (float )| -  For 2D projections, store the maximum deviation of line segment centers
 from the actual projection value.

\item  \verb|obj.ShallowCopy (vtkGeoTreeNode src)| -  Shallow and Deep copy.

\item  \verb|obj.DeepCopy (vtkGeoTreeNode src)| -  Shallow and Deep copy.

\item  \verb|bool = obj.HasData ()| -  Returns whether this node has valid data associated
 with it, or if it is an ''empty'' node.

\item  \verb|obj.DeleteData ()| -  Deletes the data associated with the node to make this
 an ''empty'' node. This is performed when the node has
 been unused for a certain amount of time.

\end{itemize}
