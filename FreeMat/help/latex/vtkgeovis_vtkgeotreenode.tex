\section{vtkGeoTreeNode}

\subsection{Usage}

 A self-referential data structure for storing geometry or imagery for
 the geospatial views. The data is organized in a quadtree. Each node
 contains a pointer to its parent and owns references to its four
 child nodes. The ID of each node is unique in its level, and encodes
 the path from the root node in its bits.

To create an instance of class vtkGeoTreeNode, simply
invoke its constructor as follows
\begin{verbatim}
  obj = vtkGeoTreeNode
\end{verbatim}
\subsection{Methods}

The class vtkGeoTreeNode has several methods that can be used.
  They are listed below.
Note that the documentation is translated automatically from the VTK sources,
and may not be completely intelligible.  When in doubt, consult the VTK website.
In the methods listed below, \verb|obj| is an instance of the vtkGeoTreeNode class.
\begin{itemize}
\item  \verb|string = obj.GetClassName ()|

\item  \verb|int = obj.IsA (string name)|

\item  \verb|vtkGeoTreeNode = obj.NewInstance ()|

\item  \verb|vtkGeoTreeNode = obj.SafeDownCast (vtkObject o)|

\item  \verb|obj.SetId (long )| -  The id uniquely specified this node.
 For this implementation I am going to store the branch path
 in the bits.

\item  \verb|long = obj.GetId ()| -  The id uniquely specified this node.
 For this implementation I am going to store the branch path
 in the bits.

\item  \verb|obj.SetLevel (int )|

\item  \verb|int = obj.GetLevel ()|

\item  \verb|obj.SetLongitudeRange (double , double )| -  Longitude and latitude range of the terrain model.

\item  \verb|obj.SetLongitudeRange (double  a[2])| -  Longitude and latitude range of the terrain model.

\item  \verb|double = obj. GetLongitudeRange ()| -  Longitude and latitude range of the terrain model.

\item  \verb|obj.SetLatitudeRange (double , double )| -  Longitude and latitude range of the terrain model.

\item  \verb|obj.SetLatitudeRange (double  a[2])| -  Longitude and latitude range of the terrain model.

\item  \verb|double = obj. GetLatitudeRange ()| -  Longitude and latitude range of the terrain model.

\item  \verb|obj.SetChild (vtkGeoTreeNode node, int idx)| -  Get a child of this node. If one is set, then they all should
 set.  No not mix subclasses.

\item  \verb|obj.SetParent (vtkGeoTreeNode node)| -  Manage links to older and newer tree nodes.
 These are used to periodically delete unused patches.

\item  \verb|obj.SetOlder (vtkGeoTreeNode node)| -  Manage links to older and newer tree nodes.
 These are used to periodically delete unused patches.

\item  \verb|vtkGeoTreeNode = obj.GetOlder ()| -  Manage links to older and newer tree nodes.
 These are used to periodically delete unused patches.

\item  \verb|obj.SetNewer (vtkGeoTreeNode node)| -  Manage links to older and newer tree nodes.
 These are used to periodically delete unused patches.

\item  \verb|vtkGeoTreeNode = obj.GetNewer ()| -  Returns whether this node has valid data associated
 with it, or if it is an ''empty'' node.

\item  \verb|bool = obj.HasData ()| -  Deletes the data associated with the node to make this
 an ''empty'' node. This is performed when the node has
 been unused for a certain amount of time.

\item  \verb|obj.DeleteData ()|

\item  \verb|int = obj.GetWhichChildAreYou ()|

\item  \verb|bool = obj.IsDescendantOf (vtkGeoTreeNode elder)| -  This method returns true if this node descends from the
 elder node.  The descision is made from the node ids, so the nodes do
 not have to be in the same tree!

\item  \verb|int = obj.CreateChildren ()|

\item  \verb|vtkGeoTreeNode = obj.GetChildTreeNode (int idx)| -  Get the parent as a vtkGeoTreeNode.
 Subclasses also implement GetParent() which returns the parent
 as the appropriate subclass type.

\item  \verb|vtkGeoTreeNode = obj.GetParentTreeNode ()| -  Shallow and Deep copy. Deep copy performs a shallow copy
 of the Child nodes.

\item  \verb|obj.ShallowCopy (vtkGeoTreeNode src)| -  Shallow and Deep copy. Deep copy performs a shallow copy
 of the Child nodes.

\item  \verb|obj.DeepCopy (vtkGeoTreeNode src)| -  Shallow and Deep copy. Deep copy performs a shallow copy
 of the Child nodes.

\end{itemize}
