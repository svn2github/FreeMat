\section{vtkButterflySubdivisionFilter}

\subsection{Usage}

 vtkButterflySubdivisionFilter is an interpolating subdivision scheme
 that creates four new triangles for each triangle in the mesh. The
 user can specify the NumberOfSubdivisions. This filter implements the
 8-point butterfly scheme described in: Zorin, D., Schroder, P., and
 Sweldens, W., ''Interpolating Subdivisions for Meshes with Arbitrary
 Topology,'' Computer Graphics Proceedings, Annual Conference Series,
 1996, ACM SIGGRAPH, pp.189-192. This scheme improves previous
 butterfly subdivisions with special treatment of vertices with valence
 other than 6.
 
 Currently, the filter only operates on triangles. Users should use the
 vtkTriangleFilter to triangulate meshes that contain polygons or
 triangle strips.
 
 The filter interpolates point data using the same scheme. New
 triangles created at a subdivision step will have the cell data of
 their parent cell.

To create an instance of class vtkButterflySubdivisionFilter, simply
invoke its constructor as follows
\begin{verbatim}
  obj = vtkButterflySubdivisionFilter
\end{verbatim}
\subsection{Methods}

The class vtkButterflySubdivisionFilter has several methods that can be used.
  They are listed below.
Note that the documentation is translated automatically from the VTK sources,
and may not be completely intelligible.  When in doubt, consult the VTK website.
In the methods listed below, \verb|obj| is an instance of the vtkButterflySubdivisionFilter class.
\begin{itemize}
\item  \verb|string = obj.GetClassName ()| -  Construct object with NumberOfSubdivisions set to 1.

\item  \verb|int = obj.IsA (string name)| -  Construct object with NumberOfSubdivisions set to 1.

\item  \verb|vtkButterflySubdivisionFilter = obj.NewInstance ()| -  Construct object with NumberOfSubdivisions set to 1.

\item  \verb|vtkButterflySubdivisionFilter = obj.SafeDownCast (vtkObject o)| -  Construct object with NumberOfSubdivisions set to 1.

\end{itemize}
