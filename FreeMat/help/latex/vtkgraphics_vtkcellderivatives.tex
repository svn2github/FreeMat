\section{vtkCellDerivatives}

\subsection{Usage}

 vtkCellDerivatives is a filter that computes derivatives of scalars
 and vectors at the center of cells. You can choose to generate
 different output including the scalar gradient (a vector), computed
 tensor vorticity (a vector), gradient of input vectors (a tensor),
 and strain matrix of the input vectors (a tensor); or you may
 choose to pass data through to the output.

 Note that it is assumed that on input scalars and vector point data
 is available, which are then used to generate cell vectors and tensors.
 (The interpolation functions of the cells are used to compute the
 derivatives which is why point data is required.)

To create an instance of class vtkCellDerivatives, simply
invoke its constructor as follows
\begin{verbatim}
  obj = vtkCellDerivatives
\end{verbatim}
\subsection{Methods}

The class vtkCellDerivatives has several methods that can be used.
  They are listed below.
Note that the documentation is translated automatically from the VTK sources,
and may not be completely intelligible.  When in doubt, consult the VTK website.
In the methods listed below, \verb|obj| is an instance of the vtkCellDerivatives class.
\begin{itemize}
\item  \verb|string = obj.GetClassName ()|

\item  \verb|int = obj.IsA (string name)|

\item  \verb|vtkCellDerivatives = obj.NewInstance ()|

\item  \verb|vtkCellDerivatives = obj.SafeDownCast (vtkObject o)|

\item  \verb|obj.SetVectorMode (int )| -  Control how the filter works to generate vector cell data. You
 can choose to pass the input cell vectors, compute the gradient
 of the input scalars, or extract the vorticity of the computed
 vector gradient tensor. By default (VectorModeToComputeGradient),
 the filter will take the gradient of the input scalar data.

\item  \verb|int = obj.GetVectorMode ()| -  Control how the filter works to generate vector cell data. You
 can choose to pass the input cell vectors, compute the gradient
 of the input scalars, or extract the vorticity of the computed
 vector gradient tensor. By default (VectorModeToComputeGradient),
 the filter will take the gradient of the input scalar data.

\item  \verb|obj.SetVectorModeToPassVectors ()| -  Control how the filter works to generate vector cell data. You
 can choose to pass the input cell vectors, compute the gradient
 of the input scalars, or extract the vorticity of the computed
 vector gradient tensor. By default (VectorModeToComputeGradient),
 the filter will take the gradient of the input scalar data.

\item  \verb|obj.SetVectorModeToComputeGradient ()| -  Control how the filter works to generate vector cell data. You
 can choose to pass the input cell vectors, compute the gradient
 of the input scalars, or extract the vorticity of the computed
 vector gradient tensor. By default (VectorModeToComputeGradient),
 the filter will take the gradient of the input scalar data.

\item  \verb|obj.SetVectorModeToComputeVorticity ()| -  Control how the filter works to generate vector cell data. You
 can choose to pass the input cell vectors, compute the gradient
 of the input scalars, or extract the vorticity of the computed
 vector gradient tensor. By default (VectorModeToComputeGradient),
 the filter will take the gradient of the input scalar data.

\item  \verb|string = obj.GetVectorModeAsString ()| -  Control how the filter works to generate vector cell data. You
 can choose to pass the input cell vectors, compute the gradient
 of the input scalars, or extract the vorticity of the computed
 vector gradient tensor. By default (VectorModeToComputeGradient),
 the filter will take the gradient of the input scalar data.

\item  \verb|obj.SetTensorMode (int )| -  Control how the filter works to generate tensor cell data. You can
 choose to pass the input cell tensors, compute the gradient of
 the input vectors, or compute the strain tensor of the vector gradient
 tensor. By default (TensorModeToComputeGradient), the filter will
 take the gradient of the vector data to construct a tensor.

\item  \verb|int = obj.GetTensorMode ()| -  Control how the filter works to generate tensor cell data. You can
 choose to pass the input cell tensors, compute the gradient of
 the input vectors, or compute the strain tensor of the vector gradient
 tensor. By default (TensorModeToComputeGradient), the filter will
 take the gradient of the vector data to construct a tensor.

\item  \verb|obj.SetTensorModeToPassTensors ()| -  Control how the filter works to generate tensor cell data. You can
 choose to pass the input cell tensors, compute the gradient of
 the input vectors, or compute the strain tensor of the vector gradient
 tensor. By default (TensorModeToComputeGradient), the filter will
 take the gradient of the vector data to construct a tensor.

\item  \verb|obj.SetTensorModeToComputeGradient ()| -  Control how the filter works to generate tensor cell data. You can
 choose to pass the input cell tensors, compute the gradient of
 the input vectors, or compute the strain tensor of the vector gradient
 tensor. By default (TensorModeToComputeGradient), the filter will
 take the gradient of the vector data to construct a tensor.

\item  \verb|obj.SetTensorModeToComputeStrain ()| -  Control how the filter works to generate tensor cell data. You can
 choose to pass the input cell tensors, compute the gradient of
 the input vectors, or compute the strain tensor of the vector gradient
 tensor. By default (TensorModeToComputeGradient), the filter will
 take the gradient of the vector data to construct a tensor.

\item  \verb|string = obj.GetTensorModeAsString ()| -  Control how the filter works to generate tensor cell data. You can
 choose to pass the input cell tensors, compute the gradient of
 the input vectors, or compute the strain tensor of the vector gradient
 tensor. By default (TensorModeToComputeGradient), the filter will
 take the gradient of the vector data to construct a tensor.

\end{itemize}
