\section{vtkConnectivityFilter}

\subsection{Usage}

 vtkConnectivityFilter is a filter that extracts cells that share common
 points and/or meet other connectivity criterion. (Cells that share
 vertices and meet other connectivity criterion such as scalar range are
 known as a region.)  The filter works in one of six ways: 1) extract the
 largest connected region in the dataset; 2) extract specified region
 numbers; 3) extract all regions sharing specified point ids; 4) extract
 all regions sharing specified cell ids; 5) extract the region closest to
 the specified point; or 6) extract all regions (used to color the data by
 region).

 vtkConnectivityFilter is generalized to handle any type of input dataset.
 It generates output data of type vtkUnstructuredGrid. If you know that
 your input type is vtkPolyData, you may wish to use
 vtkPolyDataConnectivityFilter.

 The behavior of vtkConnectivityFilter can be modified by turning on the
 boolean ivar ScalarConnectivity. If this flag is on, the connectivity
 algorithm is modified so that cells are considered connected only if 1)
 they are geometrically connected (share a point) and 2) the scalar values
 of one of the cell's points falls in the scalar range specified. This use
 of ScalarConnectivity is particularly useful for volume datasets: it can
 be used as a simple ''connected segmentation'' algorithm. For example, by
 using a seed voxel (i.e., cell) on a known anatomical structure,
 connectivity will pull out all voxels ''containing'' the anatomical
 structure. These voxels can then be contoured or processed by other
 visualization filters.

To create an instance of class vtkConnectivityFilter, simply
invoke its constructor as follows
\begin{verbatim}
  obj = vtkConnectivityFilter
\end{verbatim}
\subsection{Methods}

The class vtkConnectivityFilter has several methods that can be used.
  They are listed below.
Note that the documentation is translated automatically from the VTK sources,
and may not be completely intelligible.  When in doubt, consult the VTK website.
In the methods listed below, \verb|obj| is an instance of the vtkConnectivityFilter class.
\begin{itemize}
\item  \verb|string = obj.GetClassName ()|

\item  \verb|int = obj.IsA (string name)|

\item  \verb|vtkConnectivityFilter = obj.NewInstance ()|

\item  \verb|vtkConnectivityFilter = obj.SafeDownCast (vtkObject o)|

\item  \verb|obj.SetScalarConnectivity (int )| -  Turn on/off connectivity based on scalar value. If on, cells are connected
 only if they share points AND one of the cells scalar values falls in the
 scalar range specified.

\item  \verb|int = obj.GetScalarConnectivity ()| -  Turn on/off connectivity based on scalar value. If on, cells are connected
 only if they share points AND one of the cells scalar values falls in the
 scalar range specified.

\item  \verb|obj.ScalarConnectivityOn ()| -  Turn on/off connectivity based on scalar value. If on, cells are connected
 only if they share points AND one of the cells scalar values falls in the
 scalar range specified.

\item  \verb|obj.ScalarConnectivityOff ()| -  Turn on/off connectivity based on scalar value. If on, cells are connected
 only if they share points AND one of the cells scalar values falls in the
 scalar range specified.

\item  \verb|obj.SetScalarRange (double , double )| -  Set the scalar range to use to extract cells based on scalar connectivity.

\item  \verb|obj.SetScalarRange (double  a[2])| -  Set the scalar range to use to extract cells based on scalar connectivity.

\item  \verb|double = obj. GetScalarRange ()| -  Set the scalar range to use to extract cells based on scalar connectivity.

\item  \verb|obj.SetExtractionMode (int )| -  Control the extraction of connected surfaces.

\item  \verb|int = obj.GetExtractionModeMinValue ()| -  Control the extraction of connected surfaces.

\item  \verb|int = obj.GetExtractionModeMaxValue ()| -  Control the extraction of connected surfaces.

\item  \verb|int = obj.GetExtractionMode ()| -  Control the extraction of connected surfaces.

\item  \verb|obj.SetExtractionModeToPointSeededRegions ()| -  Control the extraction of connected surfaces.

\item  \verb|obj.SetExtractionModeToCellSeededRegions ()| -  Control the extraction of connected surfaces.

\item  \verb|obj.SetExtractionModeToLargestRegion ()| -  Control the extraction of connected surfaces.

\item  \verb|obj.SetExtractionModeToSpecifiedRegions ()| -  Control the extraction of connected surfaces.

\item  \verb|obj.SetExtractionModeToClosestPointRegion ()| -  Control the extraction of connected surfaces.

\item  \verb|obj.SetExtractionModeToAllRegions ()| -  Control the extraction of connected surfaces.

\item  \verb|string = obj.GetExtractionModeAsString ()| -  Control the extraction of connected surfaces.

\item  \verb|obj.InitializeSeedList ()| -  Initialize list of point ids/cell ids used to seed regions.

\item  \verb|obj.AddSeed (vtkIdType id)| -  Add a seed id (point or cell id). Note: ids are 0-offset.

\item  \verb|obj.DeleteSeed (vtkIdType id)| -  Delete a seed id (point or cell id). Note: ids are 0-offset.

\item  \verb|obj.InitializeSpecifiedRegionList ()| -  Initialize list of region ids to extract.

\item  \verb|obj.AddSpecifiedRegion (int id)| -  Add a region id to extract. Note: ids are 0-offset.

\item  \verb|obj.DeleteSpecifiedRegion (int id)| -  Delete a region id to extract. Note: ids are 0-offset.

\item  \verb|obj.SetClosestPoint (double , double , double )| -  Use to specify x-y-z point coordinates when extracting the region 
 closest to a specified point.

\item  \verb|obj.SetClosestPoint (double  a[3])| -  Use to specify x-y-z point coordinates when extracting the region 
 closest to a specified point.

\item  \verb|double = obj. GetClosestPoint ()| -  Use to specify x-y-z point coordinates when extracting the region 
 closest to a specified point.

\item  \verb|int = obj.GetNumberOfExtractedRegions ()| -  Obtain the number of connected regions.

\item  \verb|obj.SetColorRegions (int )| -  Turn on/off the coloring of connected regions.

\item  \verb|int = obj.GetColorRegions ()| -  Turn on/off the coloring of connected regions.

\item  \verb|obj.ColorRegionsOn ()| -  Turn on/off the coloring of connected regions.

\item  \verb|obj.ColorRegionsOff ()| -  Turn on/off the coloring of connected regions.

\end{itemize}
