\section{vtkCutter}

\subsection{Usage}

 vtkCutter is a filter to cut through data using any subclass of 
 vtkImplicitFunction. That is, a polygonal surface is created
 corresponding to the implicit function F(x,y,z) = value(s), where
 you can specify one or more values used to cut with.

 In VTK, cutting means reducing a cell of dimension N to a cut surface
 of dimension N-1. For example, a tetrahedron when cut by a plane (i.e.,
 vtkPlane implicit function) will generate triangles. (In comparison,
 clipping takes a N dimensional cell and creates N dimension primitives.)

 vtkCutter is generally used to ''slice-through'' a dataset, generating
 a surface that can be visualized. It is also possible to use vtkCutter
 to do a form of volume rendering. vtkCutter does this by generating
 multiple cut surfaces (usually planes) which are ordered (and rendered)
 from back-to-front. The surfaces are set translucent to give a 
 volumetric rendering effect.

 Note that data can be cut using either 1) the scalar values associated
 with the dataset or 2) an implicit function associated with this class.
 By default, if an implicit function is set it is used to clip the data
 set, otherwise the dataset scalars are used to perform the clipping.

To create an instance of class vtkCutter, simply
invoke its constructor as follows
\begin{verbatim}
  obj = vtkCutter
\end{verbatim}
\subsection{Methods}

The class vtkCutter has several methods that can be used.
  They are listed below.
Note that the documentation is translated automatically from the VTK sources,
and may not be completely intelligible.  When in doubt, consult the VTK website.
In the methods listed below, \verb|obj| is an instance of the vtkCutter class.
\begin{itemize}
\item  \verb|string = obj.GetClassName ()|

\item  \verb|int = obj.IsA (string name)|

\item  \verb|vtkCutter = obj.NewInstance ()|

\item  \verb|vtkCutter = obj.SafeDownCast (vtkObject o)|

\item  \verb|obj.SetValue (int i, double value)| -  Get the ith contour value.

\item  \verb|double = obj.GetValue (int i)| -  Get a pointer to an array of contour values. There will be
 GetNumberOfContours() values in the list.

\item  \verb|obj.GetValues (double contourValues)| -  Set the number of contours to place into the list. You only really
 need to use this method to reduce list size. The method SetValue()
 will automatically increase list size as needed.

\item  \verb|obj.SetNumberOfContours (int number)| -  Get the number of contours in the list of contour values.

\item  \verb|int = obj.GetNumberOfContours ()| -  Generate numContours equally spaced contour values between specified
 range. Contour values will include min/max range values.

\item  \verb|obj.GenerateValues (int numContours, double range[2])| -  Generate numContours equally spaced contour values between specified
 range. Contour values will include min/max range values.

\item  \verb|obj.GenerateValues (int numContours, double rangeStart, double rangeEnd)| -  Override GetMTime because we delegate to vtkContourValues and refer to
 vtkImplicitFunction.

\item  \verb|long = obj.GetMTime ()| -  Override GetMTime because we delegate to vtkContourValues and refer to
 vtkImplicitFunction.

\item  \verb|obj.SetCutFunction (vtkImplicitFunction )|

\item  \verb|vtkImplicitFunction = obj.GetCutFunction ()|

\item  \verb|obj.SetGenerateCutScalars (int )| -  If this flag is enabled, then the output scalar values will be
 interpolated from the implicit function values, and not the input scalar
 data.

\item  \verb|int = obj.GetGenerateCutScalars ()| -  If this flag is enabled, then the output scalar values will be
 interpolated from the implicit function values, and not the input scalar
 data.

\item  \verb|obj.GenerateCutScalarsOn ()| -  If this flag is enabled, then the output scalar values will be
 interpolated from the implicit function values, and not the input scalar
 data.

\item  \verb|obj.GenerateCutScalarsOff ()| -  If this flag is enabled, then the output scalar values will be
 interpolated from the implicit function values, and not the input scalar
 data.

\item  \verb|obj.SetLocator (vtkIncrementalPointLocator locator)| -  Specify a spatial locator for merging points. By default, 
 an instance of vtkMergePoints is used.

\item  \verb|vtkIncrementalPointLocator = obj.GetLocator ()| -  Specify a spatial locator for merging points. By default, 
 an instance of vtkMergePoints is used.

\item  \verb|obj.SetSortBy (int )| -  Set the sorting order for the generated polydata. There are two
 possibilities:
   Sort by value = 0 - This is the most efficient sort. For each cell,
      all contour values are processed. This is the default.
   Sort by cell = 1 - For each contour value, all cells are processed.
      This order should be used if the extracted polygons must be rendered
      in a back-to-front or front-to-back order. This is very problem 
      dependent.
 For most applications, the default order is fine (and faster).

 Sort by cell is going to have a problem if the input has 2D and 3D cells.
 Cell data will be scrambled becauses with 
 vtkPolyData output, verts and lines have lower cell ids than triangles.

\item  \verb|int = obj.GetSortByMinValue ()| -  Set the sorting order for the generated polydata. There are two
 possibilities:
   Sort by value = 0 - This is the most efficient sort. For each cell,
      all contour values are processed. This is the default.
   Sort by cell = 1 - For each contour value, all cells are processed.
      This order should be used if the extracted polygons must be rendered
      in a back-to-front or front-to-back order. This is very problem 
      dependent.
 For most applications, the default order is fine (and faster).

 Sort by cell is going to have a problem if the input has 2D and 3D cells.
 Cell data will be scrambled becauses with 
 vtkPolyData output, verts and lines have lower cell ids than triangles.

\item  \verb|int = obj.GetSortByMaxValue ()| -  Set the sorting order for the generated polydata. There are two
 possibilities:
   Sort by value = 0 - This is the most efficient sort. For each cell,
      all contour values are processed. This is the default.
   Sort by cell = 1 - For each contour value, all cells are processed.
      This order should be used if the extracted polygons must be rendered
      in a back-to-front or front-to-back order. This is very problem 
      dependent.
 For most applications, the default order is fine (and faster).

 Sort by cell is going to have a problem if the input has 2D and 3D cells.
 Cell data will be scrambled becauses with 
 vtkPolyData output, verts and lines have lower cell ids than triangles.

\item  \verb|int = obj.GetSortBy ()| -  Set the sorting order for the generated polydata. There are two
 possibilities:
   Sort by value = 0 - This is the most efficient sort. For each cell,
      all contour values are processed. This is the default.
   Sort by cell = 1 - For each contour value, all cells are processed.
      This order should be used if the extracted polygons must be rendered
      in a back-to-front or front-to-back order. This is very problem 
      dependent.
 For most applications, the default order is fine (and faster).

 Sort by cell is going to have a problem if the input has 2D and 3D cells.
 Cell data will be scrambled becauses with 
 vtkPolyData output, verts and lines have lower cell ids than triangles.

\item  \verb|obj.SetSortByToSortByValue ()| -  Set the sorting order for the generated polydata. There are two
 possibilities:
   Sort by value = 0 - This is the most efficient sort. For each cell,
      all contour values are processed. This is the default.
   Sort by cell = 1 - For each contour value, all cells are processed.
      This order should be used if the extracted polygons must be rendered
      in a back-to-front or front-to-back order. This is very problem 
      dependent.
 For most applications, the default order is fine (and faster).

 Sort by cell is going to have a problem if the input has 2D and 3D cells.
 Cell data will be scrambled becauses with 
 vtkPolyData output, verts and lines have lower cell ids than triangles.

\item  \verb|obj.SetSortByToSortByCell ()| -  Set the sorting order for the generated polydata. There are two
 possibilities:
   Sort by value = 0 - This is the most efficient sort. For each cell,
      all contour values are processed. This is the default.
   Sort by cell = 1 - For each contour value, all cells are processed.
      This order should be used if the extracted polygons must be rendered
      in a back-to-front or front-to-back order. This is very problem 
      dependent.
 For most applications, the default order is fine (and faster).

 Sort by cell is going to have a problem if the input has 2D and 3D cells.
 Cell data will be scrambled becauses with 
 vtkPolyData output, verts and lines have lower cell ids than triangles.

\item  \verb|string = obj.GetSortByAsString ()| -  Set the sorting order for the generated polydata. There are two
 possibilities:
   Sort by value = 0 - This is the most efficient sort. For each cell,
      all contour values are processed. This is the default.
   Sort by cell = 1 - For each contour value, all cells are processed.
      This order should be used if the extracted polygons must be rendered
      in a back-to-front or front-to-back order. This is very problem 
      dependent.
 For most applications, the default order is fine (and faster).

 Sort by cell is going to have a problem if the input has 2D and 3D cells.
 Cell data will be scrambled becauses with 
 vtkPolyData output, verts and lines have lower cell ids than triangles.

\item  \verb|obj.CreateDefaultLocator ()| -  Create default locator. Used to create one when none is specified. The 
 locator is used to merge coincident points.

\end{itemize}
