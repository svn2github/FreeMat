\section{vtkDataObjectToDataSetFilter}

\subsection{Usage}

 vtkDataObjectToDataSetFilter is an class that maps a data object (i.e., a field)
 into a concrete dataset, i.e., gives structure to the field by defining a
 geometry and topology.

 To use this filter you associate components in the input field data with
 portions of the output dataset. (A component is an array of values from
 the field.) For example, you would specify x-y-z points by assigning 
 components from the field for the x, then y, then z values of the points.
 You may also have to specify component ranges (for each z-y-z) to make 
 sure that the number of x, y, and z values is the same. Also, you may 
 want to normalize the components which helps distribute the data 
 uniformly. Once you've setup the filter to combine all the pieces of 
 data into a specified dataset (the geometry, topology, point and cell 
 data attributes), the various output methods (e.g., GetPolyData()) are
 used to retrieve the final product.

 This filter is often used in conjunction with
 vtkFieldDataToAttributeDataFilter.  vtkFieldDataToAttributeDataFilter
 takes field data and transforms it into attribute data (e.g., point and
 cell data attributes such as scalars and vectors).  To do this, use this
 filter which constructs a concrete dataset and passes the input data
 object field data to its output. and then use
 vtkFieldDataToAttributeDataFilter to generate the attribute data associated
 with the dataset.

To create an instance of class vtkDataObjectToDataSetFilter, simply
invoke its constructor as follows
\begin{verbatim}
  obj = vtkDataObjectToDataSetFilter
\end{verbatim}
\subsection{Methods}

The class vtkDataObjectToDataSetFilter has several methods that can be used.
  They are listed below.
Note that the documentation is translated automatically from the VTK sources,
and may not be completely intelligible.  When in doubt, consult the VTK website.
In the methods listed below, \verb|obj| is an instance of the vtkDataObjectToDataSetFilter class.
\begin{itemize}
\item  \verb|string = obj.GetClassName ()|

\item  \verb|int = obj.IsA (string name)|

\item  \verb|vtkDataObjectToDataSetFilter = obj.NewInstance ()|

\item  \verb|vtkDataObjectToDataSetFilter = obj.SafeDownCast (vtkObject o)|

\item  \verb|vtkDataObject = obj.GetInput ()| -  Get the input to the filter.

\item  \verb|obj.SetDataSetType (int )| -  Control what type of data is generated for output.

\item  \verb|int = obj.GetDataSetType ()| -  Control what type of data is generated for output.

\item  \verb|obj.SetDataSetTypeToPolyData ()| -  Control what type of data is generated for output.

\item  \verb|obj.SetDataSetTypeToStructuredPoints ()| -  Control what type of data is generated for output.

\item  \verb|obj.SetDataSetTypeToStructuredGrid ()| -  Control what type of data is generated for output.

\item  \verb|obj.SetDataSetTypeToRectilinearGrid ()| -  Control what type of data is generated for output.

\item  \verb|obj.SetDataSetTypeToUnstructuredGrid ()| -  Control what type of data is generated for output.

\item  \verb|vtkDataSet = obj.GetOutput ()| -  Get the output in different forms. The particular method invoked
 should be consistent with the SetDataSetType() method. (Note:
 GetOutput() will always return a type consistent with 
 SetDataSetType(). Also, GetOutput() will return NULL if the filter
 aborted due to inconsistent data.)

\item  \verb|vtkDataSet = obj.GetOutput (int idx)| -  Get the output in different forms. The particular method invoked
 should be consistent with the SetDataSetType() method. (Note:
 GetOutput() will always return a type consistent with 
 SetDataSetType(). Also, GetOutput() will return NULL if the filter
 aborted due to inconsistent data.)

\item  \verb|vtkPolyData = obj.GetPolyDataOutput ()| -  Get the output in different forms. The particular method invoked
 should be consistent with the SetDataSetType() method. (Note:
 GetOutput() will always return a type consistent with 
 SetDataSetType(). Also, GetOutput() will return NULL if the filter
 aborted due to inconsistent data.)

\item  \verb|vtkStructuredPoints = obj.GetStructuredPointsOutput ()| -  Get the output in different forms. The particular method invoked
 should be consistent with the SetDataSetType() method. (Note:
 GetOutput() will always return a type consistent with 
 SetDataSetType(). Also, GetOutput() will return NULL if the filter
 aborted due to inconsistent data.)

\item  \verb|vtkStructuredGrid = obj.GetStructuredGridOutput ()| -  Get the output in different forms. The particular method invoked
 should be consistent with the SetDataSetType() method. (Note:
 GetOutput() will always return a type consistent with 
 SetDataSetType(). Also, GetOutput() will return NULL if the filter
 aborted due to inconsistent data.)

\item  \verb|vtkUnstructuredGrid = obj.GetUnstructuredGridOutput ()| -  Get the output in different forms. The particular method invoked
 should be consistent with the SetDataSetType() method. (Note:
 GetOutput() will always return a type consistent with 
 SetDataSetType(). Also, GetOutput() will return NULL if the filter
 aborted due to inconsistent data.)

\item  \verb|vtkRectilinearGrid = obj.GetRectilinearGridOutput ()| -  Get the output in different forms. The particular method invoked
 should be consistent with the SetDataSetType() method. (Note:
 GetOutput() will always return a type consistent with 
 SetDataSetType(). Also, GetOutput() will return NULL if the filter
 aborted due to inconsistent data.)

\item  \verb|obj.SetPointComponent (int comp, string arrayName, int arrayComp, int min, int max, int normalize)| -  Define the component of the field to be used for the x, y, and z values
 of the points. Note that the parameter comp must lie between (0,2) and
 refers to the x-y-z (i.e., 0,1,2) components of the points. To define
 the field component to use you can specify an array name and the
 component in that array. The (min,max) values are the range of data in
 the component you wish to extract. (This method should be used for
 vtkPolyData, vtkUnstructuredGrid, vtkStructuredGrid, and
 vtkRectilinearGrid.) A convenience method, SetPointComponent(),is also
 provided which does not require setting the (min,max) component range or
 the normalize flag (normalize is set to DefaulatNormalize value).

\item  \verb|obj.SetPointComponent (int comp, string arrayName, int arrayComp)| -  Define the component of the field to be used for the x, y, and z values
 of the points. Note that the parameter comp must lie between (0,2) and
 refers to the x-y-z (i.e., 0,1,2) components of the points. To define
 the field component to use you can specify an array name and the
 component in that array. The (min,max) values are the range of data in
 the component you wish to extract. (This method should be used for
 vtkPolyData, vtkUnstructuredGrid, vtkStructuredGrid, and
 vtkRectilinearGrid.) A convenience method, SetPointComponent(),is also
 provided which does not require setting the (min,max) component range or
 the normalize flag (normalize is set to DefaulatNormalize value).

\item  \verb|string = obj.GetPointComponentArrayName (int comp)| -  Define the component of the field to be used for the x, y, and z values
 of the points. Note that the parameter comp must lie between (0,2) and
 refers to the x-y-z (i.e., 0,1,2) components of the points. To define
 the field component to use you can specify an array name and the
 component in that array. The (min,max) values are the range of data in
 the component you wish to extract. (This method should be used for
 vtkPolyData, vtkUnstructuredGrid, vtkStructuredGrid, and
 vtkRectilinearGrid.) A convenience method, SetPointComponent(),is also
 provided which does not require setting the (min,max) component range or
 the normalize flag (normalize is set to DefaulatNormalize value).

\item  \verb|int = obj.GetPointComponentArrayComponent (int comp)| -  Define the component of the field to be used for the x, y, and z values
 of the points. Note that the parameter comp must lie between (0,2) and
 refers to the x-y-z (i.e., 0,1,2) components of the points. To define
 the field component to use you can specify an array name and the
 component in that array. The (min,max) values are the range of data in
 the component you wish to extract. (This method should be used for
 vtkPolyData, vtkUnstructuredGrid, vtkStructuredGrid, and
 vtkRectilinearGrid.) A convenience method, SetPointComponent(),is also
 provided which does not require setting the (min,max) component range or
 the normalize flag (normalize is set to DefaulatNormalize value).

\item  \verb|int = obj.GetPointComponentMinRange (int comp)| -  Define the component of the field to be used for the x, y, and z values
 of the points. Note that the parameter comp must lie between (0,2) and
 refers to the x-y-z (i.e., 0,1,2) components of the points. To define
 the field component to use you can specify an array name and the
 component in that array. The (min,max) values are the range of data in
 the component you wish to extract. (This method should be used for
 vtkPolyData, vtkUnstructuredGrid, vtkStructuredGrid, and
 vtkRectilinearGrid.) A convenience method, SetPointComponent(),is also
 provided which does not require setting the (min,max) component range or
 the normalize flag (normalize is set to DefaulatNormalize value).

\item  \verb|int = obj.GetPointComponentMaxRange (int comp)| -  Define the component of the field to be used for the x, y, and z values
 of the points. Note that the parameter comp must lie between (0,2) and
 refers to the x-y-z (i.e., 0,1,2) components of the points. To define
 the field component to use you can specify an array name and the
 component in that array. The (min,max) values are the range of data in
 the component you wish to extract. (This method should be used for
 vtkPolyData, vtkUnstructuredGrid, vtkStructuredGrid, and
 vtkRectilinearGrid.) A convenience method, SetPointComponent(),is also
 provided which does not require setting the (min,max) component range or
 the normalize flag (normalize is set to DefaulatNormalize value).

\item  \verb|int = obj.GetPointComponentNormailzeFlag (int comp)| -  Define the component of the field to be used for the x, y, and z values
 of the points. Note that the parameter comp must lie between (0,2) and
 refers to the x-y-z (i.e., 0,1,2) components of the points. To define
 the field component to use you can specify an array name and the
 component in that array. The (min,max) values are the range of data in
 the component you wish to extract. (This method should be used for
 vtkPolyData, vtkUnstructuredGrid, vtkStructuredGrid, and
 vtkRectilinearGrid.) A convenience method, SetPointComponent(),is also
 provided which does not require setting the (min,max) component range or
 the normalize flag (normalize is set to DefaulatNormalize value).

\item  \verb|obj.SetVertsComponent (string arrayName, int arrayComp, int min, int max)| -  Define cell connectivity when creating vtkPolyData. You can define
 vertices, lines, polygons, and/or triangle strips via these methods.
 These methods are similar to those for defining points, except
 that no normalization of the data is possible. Basically, you need to
 define an array of values that (for each cell) includes the number of 
 points per cell, and then the cell connectivity. (This is the vtk file 
 format described in in the textbook or User's Guide.)

\item  \verb|obj.SetVertsComponent (string arrayName, int arrayComp)| -  Define cell connectivity when creating vtkPolyData. You can define
 vertices, lines, polygons, and/or triangle strips via these methods.
 These methods are similar to those for defining points, except
 that no normalization of the data is possible. Basically, you need to
 define an array of values that (for each cell) includes the number of 
 points per cell, and then the cell connectivity. (This is the vtk file 
 format described in in the textbook or User's Guide.)

\item  \verb|string = obj.GetVertsComponentArrayName ()| -  Define cell connectivity when creating vtkPolyData. You can define
 vertices, lines, polygons, and/or triangle strips via these methods.
 These methods are similar to those for defining points, except
 that no normalization of the data is possible. Basically, you need to
 define an array of values that (for each cell) includes the number of 
 points per cell, and then the cell connectivity. (This is the vtk file 
 format described in in the textbook or User's Guide.)

\item  \verb|int = obj.GetVertsComponentArrayComponent ()| -  Define cell connectivity when creating vtkPolyData. You can define
 vertices, lines, polygons, and/or triangle strips via these methods.
 These methods are similar to those for defining points, except
 that no normalization of the data is possible. Basically, you need to
 define an array of values that (for each cell) includes the number of 
 points per cell, and then the cell connectivity. (This is the vtk file 
 format described in in the textbook or User's Guide.)

\item  \verb|int = obj.GetVertsComponentMinRange ()| -  Define cell connectivity when creating vtkPolyData. You can define
 vertices, lines, polygons, and/or triangle strips via these methods.
 These methods are similar to those for defining points, except
 that no normalization of the data is possible. Basically, you need to
 define an array of values that (for each cell) includes the number of 
 points per cell, and then the cell connectivity. (This is the vtk file 
 format described in in the textbook or User's Guide.)

\item  \verb|int = obj.GetVertsComponentMaxRange ()| -  Define cell connectivity when creating vtkPolyData. You can define
 vertices, lines, polygons, and/or triangle strips via these methods.
 These methods are similar to those for defining points, except
 that no normalization of the data is possible. Basically, you need to
 define an array of values that (for each cell) includes the number of 
 points per cell, and then the cell connectivity. (This is the vtk file 
 format described in in the textbook or User's Guide.)

\item  \verb|obj.SetLinesComponent (string arrayName, int arrayComp, int min, int max)| -  Define cell connectivity when creating vtkPolyData. You can define
 vertices, lines, polygons, and/or triangle strips via these methods.
 These methods are similar to those for defining points, except
 that no normalization of the data is possible. Basically, you need to
 define an array of values that (for each cell) includes the number of 
 points per cell, and then the cell connectivity. (This is the vtk file 
 format described in in the textbook or User's Guide.)

\item  \verb|obj.SetLinesComponent (string arrayName, int arrayComp)| -  Define cell connectivity when creating vtkPolyData. You can define
 vertices, lines, polygons, and/or triangle strips via these methods.
 These methods are similar to those for defining points, except
 that no normalization of the data is possible. Basically, you need to
 define an array of values that (for each cell) includes the number of 
 points per cell, and then the cell connectivity. (This is the vtk file 
 format described in in the textbook or User's Guide.)

\item  \verb|string = obj.GetLinesComponentArrayName ()| -  Define cell connectivity when creating vtkPolyData. You can define
 vertices, lines, polygons, and/or triangle strips via these methods.
 These methods are similar to those for defining points, except
 that no normalization of the data is possible. Basically, you need to
 define an array of values that (for each cell) includes the number of 
 points per cell, and then the cell connectivity. (This is the vtk file 
 format described in in the textbook or User's Guide.)

\item  \verb|int = obj.GetLinesComponentArrayComponent ()| -  Define cell connectivity when creating vtkPolyData. You can define
 vertices, lines, polygons, and/or triangle strips via these methods.
 These methods are similar to those for defining points, except
 that no normalization of the data is possible. Basically, you need to
 define an array of values that (for each cell) includes the number of 
 points per cell, and then the cell connectivity. (This is the vtk file 
 format described in in the textbook or User's Guide.)

\item  \verb|int = obj.GetLinesComponentMinRange ()| -  Define cell connectivity when creating vtkPolyData. You can define
 vertices, lines, polygons, and/or triangle strips via these methods.
 These methods are similar to those for defining points, except
 that no normalization of the data is possible. Basically, you need to
 define an array of values that (for each cell) includes the number of 
 points per cell, and then the cell connectivity. (This is the vtk file 
 format described in in the textbook or User's Guide.)

\item  \verb|int = obj.GetLinesComponentMaxRange ()| -  Define cell connectivity when creating vtkPolyData. You can define
 vertices, lines, polygons, and/or triangle strips via these methods.
 These methods are similar to those for defining points, except
 that no normalization of the data is possible. Basically, you need to
 define an array of values that (for each cell) includes the number of 
 points per cell, and then the cell connectivity. (This is the vtk file 
 format described in in the textbook or User's Guide.)

\item  \verb|obj.SetPolysComponent (string arrayName, int arrayComp, int min, int max)| -  Define cell connectivity when creating vtkPolyData. You can define
 vertices, lines, polygons, and/or triangle strips via these methods.
 These methods are similar to those for defining points, except
 that no normalization of the data is possible. Basically, you need to
 define an array of values that (for each cell) includes the number of 
 points per cell, and then the cell connectivity. (This is the vtk file 
 format described in in the textbook or User's Guide.)

\item  \verb|obj.SetPolysComponent (string arrayName, int arrayComp)| -  Define cell connectivity when creating vtkPolyData. You can define
 vertices, lines, polygons, and/or triangle strips via these methods.
 These methods are similar to those for defining points, except
 that no normalization of the data is possible. Basically, you need to
 define an array of values that (for each cell) includes the number of 
 points per cell, and then the cell connectivity. (This is the vtk file 
 format described in in the textbook or User's Guide.)

\item  \verb|string = obj.GetPolysComponentArrayName ()| -  Define cell connectivity when creating vtkPolyData. You can define
 vertices, lines, polygons, and/or triangle strips via these methods.
 These methods are similar to those for defining points, except
 that no normalization of the data is possible. Basically, you need to
 define an array of values that (for each cell) includes the number of 
 points per cell, and then the cell connectivity. (This is the vtk file 
 format described in in the textbook or User's Guide.)

\item  \verb|int = obj.GetPolysComponentArrayComponent ()| -  Define cell connectivity when creating vtkPolyData. You can define
 vertices, lines, polygons, and/or triangle strips via these methods.
 These methods are similar to those for defining points, except
 that no normalization of the data is possible. Basically, you need to
 define an array of values that (for each cell) includes the number of 
 points per cell, and then the cell connectivity. (This is the vtk file 
 format described in in the textbook or User's Guide.)

\item  \verb|int = obj.GetPolysComponentMinRange ()| -  Define cell connectivity when creating vtkPolyData. You can define
 vertices, lines, polygons, and/or triangle strips via these methods.
 These methods are similar to those for defining points, except
 that no normalization of the data is possible. Basically, you need to
 define an array of values that (for each cell) includes the number of 
 points per cell, and then the cell connectivity. (This is the vtk file 
 format described in in the textbook or User's Guide.)

\item  \verb|int = obj.GetPolysComponentMaxRange ()| -  Define cell connectivity when creating vtkPolyData. You can define
 vertices, lines, polygons, and/or triangle strips via these methods.
 These methods are similar to those for defining points, except
 that no normalization of the data is possible. Basically, you need to
 define an array of values that (for each cell) includes the number of 
 points per cell, and then the cell connectivity. (This is the vtk file 
 format described in in the textbook or User's Guide.)

\item  \verb|obj.SetStripsComponent (string arrayName, int arrayComp, int min, int max)| -  Define cell connectivity when creating vtkPolyData. You can define
 vertices, lines, polygons, and/or triangle strips via these methods.
 These methods are similar to those for defining points, except
 that no normalization of the data is possible. Basically, you need to
 define an array of values that (for each cell) includes the number of 
 points per cell, and then the cell connectivity. (This is the vtk file 
 format described in in the textbook or User's Guide.)

\item  \verb|obj.SetStripsComponent (string arrayName, int arrayComp)| -  Define cell connectivity when creating vtkPolyData. You can define
 vertices, lines, polygons, and/or triangle strips via these methods.
 These methods are similar to those for defining points, except
 that no normalization of the data is possible. Basically, you need to
 define an array of values that (for each cell) includes the number of 
 points per cell, and then the cell connectivity. (This is the vtk file 
 format described in in the textbook or User's Guide.)

\item  \verb|string = obj.GetStripsComponentArrayName ()| -  Define cell connectivity when creating vtkPolyData. You can define
 vertices, lines, polygons, and/or triangle strips via these methods.
 These methods are similar to those for defining points, except
 that no normalization of the data is possible. Basically, you need to
 define an array of values that (for each cell) includes the number of 
 points per cell, and then the cell connectivity. (This is the vtk file 
 format described in in the textbook or User's Guide.)

\item  \verb|int = obj.GetStripsComponentArrayComponent ()| -  Define cell connectivity when creating vtkPolyData. You can define
 vertices, lines, polygons, and/or triangle strips via these methods.
 These methods are similar to those for defining points, except
 that no normalization of the data is possible. Basically, you need to
 define an array of values that (for each cell) includes the number of 
 points per cell, and then the cell connectivity. (This is the vtk file 
 format described in in the textbook or User's Guide.)

\item  \verb|int = obj.GetStripsComponentMinRange ()| -  Define cell connectivity when creating vtkPolyData. You can define
 vertices, lines, polygons, and/or triangle strips via these methods.
 These methods are similar to those for defining points, except
 that no normalization of the data is possible. Basically, you need to
 define an array of values that (for each cell) includes the number of 
 points per cell, and then the cell connectivity. (This is the vtk file 
 format described in in the textbook or User's Guide.)

\item  \verb|int = obj.GetStripsComponentMaxRange ()| -  Define cell connectivity when creating vtkPolyData. You can define
 vertices, lines, polygons, and/or triangle strips via these methods.
 These methods are similar to those for defining points, except
 that no normalization of the data is possible. Basically, you need to
 define an array of values that (for each cell) includes the number of 
 points per cell, and then the cell connectivity. (This is the vtk file 
 format described in in the textbook or User's Guide.)

\item  \verb|obj.SetCellTypeComponent (string arrayName, int arrayComp, int min, int max)| -  Define cell types and cell connectivity when creating unstructured grid
 data.  These methods are similar to those for defining points, except
 that no normalization of the data is possible. Basically, you need to
 define an array of cell types (an integer value per cell), and another
 array consisting (for each cell) of a number of points per cell, and
 then the cell connectivity. (This is the vtk file format described in 
 in the textbook or User's Guide.)

\item  \verb|obj.SetCellTypeComponent (string arrayName, int arrayComp)| -  Define cell types and cell connectivity when creating unstructured grid
 data.  These methods are similar to those for defining points, except
 that no normalization of the data is possible. Basically, you need to
 define an array of cell types (an integer value per cell), and another
 array consisting (for each cell) of a number of points per cell, and
 then the cell connectivity. (This is the vtk file format described in 
 in the textbook or User's Guide.)

\item  \verb|string = obj.GetCellTypeComponentArrayName ()| -  Define cell types and cell connectivity when creating unstructured grid
 data.  These methods are similar to those for defining points, except
 that no normalization of the data is possible. Basically, you need to
 define an array of cell types (an integer value per cell), and another
 array consisting (for each cell) of a number of points per cell, and
 then the cell connectivity. (This is the vtk file format described in 
 in the textbook or User's Guide.)

\item  \verb|int = obj.GetCellTypeComponentArrayComponent ()| -  Define cell types and cell connectivity when creating unstructured grid
 data.  These methods are similar to those for defining points, except
 that no normalization of the data is possible. Basically, you need to
 define an array of cell types (an integer value per cell), and another
 array consisting (for each cell) of a number of points per cell, and
 then the cell connectivity. (This is the vtk file format described in 
 in the textbook or User's Guide.)

\item  \verb|int = obj.GetCellTypeComponentMinRange ()| -  Define cell types and cell connectivity when creating unstructured grid
 data.  These methods are similar to those for defining points, except
 that no normalization of the data is possible. Basically, you need to
 define an array of cell types (an integer value per cell), and another
 array consisting (for each cell) of a number of points per cell, and
 then the cell connectivity. (This is the vtk file format described in 
 in the textbook or User's Guide.)

\item  \verb|int = obj.GetCellTypeComponentMaxRange ()| -  Define cell types and cell connectivity when creating unstructured grid
 data.  These methods are similar to those for defining points, except
 that no normalization of the data is possible. Basically, you need to
 define an array of cell types (an integer value per cell), and another
 array consisting (for each cell) of a number of points per cell, and
 then the cell connectivity. (This is the vtk file format described in 
 in the textbook or User's Guide.)

\item  \verb|obj.SetCellConnectivityComponent (string arrayName, int arrayComp, int min, int max)| -  Define cell types and cell connectivity when creating unstructured grid
 data.  These methods are similar to those for defining points, except
 that no normalization of the data is possible. Basically, you need to
 define an array of cell types (an integer value per cell), and another
 array consisting (for each cell) of a number of points per cell, and
 then the cell connectivity. (This is the vtk file format described in 
 in the textbook or User's Guide.)

\item  \verb|obj.SetCellConnectivityComponent (string arrayName, int arrayComp)| -  Define cell types and cell connectivity when creating unstructured grid
 data.  These methods are similar to those for defining points, except
 that no normalization of the data is possible. Basically, you need to
 define an array of cell types (an integer value per cell), and another
 array consisting (for each cell) of a number of points per cell, and
 then the cell connectivity. (This is the vtk file format described in 
 in the textbook or User's Guide.)

\item  \verb|string = obj.GetCellConnectivityComponentArrayName ()| -  Define cell types and cell connectivity when creating unstructured grid
 data.  These methods are similar to those for defining points, except
 that no normalization of the data is possible. Basically, you need to
 define an array of cell types (an integer value per cell), and another
 array consisting (for each cell) of a number of points per cell, and
 then the cell connectivity. (This is the vtk file format described in 
 in the textbook or User's Guide.)

\item  \verb|int = obj.GetCellConnectivityComponentArrayComponent ()| -  Define cell types and cell connectivity when creating unstructured grid
 data.  These methods are similar to those for defining points, except
 that no normalization of the data is possible. Basically, you need to
 define an array of cell types (an integer value per cell), and another
 array consisting (for each cell) of a number of points per cell, and
 then the cell connectivity. (This is the vtk file format described in 
 in the textbook or User's Guide.)

\item  \verb|int = obj.GetCellConnectivityComponentMinRange ()| -  Define cell types and cell connectivity when creating unstructured grid
 data.  These methods are similar to those for defining points, except
 that no normalization of the data is possible. Basically, you need to
 define an array of cell types (an integer value per cell), and another
 array consisting (for each cell) of a number of points per cell, and
 then the cell connectivity. (This is the vtk file format described in 
 in the textbook or User's Guide.)

\item  \verb|int = obj.GetCellConnectivityComponentMaxRange ()| -  Define cell types and cell connectivity when creating unstructured grid
 data.  These methods are similar to those for defining points, except
 that no normalization of the data is possible. Basically, you need to
 define an array of cell types (an integer value per cell), and another
 array consisting (for each cell) of a number of points per cell, and
 then the cell connectivity. (This is the vtk file format described in 
 in the textbook or User's Guide.)

\item  \verb|obj.SetDefaultNormalize (int )| -  Set the default Normalize() flag for those methods setting a default
 Normalize value (e.g., SetPointComponent).

\item  \verb|int = obj.GetDefaultNormalize ()| -  Set the default Normalize() flag for those methods setting a default
 Normalize value (e.g., SetPointComponent).

\item  \verb|obj.DefaultNormalizeOn ()| -  Set the default Normalize() flag for those methods setting a default
 Normalize value (e.g., SetPointComponent).

\item  \verb|obj.DefaultNormalizeOff ()| -  Set the default Normalize() flag for those methods setting a default
 Normalize value (e.g., SetPointComponent).

\item  \verb|obj.SetDimensions (int , int , int )|

\item  \verb|obj.SetDimensions (int  a[3])|

\item  \verb|int = obj. GetDimensions ()|

\item  \verb|obj.SetOrigin (double , double , double )|

\item  \verb|obj.SetOrigin (double  a[3])|

\item  \verb|double = obj. GetOrigin ()|

\item  \verb|obj.SetSpacing (double , double , double )|

\item  \verb|obj.SetSpacing (double  a[3])|

\item  \verb|double = obj. GetSpacing ()|

\item  \verb|obj.SetDimensionsComponent (string arrayName, int arrayComp, int min, int max)|

\item  \verb|obj.SetDimensionsComponent (string arrayName, int arrayComp)|

\item  \verb|obj.SetSpacingComponent (string arrayName, int arrayComp, int min, int max)|

\item  \verb|obj.SetSpacingComponent (string arrayName, int arrayComp)|

\item  \verb|obj.SetOriginComponent (string arrayName, int arrayComp, int min, int max)|

\item  \verb|obj.SetOriginComponent (string arrayName, int arrayComp)|

\end{itemize}
