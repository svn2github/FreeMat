\section{vtkDelaunay2D}

\subsection{Usage}

 vtkDelaunay2D is a filter that constructs a 2D Delaunay triangulation from
 a list of input points. These points may be represented by any dataset of
 type vtkPointSet and subclasses. The output of the filter is a polygonal
 dataset. Usually the output is a triangle mesh, but if a non-zero alpha
 distance value is specified (called the ''alpha'' value), then only
 triangles, edges, and vertices lying within the alpha radius are
 output. In other words, non-zero alpha values may result in arbitrary
 combinations of triangles, lines, and vertices. (The notion of alpha value
 is derived from Edelsbrunner's work on ''alpha shapes''.) Also, it is
 possible to generate ''constrained triangulations'' using this filter.
 A constrained triangulation is one where edges and loops (i.e., polygons)
 can be defined and the triangulation will preserve them (read on for 
 more information).

 The 2D Delaunay triangulation is defined as the triangulation that 
 satisfies the Delaunay criterion for n-dimensional simplexes (in this case
 n=2 and the simplexes are triangles). This criterion states that a 
 circumsphere of each simplex in a triangulation contains only the n+1 
 defining points of the simplex. (See ''The Visualization Toolkit'' text 
 for more information.) In two dimensions, this translates into an optimal 
 triangulation. That is, the maximum interior angle of any triangle is less 
 than or equal to that of any possible triangulation.
 
 Delaunay triangulations are used to build topological structures
 from unorganized (or unstructured) points. The input to this filter
 is a list of points specified in 3D, even though the triangulation
 is 2D. Thus the triangulation is constructed in the x-y plane, and
 the z coordinate is ignored (although carried through to the
 output). If you desire to triangulate in a different plane, you
 can use the vtkTransformFilter to transform the points into and
 out of the x-y plane or you can specify a transform to the Delaunay2D
 directly.  In the latter case, the input points are transformed, the
 transformed points are triangulated, and the output will use the
 triangulated topology for the original (non-transformed) points.  This
 avoids transforming the data back as would be required when using the
 vtkTransformFilter method.  Specifying a transform directly also allows
 any transform to be used: rigid, non-rigid, non-invertible, etc.

 If an input transform is used, then alpha values are applied (for the
 most part) in the original data space.  The exception is when
 BoundingTriangulation is on.  In this case, alpha values are applied in
 the original data space unless a cell uses a bounding vertex.  
 
 The Delaunay triangulation can be numerically sensitive in some cases. To
 prevent problems, try to avoid injecting points that will result in
 triangles with bad aspect ratios (1000:1 or greater). In practice this
 means inserting points that are ''widely dispersed'', and enables smooth
 transition of triangle sizes throughout the mesh. (You may even want to
 add extra points to create a better point distribution.) If numerical
 problems are present, you will see a warning message to this effect at
 the end of the triangulation process.

 To create constrained meshes, you must define an additional
 input. This input is an instance of vtkPolyData which contains
 lines, polylines, and/or polygons that define constrained edges and
 loops. Only the topology of (lines and polygons) from this second
 input are used.  The topology is assumed to reference points in the
 input point set (the one to be triangulated). In other words, the
 lines and polygons use point ids from the first input point
 set. Lines and polylines found in the input will be mesh edges in
 the output. Polygons define a loop with inside and outside
 regions. The inside of the polygon is determined by using the
 right-hand-rule, i.e., looking down the z-axis a polygon should be
 ordered counter-clockwise. Holes in a polygon should be ordered
 clockwise. If you choose to create a constrained triangulation, the
 final mesh may not satisfy the Delaunay criterion. (Noted: the
 lines/polygon edges must not intersect when projected onto the 2D
 plane.  It may not be possible to recover all edges due to not
 enough points in the triangulation, or poorly defined edges
 (coincident or excessively long).  The form of the lines or
 polygons is a list of point ids that correspond to the input point
 ids used to generate the triangulation.)

 If an input transform is used, constraints are defined in the
 ''transformed'' space.  So when the right hand rule is used for a
 polygon constraint, that operation is applied using the transformed
 points.  Since the input transform can be any transformation (rigid
 or non-rigid), care must be taken in constructing constraints when
 an input transform is used.

To create an instance of class vtkDelaunay2D, simply
invoke its constructor as follows
\begin{verbatim}
  obj = vtkDelaunay2D
\end{verbatim}
\subsection{Methods}

The class vtkDelaunay2D has several methods that can be used.
  They are listed below.
Note that the documentation is translated automatically from the VTK sources,
and may not be completely intelligible.  When in doubt, consult the VTK website.
In the methods listed below, \verb|obj| is an instance of the vtkDelaunay2D class.
\begin{itemize}
\item  \verb|string = obj.GetClassName ()|

\item  \verb|int = obj.IsA (string name)|

\item  \verb|vtkDelaunay2D = obj.NewInstance ()|

\item  \verb|vtkDelaunay2D = obj.SafeDownCast (vtkObject o)|

\item  \verb|obj.SetSource (vtkPolyData )| -  Specify the source object used to specify constrained edges and loops.
 (This is optional.) If set, and lines/polygons are defined, a constrained
 triangulation is created. The lines/polygons are assumed to reference
 points in the input point set (i.e. point ids are identical in the
 input and source).
 Old style. See SetSourceConnection.

\item  \verb|obj.SetSourceConnection (vtkAlgorithmOutput algOutput)| -  Specify the source object used to specify constrained edges and loops.
 (This is optional.) If set, and lines/polygons are defined, a constrained
 triangulation is created. The lines/polygons are assumed to reference
 points in the input point set (i.e. point ids are identical in the
 input and source).
 New style. This method is equivalent to SetInputConnection(1, algOutput).

\item  \verb|vtkPolyData = obj.GetSource ()| -  Get a pointer to the source object.

\item  \verb|obj.SetAlpha (double )| -  Specify alpha (or distance) value to control output of this filter.
 For a non-zero alpha value, only edges or triangles contained within
 a sphere centered at mesh vertices will be output. Otherwise, only
 triangles will be output.

\item  \verb|double = obj.GetAlphaMinValue ()| -  Specify alpha (or distance) value to control output of this filter.
 For a non-zero alpha value, only edges or triangles contained within
 a sphere centered at mesh vertices will be output. Otherwise, only
 triangles will be output.

\item  \verb|double = obj.GetAlphaMaxValue ()| -  Specify alpha (or distance) value to control output of this filter.
 For a non-zero alpha value, only edges or triangles contained within
 a sphere centered at mesh vertices will be output. Otherwise, only
 triangles will be output.

\item  \verb|double = obj.GetAlpha ()| -  Specify alpha (or distance) value to control output of this filter.
 For a non-zero alpha value, only edges or triangles contained within
 a sphere centered at mesh vertices will be output. Otherwise, only
 triangles will be output.

\item  \verb|obj.SetTolerance (double )| -  Specify a tolerance to control discarding of closely spaced points.
 This tolerance is specified as a fraction of the diagonal length of
 the bounding box of the points.

\item  \verb|double = obj.GetToleranceMinValue ()| -  Specify a tolerance to control discarding of closely spaced points.
 This tolerance is specified as a fraction of the diagonal length of
 the bounding box of the points.

\item  \verb|double = obj.GetToleranceMaxValue ()| -  Specify a tolerance to control discarding of closely spaced points.
 This tolerance is specified as a fraction of the diagonal length of
 the bounding box of the points.

\item  \verb|double = obj.GetTolerance ()| -  Specify a tolerance to control discarding of closely spaced points.
 This tolerance is specified as a fraction of the diagonal length of
 the bounding box of the points.

\item  \verb|obj.SetOffset (double )| -  Specify a multiplier to control the size of the initial, bounding
 Delaunay triangulation.

\item  \verb|double = obj.GetOffsetMinValue ()| -  Specify a multiplier to control the size of the initial, bounding
 Delaunay triangulation.

\item  \verb|double = obj.GetOffsetMaxValue ()| -  Specify a multiplier to control the size of the initial, bounding
 Delaunay triangulation.

\item  \verb|double = obj.GetOffset ()| -  Specify a multiplier to control the size of the initial, bounding
 Delaunay triangulation.

\item  \verb|obj.SetBoundingTriangulation (int )| -  Boolean controls whether bounding triangulation points (and associated
 triangles) are included in the output. (These are introduced as an
 initial triangulation to begin the triangulation process. This feature
 is nice for debugging output.)

\item  \verb|int = obj.GetBoundingTriangulation ()| -  Boolean controls whether bounding triangulation points (and associated
 triangles) are included in the output. (These are introduced as an
 initial triangulation to begin the triangulation process. This feature
 is nice for debugging output.)

\item  \verb|obj.BoundingTriangulationOn ()| -  Boolean controls whether bounding triangulation points (and associated
 triangles) are included in the output. (These are introduced as an
 initial triangulation to begin the triangulation process. This feature
 is nice for debugging output.)

\item  \verb|obj.BoundingTriangulationOff ()| -  Boolean controls whether bounding triangulation points (and associated
 triangles) are included in the output. (These are introduced as an
 initial triangulation to begin the triangulation process. This feature
 is nice for debugging output.)

\item  \verb|obj.SetTransform (vtkAbstractTransform )| -  Set / get the transform which is applied to points to generate a
 2D problem.  This maps a 3D dataset into a 2D dataset where
 triangulation can be done on the XY plane.  The points are
 transformed and triangulated.  The topology of triangulated
 points is used as the output topology.  The output points are the
 original (untransformed) points.  The transform can be any
 subclass of vtkAbstractTransform (thus it does not need to be a
 linear or invertible transform).

\item  \verb|vtkAbstractTransform = obj.GetTransform ()| -  Set / get the transform which is applied to points to generate a
 2D problem.  This maps a 3D dataset into a 2D dataset where
 triangulation can be done on the XY plane.  The points are
 transformed and triangulated.  The topology of triangulated
 points is used as the output topology.  The output points are the
 original (untransformed) points.  The transform can be any
 subclass of vtkAbstractTransform (thus it does not need to be a
 linear or invertible transform).

\item  \verb|obj.SetProjectionPlaneMode (int )| -  Define

\item  \verb|int = obj.GetProjectionPlaneModeMinValue ()| -  Define

\item  \verb|int = obj.GetProjectionPlaneModeMaxValue ()| -  Define

\item  \verb|int = obj.GetProjectionPlaneMode ()| -  Define

\end{itemize}
