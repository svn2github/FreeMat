\section{vtkDicer}

\subsection{Usage}

 Subclasses of vtkDicer divides the input dataset into separate
 pieces.  These pieces can then be operated on by other filters
 (e.g., vtkThreshold). One application is to break very large
 polygonal models into pieces and performing viewing and occlusion
 culling on the pieces. Multiple pieces can also be streamed through
 the visualization pipeline.

 To use this filter, you must specify the execution mode of the
 filter; i.e., set the way that the piece size is controlled (do
 this by setting the DiceMode ivar). The filter does not change the
 geometry or topology of the input dataset, rather it generates
 integer numbers that indicate which piece a particular point
 belongs to (i.e., it modifies the point and cell attribute
 data). The integer number can be placed into the output scalar
 data, or the output field data.

To create an instance of class vtkDicer, simply
invoke its constructor as follows
\begin{verbatim}
  obj = vtkDicer
\end{verbatim}
\subsection{Methods}

The class vtkDicer has several methods that can be used.
  They are listed below.
Note that the documentation is translated automatically from the VTK sources,
and may not be completely intelligible.  When in doubt, consult the VTK website.
In the methods listed below, \verb|obj| is an instance of the vtkDicer class.
\begin{itemize}
\item  \verb|string = obj.GetClassName ()|

\item  \verb|int = obj.IsA (string name)|

\item  \verb|vtkDicer = obj.NewInstance ()|

\item  \verb|vtkDicer = obj.SafeDownCast (vtkObject o)|

\item  \verb|obj.SetFieldData (int )| -  Set/Get the flag which controls whether to generate point scalar
 data or point field data. If this flag is off, scalar data is
 generated.  Otherwise, field data is generated. Note that the
 generated the data are integer numbers indicating which piece a
 particular point belongs to.

\item  \verb|int = obj.GetFieldData ()| -  Set/Get the flag which controls whether to generate point scalar
 data or point field data. If this flag is off, scalar data is
 generated.  Otherwise, field data is generated. Note that the
 generated the data are integer numbers indicating which piece a
 particular point belongs to.

\item  \verb|obj.FieldDataOn ()| -  Set/Get the flag which controls whether to generate point scalar
 data or point field data. If this flag is off, scalar data is
 generated.  Otherwise, field data is generated. Note that the
 generated the data are integer numbers indicating which piece a
 particular point belongs to.

\item  \verb|obj.FieldDataOff ()| -  Set/Get the flag which controls whether to generate point scalar
 data or point field data. If this flag is off, scalar data is
 generated.  Otherwise, field data is generated. Note that the
 generated the data are integer numbers indicating which piece a
 particular point belongs to.

\item  \verb|obj.SetDiceMode (int )| -  Specify the method to determine how many pieces the data should be
 broken into. By default, the number of points per piece is used.

\item  \verb|int = obj.GetDiceModeMinValue ()| -  Specify the method to determine how many pieces the data should be
 broken into. By default, the number of points per piece is used.

\item  \verb|int = obj.GetDiceModeMaxValue ()| -  Specify the method to determine how many pieces the data should be
 broken into. By default, the number of points per piece is used.

\item  \verb|int = obj.GetDiceMode ()| -  Specify the method to determine how many pieces the data should be
 broken into. By default, the number of points per piece is used.

\item  \verb|obj.SetDiceModeToNumberOfPointsPerPiece ()| -  Specify the method to determine how many pieces the data should be
 broken into. By default, the number of points per piece is used.

\item  \verb|obj.SetDiceModeToSpecifiedNumberOfPieces ()| -  Specify the method to determine how many pieces the data should be
 broken into. By default, the number of points per piece is used.

\item  \verb|obj.SetDiceModeToMemoryLimitPerPiece ()| -  Specify the method to determine how many pieces the data should be
 broken into. By default, the number of points per piece is used.

\item  \verb|int = obj.GetNumberOfActualPieces ()| -  Use the following method after the filter has updated to
 determine the actual number of pieces the data was separated
 into.

\item  \verb|obj.SetNumberOfPointsPerPiece (int )| -  Control piece size based on the maximum number of points per piece.
 (This ivar has effect only when the DiceMode is set to 
 SetDiceModeToNumberOfPoints().)

\item  \verb|int = obj.GetNumberOfPointsPerPieceMinValue ()| -  Control piece size based on the maximum number of points per piece.
 (This ivar has effect only when the DiceMode is set to 
 SetDiceModeToNumberOfPoints().)

\item  \verb|int = obj.GetNumberOfPointsPerPieceMaxValue ()| -  Control piece size based on the maximum number of points per piece.
 (This ivar has effect only when the DiceMode is set to 
 SetDiceModeToNumberOfPoints().)

\item  \verb|int = obj.GetNumberOfPointsPerPiece ()| -  Control piece size based on the maximum number of points per piece.
 (This ivar has effect only when the DiceMode is set to 
 SetDiceModeToNumberOfPoints().)

\item  \verb|obj.SetNumberOfPieces (int )| -  Set/Get the number of pieces the object is to be separated into.
 (This ivar has effect only when the DiceMode is set to
 SetDiceModeToSpecifiedNumber()). Note that the ivar
 NumberOfPieces is a target - depending on the particulars of the
 data, more or less number of pieces than the target value may be
 created.

\item  \verb|int = obj.GetNumberOfPiecesMinValue ()| -  Set/Get the number of pieces the object is to be separated into.
 (This ivar has effect only when the DiceMode is set to
 SetDiceModeToSpecifiedNumber()). Note that the ivar
 NumberOfPieces is a target - depending on the particulars of the
 data, more or less number of pieces than the target value may be
 created.

\item  \verb|int = obj.GetNumberOfPiecesMaxValue ()| -  Set/Get the number of pieces the object is to be separated into.
 (This ivar has effect only when the DiceMode is set to
 SetDiceModeToSpecifiedNumber()). Note that the ivar
 NumberOfPieces is a target - depending on the particulars of the
 data, more or less number of pieces than the target value may be
 created.

\item  \verb|int = obj.GetNumberOfPieces ()| -  Set/Get the number of pieces the object is to be separated into.
 (This ivar has effect only when the DiceMode is set to
 SetDiceModeToSpecifiedNumber()). Note that the ivar
 NumberOfPieces is a target - depending on the particulars of the
 data, more or less number of pieces than the target value may be
 created.

\item  \verb|obj.SetMemoryLimit (long )| -  Control piece size based on a memory limit.  (This ivar has
 effect only when the DiceMode is set to
 SetDiceModeToMemoryLimit()). The memory limit should be set in
 kilobytes.

\item  \verb|GetMemoryLimitMinValue = obj.()| -  Control piece size based on a memory limit.  (This ivar has
 effect only when the DiceMode is set to
 SetDiceModeToMemoryLimit()). The memory limit should be set in
 kilobytes.

\item  \verb|GetMemoryLimitMaxValue = obj.()| -  Control piece size based on a memory limit.  (This ivar has
 effect only when the DiceMode is set to
 SetDiceModeToMemoryLimit()). The memory limit should be set in
 kilobytes.

\item  \verb|long = obj.GetMemoryLimit ()| -  Control piece size based on a memory limit.  (This ivar has
 effect only when the DiceMode is set to
 SetDiceModeToMemoryLimit()). The memory limit should be set in
 kilobytes.

\end{itemize}
