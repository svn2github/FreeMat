\section{vtkDijkstraImageGeodesicPath}

\subsection{Usage}

 Takes as input a polyline and an image representing a 2D cost function
 and performs a single source shortest path calculation. 
 Dijkstra's algorithm is used. The implementation is 
 similar to the one described in Introduction to Algorithms (Second Edition)
 by Thomas H. Cormen, Charles E. Leiserson, Ronald L. Rivest, and 
 Cliff Stein, published by MIT Press and McGraw-Hill. Some minor 
 enhancement are added though. All vertices are not pushed on the heap
 at start, instead a front set is maintained. The heap is implemented as 
 a binary heap. The output of the filter is a set of lines describing 
 the shortest path from StartVertex to EndVertex.  See parent class
 vtkDijkstraGraphGeodesicPath for the implementation.


To create an instance of class vtkDijkstraImageGeodesicPath, simply
invoke its constructor as follows
\begin{verbatim}
  obj = vtkDijkstraImageGeodesicPath
\end{verbatim}
\subsection{Methods}

The class vtkDijkstraImageGeodesicPath has several methods that can be used.
  They are listed below.
Note that the documentation is translated automatically from the VTK sources,
and may not be completely intelligible.  When in doubt, consult the VTK website.
In the methods listed below, \verb|obj| is an instance of the vtkDijkstraImageGeodesicPath class.
\begin{itemize}
\item  \verb|string = obj.GetClassName ()| -  Standard methids for printing and determining type information.

\item  \verb|int = obj.IsA (string name)| -  Standard methids for printing and determining type information.

\item  \verb|vtkDijkstraImageGeodesicPath = obj.NewInstance ()| -  Standard methids for printing and determining type information.

\item  \verb|vtkDijkstraImageGeodesicPath = obj.SafeDownCast (vtkObject o)| -  Standard methids for printing and determining type information.

\item  \verb|obj.SetInput (vtkDataObject )| -  Specify the image object which is used as a cost function.

\item  \verb|vtkImageData = obj.GetInputAsImageData ()| -  Specify the image object which is used as a cost function.

\item  \verb|obj.SetImageWeight (double )| -  Image cost weight.

\item  \verb|double = obj.GetImageWeight ()| -  Image cost weight.

\item  \verb|obj.SetEdgeLengthWeight (double )| -  Edge length cost weight.

\item  \verb|double = obj.GetEdgeLengthWeight ()| -  Edge length cost weight.

\item  \verb|obj.SetCurvatureWeight (double )| -  Curvature cost weight.

\item  \verb|double = obj.GetCurvatureWeightMinValue ()| -  Curvature cost weight.

\item  \verb|double = obj.GetCurvatureWeightMaxValue ()| -  Curvature cost weight.

\item  \verb|double = obj.GetCurvatureWeight ()| -  Curvature cost weight.

\end{itemize}
