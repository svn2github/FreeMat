\section{vtkExtractUnstructuredGrid}

\subsection{Usage}

 vtkExtractUnstructuredGrid is a general-purpose filter to
 extract geometry (and associated data) from an unstructured grid
 dataset. The extraction process is controlled by specifying a range
 of point ids, cell ids, or a bounding box (referred to as ''Extent'').
 Those cells lying within these regions are sent to the output.
 The user has the choice of merging coincident points (Merging is on)
 or using the original point set (Merging is off).

To create an instance of class vtkExtractUnstructuredGrid, simply
invoke its constructor as follows
\begin{verbatim}
  obj = vtkExtractUnstructuredGrid
\end{verbatim}
\subsection{Methods}

The class vtkExtractUnstructuredGrid has several methods that can be used.
  They are listed below.
Note that the documentation is translated automatically from the VTK sources,
and may not be completely intelligible.  When in doubt, consult the VTK website.
In the methods listed below, \verb|obj| is an instance of the vtkExtractUnstructuredGrid class.
\begin{itemize}
\item  \verb|string = obj.GetClassName ()|

\item  \verb|int = obj.IsA (string name)|

\item  \verb|vtkExtractUnstructuredGrid = obj.NewInstance ()|

\item  \verb|vtkExtractUnstructuredGrid = obj.SafeDownCast (vtkObject o)|

\item  \verb|obj.SetPointClipping (int )| -  Turn on/off selection of geometry by point id.

\item  \verb|int = obj.GetPointClipping ()| -  Turn on/off selection of geometry by point id.

\item  \verb|obj.PointClippingOn ()| -  Turn on/off selection of geometry by point id.

\item  \verb|obj.PointClippingOff ()| -  Turn on/off selection of geometry by point id.

\item  \verb|obj.SetCellClipping (int )| -  Turn on/off selection of geometry by cell id.

\item  \verb|int = obj.GetCellClipping ()| -  Turn on/off selection of geometry by cell id.

\item  \verb|obj.CellClippingOn ()| -  Turn on/off selection of geometry by cell id.

\item  \verb|obj.CellClippingOff ()| -  Turn on/off selection of geometry by cell id.

\item  \verb|obj.SetExtentClipping (int )| -  Turn on/off selection of geometry via bounding box.

\item  \verb|int = obj.GetExtentClipping ()| -  Turn on/off selection of geometry via bounding box.

\item  \verb|obj.ExtentClippingOn ()| -  Turn on/off selection of geometry via bounding box.

\item  \verb|obj.ExtentClippingOff ()| -  Turn on/off selection of geometry via bounding box.

\item  \verb|obj.SetPointMinimum (vtkIdType )| -  Specify the minimum point id for point id selection.

\item  \verb|vtkIdType = obj.GetPointMinimumMinValue ()| -  Specify the minimum point id for point id selection.

\item  \verb|vtkIdType = obj.GetPointMinimumMaxValue ()| -  Specify the minimum point id for point id selection.

\item  \verb|vtkIdType = obj.GetPointMinimum ()| -  Specify the minimum point id for point id selection.

\item  \verb|obj.SetPointMaximum (vtkIdType )| -  Specify the maximum point id for point id selection.

\item  \verb|vtkIdType = obj.GetPointMaximumMinValue ()| -  Specify the maximum point id for point id selection.

\item  \verb|vtkIdType = obj.GetPointMaximumMaxValue ()| -  Specify the maximum point id for point id selection.

\item  \verb|vtkIdType = obj.GetPointMaximum ()| -  Specify the maximum point id for point id selection.

\item  \verb|obj.SetCellMinimum (vtkIdType )| -  Specify the minimum cell id for point id selection.

\item  \verb|vtkIdType = obj.GetCellMinimumMinValue ()| -  Specify the minimum cell id for point id selection.

\item  \verb|vtkIdType = obj.GetCellMinimumMaxValue ()| -  Specify the minimum cell id for point id selection.

\item  \verb|vtkIdType = obj.GetCellMinimum ()| -  Specify the minimum cell id for point id selection.

\item  \verb|obj.SetCellMaximum (vtkIdType )| -  Specify the maximum cell id for point id selection.

\item  \verb|vtkIdType = obj.GetCellMaximumMinValue ()| -  Specify the maximum cell id for point id selection.

\item  \verb|vtkIdType = obj.GetCellMaximumMaxValue ()| -  Specify the maximum cell id for point id selection.

\item  \verb|vtkIdType = obj.GetCellMaximum ()| -  Specify the maximum cell id for point id selection.

\item  \verb|obj.SetExtent (double xMin, double xMax, double yMin, double yMax, double zMin, double zMax)| -  Specify a (xmin,xmax, ymin,ymax, zmin,zmax) bounding box to clip data.

\item  \verb|obj.SetExtent (double extent[6])| -  Set / get a (xmin,xmax, ymin,ymax, zmin,zmax) bounding box to clip data.

\item  \verb|double = obj.GetExtent ()| -  Set / get a (xmin,xmax, ymin,ymax, zmin,zmax) bounding box to clip data.

\item  \verb|obj.SetMerging (int )| -  Turn on/off merging of coincident points. Note that is merging is
 on, points with different point attributes (e.g., normals) are merged,
 which may cause rendering artifacts.

\item  \verb|int = obj.GetMerging ()| -  Turn on/off merging of coincident points. Note that is merging is
 on, points with different point attributes (e.g., normals) are merged,
 which may cause rendering artifacts.

\item  \verb|obj.MergingOn ()| -  Turn on/off merging of coincident points. Note that is merging is
 on, points with different point attributes (e.g., normals) are merged,
 which may cause rendering artifacts.

\item  \verb|obj.MergingOff ()| -  Turn on/off merging of coincident points. Note that is merging is
 on, points with different point attributes (e.g., normals) are merged,
 which may cause rendering artifacts.

\item  \verb|obj.SetLocator (vtkIncrementalPointLocator locator)| -  Set / get a spatial locator for merging points. By
 default an instance of vtkMergePoints is used.

\item  \verb|vtkIncrementalPointLocator = obj.GetLocator ()| -  Set / get a spatial locator for merging points. By
 default an instance of vtkMergePoints is used.

\item  \verb|obj.CreateDefaultLocator ()| -  Create default locator. Used to create one when none is specified.

\item  \verb|long = obj.GetMTime ()| -  Return the MTime also considering the locator.

\end{itemize}
