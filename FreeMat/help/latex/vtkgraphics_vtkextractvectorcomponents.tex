\section{vtkExtractVectorComponents}

\subsection{Usage}

 vtkExtractVectorComponents is a filter that extracts vector components as
 separate scalars. This is accomplished by creating three different outputs.
 Each output is the same as the input, except that the scalar values will be
 one of the three components of the vector. These can be found in the
 VxComponent, VyComponent, and VzComponent.
 Alternatively, if the ExtractToFieldData flag is set, the filter will
 put all the components in the field data. The first component will be
 the scalar and the others will be non-attribute arrays.

To create an instance of class vtkExtractVectorComponents, simply
invoke its constructor as follows
\begin{verbatim}
  obj = vtkExtractVectorComponents
\end{verbatim}
\subsection{Methods}

The class vtkExtractVectorComponents has several methods that can be used.
  They are listed below.
Note that the documentation is translated automatically from the VTK sources,
and may not be completely intelligible.  When in doubt, consult the VTK website.
In the methods listed below, \verb|obj| is an instance of the vtkExtractVectorComponents class.
\begin{itemize}
\item  \verb|string = obj.GetClassName ()|

\item  \verb|int = obj.IsA (string name)|

\item  \verb|vtkExtractVectorComponents = obj.NewInstance ()|

\item  \verb|vtkExtractVectorComponents = obj.SafeDownCast (vtkObject o)|

\item  \verb|obj.SetInput (vtkDataSet input)| -  Specify the input data or filter.

\item  \verb|vtkDataSet = obj.GetVxComponent ()| -  Get the output dataset representing velocity x-component. If output is
 NULL then input hasn't been set, which is necessary for abstract
 objects. (Note: this method returns the same information as the
 GetOutput() method with an index of 0.)

\item  \verb|vtkDataSet = obj.GetVyComponent ()| -  Get the output dataset representing velocity y-component. If output is
 NULL then input hasn't been set, which is necessary for abstract
 objects. (Note: this method returns the same information as the
 GetOutput() method with an index of 1.)
 Note that if ExtractToFieldData is true, this output will be empty.

\item  \verb|vtkDataSet = obj.GetVzComponent ()| -  Get the output dataset representing velocity z-component. If output is
 NULL then input hasn't been set, which is necessary for abstract
 objects. (Note: this method returns the same information as the
 GetOutput() method with an index of 2.)
 Note that if ExtractToFieldData is true, this output will be empty.

\item  \verb|obj.SetExtractToFieldData (int )| -  Determines whether the vector components will be put
 in separate outputs or in the first output's field data

\item  \verb|int = obj.GetExtractToFieldData ()| -  Determines whether the vector components will be put
 in separate outputs or in the first output's field data

\item  \verb|obj.ExtractToFieldDataOn ()| -  Determines whether the vector components will be put
 in separate outputs or in the first output's field data

\item  \verb|obj.ExtractToFieldDataOff ()| -  Determines whether the vector components will be put
 in separate outputs or in the first output's field data

\end{itemize}
