\section{vtkFeatureEdges}

\subsection{Usage}

 vtkFeatureEdges is a filter to extract special types of edges from
 input polygonal data. These edges are either 1) boundary (used by 
 one polygon) or a line cell; 2) non-manifold (used by three or more 
 polygons); 3) feature edges (edges used by two triangles and whose
 dihedral angle > FeatureAngle); or 4) manifold edges (edges used by
 exactly two polygons). These edges may be extracted in any
 combination. Edges may also be ''colored'' (i.e., scalar values assigned)
 based on edge type. The cell coloring is assigned to the cell data of
 the extracted edges.

To create an instance of class vtkFeatureEdges, simply
invoke its constructor as follows
\begin{verbatim}
  obj = vtkFeatureEdges
\end{verbatim}
\subsection{Methods}

The class vtkFeatureEdges has several methods that can be used.
  They are listed below.
Note that the documentation is translated automatically from the VTK sources,
and may not be completely intelligible.  When in doubt, consult the VTK website.
In the methods listed below, \verb|obj| is an instance of the vtkFeatureEdges class.
\begin{itemize}
\item  \verb|string = obj.GetClassName ()|

\item  \verb|int = obj.IsA (string name)|

\item  \verb|vtkFeatureEdges = obj.NewInstance ()|

\item  \verb|vtkFeatureEdges = obj.SafeDownCast (vtkObject o)|

\item  \verb|obj.SetBoundaryEdges (int )| -  Turn on/off the extraction of boundary edges.

\item  \verb|int = obj.GetBoundaryEdges ()| -  Turn on/off the extraction of boundary edges.

\item  \verb|obj.BoundaryEdgesOn ()| -  Turn on/off the extraction of boundary edges.

\item  \verb|obj.BoundaryEdgesOff ()| -  Turn on/off the extraction of boundary edges.

\item  \verb|obj.SetFeatureEdges (int )| -  Turn on/off the extraction of feature edges.

\item  \verb|int = obj.GetFeatureEdges ()| -  Turn on/off the extraction of feature edges.

\item  \verb|obj.FeatureEdgesOn ()| -  Turn on/off the extraction of feature edges.

\item  \verb|obj.FeatureEdgesOff ()| -  Turn on/off the extraction of feature edges.

\item  \verb|obj.SetFeatureAngle (double )| -  Specify the feature angle for extracting feature edges.

\item  \verb|double = obj.GetFeatureAngleMinValue ()| -  Specify the feature angle for extracting feature edges.

\item  \verb|double = obj.GetFeatureAngleMaxValue ()| -  Specify the feature angle for extracting feature edges.

\item  \verb|double = obj.GetFeatureAngle ()| -  Specify the feature angle for extracting feature edges.

\item  \verb|obj.SetNonManifoldEdges (int )| -  Turn on/off the extraction of non-manifold edges.

\item  \verb|int = obj.GetNonManifoldEdges ()| -  Turn on/off the extraction of non-manifold edges.

\item  \verb|obj.NonManifoldEdgesOn ()| -  Turn on/off the extraction of non-manifold edges.

\item  \verb|obj.NonManifoldEdgesOff ()| -  Turn on/off the extraction of non-manifold edges.

\item  \verb|obj.SetManifoldEdges (int )| -  Turn on/off the extraction of manifold edges.

\item  \verb|int = obj.GetManifoldEdges ()| -  Turn on/off the extraction of manifold edges.

\item  \verb|obj.ManifoldEdgesOn ()| -  Turn on/off the extraction of manifold edges.

\item  \verb|obj.ManifoldEdgesOff ()| -  Turn on/off the extraction of manifold edges.

\item  \verb|obj.SetColoring (int )| -  Turn on/off the coloring of edges by type.

\item  \verb|int = obj.GetColoring ()| -  Turn on/off the coloring of edges by type.

\item  \verb|obj.ColoringOn ()| -  Turn on/off the coloring of edges by type.

\item  \verb|obj.ColoringOff ()| -  Turn on/off the coloring of edges by type.

\item  \verb|obj.SetLocator (vtkIncrementalPointLocator locator)| -  Set / get a spatial locator for merging points. By
 default an instance of vtkMergePoints is used.

\item  \verb|vtkIncrementalPointLocator = obj.GetLocator ()| -  Set / get a spatial locator for merging points. By
 default an instance of vtkMergePoints is used.

\item  \verb|obj.CreateDefaultLocator ()| -  Create default locator. Used to create one when none is specified.

\item  \verb|long = obj.GetMTime ()| -  Return MTime also considering the locator.

\end{itemize}
