\section{vtkGeometryFilter}

\subsection{Usage}

 vtkGeometryFilter is a general-purpose filter to extract geometry (and 
 associated data) from any type of dataset. Geometry is obtained as 
 follows: all 0D, 1D, and 2D cells are extracted. All 2D faces that are 
 used by only one 3D cell (i.e., boundary faces) are extracted. It also is 
 possible to specify conditions on point ids, cell ids, and on 
 bounding box (referred to as ''Extent'') to control the extraction process.

 This filter also may be used to convert any type of data to polygonal
 type. The conversion process may be less than satisfactory for some 3D
 datasets. For example, this filter will extract the outer surface of a 
 volume or structured grid dataset. (For structured data you may want to
 use vtkImageDataGeometryFilter, vtkStructuredGridGeometryFilter,
 vtkExtractUnstructuredGrid, vtkRectilinearGridGeometryFilter, or 
 vtkExtractVOI.)

To create an instance of class vtkGeometryFilter, simply
invoke its constructor as follows
\begin{verbatim}
  obj = vtkGeometryFilter
\end{verbatim}
\subsection{Methods}

The class vtkGeometryFilter has several methods that can be used.
  They are listed below.
Note that the documentation is translated automatically from the VTK sources,
and may not be completely intelligible.  When in doubt, consult the VTK website.
In the methods listed below, \verb|obj| is an instance of the vtkGeometryFilter class.
\begin{itemize}
\item  \verb|string = obj.GetClassName ()|

\item  \verb|int = obj.IsA (string name)|

\item  \verb|vtkGeometryFilter = obj.NewInstance ()|

\item  \verb|vtkGeometryFilter = obj.SafeDownCast (vtkObject o)|

\item  \verb|obj.SetPointClipping (int )| -  Turn on/off selection of geometry by point id.

\item  \verb|int = obj.GetPointClipping ()| -  Turn on/off selection of geometry by point id.

\item  \verb|obj.PointClippingOn ()| -  Turn on/off selection of geometry by point id.

\item  \verb|obj.PointClippingOff ()| -  Turn on/off selection of geometry by point id.

\item  \verb|obj.SetCellClipping (int )| -  Turn on/off selection of geometry by cell id.

\item  \verb|int = obj.GetCellClipping ()| -  Turn on/off selection of geometry by cell id.

\item  \verb|obj.CellClippingOn ()| -  Turn on/off selection of geometry by cell id.

\item  \verb|obj.CellClippingOff ()| -  Turn on/off selection of geometry by cell id.

\item  \verb|obj.SetExtentClipping (int )| -  Turn on/off selection of geometry via bounding box.

\item  \verb|int = obj.GetExtentClipping ()| -  Turn on/off selection of geometry via bounding box.

\item  \verb|obj.ExtentClippingOn ()| -  Turn on/off selection of geometry via bounding box.

\item  \verb|obj.ExtentClippingOff ()| -  Turn on/off selection of geometry via bounding box.

\item  \verb|obj.SetPointMinimum (vtkIdType )| -  Specify the minimum point id for point id selection.

\item  \verb|vtkIdType = obj.GetPointMinimumMinValue ()| -  Specify the minimum point id for point id selection.

\item  \verb|vtkIdType = obj.GetPointMinimumMaxValue ()| -  Specify the minimum point id for point id selection.

\item  \verb|vtkIdType = obj.GetPointMinimum ()| -  Specify the minimum point id for point id selection.

\item  \verb|obj.SetPointMaximum (vtkIdType )| -  Specify the maximum point id for point id selection.

\item  \verb|vtkIdType = obj.GetPointMaximumMinValue ()| -  Specify the maximum point id for point id selection.

\item  \verb|vtkIdType = obj.GetPointMaximumMaxValue ()| -  Specify the maximum point id for point id selection.

\item  \verb|vtkIdType = obj.GetPointMaximum ()| -  Specify the maximum point id for point id selection.

\item  \verb|obj.SetCellMinimum (vtkIdType )| -  Specify the minimum cell id for point id selection.

\item  \verb|vtkIdType = obj.GetCellMinimumMinValue ()| -  Specify the minimum cell id for point id selection.

\item  \verb|vtkIdType = obj.GetCellMinimumMaxValue ()| -  Specify the minimum cell id for point id selection.

\item  \verb|vtkIdType = obj.GetCellMinimum ()| -  Specify the minimum cell id for point id selection.

\item  \verb|obj.SetCellMaximum (vtkIdType )| -  Specify the maximum cell id for point id selection.

\item  \verb|vtkIdType = obj.GetCellMaximumMinValue ()| -  Specify the maximum cell id for point id selection.

\item  \verb|vtkIdType = obj.GetCellMaximumMaxValue ()| -  Specify the maximum cell id for point id selection.

\item  \verb|vtkIdType = obj.GetCellMaximum ()| -  Specify the maximum cell id for point id selection.

\item  \verb|obj.SetExtent (double xMin, double xMax, double yMin, double yMax, double zMin, double zMax)| -  Specify a (xmin,xmax, ymin,ymax, zmin,zmax) bounding box to clip data.

\item  \verb|obj.SetExtent (double extent[6])| -  Set / get a (xmin,xmax, ymin,ymax, zmin,zmax) bounding box to clip data.

\item  \verb|double = obj.GetExtent ()| -  Set / get a (xmin,xmax, ymin,ymax, zmin,zmax) bounding box to clip data.

\item  \verb|obj.SetMerging (int )| -  Turn on/off merging of coincident points. Note that is merging is
 on, points with different point attributes (e.g., normals) are merged,
 which may cause rendering artifacts.

\item  \verb|int = obj.GetMerging ()| -  Turn on/off merging of coincident points. Note that is merging is
 on, points with different point attributes (e.g., normals) are merged,
 which may cause rendering artifacts.

\item  \verb|obj.MergingOn ()| -  Turn on/off merging of coincident points. Note that is merging is
 on, points with different point attributes (e.g., normals) are merged,
 which may cause rendering artifacts.

\item  \verb|obj.MergingOff ()| -  Turn on/off merging of coincident points. Note that is merging is
 on, points with different point attributes (e.g., normals) are merged,
 which may cause rendering artifacts.

\item  \verb|obj.SetLocator (vtkIncrementalPointLocator locator)| -  Set / get a spatial locator for merging points. By
 default an instance of vtkMergePoints is used.

\item  \verb|vtkIncrementalPointLocator = obj.GetLocator ()| -  Set / get a spatial locator for merging points. By
 default an instance of vtkMergePoints is used.

\item  \verb|obj.CreateDefaultLocator ()| -  Create default locator. Used to create one when none is specified.

\item  \verb|long = obj.GetMTime ()| -  Return the MTime also considering the locator.

\end{itemize}
