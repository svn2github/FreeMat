\section{vtkHyperStreamline}

\subsection{Usage}

 vtkHyperStreamline is a filter that integrates through a tensor field to 
 generate a hyperstreamline. The integration is along the maximum eigenvector
 and the cross section of the hyperstreamline is defined by the two other
 eigenvectors. Thus the shape of the hyperstreamline is ''tube-like'', with 
 the cross section being elliptical. Hyperstreamlines are used to visualize
 tensor fields.

 The starting point of a hyperstreamline can be defined in one of two ways. 
 First, you may specify an initial position. This is a x-y-z global 
 coordinate. The second option is to specify a starting location. This is 
 cellId, subId, and  cell parametric coordinates.

 The integration of the hyperstreamline occurs through the major eigenvector 
 field. IntegrationStepLength controls the step length within each cell 
 (i.e., this is the fraction of the cell length). The length of the 
 hyperstreamline is controlled by MaximumPropagationDistance. This parameter
 is the length of the hyperstreamline in units of distance. The tube itself 
 is composed of many small sub-tubes - NumberOfSides controls the number of 
 sides in the tube, and StepLength controls the length of the sub-tubes.

 Because hyperstreamlines are often created near regions of singularities, it
 is possible to control the scaling of the tube cross section by using a 
 logarithmic scale. Use LogScalingOn to turn this capability on. The Radius 
 value controls the initial radius of the tube.

To create an instance of class vtkHyperStreamline, simply
invoke its constructor as follows
\begin{verbatim}
  obj = vtkHyperStreamline
\end{verbatim}
\subsection{Methods}

The class vtkHyperStreamline has several methods that can be used.
  They are listed below.
Note that the documentation is translated automatically from the VTK sources,
and may not be completely intelligible.  When in doubt, consult the VTK website.
In the methods listed below, \verb|obj| is an instance of the vtkHyperStreamline class.
\begin{itemize}
\item  \verb|string = obj.GetClassName ()|

\item  \verb|int = obj.IsA (string name)|

\item  \verb|vtkHyperStreamline = obj.NewInstance ()|

\item  \verb|vtkHyperStreamline = obj.SafeDownCast (vtkObject o)|

\item  \verb|obj.SetStartLocation (vtkIdType cellId, int subId, double pcoords[3])| -  Specify the start of the hyperstreamline in the cell coordinate system. 
 That is, cellId and subId (if composite cell), and parametric coordinates.

\item  \verb|obj.SetStartLocation (vtkIdType cellId, int subId, double r, double s, double t)| -  Specify the start of the hyperstreamline in the cell coordinate system. 
 That is, cellId and subId (if composite cell), and parametric coordinates.

\item  \verb|obj.SetStartPosition (double x[3])| -  Specify the start of the hyperstreamline in the global coordinate system. 
 Starting from position implies that a search must be performed to find 
 initial cell to start integration from.

\item  \verb|obj.SetStartPosition (double x, double y, double z)| -  Specify the start of the hyperstreamline in the global coordinate system. 
 Starting from position implies that a search must be performed to find 
 initial cell to start integration from.

\item  \verb|double = obj.GetStartPosition ()| -  Get the start position of the hyperstreamline in global x-y-z coordinates.

\item  \verb|obj.SetMaximumPropagationDistance (double )| -  Set / get the maximum length of the hyperstreamline expressed as absolute
 distance (i.e., arc length) value.

\item  \verb|double = obj.GetMaximumPropagationDistanceMinValue ()| -  Set / get the maximum length of the hyperstreamline expressed as absolute
 distance (i.e., arc length) value.

\item  \verb|double = obj.GetMaximumPropagationDistanceMaxValue ()| -  Set / get the maximum length of the hyperstreamline expressed as absolute
 distance (i.e., arc length) value.

\item  \verb|double = obj.GetMaximumPropagationDistance ()| -  Set / get the maximum length of the hyperstreamline expressed as absolute
 distance (i.e., arc length) value.

\item  \verb|obj.SetIntegrationEigenvector (int )| -  Set / get the eigenvector field through which to ingrate. It is
 possible to integrate using the major, medium or minor
 eigenvector field.  The major eigenvector is the eigenvector
 whose corresponding eigenvalue is closest to positive infinity.
 The minor eigenvector is the eigenvector whose corresponding
 eigenvalue is closest to negative infinity.  The medium
 eigenvector is the eigenvector whose corresponding eigenvalue is
 between the major and minor eigenvalues.

\item  \verb|int = obj.GetIntegrationEigenvectorMinValue ()| -  Set / get the eigenvector field through which to ingrate. It is
 possible to integrate using the major, medium or minor
 eigenvector field.  The major eigenvector is the eigenvector
 whose corresponding eigenvalue is closest to positive infinity.
 The minor eigenvector is the eigenvector whose corresponding
 eigenvalue is closest to negative infinity.  The medium
 eigenvector is the eigenvector whose corresponding eigenvalue is
 between the major and minor eigenvalues.

\item  \verb|int = obj.GetIntegrationEigenvectorMaxValue ()| -  Set / get the eigenvector field through which to ingrate. It is
 possible to integrate using the major, medium or minor
 eigenvector field.  The major eigenvector is the eigenvector
 whose corresponding eigenvalue is closest to positive infinity.
 The minor eigenvector is the eigenvector whose corresponding
 eigenvalue is closest to negative infinity.  The medium
 eigenvector is the eigenvector whose corresponding eigenvalue is
 between the major and minor eigenvalues.

\item  \verb|int = obj.GetIntegrationEigenvector ()| -  Set / get the eigenvector field through which to ingrate. It is
 possible to integrate using the major, medium or minor
 eigenvector field.  The major eigenvector is the eigenvector
 whose corresponding eigenvalue is closest to positive infinity.
 The minor eigenvector is the eigenvector whose corresponding
 eigenvalue is closest to negative infinity.  The medium
 eigenvector is the eigenvector whose corresponding eigenvalue is
 between the major and minor eigenvalues.

\item  \verb|obj.SetIntegrationEigenvectorToMajor ()| -  Set / get the eigenvector field through which to ingrate. It is
 possible to integrate using the major, medium or minor
 eigenvector field.  The major eigenvector is the eigenvector
 whose corresponding eigenvalue is closest to positive infinity.
 The minor eigenvector is the eigenvector whose corresponding
 eigenvalue is closest to negative infinity.  The medium
 eigenvector is the eigenvector whose corresponding eigenvalue is
 between the major and minor eigenvalues.

\item  \verb|obj.SetIntegrationEigenvectorToMedium ()| -  Set / get the eigenvector field through which to ingrate. It is
 possible to integrate using the major, medium or minor
 eigenvector field.  The major eigenvector is the eigenvector
 whose corresponding eigenvalue is closest to positive infinity.
 The minor eigenvector is the eigenvector whose corresponding
 eigenvalue is closest to negative infinity.  The medium
 eigenvector is the eigenvector whose corresponding eigenvalue is
 between the major and minor eigenvalues.

\item  \verb|obj.SetIntegrationEigenvectorToMinor ()| -  Set / get the eigenvector field through which to ingrate. It is
 possible to integrate using the major, medium or minor
 eigenvector field.  The major eigenvector is the eigenvector
 whose corresponding eigenvalue is closest to positive infinity.
 The minor eigenvector is the eigenvector whose corresponding
 eigenvalue is closest to negative infinity.  The medium
 eigenvector is the eigenvector whose corresponding eigenvalue is
 between the major and minor eigenvalues.

\item  \verb|obj.IntegrateMajorEigenvector ()| -  Use the major eigenvector field as the vector field through which
 to integrate.  The major eigenvector is the eigenvector whose
 corresponding eigenvalue is closest to positive infinity.  

\item  \verb|obj.IntegrateMediumEigenvector ()| -  Use the medium eigenvector field as the vector field through which
 to integrate. The medium eigenvector is the eigenvector whose
 corresponding eigenvalue is between the major and minor
 eigenvalues.

\item  \verb|obj.IntegrateMinorEigenvector ()| -  Use the minor eigenvector field as the vector field through which
 to integrate. The minor eigenvector is the eigenvector whose
 corresponding eigenvalue is closest to negative infinity.

\item  \verb|obj.SetIntegrationStepLength (double )| -  Set / get a nominal integration step size (expressed as a fraction of
 the size of each cell).

\item  \verb|double = obj.GetIntegrationStepLengthMinValue ()| -  Set / get a nominal integration step size (expressed as a fraction of
 the size of each cell).

\item  \verb|double = obj.GetIntegrationStepLengthMaxValue ()| -  Set / get a nominal integration step size (expressed as a fraction of
 the size of each cell).

\item  \verb|double = obj.GetIntegrationStepLength ()| -  Set / get a nominal integration step size (expressed as a fraction of
 the size of each cell).

\item  \verb|obj.SetStepLength (double )| -  Set / get the length of a tube segment composing the
 hyperstreamline. The length is specified as a fraction of the
 diagonal length of the input bounding box.

\item  \verb|double = obj.GetStepLengthMinValue ()| -  Set / get the length of a tube segment composing the
 hyperstreamline. The length is specified as a fraction of the
 diagonal length of the input bounding box.

\item  \verb|double = obj.GetStepLengthMaxValue ()| -  Set / get the length of a tube segment composing the
 hyperstreamline. The length is specified as a fraction of the
 diagonal length of the input bounding box.

\item  \verb|double = obj.GetStepLength ()| -  Set / get the length of a tube segment composing the
 hyperstreamline. The length is specified as a fraction of the
 diagonal length of the input bounding box.

\item  \verb|obj.SetIntegrationDirection (int )| -  Specify the direction in which to integrate the hyperstreamline.

\item  \verb|int = obj.GetIntegrationDirectionMinValue ()| -  Specify the direction in which to integrate the hyperstreamline.

\item  \verb|int = obj.GetIntegrationDirectionMaxValue ()| -  Specify the direction in which to integrate the hyperstreamline.

\item  \verb|int = obj.GetIntegrationDirection ()| -  Specify the direction in which to integrate the hyperstreamline.

\item  \verb|obj.SetIntegrationDirectionToForward ()| -  Specify the direction in which to integrate the hyperstreamline.

\item  \verb|obj.SetIntegrationDirectionToBackward ()| -  Specify the direction in which to integrate the hyperstreamline.

\item  \verb|obj.SetIntegrationDirectionToIntegrateBothDirections ()| -  Specify the direction in which to integrate the hyperstreamline.

\item  \verb|obj.SetTerminalEigenvalue (double )| -  Set/get terminal eigenvalue.  If major eigenvalue falls below this
 value, hyperstreamline terminates propagation.

\item  \verb|double = obj.GetTerminalEigenvalueMinValue ()| -  Set/get terminal eigenvalue.  If major eigenvalue falls below this
 value, hyperstreamline terminates propagation.

\item  \verb|double = obj.GetTerminalEigenvalueMaxValue ()| -  Set/get terminal eigenvalue.  If major eigenvalue falls below this
 value, hyperstreamline terminates propagation.

\item  \verb|double = obj.GetTerminalEigenvalue ()| -  Set/get terminal eigenvalue.  If major eigenvalue falls below this
 value, hyperstreamline terminates propagation.

\item  \verb|obj.SetNumberOfSides (int )| -  Set / get the number of sides for the hyperstreamlines. At a minimum,
 number of sides is 3.

\item  \verb|int = obj.GetNumberOfSidesMinValue ()| -  Set / get the number of sides for the hyperstreamlines. At a minimum,
 number of sides is 3.

\item  \verb|int = obj.GetNumberOfSidesMaxValue ()| -  Set / get the number of sides for the hyperstreamlines. At a minimum,
 number of sides is 3.

\item  \verb|int = obj.GetNumberOfSides ()| -  Set / get the number of sides for the hyperstreamlines. At a minimum,
 number of sides is 3.

\item  \verb|obj.SetRadius (double )| -  Set / get the initial tube radius. This is the maximum ''elliptical''
 radius at the beginning of the tube. Radius varies based on ratio of
 eigenvalues.  Note that tube section is actually elliptical and may
 become a point or line in cross section in some cases.

\item  \verb|double = obj.GetRadiusMinValue ()| -  Set / get the initial tube radius. This is the maximum ''elliptical''
 radius at the beginning of the tube. Radius varies based on ratio of
 eigenvalues.  Note that tube section is actually elliptical and may
 become a point or line in cross section in some cases.

\item  \verb|double = obj.GetRadiusMaxValue ()| -  Set / get the initial tube radius. This is the maximum ''elliptical''
 radius at the beginning of the tube. Radius varies based on ratio of
 eigenvalues.  Note that tube section is actually elliptical and may
 become a point or line in cross section in some cases.

\item  \verb|double = obj.GetRadius ()| -  Set / get the initial tube radius. This is the maximum ''elliptical''
 radius at the beginning of the tube. Radius varies based on ratio of
 eigenvalues.  Note that tube section is actually elliptical and may
 become a point or line in cross section in some cases.

\item  \verb|obj.SetLogScaling (int )| -  Turn on/off logarithmic scaling. If scaling is on, the log base 10
 of the computed eigenvalues are used to scale the cross section radii.

\item  \verb|int = obj.GetLogScaling ()| -  Turn on/off logarithmic scaling. If scaling is on, the log base 10
 of the computed eigenvalues are used to scale the cross section radii.

\item  \verb|obj.LogScalingOn ()| -  Turn on/off logarithmic scaling. If scaling is on, the log base 10
 of the computed eigenvalues are used to scale the cross section radii.

\item  \verb|obj.LogScalingOff ()| -  Turn on/off logarithmic scaling. If scaling is on, the log base 10
 of the computed eigenvalues are used to scale the cross section radii.

\end{itemize}
