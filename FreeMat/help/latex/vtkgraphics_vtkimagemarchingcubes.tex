\section{vtkImageMarchingCubes}

\subsection{Usage}

 vtkImageMarchingCubes is a filter that takes as input images (e.g., 3D
 image region) and generates on output one or more isosurfaces.
 One or more contour values must be specified to generate the isosurfaces.
 Alternatively, you can specify a min/max scalar range and the number of
 contours to generate a series of evenly spaced contour values.
 This filter can stream, so that the entire volume need not be loaded at
 once.  Streaming is controlled using the instance variable
 InputMemoryLimit, which has units KBytes.

To create an instance of class vtkImageMarchingCubes, simply
invoke its constructor as follows
\begin{verbatim}
  obj = vtkImageMarchingCubes
\end{verbatim}
\subsection{Methods}

The class vtkImageMarchingCubes has several methods that can be used.
  They are listed below.
Note that the documentation is translated automatically from the VTK sources,
and may not be completely intelligible.  When in doubt, consult the VTK website.
In the methods listed below, \verb|obj| is an instance of the vtkImageMarchingCubes class.
\begin{itemize}
\item  \verb|string = obj.GetClassName ()|

\item  \verb|int = obj.IsA (string name)|

\item  \verb|vtkImageMarchingCubes = obj.NewInstance ()|

\item  \verb|vtkImageMarchingCubes = obj.SafeDownCast (vtkObject o)|

\item  \verb|obj.SetValue (int i, double value)| -  Methods to set contour values

\item  \verb|double = obj.GetValue (int i)| -  Methods to set contour values

\item  \verb|obj.GetValues (double contourValues)| -  Methods to set contour values

\item  \verb|obj.SetNumberOfContours (int number)| -  Methods to set contour values

\item  \verb|int = obj.GetNumberOfContours ()| -  Methods to set contour values

\item  \verb|obj.GenerateValues (int numContours, double range[2])| -  Methods to set contour values

\item  \verb|obj.GenerateValues (int numContours, double rangeStart, double rangeEnd)| -  Methods to set contour values

\item  \verb|long = obj.GetMTime ()| -  Because we delegate to vtkContourValues \& refer to vtkImplicitFunction

\item  \verb|obj.SetComputeScalars (int )| -  Set/Get the computation of scalars.

\item  \verb|int = obj.GetComputeScalars ()| -  Set/Get the computation of scalars.

\item  \verb|obj.ComputeScalarsOn ()| -  Set/Get the computation of scalars.

\item  \verb|obj.ComputeScalarsOff ()| -  Set/Get the computation of scalars.

\item  \verb|obj.SetComputeNormals (int )| -  Set/Get the computation of normals. Normal computation is fairly expensive
 in both time and storage. If the output data will be processed by filters
 that modify topology or geometry, it may be wise to turn Normals and Gradients off.

\item  \verb|int = obj.GetComputeNormals ()| -  Set/Get the computation of normals. Normal computation is fairly expensive
 in both time and storage. If the output data will be processed by filters
 that modify topology or geometry, it may be wise to turn Normals and Gradients off.

\item  \verb|obj.ComputeNormalsOn ()| -  Set/Get the computation of normals. Normal computation is fairly expensive
 in both time and storage. If the output data will be processed by filters
 that modify topology or geometry, it may be wise to turn Normals and Gradients off.

\item  \verb|obj.ComputeNormalsOff ()| -  Set/Get the computation of normals. Normal computation is fairly expensive
 in both time and storage. If the output data will be processed by filters
 that modify topology or geometry, it may be wise to turn Normals and Gradients off.

\item  \verb|obj.SetComputeGradients (int )| -  Set/Get the computation of gradients. Gradient computation is fairly expensive
 in both time and storage. Note that if ComputeNormals is on, gradients will
 have to be calculated, but will not be stored in the output dataset.
 If the output data will be processed by filters that modify topology or
 geometry, it may be wise to turn Normals and Gradients off.

\item  \verb|int = obj.GetComputeGradients ()| -  Set/Get the computation of gradients. Gradient computation is fairly expensive
 in both time and storage. Note that if ComputeNormals is on, gradients will
 have to be calculated, but will not be stored in the output dataset.
 If the output data will be processed by filters that modify topology or
 geometry, it may be wise to turn Normals and Gradients off.

\item  \verb|obj.ComputeGradientsOn ()| -  Set/Get the computation of gradients. Gradient computation is fairly expensive
 in both time and storage. Note that if ComputeNormals is on, gradients will
 have to be calculated, but will not be stored in the output dataset.
 If the output data will be processed by filters that modify topology or
 geometry, it may be wise to turn Normals and Gradients off.

\item  \verb|obj.ComputeGradientsOff ()| -  Set/Get the computation of gradients. Gradient computation is fairly expensive
 in both time and storage. Note that if ComputeNormals is on, gradients will
 have to be calculated, but will not be stored in the output dataset.
 If the output data will be processed by filters that modify topology or
 geometry, it may be wise to turn Normals and Gradients off.

\item  \verb|int = obj.GetLocatorPoint (int cellX, int cellY, int edge)|

\item  \verb|obj.AddLocatorPoint (int cellX, int cellY, int edge, int ptId)|

\item  \verb|obj.IncrementLocatorZ ()|

\item  \verb|obj.SetInputMemoryLimit (int )| -  The InputMemoryLimit determines the chunk size (the number of slices
 requested at each iteration).  The units of this limit is KiloBytes.
 For now, only the Z axis is split.

\item  \verb|int = obj.GetInputMemoryLimit ()| -  The InputMemoryLimit determines the chunk size (the number of slices
 requested at each iteration).  The units of this limit is KiloBytes.
 For now, only the Z axis is split.

\end{itemize}
