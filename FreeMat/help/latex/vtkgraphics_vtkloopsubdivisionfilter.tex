\section{vtkLoopSubdivisionFilter}

\subsection{Usage}

 vtkLoopSubdivisionFilter is an approximating subdivision scheme that
 creates four new triangles for each triangle in the mesh. The user can
 specify the NumberOfSubdivisions. Loop's subdivision scheme is
 described in: Loop, C., ''Smooth Subdivision surfaces based on
 triangles,'', Masters Thesis, University of Utah, August 1987.
 For a nice summary of the technique see, Hoppe, H., et. al,
 ''Piecewise Smooth Surface Reconstruction,:, Proceedings of Siggraph 94
 (Orlando, Florida, July 24-29, 1994). In COmputer Graphics
 Proceedings, Annual COnference Series, 1994, ACM SIGGRAPH,
 pp. 295-302.
 <P>
 The filter only operates on triangles. Users should use the
 vtkTriangleFilter to triangulate meshes that contain polygons or
 triangle strips.
 <P>
 The filter approximates point data using the same scheme. New
 triangles create at a subdivision step will have the cell data of
 their parent cell.

To create an instance of class vtkLoopSubdivisionFilter, simply
invoke its constructor as follows
\begin{verbatim}
  obj = vtkLoopSubdivisionFilter
\end{verbatim}
\subsection{Methods}

The class vtkLoopSubdivisionFilter has several methods that can be used.
  They are listed below.
Note that the documentation is translated automatically from the VTK sources,
and may not be completely intelligible.  When in doubt, consult the VTK website.
In the methods listed below, \verb|obj| is an instance of the vtkLoopSubdivisionFilter class.
\begin{itemize}
\item  \verb|string = obj.GetClassName ()| -  Construct object with NumberOfSubdivisions set to 1.

\item  \verb|int = obj.IsA (string name)| -  Construct object with NumberOfSubdivisions set to 1.

\item  \verb|vtkLoopSubdivisionFilter = obj.NewInstance ()| -  Construct object with NumberOfSubdivisions set to 1.

\item  \verb|vtkLoopSubdivisionFilter = obj.SafeDownCast (vtkObject o)| -  Construct object with NumberOfSubdivisions set to 1.

\end{itemize}
