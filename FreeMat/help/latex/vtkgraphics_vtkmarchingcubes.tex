\section{vtkMarchingCubes}

\subsection{Usage}

 vtkMarchingCubes is a filter that takes as input a volume (e.g., 3D
 structured point set) and generates on output one or more isosurfaces.
 One or more contour values must be specified to generate the isosurfaces.
 Alternatively, you can specify a min/max scalar range and the number of
 contours to generate a series of evenly spaced contour values.

To create an instance of class vtkMarchingCubes, simply
invoke its constructor as follows
\begin{verbatim}
  obj = vtkMarchingCubes
\end{verbatim}
\subsection{Methods}

The class vtkMarchingCubes has several methods that can be used.
  They are listed below.
Note that the documentation is translated automatically from the VTK sources,
and may not be completely intelligible.  When in doubt, consult the VTK website.
In the methods listed below, \verb|obj| is an instance of the vtkMarchingCubes class.
\begin{itemize}
\item  \verb|string = obj.GetClassName ()|

\item  \verb|int = obj.IsA (string name)|

\item  \verb|vtkMarchingCubes = obj.NewInstance ()|

\item  \verb|vtkMarchingCubes = obj.SafeDownCast (vtkObject o)|

\item  \verb|obj.SetValue (int i, double value)|

\item  \verb|double = obj.GetValue (int i)|

\item  \verb|obj.GetValues (double contourValues)|

\item  \verb|obj.SetNumberOfContours (int number)|

\item  \verb|int = obj.GetNumberOfContours ()|

\item  \verb|obj.GenerateValues (int numContours, double range[2])|

\item  \verb|obj.GenerateValues (int numContours, double rangeStart, double rangeEnd)|

\item  \verb|long = obj.GetMTime ()|

\item  \verb|obj.SetComputeNormals (int )| -  Set/Get the computation of normals. Normal computation is fairly
 expensive in both time and storage. If the output data will be
 processed by filters that modify topology or geometry, it may be
 wise to turn Normals and Gradients off.

\item  \verb|int = obj.GetComputeNormals ()| -  Set/Get the computation of normals. Normal computation is fairly
 expensive in both time and storage. If the output data will be
 processed by filters that modify topology or geometry, it may be
 wise to turn Normals and Gradients off.

\item  \verb|obj.ComputeNormalsOn ()| -  Set/Get the computation of normals. Normal computation is fairly
 expensive in both time and storage. If the output data will be
 processed by filters that modify topology or geometry, it may be
 wise to turn Normals and Gradients off.

\item  \verb|obj.ComputeNormalsOff ()| -  Set/Get the computation of normals. Normal computation is fairly
 expensive in both time and storage. If the output data will be
 processed by filters that modify topology or geometry, it may be
 wise to turn Normals and Gradients off.

\item  \verb|obj.SetComputeGradients (int )| -  Set/Get the computation of gradients. Gradient computation is
 fairly expensive in both time and storage. Note that if
 ComputeNormals is on, gradients will have to be calculated, but
 will not be stored in the output dataset.  If the output data
 will be processed by filters that modify topology or geometry, it
 may be wise to turn Normals and Gradients off.

\item  \verb|int = obj.GetComputeGradients ()| -  Set/Get the computation of gradients. Gradient computation is
 fairly expensive in both time and storage. Note that if
 ComputeNormals is on, gradients will have to be calculated, but
 will not be stored in the output dataset.  If the output data
 will be processed by filters that modify topology or geometry, it
 may be wise to turn Normals and Gradients off.

\item  \verb|obj.ComputeGradientsOn ()| -  Set/Get the computation of gradients. Gradient computation is
 fairly expensive in both time and storage. Note that if
 ComputeNormals is on, gradients will have to be calculated, but
 will not be stored in the output dataset.  If the output data
 will be processed by filters that modify topology or geometry, it
 may be wise to turn Normals and Gradients off.

\item  \verb|obj.ComputeGradientsOff ()| -  Set/Get the computation of gradients. Gradient computation is
 fairly expensive in both time and storage. Note that if
 ComputeNormals is on, gradients will have to be calculated, but
 will not be stored in the output dataset.  If the output data
 will be processed by filters that modify topology or geometry, it
 may be wise to turn Normals and Gradients off.

\item  \verb|obj.SetComputeScalars (int )| -  Set/Get the computation of scalars.

\item  \verb|int = obj.GetComputeScalars ()| -  Set/Get the computation of scalars.

\item  \verb|obj.ComputeScalarsOn ()| -  Set/Get the computation of scalars.

\item  \verb|obj.ComputeScalarsOff ()| -  Set/Get the computation of scalars.

\item  \verb|obj.SetLocator (vtkIncrementalPointLocator locator)| -  Overide the default locator.  Useful for changing the number of
 bins for performance or specifying a more aggressive locator.

\item  \verb|vtkIncrementalPointLocator = obj.GetLocator ()| -  Overide the default locator.  Useful for changing the number of
 bins for performance or specifying a more aggressive locator.

\item  \verb|obj.CreateDefaultLocator ()| -  Create default locator. Used to create one when none is
 specified. The locator is used to merge coincident points.

\end{itemize}
