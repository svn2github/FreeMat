\section{vtkMaskPoints}

\subsection{Usage}

 vtkMaskPoints is a filter that passes through points and point attributes 
 from input dataset. (Other geometry is not passed through.) It is 
 possible to mask every nth point, and to specify an initial offset
 to begin masking from. A special random mode feature enables random 
 selection of points. The filter can also generate vertices (topological
 primitives) as well as points. This is useful because vertices are
 rendered while points are not.

To create an instance of class vtkMaskPoints, simply
invoke its constructor as follows
\begin{verbatim}
  obj = vtkMaskPoints
\end{verbatim}
\subsection{Methods}

The class vtkMaskPoints has several methods that can be used.
  They are listed below.
Note that the documentation is translated automatically from the VTK sources,
and may not be completely intelligible.  When in doubt, consult the VTK website.
In the methods listed below, \verb|obj| is an instance of the vtkMaskPoints class.
\begin{itemize}
\item  \verb|string = obj.GetClassName ()|

\item  \verb|int = obj.IsA (string name)|

\item  \verb|vtkMaskPoints = obj.NewInstance ()|

\item  \verb|vtkMaskPoints = obj.SafeDownCast (vtkObject o)|

\item  \verb|obj.SetOnRatio (int )| -  Turn on every nth point.

\item  \verb|int = obj.GetOnRatioMinValue ()| -  Turn on every nth point.

\item  \verb|int = obj.GetOnRatioMaxValue ()| -  Turn on every nth point.

\item  \verb|int = obj.GetOnRatio ()| -  Turn on every nth point.

\item  \verb|obj.SetMaximumNumberOfPoints (vtkIdType )| -  Limit the number of points that can be passed through.

\item  \verb|vtkIdType = obj.GetMaximumNumberOfPointsMinValue ()| -  Limit the number of points that can be passed through.

\item  \verb|vtkIdType = obj.GetMaximumNumberOfPointsMaxValue ()| -  Limit the number of points that can be passed through.

\item  \verb|vtkIdType = obj.GetMaximumNumberOfPoints ()| -  Limit the number of points that can be passed through.

\item  \verb|obj.SetOffset (vtkIdType )| -  Start with this point.

\item  \verb|vtkIdType = obj.GetOffsetMinValue ()| -  Start with this point.

\item  \verb|vtkIdType = obj.GetOffsetMaxValue ()| -  Start with this point.

\item  \verb|vtkIdType = obj.GetOffset ()| -  Start with this point.

\item  \verb|obj.SetRandomMode (int )| -  Special flag causes randomization of point selection. If this mode is on,
 statistically every nth point (i.e., OnRatio) will be displayed.

\item  \verb|int = obj.GetRandomMode ()| -  Special flag causes randomization of point selection. If this mode is on,
 statistically every nth point (i.e., OnRatio) will be displayed.

\item  \verb|obj.RandomModeOn ()| -  Special flag causes randomization of point selection. If this mode is on,
 statistically every nth point (i.e., OnRatio) will be displayed.

\item  \verb|obj.RandomModeOff ()| -  Special flag causes randomization of point selection. If this mode is on,
 statistically every nth point (i.e., OnRatio) will be displayed.

\item  \verb|obj.SetGenerateVertices (int )| -  Generate output polydata vertices as well as points. A useful
 convenience method because vertices are drawn (they are topology) while
 points are not (they are geometry). By default this method is off.

\item  \verb|int = obj.GetGenerateVertices ()| -  Generate output polydata vertices as well as points. A useful
 convenience method because vertices are drawn (they are topology) while
 points are not (they are geometry). By default this method is off.

\item  \verb|obj.GenerateVerticesOn ()| -  Generate output polydata vertices as well as points. A useful
 convenience method because vertices are drawn (they are topology) while
 points are not (they are geometry). By default this method is off.

\item  \verb|obj.GenerateVerticesOff ()| -  Generate output polydata vertices as well as points. A useful
 convenience method because vertices are drawn (they are topology) while
 points are not (they are geometry). By default this method is off.

\item  \verb|obj.SetSingleVertexPerCell (int )| -  When vertex generation is enabled, by default vertices are produced
 as multi-vertex cells (more than one per cell), if you wish to have a single
 vertex per cell, enable this flag.

\item  \verb|int = obj.GetSingleVertexPerCell ()| -  When vertex generation is enabled, by default vertices are produced
 as multi-vertex cells (more than one per cell), if you wish to have a single
 vertex per cell, enable this flag.

\item  \verb|obj.SingleVertexPerCellOn ()| -  When vertex generation is enabled, by default vertices are produced
 as multi-vertex cells (more than one per cell), if you wish to have a single
 vertex per cell, enable this flag.

\item  \verb|obj.SingleVertexPerCellOff ()| -  When vertex generation is enabled, by default vertices are produced
 as multi-vertex cells (more than one per cell), if you wish to have a single
 vertex per cell, enable this flag.

\end{itemize}
