\section{vtkMergeDataObjectFilter}

\subsection{Usage}

 vtkMergeDataObjectFilter is a filter that merges the field from a
 vtkDataObject with a vtkDataSet. The resulting combined dataset can
 then be processed by other filters (e.g.,
 vtkFieldDataToAttributeDataFilter) to create attribute data like
 scalars, vectors, etc.

 The filter operates as follows. The field data from the
 vtkDataObject is merged with the input's vtkDataSet and then placed
 in the output. You can choose to place the field data into the cell
 data field, the point data field, or the datasets field (i.e., the
 one inherited from vtkDataSet's superclass vtkDataObject). All this
 data shuffling occurs via reference counting, therefore memory is
 not copied.

 One of the uses of this filter is to allow you to read/generate the
 structure of a dataset independent of the attributes. So, for
 example, you could store the dataset geometry/topology in one file,
 and field data in another. Then use this filter in combination with
 vtkFieldDataToAttributeData to create a dataset ready for
 processing in the visualization pipeline.

To create an instance of class vtkMergeDataObjectFilter, simply
invoke its constructor as follows
\begin{verbatim}
  obj = vtkMergeDataObjectFilter
\end{verbatim}
\subsection{Methods}

The class vtkMergeDataObjectFilter has several methods that can be used.
  They are listed below.
Note that the documentation is translated automatically from the VTK sources,
and may not be completely intelligible.  When in doubt, consult the VTK website.
In the methods listed below, \verb|obj| is an instance of the vtkMergeDataObjectFilter class.
\begin{itemize}
\item  \verb|string = obj.GetClassName ()|

\item  \verb|int = obj.IsA (string name)|

\item  \verb|vtkMergeDataObjectFilter = obj.NewInstance ()|

\item  \verb|vtkMergeDataObjectFilter = obj.SafeDownCast (vtkObject o)|

\item  \verb|obj.SetDataObject (vtkDataObject object)| -  Specify the data object to merge with the input dataset.

\item  \verb|vtkDataObject = obj.GetDataObject ()| -  Specify the data object to merge with the input dataset.

\item  \verb|obj.SetOutputField (int )| -  Specify where to place the field data during the merge process.  There
 are three choices: the field data associated with the vtkDataObject
 superclass; the point field attribute data; and the cell field attribute
 data.

\item  \verb|int = obj.GetOutputField ()| -  Specify where to place the field data during the merge process.  There
 are three choices: the field data associated with the vtkDataObject
 superclass; the point field attribute data; and the cell field attribute
 data.

\item  \verb|obj.SetOutputFieldToDataObjectField ()| -  Specify where to place the field data during the merge process.  There
 are three choices: the field data associated with the vtkDataObject
 superclass; the point field attribute data; and the cell field attribute
 data.

\item  \verb|obj.SetOutputFieldToPointDataField ()| -  Specify where to place the field data during the merge process.  There
 are three choices: the field data associated with the vtkDataObject
 superclass; the point field attribute data; and the cell field attribute
 data.

\item  \verb|obj.SetOutputFieldToCellDataField ()| -  Specify where to place the field data during the merge process.  There
 are three choices: the field data associated with the vtkDataObject
 superclass; the point field attribute data; and the cell field attribute
 data.

\end{itemize}
