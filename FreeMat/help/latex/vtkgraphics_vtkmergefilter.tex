\section{vtkMergeFilter}

\subsection{Usage}

 vtkMergeFilter is a filter that extracts separate components of data from
 different datasets and merges them into a single dataset. The output from
 this filter is of the same type as the input (i.e., vtkDataSet.) It treats 
 both cell and point data set attributes.

To create an instance of class vtkMergeFilter, simply
invoke its constructor as follows
\begin{verbatim}
  obj = vtkMergeFilter
\end{verbatim}
\subsection{Methods}

The class vtkMergeFilter has several methods that can be used.
  They are listed below.
Note that the documentation is translated automatically from the VTK sources,
and may not be completely intelligible.  When in doubt, consult the VTK website.
In the methods listed below, \verb|obj| is an instance of the vtkMergeFilter class.
\begin{itemize}
\item  \verb|string = obj.GetClassName ()|

\item  \verb|int = obj.IsA (string name)|

\item  \verb|vtkMergeFilter = obj.NewInstance ()|

\item  \verb|vtkMergeFilter = obj.SafeDownCast (vtkObject o)|

\item  \verb|obj.SetGeometry (vtkDataSet input)| -  Specify object from which to extract geometry information.
 Old style. Use SetGeometryConnection() instead.

\item  \verb|vtkDataSet = obj.GetGeometry ()| -  Specify object from which to extract geometry information.
 Old style. Use SetGeometryConnection() instead.

\item  \verb|obj.SetGeometryConnection (vtkAlgorithmOutput algOutput)| -  Specify object from which to extract scalar information.
 Old style. Use SetScalarsConnection() instead.

\item  \verb|obj.SetScalars (vtkDataSet )| -  Specify object from which to extract scalar information.
 Old style. Use SetScalarsConnection() instead.

\item  \verb|vtkDataSet = obj.GetScalars ()| -  Specify object from which to extract scalar information.
 Old style. Use SetScalarsConnection() instead.

\item  \verb|obj.SetScalarsConnection (vtkAlgorithmOutput algOutput)| -  Set / get the object from which to extract vector information.
 Old style. Use SetVectorsConnection() instead.

\item  \verb|obj.SetVectors (vtkDataSet )| -  Set / get the object from which to extract vector information.
 Old style. Use SetVectorsConnection() instead.

\item  \verb|vtkDataSet = obj.GetVectors ()| -  Set / get the object from which to extract vector information.
 Old style. Use SetVectorsConnection() instead.

\item  \verb|obj.SetVectorsConnection (vtkAlgorithmOutput algOutput)| -  Set / get the object from which to extract normal information.
 Old style. Use SetNormalsConnection() instead.

\item  \verb|obj.SetNormals (vtkDataSet )| -  Set / get the object from which to extract normal information.
 Old style. Use SetNormalsConnection() instead.

\item  \verb|vtkDataSet = obj.GetNormals ()| -  Set / get the object from which to extract normal information.
 Old style. Use SetNormalsConnection() instead.

\item  \verb|obj.SetNormalsConnection (vtkAlgorithmOutput algOutput)| -  Set / get the object from which to extract texture coordinates
 information.
 Old style. Use SetTCoordsConnection() instead.

\item  \verb|obj.SetTCoords (vtkDataSet )| -  Set / get the object from which to extract texture coordinates
 information.
 Old style. Use SetTCoordsConnection() instead.

\item  \verb|vtkDataSet = obj.GetTCoords ()| -  Set / get the object from which to extract texture coordinates
 information.
 Old style. Use SetTCoordsConnection() instead.

\item  \verb|obj.SetTCoordsConnection (vtkAlgorithmOutput algOutput)| -  Set / get the object from which to extract tensor data.
 Old style. Use SetTensorsConnection() instead.

\item  \verb|obj.SetTensors (vtkDataSet )| -  Set / get the object from which to extract tensor data.
 Old style. Use SetTensorsConnection() instead.

\item  \verb|vtkDataSet = obj.GetTensors ()| -  Set / get the object from which to extract tensor data.
 Old style. Use SetTensorsConnection() instead.

\item  \verb|obj.SetTensorsConnection (vtkAlgorithmOutput algOutput)| -  Set the object from which to extract a field and the name
 of the field. Note that this does not create pipeline
 connectivity.

\item  \verb|obj.AddField (string name, vtkDataSet input)| -  Set the object from which to extract a field and the name
 of the field. Note that this does not create pipeline
 connectivity.

\end{itemize}
