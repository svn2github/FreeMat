\section{vtkPolyDataNormals}

\subsection{Usage}

 vtkPolyDataNormals is a filter that computes point normals for a polygonal 
 mesh. The filter can reorder polygons to insure consistent orientation
 across polygon neighbors. Sharp edges can be split and points duplicated
 with separate normals to give crisp (rendered) surface definition. It is
 also possible to globally flip the normal orientation.

 The algorithm works by determining normals for each polygon and then
 averaging them at shared points. When sharp edges are present, the edges
 are split and new points generated to prevent blurry edges (due to 
 Gouraud shading).

To create an instance of class vtkPolyDataNormals, simply
invoke its constructor as follows
\begin{verbatim}
  obj = vtkPolyDataNormals
\end{verbatim}
\subsection{Methods}

The class vtkPolyDataNormals has several methods that can be used.
  They are listed below.
Note that the documentation is translated automatically from the VTK sources,
and may not be completely intelligible.  When in doubt, consult the VTK website.
In the methods listed below, \verb|obj| is an instance of the vtkPolyDataNormals class.
\begin{itemize}
\item  \verb|string = obj.GetClassName ()|

\item  \verb|int = obj.IsA (string name)|

\item  \verb|vtkPolyDataNormals = obj.NewInstance ()|

\item  \verb|vtkPolyDataNormals = obj.SafeDownCast (vtkObject o)|

\item  \verb|obj.SetFeatureAngle (double )| -  Specify the angle that defines a sharp edge. If the difference in
 angle across neighboring polygons is greater than this value, the
 shared edge is considered ''sharp''.

\item  \verb|double = obj.GetFeatureAngleMinValue ()| -  Specify the angle that defines a sharp edge. If the difference in
 angle across neighboring polygons is greater than this value, the
 shared edge is considered ''sharp''.

\item  \verb|double = obj.GetFeatureAngleMaxValue ()| -  Specify the angle that defines a sharp edge. If the difference in
 angle across neighboring polygons is greater than this value, the
 shared edge is considered ''sharp''.

\item  \verb|double = obj.GetFeatureAngle ()| -  Specify the angle that defines a sharp edge. If the difference in
 angle across neighboring polygons is greater than this value, the
 shared edge is considered ''sharp''.

\item  \verb|obj.SetSplitting (int )| -  Turn on/off the splitting of sharp edges.

\item  \verb|int = obj.GetSplitting ()| -  Turn on/off the splitting of sharp edges.

\item  \verb|obj.SplittingOn ()| -  Turn on/off the splitting of sharp edges.

\item  \verb|obj.SplittingOff ()| -  Turn on/off the splitting of sharp edges.

\item  \verb|obj.SetConsistency (int )| -  Turn on/off the enforcement of consistent polygon ordering.

\item  \verb|int = obj.GetConsistency ()| -  Turn on/off the enforcement of consistent polygon ordering.

\item  \verb|obj.ConsistencyOn ()| -  Turn on/off the enforcement of consistent polygon ordering.

\item  \verb|obj.ConsistencyOff ()| -  Turn on/off the enforcement of consistent polygon ordering.

\item  \verb|obj.SetAutoOrientNormals (int )| -  Turn on/off the automatic determination of correct normal
 orientation. NOTE: This assumes a completely closed surface
 (i.e. no boundary edges) and no non-manifold edges. If these
 constraints do not hold, all bets are off. This option adds some
 computational complexity, and is useful if you don't want to have
 to inspect the rendered image to determine whether to turn on the 
 FlipNormals flag. However, this flag can work with the FlipNormals 
 flag, and if both are set, all the normals in the output will 
 point ''inward''.

\item  \verb|int = obj.GetAutoOrientNormals ()| -  Turn on/off the automatic determination of correct normal
 orientation. NOTE: This assumes a completely closed surface
 (i.e. no boundary edges) and no non-manifold edges. If these
 constraints do not hold, all bets are off. This option adds some
 computational complexity, and is useful if you don't want to have
 to inspect the rendered image to determine whether to turn on the 
 FlipNormals flag. However, this flag can work with the FlipNormals 
 flag, and if both are set, all the normals in the output will 
 point ''inward''.

\item  \verb|obj.AutoOrientNormalsOn ()| -  Turn on/off the automatic determination of correct normal
 orientation. NOTE: This assumes a completely closed surface
 (i.e. no boundary edges) and no non-manifold edges. If these
 constraints do not hold, all bets are off. This option adds some
 computational complexity, and is useful if you don't want to have
 to inspect the rendered image to determine whether to turn on the 
 FlipNormals flag. However, this flag can work with the FlipNormals 
 flag, and if both are set, all the normals in the output will 
 point ''inward''.

\item  \verb|obj.AutoOrientNormalsOff ()| -  Turn on/off the automatic determination of correct normal
 orientation. NOTE: This assumes a completely closed surface
 (i.e. no boundary edges) and no non-manifold edges. If these
 constraints do not hold, all bets are off. This option adds some
 computational complexity, and is useful if you don't want to have
 to inspect the rendered image to determine whether to turn on the 
 FlipNormals flag. However, this flag can work with the FlipNormals 
 flag, and if both are set, all the normals in the output will 
 point ''inward''.

\item  \verb|obj.SetComputePointNormals (int )| -  Turn on/off the computation of point normals.

\item  \verb|int = obj.GetComputePointNormals ()| -  Turn on/off the computation of point normals.

\item  \verb|obj.ComputePointNormalsOn ()| -  Turn on/off the computation of point normals.

\item  \verb|obj.ComputePointNormalsOff ()| -  Turn on/off the computation of point normals.

\item  \verb|obj.SetComputeCellNormals (int )| -  Turn on/off the computation of cell normals.

\item  \verb|int = obj.GetComputeCellNormals ()| -  Turn on/off the computation of cell normals.

\item  \verb|obj.ComputeCellNormalsOn ()| -  Turn on/off the computation of cell normals.

\item  \verb|obj.ComputeCellNormalsOff ()| -  Turn on/off the computation of cell normals.

\item  \verb|obj.SetFlipNormals (int )| -  Turn on/off the global flipping of normal orientation. Flipping
 reverves the meaning of front and back for Frontface and Backface
 culling in vtkProperty.  Flipping modifies both the normal
 direction and the order of a cell's points.

\item  \verb|int = obj.GetFlipNormals ()| -  Turn on/off the global flipping of normal orientation. Flipping
 reverves the meaning of front and back for Frontface and Backface
 culling in vtkProperty.  Flipping modifies both the normal
 direction and the order of a cell's points.

\item  \verb|obj.FlipNormalsOn ()| -  Turn on/off the global flipping of normal orientation. Flipping
 reverves the meaning of front and back for Frontface and Backface
 culling in vtkProperty.  Flipping modifies both the normal
 direction and the order of a cell's points.

\item  \verb|obj.FlipNormalsOff ()| -  Turn on/off the global flipping of normal orientation. Flipping
 reverves the meaning of front and back for Frontface and Backface
 culling in vtkProperty.  Flipping modifies both the normal
 direction and the order of a cell's points.

\item  \verb|obj.SetNonManifoldTraversal (int )| -  Turn on/off traversal across non-manifold edges. This will prevent
 problems where the consistency of polygonal ordering is corrupted due
 to topological loops.

\item  \verb|int = obj.GetNonManifoldTraversal ()| -  Turn on/off traversal across non-manifold edges. This will prevent
 problems where the consistency of polygonal ordering is corrupted due
 to topological loops.

\item  \verb|obj.NonManifoldTraversalOn ()| -  Turn on/off traversal across non-manifold edges. This will prevent
 problems where the consistency of polygonal ordering is corrupted due
 to topological loops.

\item  \verb|obj.NonManifoldTraversalOff ()| -  Turn on/off traversal across non-manifold edges. This will prevent
 problems where the consistency of polygonal ordering is corrupted due
 to topological loops.

\end{itemize}
