\section{vtkPolyDataStreamer}

\subsection{Usage}

 vtkPolyDataStreamer initiates streaming by requesting pieces from its 
 single input it appends these pieces it to the requested output.
 Note that since vtkPolyDataStreamer uses an append filter, all the
 polygons generated have to be kept in memory before rendering. If
 these do not fit in the memory, it is possible to make the vtkPolyDataMapper
 stream. Since the mapper will render each piece separately, all the
 polygons do not have to stored in memory.
 .SECTION Note
 The output may be slightly different if the pipeline does not handle 
 ghost cells properly (i.e. you might see seames between the pieces).

To create an instance of class vtkPolyDataStreamer, simply
invoke its constructor as follows
\begin{verbatim}
  obj = vtkPolyDataStreamer
\end{verbatim}
\subsection{Methods}

The class vtkPolyDataStreamer has several methods that can be used.
  They are listed below.
Note that the documentation is translated automatically from the VTK sources,
and may not be completely intelligible.  When in doubt, consult the VTK website.
In the methods listed below, \verb|obj| is an instance of the vtkPolyDataStreamer class.
\begin{itemize}
\item  \verb|string = obj.GetClassName ()|

\item  \verb|int = obj.IsA (string name)|

\item  \verb|vtkPolyDataStreamer = obj.NewInstance ()|

\item  \verb|vtkPolyDataStreamer = obj.SafeDownCast (vtkObject o)|

\item  \verb|obj.SetNumberOfStreamDivisions (int num)| -  Set the number of pieces to divide the problem into.

\item  \verb|int = obj.GetNumberOfStreamDivisions ()| -  Set the number of pieces to divide the problem into.

\item  \verb|obj.SetColorByPiece (int )| -  By default, this option is off.  When it is on, cell scalars are generated
 based on which piece they are in.

\item  \verb|int = obj.GetColorByPiece ()| -  By default, this option is off.  When it is on, cell scalars are generated
 based on which piece they are in.

\item  \verb|obj.ColorByPieceOn ()| -  By default, this option is off.  When it is on, cell scalars are generated
 based on which piece they are in.

\item  \verb|obj.ColorByPieceOff ()| -  By default, this option is off.  When it is on, cell scalars are generated
 based on which piece they are in.

\end{itemize}
