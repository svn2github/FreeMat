\section{vtkRecursiveDividingCubes}

\subsection{Usage}

 vtkRecursiveDividingCubes is a filter that generates points lying on a 
 surface of constant scalar value (i.e., an isosurface). Dense point 
 clouds (i.e., at screen resolution) will appear as a surface. Less dense 
 clouds can be used as a source to generate streamlines or to generate 
 ''transparent'' surfaces.

 This implementation differs from vtkDividingCubes in that it uses a 
 recursive procedure. In many cases this can result in generating 
 more points than the procedural implementation of vtkDividingCubes. This is
 because the recursive procedure divides voxels by multiples of powers of 
 two. This can over-constrain subdivision. One of the advantages of the 
 recursive technique is that the recursion is terminated earlier, which in
 some cases can be more efficient.

To create an instance of class vtkRecursiveDividingCubes, simply
invoke its constructor as follows
\begin{verbatim}
  obj = vtkRecursiveDividingCubes
\end{verbatim}
\subsection{Methods}

The class vtkRecursiveDividingCubes has several methods that can be used.
  They are listed below.
Note that the documentation is translated automatically from the VTK sources,
and may not be completely intelligible.  When in doubt, consult the VTK website.
In the methods listed below, \verb|obj| is an instance of the vtkRecursiveDividingCubes class.
\begin{itemize}
\item  \verb|string = obj.GetClassName ()|

\item  \verb|int = obj.IsA (string name)|

\item  \verb|vtkRecursiveDividingCubes = obj.NewInstance ()|

\item  \verb|vtkRecursiveDividingCubes = obj.SafeDownCast (vtkObject o)|

\item  \verb|obj.SetValue (double )| -  Set isosurface value.

\item  \verb|double = obj.GetValue ()| -  Set isosurface value.

\item  \verb|obj.SetDistance (double )| -  Specify sub-voxel size at which to generate point.

\item  \verb|double = obj.GetDistanceMinValue ()| -  Specify sub-voxel size at which to generate point.

\item  \verb|double = obj.GetDistanceMaxValue ()| -  Specify sub-voxel size at which to generate point.

\item  \verb|double = obj.GetDistance ()| -  Specify sub-voxel size at which to generate point.

\item  \verb|obj.SetIncrement (int )| -  Every ''Increment'' point is added to the list of points. This parameter, if
 set to a large value, can be used to limit the number of points while
 retaining good accuracy.

\item  \verb|int = obj.GetIncrementMinValue ()| -  Every ''Increment'' point is added to the list of points. This parameter, if
 set to a large value, can be used to limit the number of points while
 retaining good accuracy.

\item  \verb|int = obj.GetIncrementMaxValue ()| -  Every ''Increment'' point is added to the list of points. This parameter, if
 set to a large value, can be used to limit the number of points while
 retaining good accuracy.

\item  \verb|int = obj.GetIncrement ()| -  Every ''Increment'' point is added to the list of points. This parameter, if
 set to a large value, can be used to limit the number of points while
 retaining good accuracy.

\end{itemize}
