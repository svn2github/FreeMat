\section{vtkRotationalExtrusionFilter}

\subsection{Usage}

 vtkRotationalExtrusionFilter is a modeling filter. It takes polygonal 
 data as input and generates polygonal data on output. The input dataset 
 is swept around the z-axis to create new polygonal primitives. These 
 primitives form a ''skirt'' or swept surface. For example, sweeping a
 line results in a cylindrical shell, and sweeping a circle creates a 
 torus.

 There are a number of control parameters for this filter. You can 
 control whether the sweep of a 2D object (i.e., polygon or triangle 
 strip) is capped with the generating geometry via the ''Capping'' instance
 variable. Also, you can control the angle of rotation, and whether 
 translation along the z-axis is performed along with the rotation.
 (Translation is useful for creating ''springs''.) You also can adjust 
 the radius of the generating geometry using the ''DeltaRotation'' instance 
 variable.

 The skirt is generated by locating certain topological features. Free 
 edges (edges of polygons or triangle strips only used by one polygon or
 triangle strips) generate surfaces. This is true also of lines or 
 polylines. Vertices generate lines.

 This filter can be used to model axisymmetric objects like cylinders,
 bottles, and wine glasses; or translational/rotational symmetric objects
 like springs or corkscrews.

To create an instance of class vtkRotationalExtrusionFilter, simply
invoke its constructor as follows
\begin{verbatim}
  obj = vtkRotationalExtrusionFilter
\end{verbatim}
\subsection{Methods}

The class vtkRotationalExtrusionFilter has several methods that can be used.
  They are listed below.
Note that the documentation is translated automatically from the VTK sources,
and may not be completely intelligible.  When in doubt, consult the VTK website.
In the methods listed below, \verb|obj| is an instance of the vtkRotationalExtrusionFilter class.
\begin{itemize}
\item  \verb|string = obj.GetClassName ()|

\item  \verb|int = obj.IsA (string name)|

\item  \verb|vtkRotationalExtrusionFilter = obj.NewInstance ()|

\item  \verb|vtkRotationalExtrusionFilter = obj.SafeDownCast (vtkObject o)|

\item  \verb|obj.SetResolution (int )| -  Set/Get resolution of sweep operation. Resolution controls the number
 of intermediate node points.

\item  \verb|int = obj.GetResolutionMinValue ()| -  Set/Get resolution of sweep operation. Resolution controls the number
 of intermediate node points.

\item  \verb|int = obj.GetResolutionMaxValue ()| -  Set/Get resolution of sweep operation. Resolution controls the number
 of intermediate node points.

\item  \verb|int = obj.GetResolution ()| -  Set/Get resolution of sweep operation. Resolution controls the number
 of intermediate node points.

\item  \verb|obj.SetCapping (int )| -  Turn on/off the capping of the skirt.

\item  \verb|int = obj.GetCapping ()| -  Turn on/off the capping of the skirt.

\item  \verb|obj.CappingOn ()| -  Turn on/off the capping of the skirt.

\item  \verb|obj.CappingOff ()| -  Turn on/off the capping of the skirt.

\item  \verb|obj.SetAngle (double )| -  Set/Get angle of rotation.

\item  \verb|double = obj.GetAngle ()| -  Set/Get angle of rotation.

\item  \verb|obj.SetTranslation (double )| -  Set/Get total amount of translation along the z-axis.

\item  \verb|double = obj.GetTranslation ()| -  Set/Get total amount of translation along the z-axis.

\item  \verb|obj.SetDeltaRadius (double )| -  Set/Get change in radius during sweep process.

\item  \verb|double = obj.GetDeltaRadius ()| -  Set/Get change in radius during sweep process.

\end{itemize}
