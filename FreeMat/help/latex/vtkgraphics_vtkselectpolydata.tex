\section{vtkSelectPolyData}

\subsection{Usage}

 vtkSelectPolyData is a filter that selects polygonal data based on
 defining a ''loop'' and indicating the region inside of the loop. The
 mesh within the loop consists of complete cells (the cells are not
 cut). Alternatively, this filter can be used to generate scalars.
 These scalar values, which are a distance measure to the loop, can
 be used to clip, contour. or extract data (i.e., anything that an
 implicit function can do). 

 The loop is defined by an array of x-y-z point coordinates.
 (Coordinates should be in the same coordinate space as the input
 polygonal data.) The loop can be concave and non-planar, but not
 self-intersecting. The input to the filter is a polygonal mesh
 (only surface primitives such as triangle strips and polygons); the
 output is either a) a portion of the original mesh laying within
 the selection loop (GenerateSelectionScalarsOff); or b) the same
 polygonal mesh with the addition of scalar values
 (GenerateSelectionScalarsOn).

 The algorithm works as follows. For each point coordinate in the
 loop, the closest point in the mesh is found. The result is a loop
 of closest point ids from the mesh. Then, the edges in the mesh
 connecting the closest points (and laying along the lines forming
 the loop) are found. A greedy edge tracking procedure is used as
 follows. At the current point, the mesh edge oriented in the
 direction of and whose end point is closest to the line is
 chosen. The edge is followed to the new end point, and the
 procedure is repeated. This process continues until the entire loop
 has been created. 
 
 To determine what portion of the mesh is inside and outside of the
 loop, three options are possible. 1) the smallest connected region,
 2) the largest connected region, and 3) the connected region
 closest to a user specified point. (Set the ivar SelectionMode.)
 
 Once the loop is computed as above, the GenerateSelectionScalars
 controls the output of the filter. If on, then scalar values are
 generated based on distance to the loop lines. Otherwise, the cells
 laying inside the selection loop are output. By default, the mesh
 lying within the loop is output; however, if InsideOut is on, then
 the portion of the mesh lying outside of the loop is output.

 The filter can be configured to generate the unselected portions of
 the mesh as output by setting GenerateUnselectedOutput. Use the
 method GetUnselectedOutput to access this output. (Note: this flag
 is pertinent only when GenerateSelectionScalars is off.)

To create an instance of class vtkSelectPolyData, simply
invoke its constructor as follows
\begin{verbatim}
  obj = vtkSelectPolyData
\end{verbatim}
\subsection{Methods}

The class vtkSelectPolyData has several methods that can be used.
  They are listed below.
Note that the documentation is translated automatically from the VTK sources,
and may not be completely intelligible.  When in doubt, consult the VTK website.
In the methods listed below, \verb|obj| is an instance of the vtkSelectPolyData class.
\begin{itemize}
\item  \verb|string = obj.GetClassName ()|

\item  \verb|int = obj.IsA (string name)|

\item  \verb|vtkSelectPolyData = obj.NewInstance ()|

\item  \verb|vtkSelectPolyData = obj.SafeDownCast (vtkObject o)|

\item  \verb|obj.SetGenerateSelectionScalars (int )| -  Set/Get the flag to control behavior of the filter. If
 GenerateSelectionScalars is on, then the output of the filter
 is the same as the input, except that scalars are generated.
 If off, the filter outputs the cells laying inside the loop, and
 does not generate scalars.

\item  \verb|int = obj.GetGenerateSelectionScalars ()| -  Set/Get the flag to control behavior of the filter. If
 GenerateSelectionScalars is on, then the output of the filter
 is the same as the input, except that scalars are generated.
 If off, the filter outputs the cells laying inside the loop, and
 does not generate scalars.

\item  \verb|obj.GenerateSelectionScalarsOn ()| -  Set/Get the flag to control behavior of the filter. If
 GenerateSelectionScalars is on, then the output of the filter
 is the same as the input, except that scalars are generated.
 If off, the filter outputs the cells laying inside the loop, and
 does not generate scalars.

\item  \verb|obj.GenerateSelectionScalarsOff ()| -  Set/Get the flag to control behavior of the filter. If
 GenerateSelectionScalars is on, then the output of the filter
 is the same as the input, except that scalars are generated.
 If off, the filter outputs the cells laying inside the loop, and
 does not generate scalars.

\item  \verb|obj.SetInsideOut (int )| -  Set/Get the InsideOut flag. When off, the mesh within the loop is
 extracted. When on, the mesh outside the loop is extracted.

\item  \verb|int = obj.GetInsideOut ()| -  Set/Get the InsideOut flag. When off, the mesh within the loop is
 extracted. When on, the mesh outside the loop is extracted.

\item  \verb|obj.InsideOutOn ()| -  Set/Get the InsideOut flag. When off, the mesh within the loop is
 extracted. When on, the mesh outside the loop is extracted.

\item  \verb|obj.InsideOutOff ()| -  Set/Get the InsideOut flag. When off, the mesh within the loop is
 extracted. When on, the mesh outside the loop is extracted.

\item  \verb|obj.SetLoop (vtkPoints )| -  Set/Get the array of point coordinates defining the loop. There must
 be at least three points used to define a loop.

\item  \verb|vtkPoints = obj.GetLoop ()| -  Set/Get the array of point coordinates defining the loop. There must
 be at least three points used to define a loop.

\item  \verb|obj.SetSelectionMode (int )| -  Control how inside/outside of loop is defined.

\item  \verb|int = obj.GetSelectionModeMinValue ()| -  Control how inside/outside of loop is defined.

\item  \verb|int = obj.GetSelectionModeMaxValue ()| -  Control how inside/outside of loop is defined.

\item  \verb|int = obj.GetSelectionMode ()| -  Control how inside/outside of loop is defined.

\item  \verb|obj.SetSelectionModeToSmallestRegion ()| -  Control how inside/outside of loop is defined.

\item  \verb|obj.SetSelectionModeToLargestRegion ()| -  Control how inside/outside of loop is defined.

\item  \verb|obj.SetSelectionModeToClosestPointRegion ()| -  Control how inside/outside of loop is defined.

\item  \verb|string = obj.GetSelectionModeAsString ()| -  Control how inside/outside of loop is defined.

\item  \verb|obj.SetGenerateUnselectedOutput (int )| -  Control whether a second output is generated. The second output
 contains the polygonal data that's not been selected.

\item  \verb|int = obj.GetGenerateUnselectedOutput ()| -  Control whether a second output is generated. The second output
 contains the polygonal data that's not been selected.

\item  \verb|obj.GenerateUnselectedOutputOn ()| -  Control whether a second output is generated. The second output
 contains the polygonal data that's not been selected.

\item  \verb|obj.GenerateUnselectedOutputOff ()| -  Control whether a second output is generated. The second output
 contains the polygonal data that's not been selected.

\item  \verb|vtkPolyData = obj.GetUnselectedOutput ()| -  Return output that hasn't been selected (if GenreateUnselectedOutput is
 enabled).

\item  \verb|vtkPolyData = obj.GetSelectionEdges ()| -  Return the (mesh) edges of the selection region.

\item  \verb|long = obj.GetMTime ()|

\end{itemize}
