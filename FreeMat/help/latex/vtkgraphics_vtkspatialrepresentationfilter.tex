\section{vtkSpatialRepresentationFilter}

\subsection{Usage}

 vtkSpatialRepresentationFilter generates an polygonal representation of a
 spatial search (vtkLocator) object. The representation varies depending
 upon the nature of the spatial search object. For example, the
 representation for vtkOBBTree is a collection of oriented bounding
 boxes. Ths input to this filter is a dataset of any type, and the output
 is polygonal data. You must also specify the spatial search object to
 use.

 Generally spatial search objects are used for collision detection and
 other geometric operations, but in this filter one or more levels of
 spatial searchers can be generated to form a geometric approximation to
 the input data. This is a form of data simplification, generally used to
 accelerate the rendering process. Or, this filter can be used as a
 debugging/ visualization aid for spatial search objects.
 
 This filter can generate one or more output vtkPolyData corresponding to
 different levels in the spatial search tree. The output data is retrieved 
 using the GetOutput(id) method, where id ranges from 0 (root level) 
 to Level. Note that the output for level ''id'' is not computed unless a 
 GetOutput(id) method is issued. Thus, if you desire three levels of output 
 (say 2,4,7), you would have to invoke GetOutput(2), GetOutput(4), and 
 GetOutput(7). (Also note that the Level ivar is computed automatically 
 depending on the size and nature of the input data.) There is also 
 another GetOutput() method that takes no parameters. This method returns 
 the leafs of the spatial search tree, which may be at different levels.

To create an instance of class vtkSpatialRepresentationFilter, simply
invoke its constructor as follows
\begin{verbatim}
  obj = vtkSpatialRepresentationFilter
\end{verbatim}
\subsection{Methods}

The class vtkSpatialRepresentationFilter has several methods that can be used.
  They are listed below.
Note that the documentation is translated automatically from the VTK sources,
and may not be completely intelligible.  When in doubt, consult the VTK website.
In the methods listed below, \verb|obj| is an instance of the vtkSpatialRepresentationFilter class.
\begin{itemize}
\item  \verb|string = obj.GetClassName ()|

\item  \verb|int = obj.IsA (string name)|

\item  \verb|vtkSpatialRepresentationFilter = obj.NewInstance ()|

\item  \verb|vtkSpatialRepresentationFilter = obj.SafeDownCast (vtkObject o)|

\item  \verb|obj.SetSpatialRepresentation (vtkLocator )| -  Set/Get the locator that will be used to generate the representation.

\item  \verb|vtkLocator = obj.GetSpatialRepresentation ()| -  Set/Get the locator that will be used to generate the representation.

\item  \verb|int = obj.GetLevel ()| -  Get the maximum number of outputs actually available.

\item  \verb|vtkPolyData = obj.GetOutput (int level)| -  A special form of the GetOutput() method that returns multiple outputs.

\item  \verb|vtkPolyData = obj.GetOutput ()| -  Output of terminal nodes/leaves.

\item  \verb|obj.ResetOutput ()| -  Reset requested output levels

\item  \verb|obj.SetInput (vtkDataSet input)| -  Set / get the input data or filter.

\item  \verb|vtkDataSet = obj.GetInput ()| -  Set / get the input data or filter.

\end{itemize}
