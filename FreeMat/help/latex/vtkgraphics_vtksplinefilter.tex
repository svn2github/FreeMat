\section{vtkSplineFilter}

\subsection{Usage}

 vtkSplineFilter is a filter that generates an output polylines from an
 input set of polylines. The polylines are uniformly subdivided and produced
 with the help of a vtkSpline class that the user can specify (by default a
 vtkCardinalSpline is used). The number of subdivisions of the line can be
 controlled in several ways. The user can either specify the number of
 subdivisions or a length of each subdivision can be provided (and the
 class will figure out how many subdivisions is required over the whole
 polyline). The maximum number of subdivisions can also be set.

 The output of this filter is a polyline per input polyline (or line). New
 points and texture coordinates are created. Point data is interpolated and
 cell data passed on. Any polylines with less than two points, or who have
 coincident points, are ignored.

To create an instance of class vtkSplineFilter, simply
invoke its constructor as follows
\begin{verbatim}
  obj = vtkSplineFilter
\end{verbatim}
\subsection{Methods}

The class vtkSplineFilter has several methods that can be used.
  They are listed below.
Note that the documentation is translated automatically from the VTK sources,
and may not be completely intelligible.  When in doubt, consult the VTK website.
In the methods listed below, \verb|obj| is an instance of the vtkSplineFilter class.
\begin{itemize}
\item  \verb|string = obj.GetClassName ()|

\item  \verb|int = obj.IsA (string name)|

\item  \verb|vtkSplineFilter = obj.NewInstance ()|

\item  \verb|vtkSplineFilter = obj.SafeDownCast (vtkObject o)|

\item  \verb|obj.SetMaximumNumberOfSubdivisions (int )| -  Set the maximum number of subdivisions that are created for each
 polyline.

\item  \verb|int = obj.GetMaximumNumberOfSubdivisionsMinValue ()| -  Set the maximum number of subdivisions that are created for each
 polyline.

\item  \verb|int = obj.GetMaximumNumberOfSubdivisionsMaxValue ()| -  Set the maximum number of subdivisions that are created for each
 polyline.

\item  \verb|int = obj.GetMaximumNumberOfSubdivisions ()| -  Set the maximum number of subdivisions that are created for each
 polyline.

\item  \verb|obj.SetSubdivide (int )| -  Specify how the number of subdivisions is determined.

\item  \verb|int = obj.GetSubdivideMinValue ()| -  Specify how the number of subdivisions is determined.

\item  \verb|int = obj.GetSubdivideMaxValue ()| -  Specify how the number of subdivisions is determined.

\item  \verb|int = obj.GetSubdivide ()| -  Specify how the number of subdivisions is determined.

\item  \verb|obj.SetSubdivideToSpecified ()| -  Specify how the number of subdivisions is determined.

\item  \verb|obj.SetSubdivideToLength ()| -  Specify how the number of subdivisions is determined.

\item  \verb|string = obj.GetSubdivideAsString ()| -  Specify how the number of subdivisions is determined.

\item  \verb|obj.SetNumberOfSubdivisions (int )| -  Set the number of subdivisions that are created for the
 polyline. This method only has effect if Subdivisions is set
 to SetSubdivisionsToSpecify().

\item  \verb|int = obj.GetNumberOfSubdivisionsMinValue ()| -  Set the number of subdivisions that are created for the
 polyline. This method only has effect if Subdivisions is set
 to SetSubdivisionsToSpecify().

\item  \verb|int = obj.GetNumberOfSubdivisionsMaxValue ()| -  Set the number of subdivisions that are created for the
 polyline. This method only has effect if Subdivisions is set
 to SetSubdivisionsToSpecify().

\item  \verb|int = obj.GetNumberOfSubdivisions ()| -  Set the number of subdivisions that are created for the
 polyline. This method only has effect if Subdivisions is set
 to SetSubdivisionsToSpecify().

\item  \verb|obj.SetLength (double )| -  Control the number of subdivisions that are created for the
 polyline based on an absolute length. The length of the spline
 is divided by this length to determine the number of subdivisions.

\item  \verb|double = obj.GetLengthMinValue ()| -  Control the number of subdivisions that are created for the
 polyline based on an absolute length. The length of the spline
 is divided by this length to determine the number of subdivisions.

\item  \verb|double = obj.GetLengthMaxValue ()| -  Control the number of subdivisions that are created for the
 polyline based on an absolute length. The length of the spline
 is divided by this length to determine the number of subdivisions.

\item  \verb|double = obj.GetLength ()| -  Control the number of subdivisions that are created for the
 polyline based on an absolute length. The length of the spline
 is divided by this length to determine the number of subdivisions.

\item  \verb|obj.SetSpline (vtkSpline )| -  Specify an instance of vtkSpline to use to perform the interpolation.

\item  \verb|vtkSpline = obj.GetSpline ()| -  Specify an instance of vtkSpline to use to perform the interpolation.

\item  \verb|obj.SetGenerateTCoords (int )| -  Control whether and how texture coordinates are produced. This is
 useful for striping the output polyline. The texture coordinates
 can be generated in three ways: a normalized (0,1) generation;
 based on the length (divided by the texture length); and by using
 the input scalar values.

\item  \verb|int = obj.GetGenerateTCoordsMinValue ()| -  Control whether and how texture coordinates are produced. This is
 useful for striping the output polyline. The texture coordinates
 can be generated in three ways: a normalized (0,1) generation;
 based on the length (divided by the texture length); and by using
 the input scalar values.

\item  \verb|int = obj.GetGenerateTCoordsMaxValue ()| -  Control whether and how texture coordinates are produced. This is
 useful for striping the output polyline. The texture coordinates
 can be generated in three ways: a normalized (0,1) generation;
 based on the length (divided by the texture length); and by using
 the input scalar values.

\item  \verb|int = obj.GetGenerateTCoords ()| -  Control whether and how texture coordinates are produced. This is
 useful for striping the output polyline. The texture coordinates
 can be generated in three ways: a normalized (0,1) generation;
 based on the length (divided by the texture length); and by using
 the input scalar values.

\item  \verb|obj.SetGenerateTCoordsToOff ()| -  Control whether and how texture coordinates are produced. This is
 useful for striping the output polyline. The texture coordinates
 can be generated in three ways: a normalized (0,1) generation;
 based on the length (divided by the texture length); and by using
 the input scalar values.

\item  \verb|obj.SetGenerateTCoordsToNormalizedLength ()| -  Control whether and how texture coordinates are produced. This is
 useful for striping the output polyline. The texture coordinates
 can be generated in three ways: a normalized (0,1) generation;
 based on the length (divided by the texture length); and by using
 the input scalar values.

\item  \verb|obj.SetGenerateTCoordsToUseLength ()| -  Control whether and how texture coordinates are produced. This is
 useful for striping the output polyline. The texture coordinates
 can be generated in three ways: a normalized (0,1) generation;
 based on the length (divided by the texture length); and by using
 the input scalar values.

\item  \verb|obj.SetGenerateTCoordsToUseScalars ()| -  Control whether and how texture coordinates are produced. This is
 useful for striping the output polyline. The texture coordinates
 can be generated in three ways: a normalized (0,1) generation;
 based on the length (divided by the texture length); and by using
 the input scalar values.

\item  \verb|string = obj.GetGenerateTCoordsAsString ()| -  Control whether and how texture coordinates are produced. This is
 useful for striping the output polyline. The texture coordinates
 can be generated in three ways: a normalized (0,1) generation;
 based on the length (divided by the texture length); and by using
 the input scalar values.

\item  \verb|obj.SetTextureLength (double )| -  Control the conversion of units during the texture coordinates
 calculation. The TextureLength indicates what length (whether 
 calculated from scalars or length) is mapped to the [0,1)
 texture space.

\item  \verb|double = obj.GetTextureLengthMinValue ()| -  Control the conversion of units during the texture coordinates
 calculation. The TextureLength indicates what length (whether 
 calculated from scalars or length) is mapped to the [0,1)
 texture space.

\item  \verb|double = obj.GetTextureLengthMaxValue ()| -  Control the conversion of units during the texture coordinates
 calculation. The TextureLength indicates what length (whether 
 calculated from scalars or length) is mapped to the [0,1)
 texture space.

\item  \verb|double = obj.GetTextureLength ()| -  Control the conversion of units during the texture coordinates
 calculation. The TextureLength indicates what length (whether 
 calculated from scalars or length) is mapped to the [0,1)
 texture space.

\end{itemize}
