\section{vtkStripper}

\subsection{Usage}


To create an instance of class vtkStripper, simply
invoke its constructor as follows
\begin{verbatim}
  obj = vtkStripper
\end{verbatim}
\subsection{Methods}

The class vtkStripper has several methods that can be used.
  They are listed below.
Note that the documentation is translated automatically from the VTK sources,
and may not be completely intelligible.  When in doubt, consult the VTK website.
In the methods listed below, \verb|obj| is an instance of the vtkStripper class.
\begin{itemize}
\item  \verb|string = obj.GetClassName ()|

\item  \verb|int = obj.IsA (string name)|

\item  \verb|vtkStripper = obj.NewInstance ()|

\item  \verb|vtkStripper = obj.SafeDownCast (vtkObject o)|

\item  \verb|obj.SetMaximumLength (int )| -  Specify the maximum number of triangles in a triangle strip,
 and/or the maximum number of lines in a poly-line.

\item  \verb|int = obj.GetMaximumLengthMinValue ()| -  Specify the maximum number of triangles in a triangle strip,
 and/or the maximum number of lines in a poly-line.

\item  \verb|int = obj.GetMaximumLengthMaxValue ()| -  Specify the maximum number of triangles in a triangle strip,
 and/or the maximum number of lines in a poly-line.

\item  \verb|int = obj.GetMaximumLength ()| -  Specify the maximum number of triangles in a triangle strip,
 and/or the maximum number of lines in a poly-line.

\item  \verb|obj.PassCellDataAsFieldDataOn ()| -  Enable/Disable passing of the CellData in the input to
 the output as FieldData. Note the field data is tranformed.

\item  \verb|obj.PassCellDataAsFieldDataOff ()| -  Enable/Disable passing of the CellData in the input to
 the output as FieldData. Note the field data is tranformed.

\item  \verb|obj.SetPassCellDataAsFieldData (int )| -  Enable/Disable passing of the CellData in the input to
 the output as FieldData. Note the field data is tranformed.

\item  \verb|int = obj.GetPassCellDataAsFieldData ()| -  Enable/Disable passing of the CellData in the input to
 the output as FieldData. Note the field data is tranformed.

\item  \verb|obj.SetPassThroughCellIds (int )| -  If on, the output polygonal dataset will have a celldata array that 
 holds the cell index of the original 3D cell that produced each output
 cell. This is useful for picking. The default is off to conserve 
 memory.

\item  \verb|int = obj.GetPassThroughCellIds ()| -  If on, the output polygonal dataset will have a celldata array that 
 holds the cell index of the original 3D cell that produced each output
 cell. This is useful for picking. The default is off to conserve 
 memory.

\item  \verb|obj.PassThroughCellIdsOn ()| -  If on, the output polygonal dataset will have a celldata array that 
 holds the cell index of the original 3D cell that produced each output
 cell. This is useful for picking. The default is off to conserve 
 memory.

\item  \verb|obj.PassThroughCellIdsOff ()| -  If on, the output polygonal dataset will have a celldata array that 
 holds the cell index of the original 3D cell that produced each output
 cell. This is useful for picking. The default is off to conserve 
 memory.

\item  \verb|obj.SetPassThroughPointIds (int )| -  If on, the output polygonal dataset will have a pointdata array that 
 holds the point index of the original vertex that produced each output
 vertex. This is useful for picking. The default is off to conserve 
 memory.

\item  \verb|int = obj.GetPassThroughPointIds ()| -  If on, the output polygonal dataset will have a pointdata array that 
 holds the point index of the original vertex that produced each output
 vertex. This is useful for picking. The default is off to conserve 
 memory.

\item  \verb|obj.PassThroughPointIdsOn ()| -  If on, the output polygonal dataset will have a pointdata array that 
 holds the point index of the original vertex that produced each output
 vertex. This is useful for picking. The default is off to conserve 
 memory.

\item  \verb|obj.PassThroughPointIdsOff ()| -  If on, the output polygonal dataset will have a pointdata array that 
 holds the point index of the original vertex that produced each output
 vertex. This is useful for picking. The default is off to conserve 
 memory.

\end{itemize}
