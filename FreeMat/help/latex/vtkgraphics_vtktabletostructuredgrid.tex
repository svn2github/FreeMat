\section{vtkTableToStructuredGrid}

\subsection{Usage}

 vtkTableToStructuredGrid is a filter that converts an input
 vtkTable to a vtkStructuredGrid. It provides API to select columns to use as
 points in the output structured grid. The specified dimensions of the output
 (specified using SetWholeExtent()) must match the number of rows in the input
 table.

To create an instance of class vtkTableToStructuredGrid, simply
invoke its constructor as follows
\begin{verbatim}
  obj = vtkTableToStructuredGrid
\end{verbatim}
\subsection{Methods}

The class vtkTableToStructuredGrid has several methods that can be used.
  They are listed below.
Note that the documentation is translated automatically from the VTK sources,
and may not be completely intelligible.  When in doubt, consult the VTK website.
In the methods listed below, \verb|obj| is an instance of the vtkTableToStructuredGrid class.
\begin{itemize}
\item  \verb|string = obj.GetClassName ()|

\item  \verb|int = obj.IsA (string name)|

\item  \verb|vtkTableToStructuredGrid = obj.NewInstance ()|

\item  \verb|vtkTableToStructuredGrid = obj.SafeDownCast (vtkObject o)|

\item  \verb|obj.SetWholeExtent (int , int , int , int , int , int )| -  Get/Set the whole extents for the image to produce. The size of the image
 must match the number of rows in the input table.

\item  \verb|obj.SetWholeExtent (int  a[6])| -  Get/Set the whole extents for the image to produce. The size of the image
 must match the number of rows in the input table.

\item  \verb|int = obj. GetWholeExtent ()| -  Get/Set the whole extents for the image to produce. The size of the image
 must match the number of rows in the input table.

\item  \verb|obj.SetXColumn (string )| -  Set the name of the column to use as the X coordinate for the points. 

\item  \verb|string = obj.GetXColumn ()| -  Set the name of the column to use as the X coordinate for the points. 

\item  \verb|obj.SetXComponent (int )| -  Specify the component for the column specified using SetXColumn() to
 use as the xcoordinate in case the column is a multi-component array.
 Default is 0.

\item  \verb|int = obj.GetXComponentMinValue ()| -  Specify the component for the column specified using SetXColumn() to
 use as the xcoordinate in case the column is a multi-component array.
 Default is 0.

\item  \verb|int = obj.GetXComponentMaxValue ()| -  Specify the component for the column specified using SetXColumn() to
 use as the xcoordinate in case the column is a multi-component array.
 Default is 0.

\item  \verb|int = obj.GetXComponent ()| -  Specify the component for the column specified using SetXColumn() to
 use as the xcoordinate in case the column is a multi-component array.
 Default is 0.

\item  \verb|obj.SetYColumn (string )| -  Set the name of the column to use as the Y coordinate for the points. 
 Default is 0.

\item  \verb|string = obj.GetYColumn ()| -  Set the name of the column to use as the Y coordinate for the points. 
 Default is 0.

\item  \verb|obj.SetYComponent (int )| -  Specify the component for the column specified using SetYColumn() to
 use as the Ycoordinate in case the column is a multi-component array.

\item  \verb|int = obj.GetYComponentMinValue ()| -  Specify the component for the column specified using SetYColumn() to
 use as the Ycoordinate in case the column is a multi-component array.

\item  \verb|int = obj.GetYComponentMaxValue ()| -  Specify the component for the column specified using SetYColumn() to
 use as the Ycoordinate in case the column is a multi-component array.

\item  \verb|int = obj.GetYComponent ()| -  Specify the component for the column specified using SetYColumn() to
 use as the Ycoordinate in case the column is a multi-component array.

\item  \verb|obj.SetZColumn (string )| -  Set the name of the column to use as the Z coordinate for the points.
 Default is 0.

\item  \verb|string = obj.GetZColumn ()| -  Set the name of the column to use as the Z coordinate for the points.
 Default is 0.

\item  \verb|obj.SetZComponent (int )| -  Specify the component for the column specified using SetZColumn() to
 use as the Zcoordinate in case the column is a multi-component array.

\item  \verb|int = obj.GetZComponentMinValue ()| -  Specify the component for the column specified using SetZColumn() to
 use as the Zcoordinate in case the column is a multi-component array.

\item  \verb|int = obj.GetZComponentMaxValue ()| -  Specify the component for the column specified using SetZColumn() to
 use as the Zcoordinate in case the column is a multi-component array.

\item  \verb|int = obj.GetZComponent ()| -  Specify the component for the column specified using SetZColumn() to
 use as the Zcoordinate in case the column is a multi-component array.

\end{itemize}
