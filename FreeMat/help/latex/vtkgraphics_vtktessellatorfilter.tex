\section{vtkTessellatorFilter}

\subsection{Usage}

 This class approximates nonlinear FEM elements with linear simplices.

 <b>Warning</b>: This class is temporary and will go away at some point
 after ParaView 1.4.0.

 This filter rifles through all the cells in an input vtkDataSet. It
 tesselates each cell and uses the vtkStreamingTessellator and
 vtkDataSetEdgeSubdivisionCriterion classes to generate simplices that
 approximate the nonlinear mesh using some approximation metric (encoded
 in the particular vtkDataSetEdgeSubdivisionCriterion::EvaluateEdge
 implementation). The simplices are placed into the filter's output
 vtkDataSet object by the callback routines AddATetrahedron,
 AddATriangle, and AddALine, which are registered with the triangulator.

 The output mesh will have geometry and any fields specified as
 attributes in the input mesh's point data.  The attribute's copy flags
 are honored, except for normals.

 .SECTION Internals

 The filter's main member function is RequestData(). This function first
 calls SetupOutput() which allocates arrays and some temporary variables
 for the primitive callbacks (OutputTriangle and OutputLine which are
 called by AddATriangle and AddALine, respectively).  Each cell is given
 an initial tesselation, which results in one or more calls to
 OutputTetrahedron, OutputTriangle or OutputLine to add elements to the
 OutputMesh. Finally, Teardown() is called to free the filter's working
 space.


To create an instance of class vtkTessellatorFilter, simply
invoke its constructor as follows
\begin{verbatim}
  obj = vtkTessellatorFilter
\end{verbatim}
\subsection{Methods}

The class vtkTessellatorFilter has several methods that can be used.
  They are listed below.
Note that the documentation is translated automatically from the VTK sources,
and may not be completely intelligible.  When in doubt, consult the VTK website.
In the methods listed below, \verb|obj| is an instance of the vtkTessellatorFilter class.
\begin{itemize}
\item  \verb|string = obj.GetClassName ()|

\item  \verb|int = obj.IsA (string name)|

\item  \verb|vtkTessellatorFilter = obj.NewInstance ()|

\item  \verb|vtkTessellatorFilter = obj.SafeDownCast (vtkObject o)|

\item  \verb|obj.SetTessellator (vtkStreamingTessellator )|

\item  \verb|vtkStreamingTessellator = obj.GetTessellator ()|

\item  \verb|obj.SetSubdivider (vtkDataSetEdgeSubdivisionCriterion )|

\item  \verb|vtkDataSetEdgeSubdivisionCriterion = obj.GetSubdivider ()|

\item  \verb|long = obj.GetMTime ()|

\item  \verb|obj.SetOutputDimension (int )| -  Set the dimension of the output tessellation.
 Cells in dimensions higher than the given value will have
 their boundaries of dimension  OutputDimension tessellated.
 For example, if  OutputDimension is 2, a hexahedron's
 quadrilateral faces would be tessellated rather than its
 interior.

\item  \verb|int = obj.GetOutputDimensionMinValue ()| -  Set the dimension of the output tessellation.
 Cells in dimensions higher than the given value will have
 their boundaries of dimension  OutputDimension tessellated.
 For example, if  OutputDimension is 2, a hexahedron's
 quadrilateral faces would be tessellated rather than its
 interior.

\item  \verb|int = obj.GetOutputDimensionMaxValue ()| -  Set the dimension of the output tessellation.
 Cells in dimensions higher than the given value will have
 their boundaries of dimension  OutputDimension tessellated.
 For example, if  OutputDimension is 2, a hexahedron's
 quadrilateral faces would be tessellated rather than its
 interior.

\item  \verb|int = obj.GetOutputDimension ()| -  Set the dimension of the output tessellation.
 Cells in dimensions higher than the given value will have
 their boundaries of dimension  OutputDimension tessellated.
 For example, if  OutputDimension is 2, a hexahedron's
 quadrilateral faces would be tessellated rather than its
 interior.

\item  \verb|obj.SetMaximumNumberOfSubdivisions (int num\_subdiv\_in)| -  These are convenience routines for setting properties maintained by the
 tessellator and subdivider. They are implemented here for ParaView's
 sake.

\item  \verb|int = obj.GetMaximumNumberOfSubdivisions ()| -  These are convenience routines for setting properties maintained by the
 tessellator and subdivider. They are implemented here for ParaView's
 sake.

\item  \verb|obj.SetChordError (double ce)| -  These are convenience routines for setting properties maintained by the
 tessellator and subdivider. They are implemented here for ParaView's
 sake.

\item  \verb|double = obj.GetChordError ()| -  These are convenience routines for setting properties maintained by the
 tessellator and subdivider. They are implemented here for ParaView's
 sake.

\item  \verb|obj.ResetFieldCriteria ()| -  These methods are for the ParaView client.

\item  \verb|obj.SetFieldCriterion (int field, double chord)| -  These methods are for the ParaView client.

\item  \verb|int = obj.GetMergePoints ()| -  The adaptive tessellation will output vertices that are not shared
 among cells, even where they should be. This can be corrected to
 some extents with a vtkMergeFilter.
 By default, the filter is off and vertices will not be shared.

\item  \verb|obj.SetMergePoints (int )| -  The adaptive tessellation will output vertices that are not shared
 among cells, even where they should be. This can be corrected to
 some extents with a vtkMergeFilter.
 By default, the filter is off and vertices will not be shared.

\item  \verb|obj.MergePointsOn ()| -  The adaptive tessellation will output vertices that are not shared
 among cells, even where they should be. This can be corrected to
 some extents with a vtkMergeFilter.
 By default, the filter is off and vertices will not be shared.

\item  \verb|obj.MergePointsOff ()| -  The adaptive tessellation will output vertices that are not shared
 among cells, even where they should be. This can be corrected to
 some extents with a vtkMergeFilter.
 By default, the filter is off and vertices will not be shared.

\end{itemize}
