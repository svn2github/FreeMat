\section{vtkThreshold}

\subsection{Usage}

 vtkThreshold is a filter that extracts cells from any dataset type that
 satisfy a threshold criterion. A cell satisfies the criterion if the
 scalar value of (every or any) point satisfies the criterion. The
 criterion can take three forms: 1) greater than a particular value; 2)
 less than a particular value; or 3) between two values. The output of this
 filter is an unstructured grid.

 Note that scalar values are available from the point and cell attribute
 data.  By default, point data is used to obtain scalars, but you can
 control this behavior. See the AttributeMode ivar below.

 By default only the first scalar value is used in the decision. Use the ComponentMode
 and SelectedComponent ivars to control this behavior.

To create an instance of class vtkThreshold, simply
invoke its constructor as follows
\begin{verbatim}
  obj = vtkThreshold
\end{verbatim}
\subsection{Methods}

The class vtkThreshold has several methods that can be used.
  They are listed below.
Note that the documentation is translated automatically from the VTK sources,
and may not be completely intelligible.  When in doubt, consult the VTK website.
In the methods listed below, \verb|obj| is an instance of the vtkThreshold class.
\begin{itemize}
\item  \verb|string = obj.GetClassName ()|

\item  \verb|int = obj.IsA (string name)|

\item  \verb|vtkThreshold = obj.NewInstance ()|

\item  \verb|vtkThreshold = obj.SafeDownCast (vtkObject o)|

\item  \verb|obj.ThresholdByLower (double lower)| -  Criterion is cells whose scalars are less or equal to lower threshold.

\item  \verb|obj.ThresholdByUpper (double upper)| -  Criterion is cells whose scalars are greater or equal to upper threshold.

\item  \verb|obj.ThresholdBetween (double lower, double upper)| -  Criterion is cells whose scalars are between lower and upper thresholds
 (inclusive of the end values).

\item  \verb|double = obj.GetUpperThreshold ()| -  Get the Upper and Lower thresholds.

\item  \verb|double = obj.GetLowerThreshold ()| -  Get the Upper and Lower thresholds.

\item  \verb|obj.SetAttributeMode (int )| -  Control how the filter works with scalar point data and cell attribute
 data.  By default (AttributeModeToDefault), the filter will use point
 data, and if no point data is available, then cell data is
 used. Alternatively you can explicitly set the filter to use point data
 (AttributeModeToUsePointData) or cell data (AttributeModeToUseCellData).

\item  \verb|int = obj.GetAttributeMode ()| -  Control how the filter works with scalar point data and cell attribute
 data.  By default (AttributeModeToDefault), the filter will use point
 data, and if no point data is available, then cell data is
 used. Alternatively you can explicitly set the filter to use point data
 (AttributeModeToUsePointData) or cell data (AttributeModeToUseCellData).

\item  \verb|obj.SetAttributeModeToDefault ()| -  Control how the filter works with scalar point data and cell attribute
 data.  By default (AttributeModeToDefault), the filter will use point
 data, and if no point data is available, then cell data is
 used. Alternatively you can explicitly set the filter to use point data
 (AttributeModeToUsePointData) or cell data (AttributeModeToUseCellData).

\item  \verb|obj.SetAttributeModeToUsePointData ()| -  Control how the filter works with scalar point data and cell attribute
 data.  By default (AttributeModeToDefault), the filter will use point
 data, and if no point data is available, then cell data is
 used. Alternatively you can explicitly set the filter to use point data
 (AttributeModeToUsePointData) or cell data (AttributeModeToUseCellData).

\item  \verb|obj.SetAttributeModeToUseCellData ()| -  Control how the filter works with scalar point data and cell attribute
 data.  By default (AttributeModeToDefault), the filter will use point
 data, and if no point data is available, then cell data is
 used. Alternatively you can explicitly set the filter to use point data
 (AttributeModeToUsePointData) or cell data (AttributeModeToUseCellData).

\item  \verb|string = obj.GetAttributeModeAsString ()| -  Control how the filter works with scalar point data and cell attribute
 data.  By default (AttributeModeToDefault), the filter will use point
 data, and if no point data is available, then cell data is
 used. Alternatively you can explicitly set the filter to use point data
 (AttributeModeToUsePointData) or cell data (AttributeModeToUseCellData).

\item  \verb|obj.SetComponentMode (int )| -  Control how the decision of in / out is made with multi-component data.
 The choices are to use the selected component (specified in the
 SelectedComponent ivar), or to look at all components. When looking at
 all components, the evaluation can pass if all the components satisfy
 the rule (UseAll) or if any satisfy is (UseAny). The default value is
 UseSelected.

\item  \verb|int = obj.GetComponentModeMinValue ()| -  Control how the decision of in / out is made with multi-component data.
 The choices are to use the selected component (specified in the
 SelectedComponent ivar), or to look at all components. When looking at
 all components, the evaluation can pass if all the components satisfy
 the rule (UseAll) or if any satisfy is (UseAny). The default value is
 UseSelected.

\item  \verb|int = obj.GetComponentModeMaxValue ()| -  Control how the decision of in / out is made with multi-component data.
 The choices are to use the selected component (specified in the
 SelectedComponent ivar), or to look at all components. When looking at
 all components, the evaluation can pass if all the components satisfy
 the rule (UseAll) or if any satisfy is (UseAny). The default value is
 UseSelected.

\item  \verb|int = obj.GetComponentMode ()| -  Control how the decision of in / out is made with multi-component data.
 The choices are to use the selected component (specified in the
 SelectedComponent ivar), or to look at all components. When looking at
 all components, the evaluation can pass if all the components satisfy
 the rule (UseAll) or if any satisfy is (UseAny). The default value is
 UseSelected.

\item  \verb|obj.SetComponentModeToUseSelected ()| -  Control how the decision of in / out is made with multi-component data.
 The choices are to use the selected component (specified in the
 SelectedComponent ivar), or to look at all components. When looking at
 all components, the evaluation can pass if all the components satisfy
 the rule (UseAll) or if any satisfy is (UseAny). The default value is
 UseSelected.

\item  \verb|obj.SetComponentModeToUseAll ()| -  Control how the decision of in / out is made with multi-component data.
 The choices are to use the selected component (specified in the
 SelectedComponent ivar), or to look at all components. When looking at
 all components, the evaluation can pass if all the components satisfy
 the rule (UseAll) or if any satisfy is (UseAny). The default value is
 UseSelected.

\item  \verb|obj.SetComponentModeToUseAny ()| -  Control how the decision of in / out is made with multi-component data.
 The choices are to use the selected component (specified in the
 SelectedComponent ivar), or to look at all components. When looking at
 all components, the evaluation can pass if all the components satisfy
 the rule (UseAll) or if any satisfy is (UseAny). The default value is
 UseSelected.

\item  \verb|string = obj.GetComponentModeAsString ()| -  Control how the decision of in / out is made with multi-component data.
 The choices are to use the selected component (specified in the
 SelectedComponent ivar), or to look at all components. When looking at
 all components, the evaluation can pass if all the components satisfy
 the rule (UseAll) or if any satisfy is (UseAny). The default value is
 UseSelected.

\item  \verb|obj.SetSelectedComponent (int )| -  When the component mode is UseSelected, this ivar indicated the selected
 component. The default value is 0.

\item  \verb|int = obj.GetSelectedComponentMinValue ()| -  When the component mode is UseSelected, this ivar indicated the selected
 component. The default value is 0.

\item  \verb|int = obj.GetSelectedComponentMaxValue ()| -  When the component mode is UseSelected, this ivar indicated the selected
 component. The default value is 0.

\item  \verb|int = obj.GetSelectedComponent ()| -  When the component mode is UseSelected, this ivar indicated the selected
 component. The default value is 0.

\item  \verb|obj.SetAllScalars (int )| -  If using scalars from point data, all scalars for all points in a cell 
 must satisfy the threshold criterion if AllScalars is set. Otherwise, 
 just a single scalar value satisfying the threshold criterion enables
 will extract the cell.

\item  \verb|int = obj.GetAllScalars ()| -  If using scalars from point data, all scalars for all points in a cell 
 must satisfy the threshold criterion if AllScalars is set. Otherwise, 
 just a single scalar value satisfying the threshold criterion enables
 will extract the cell.

\item  \verb|obj.AllScalarsOn ()| -  If using scalars from point data, all scalars for all points in a cell 
 must satisfy the threshold criterion if AllScalars is set. Otherwise, 
 just a single scalar value satisfying the threshold criterion enables
 will extract the cell.

\item  \verb|obj.AllScalarsOff ()| -  If using scalars from point data, all scalars for all points in a cell 
 must satisfy the threshold criterion if AllScalars is set. Otherwise, 
 just a single scalar value satisfying the threshold criterion enables
 will extract the cell.

\item  \verb|obj.SetPointsDataTypeToDouble ()| -  Set the data type of the output points (See the data types defined in 
 vtkType.h). The default data type is float.

\item  \verb|obj.SetPointsDataTypeToFloat ()| -  Set the data type of the output points (See the data types defined in 
 vtkType.h). The default data type is float.

\item  \verb|obj.SetPointsDataType (int )| -  Set the data type of the output points (See the data types defined in 
 vtkType.h). The default data type is float.

\item  \verb|int = obj.GetPointsDataType ()| -  Set the data type of the output points (See the data types defined in 
 vtkType.h). The default data type is float.

\end{itemize}
