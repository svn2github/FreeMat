\section{vtkTransformCoordinateSystems}

\subsection{Usage}

 This filter transforms points from one coordinate system to another. The user
 must specify the coordinate systems in which the input and output are
 specified. The user must also specify the VTK viewport (i.e., renderer) in
 which the transformation occurs.


To create an instance of class vtkTransformCoordinateSystems, simply
invoke its constructor as follows
\begin{verbatim}
  obj = vtkTransformCoordinateSystems
\end{verbatim}
\subsection{Methods}

The class vtkTransformCoordinateSystems has several methods that can be used.
  They are listed below.
Note that the documentation is translated automatically from the VTK sources,
and may not be completely intelligible.  When in doubt, consult the VTK website.
In the methods listed below, \verb|obj| is an instance of the vtkTransformCoordinateSystems class.
\begin{itemize}
\item  \verb|string = obj.GetClassName ()| -  Standard methods for type information and printing.

\item  \verb|int = obj.IsA (string name)| -  Standard methods for type information and printing.

\item  \verb|vtkTransformCoordinateSystems = obj.NewInstance ()| -  Standard methods for type information and printing.

\item  \verb|vtkTransformCoordinateSystems = obj.SafeDownCast (vtkObject o)| -  Standard methods for type information and printing.

\item  \verb|obj.SetInputCoordinateSystem (int )| -  Set/get the coordinate system in which the input is specified.
 The current options are World, Viewport, and Display. By default the
 input coordinate system is World.

\item  \verb|int = obj.GetInputCoordinateSystem ()| -  Set/get the coordinate system in which the input is specified.
 The current options are World, Viewport, and Display. By default the
 input coordinate system is World.

\item  \verb|obj.SetInputCoordinateSystemToDisplay ()| -  Set/get the coordinate system in which the input is specified.
 The current options are World, Viewport, and Display. By default the
 input coordinate system is World.

\item  \verb|obj.SetInputCoordinateSystemToViewport ()| -  Set/get the coordinate system in which the input is specified.
 The current options are World, Viewport, and Display. By default the
 input coordinate system is World.

\item  \verb|obj.SetInputCoordinateSystemToWorld ()| -  Set/get the coordinate system to which to transform the output.
 The current options are World, Viewport, and Display. By default the
 output coordinate system is Display.

\item  \verb|obj.SetOutputCoordinateSystem (int )| -  Set/get the coordinate system to which to transform the output.
 The current options are World, Viewport, and Display. By default the
 output coordinate system is Display.

\item  \verb|int = obj.GetOutputCoordinateSystem ()| -  Set/get the coordinate system to which to transform the output.
 The current options are World, Viewport, and Display. By default the
 output coordinate system is Display.

\item  \verb|obj.SetOutputCoordinateSystemToDisplay ()| -  Set/get the coordinate system to which to transform the output.
 The current options are World, Viewport, and Display. By default the
 output coordinate system is Display.

\item  \verb|obj.SetOutputCoordinateSystemToViewport ()| -  Set/get the coordinate system to which to transform the output.
 The current options are World, Viewport, and Display. By default the
 output coordinate system is Display.

\item  \verb|obj.SetOutputCoordinateSystemToWorld ()| -  Return the MTime also considering the instance of vtkCoordinate.

\item  \verb|long = obj.GetMTime ()| -  Return the MTime also considering the instance of vtkCoordinate.

\item  \verb|obj.SetViewport (vtkViewport viewport)| -  In order for successful coordinate transformation to occur, an
 instance of vtkViewport (e.g., a VTK renderer) must be specified.
 NOTE: this is a raw pointer, not a weak pointer not a reference counted
 object to avoid reference cycle loop between rendering classes and filter
 classes.

\item  \verb|vtkViewport = obj.GetViewport ()| -  In order for successful coordinate transformation to occur, an
 instance of vtkViewport (e.g., a VTK renderer) must be specified.
 NOTE: this is a raw pointer, not a weak pointer not a reference counted
 object to avoid reference cycle loop between rendering classes and filter
 classes.

\end{itemize}
