\section{vtkTubeFilter}

\subsection{Usage}

 vtkTubeFilter is a filter that generates a tube around each input line. 
 The tubes are made up of triangle strips and rotate around the tube with
 the rotation of the line normals. (If no normals are present, they are
 computed automatically.) The radius of the tube can be set to vary with 
 scalar or vector value. If the radius varies with scalar value the radius
 is linearly adjusted. If the radius varies with vector value, a mass
 flux preserving variation is used. The number of sides for the tube also 
 can be specified. You can also specify which of the sides are visible. This
 is useful for generating interesting striping effects. Other options
 include the ability to cap the tube and generate texture coordinates.
 Texture coordinates can be used with an associated texture map to create
 interesting effects such as marking the tube with stripes corresponding
 to length or time.

 This filter is typically used to create thick or dramatic lines. Another
 common use is to combine this filter with vtkStreamLine to generate
 streamtubes.

To create an instance of class vtkTubeFilter, simply
invoke its constructor as follows
\begin{verbatim}
  obj = vtkTubeFilter
\end{verbatim}
\subsection{Methods}

The class vtkTubeFilter has several methods that can be used.
  They are listed below.
Note that the documentation is translated automatically from the VTK sources,
and may not be completely intelligible.  When in doubt, consult the VTK website.
In the methods listed below, \verb|obj| is an instance of the vtkTubeFilter class.
\begin{itemize}
\item  \verb|string = obj.GetClassName ()|

\item  \verb|int = obj.IsA (string name)|

\item  \verb|vtkTubeFilter = obj.NewInstance ()|

\item  \verb|vtkTubeFilter = obj.SafeDownCast (vtkObject o)|

\item  \verb|obj.SetRadius (double )| -  Set the minimum tube radius (minimum because the tube radius may vary).

\item  \verb|double = obj.GetRadiusMinValue ()| -  Set the minimum tube radius (minimum because the tube radius may vary).

\item  \verb|double = obj.GetRadiusMaxValue ()| -  Set the minimum tube radius (minimum because the tube radius may vary).

\item  \verb|double = obj.GetRadius ()| -  Set the minimum tube radius (minimum because the tube radius may vary).

\item  \verb|obj.SetVaryRadius (int )| -  Turn on/off the variation of tube radius with scalar value.

\item  \verb|int = obj.GetVaryRadiusMinValue ()| -  Turn on/off the variation of tube radius with scalar value.

\item  \verb|int = obj.GetVaryRadiusMaxValue ()| -  Turn on/off the variation of tube radius with scalar value.

\item  \verb|int = obj.GetVaryRadius ()| -  Turn on/off the variation of tube radius with scalar value.

\item  \verb|obj.SetVaryRadiusToVaryRadiusOff ()| -  Turn on/off the variation of tube radius with scalar value.

\item  \verb|obj.SetVaryRadiusToVaryRadiusByScalar ()| -  Turn on/off the variation of tube radius with scalar value.

\item  \verb|obj.SetVaryRadiusToVaryRadiusByVector ()| -  Turn on/off the variation of tube radius with scalar value.

\item  \verb|obj.SetVaryRadiusToVaryRadiusByAbsoluteScalar ()| -  Turn on/off the variation of tube radius with scalar value.

\item  \verb|string = obj.GetVaryRadiusAsString ()| -  Turn on/off the variation of tube radius with scalar value.

\item  \verb|obj.SetNumberOfSides (int )| -  Set the number of sides for the tube. At a minimum, number of sides is 3.

\item  \verb|int = obj.GetNumberOfSidesMinValue ()| -  Set the number of sides for the tube. At a minimum, number of sides is 3.

\item  \verb|int = obj.GetNumberOfSidesMaxValue ()| -  Set the number of sides for the tube. At a minimum, number of sides is 3.

\item  \verb|int = obj.GetNumberOfSides ()| -  Set the number of sides for the tube. At a minimum, number of sides is 3.

\item  \verb|obj.SetRadiusFactor (double )| -  Set the maximum tube radius in terms of a multiple of the minimum radius.

\item  \verb|double = obj.GetRadiusFactor ()| -  Set the maximum tube radius in terms of a multiple of the minimum radius.

\item  \verb|obj.SetDefaultNormal (double , double , double )| -  Set the default normal to use if no normals are supplied, and the
 DefaultNormalOn is set.

\item  \verb|obj.SetDefaultNormal (double  a[3])| -  Set the default normal to use if no normals are supplied, and the
 DefaultNormalOn is set.

\item  \verb|double = obj. GetDefaultNormal ()| -  Set the default normal to use if no normals are supplied, and the
 DefaultNormalOn is set.

\item  \verb|obj.SetUseDefaultNormal (int )| -  Set a boolean to control whether to use default normals.
 DefaultNormalOn is set.

\item  \verb|int = obj.GetUseDefaultNormal ()| -  Set a boolean to control whether to use default normals.
 DefaultNormalOn is set.

\item  \verb|obj.UseDefaultNormalOn ()| -  Set a boolean to control whether to use default normals.
 DefaultNormalOn is set.

\item  \verb|obj.UseDefaultNormalOff ()| -  Set a boolean to control whether to use default normals.
 DefaultNormalOn is set.

\item  \verb|obj.SetSidesShareVertices (int )| -  Set a boolean to control whether tube sides should share vertices.
 This creates independent strips, with constant normals so the
 tube is always faceted in appearance.

\item  \verb|int = obj.GetSidesShareVertices ()| -  Set a boolean to control whether tube sides should share vertices.
 This creates independent strips, with constant normals so the
 tube is always faceted in appearance.

\item  \verb|obj.SidesShareVerticesOn ()| -  Set a boolean to control whether tube sides should share vertices.
 This creates independent strips, with constant normals so the
 tube is always faceted in appearance.

\item  \verb|obj.SidesShareVerticesOff ()| -  Set a boolean to control whether tube sides should share vertices.
 This creates independent strips, with constant normals so the
 tube is always faceted in appearance.

\item  \verb|obj.SetCapping (int )| -  Turn on/off whether to cap the ends with polygons.

\item  \verb|int = obj.GetCapping ()| -  Turn on/off whether to cap the ends with polygons.

\item  \verb|obj.CappingOn ()| -  Turn on/off whether to cap the ends with polygons.

\item  \verb|obj.CappingOff ()| -  Turn on/off whether to cap the ends with polygons.

\item  \verb|obj.SetOnRatio (int )| -  Control the striping of the tubes. If OnRatio is greater than 1,
 then every nth tube side is turned on, beginning with the Offset
 side.

\item  \verb|int = obj.GetOnRatioMinValue ()| -  Control the striping of the tubes. If OnRatio is greater than 1,
 then every nth tube side is turned on, beginning with the Offset
 side.

\item  \verb|int = obj.GetOnRatioMaxValue ()| -  Control the striping of the tubes. If OnRatio is greater than 1,
 then every nth tube side is turned on, beginning with the Offset
 side.

\item  \verb|int = obj.GetOnRatio ()| -  Control the striping of the tubes. If OnRatio is greater than 1,
 then every nth tube side is turned on, beginning with the Offset
 side.

\item  \verb|obj.SetOffset (int )| -  Control the striping of the tubes. The offset sets the
 first tube side that is visible. Offset is generally used with
 OnRatio to create nifty striping effects.

\item  \verb|int = obj.GetOffsetMinValue ()| -  Control the striping of the tubes. The offset sets the
 first tube side that is visible. Offset is generally used with
 OnRatio to create nifty striping effects.

\item  \verb|int = obj.GetOffsetMaxValue ()| -  Control the striping of the tubes. The offset sets the
 first tube side that is visible. Offset is generally used with
 OnRatio to create nifty striping effects.

\item  \verb|int = obj.GetOffset ()| -  Control the striping of the tubes. The offset sets the
 first tube side that is visible. Offset is generally used with
 OnRatio to create nifty striping effects.

\item  \verb|obj.SetGenerateTCoords (int )| -  Control whether and how texture coordinates are produced. This is
 useful for striping the tube with length textures, etc. If you
 use scalars to create the texture, the scalars are assumed to be
 monotonically increasing (or decreasing).

\item  \verb|int = obj.GetGenerateTCoordsMinValue ()| -  Control whether and how texture coordinates are produced. This is
 useful for striping the tube with length textures, etc. If you
 use scalars to create the texture, the scalars are assumed to be
 monotonically increasing (or decreasing).

\item  \verb|int = obj.GetGenerateTCoordsMaxValue ()| -  Control whether and how texture coordinates are produced. This is
 useful for striping the tube with length textures, etc. If you
 use scalars to create the texture, the scalars are assumed to be
 monotonically increasing (or decreasing).

\item  \verb|int = obj.GetGenerateTCoords ()| -  Control whether and how texture coordinates are produced. This is
 useful for striping the tube with length textures, etc. If you
 use scalars to create the texture, the scalars are assumed to be
 monotonically increasing (or decreasing).

\item  \verb|obj.SetGenerateTCoordsToOff ()| -  Control whether and how texture coordinates are produced. This is
 useful for striping the tube with length textures, etc. If you
 use scalars to create the texture, the scalars are assumed to be
 monotonically increasing (or decreasing).

\item  \verb|obj.SetGenerateTCoordsToNormalizedLength ()| -  Control whether and how texture coordinates are produced. This is
 useful for striping the tube with length textures, etc. If you
 use scalars to create the texture, the scalars are assumed to be
 monotonically increasing (or decreasing).

\item  \verb|obj.SetGenerateTCoordsToUseLength ()| -  Control whether and how texture coordinates are produced. This is
 useful for striping the tube with length textures, etc. If you
 use scalars to create the texture, the scalars are assumed to be
 monotonically increasing (or decreasing).

\item  \verb|obj.SetGenerateTCoordsToUseScalars ()| -  Control whether and how texture coordinates are produced. This is
 useful for striping the tube with length textures, etc. If you
 use scalars to create the texture, the scalars are assumed to be
 monotonically increasing (or decreasing).

\item  \verb|string = obj.GetGenerateTCoordsAsString ()| -  Control whether and how texture coordinates are produced. This is
 useful for striping the tube with length textures, etc. If you
 use scalars to create the texture, the scalars are assumed to be
 monotonically increasing (or decreasing).

\item  \verb|obj.SetTextureLength (double )| -  Control the conversion of units during the texture coordinates
 calculation. The TextureLength indicates what length (whether 
 calculated from scalars or length) is mapped to the [0,1)
 texture space.

\item  \verb|double = obj.GetTextureLengthMinValue ()| -  Control the conversion of units during the texture coordinates
 calculation. The TextureLength indicates what length (whether 
 calculated from scalars or length) is mapped to the [0,1)
 texture space.

\item  \verb|double = obj.GetTextureLengthMaxValue ()| -  Control the conversion of units during the texture coordinates
 calculation. The TextureLength indicates what length (whether 
 calculated from scalars or length) is mapped to the [0,1)
 texture space.

\item  \verb|double = obj.GetTextureLength ()| -  Control the conversion of units during the texture coordinates
 calculation. The TextureLength indicates what length (whether 
 calculated from scalars or length) is mapped to the [0,1)
 texture space.

\end{itemize}
