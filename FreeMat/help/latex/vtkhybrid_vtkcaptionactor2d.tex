\section{vtkCaptionActor2D}

\subsection{Usage}

 vtkCaptionActor2D is a hybrid 2D/3D actor that is used to associate text 
 with a point (the AttachmentPoint) in the scene. The caption can be 
 drawn with a rectangular border and a leader connecting 
 the caption to the attachment point. Optionally, the leader can be 
 glyphed at its endpoint to create arrow heads or other indicators.
 
 To use the caption actor, you normally specify the Position and Position2
 coordinates (these are inherited from the vtkActor2D superclass). (Note
 that Position2 can be set using vtkActor2D's SetWidth() and SetHeight()
 methods.)  Position and Position2 define the size of the caption, and a
 third point, the AttachmentPoint, defines a point that the caption is
 associated with.  You must also define the caption text, 
 whether you want a border around the caption, and whether you want a
 leader from the caption to the attachment point. The font attributes of 
 the text can be set through the vtkTextProperty associated to this actor.
 You also indicate whether you want
 the leader to be 2D or 3D. (2D leaders are always drawn over the
 underlying geometry. 3D leaders may be occluded by the geometry.) The
 leader may also be terminated by an optional glyph (e.g., arrow).

 The trickiest part about using this class is setting Position, Position2,
 and AttachmentPoint correctly. These instance variables are
 vtkCoordinates, and can be set up in various ways. In default usage, the
 AttachmentPoint is defined in the world coordinate system, Position is the
 lower-left corner of the caption and relative to AttachmentPoint (defined
 in display coordaintes, i.e., pixels), and Position2 is relative to
 Position and is the upper-right corner (also in display
 coordinates). However, the user has full control over the coordinates, and
 can do things like place the caption in a fixed position in the renderer,
 with the leader moving with the AttachmentPoint.

To create an instance of class vtkCaptionActor2D, simply
invoke its constructor as follows
\begin{verbatim}
  obj = vtkCaptionActor2D
\end{verbatim}
\subsection{Methods}

The class vtkCaptionActor2D has several methods that can be used.
  They are listed below.
Note that the documentation is translated automatically from the VTK sources,
and may not be completely intelligible.  When in doubt, consult the VTK website.
In the methods listed below, \verb|obj| is an instance of the vtkCaptionActor2D class.
\begin{itemize}
\item  \verb|string = obj.GetClassName ()|

\item  \verb|int = obj.IsA (string name)|

\item  \verb|vtkCaptionActor2D = obj.NewInstance ()|

\item  \verb|vtkCaptionActor2D = obj.SafeDownCast (vtkObject o)|

\item  \verb|obj.SetCaption (string )| -  Define the text to be placed in the caption. The text can be multiple
 lines (separated by ''\\n'').

\item  \verb|string = obj.GetCaption ()| -  Define the text to be placed in the caption. The text can be multiple
 lines (separated by ''\\n'').

\item  \verb|vtkCoordinate = obj.GetAttachmentPointCoordinate ()| -  Set/Get the attachment point for the caption. By default, the attachment
 point is defined in world coordinates, but this can be changed using
 vtkCoordinate methods.

\item  \verb|obj.SetAttachmentPoint (double, double, double)| -  Set/Get the attachment point for the caption. By default, the attachment
 point is defined in world coordinates, but this can be changed using
 vtkCoordinate methods.

\item  \verb|obj.SetAttachmentPoint (double a[3])| -  Set/Get the attachment point for the caption. By default, the attachment
 point is defined in world coordinates, but this can be changed using
 vtkCoordinate methods.

\item  \verb|double = obj.GetAttachmentPoint ()| -  Set/Get the attachment point for the caption. By default, the attachment
 point is defined in world coordinates, but this can be changed using
 vtkCoordinate methods.

\item  \verb|obj.SetBorder (int )| -  Enable/disable the placement of a border around the text.

\item  \verb|int = obj.GetBorder ()| -  Enable/disable the placement of a border around the text.

\item  \verb|obj.BorderOn ()| -  Enable/disable the placement of a border around the text.

\item  \verb|obj.BorderOff ()| -  Enable/disable the placement of a border around the text.

\item  \verb|obj.SetLeader (int )| -  Enable/disable drawing a ''line'' from the caption to the 
 attachment point.

\item  \verb|int = obj.GetLeader ()| -  Enable/disable drawing a ''line'' from the caption to the 
 attachment point.

\item  \verb|obj.LeaderOn ()| -  Enable/disable drawing a ''line'' from the caption to the 
 attachment point.

\item  \verb|obj.LeaderOff ()| -  Enable/disable drawing a ''line'' from the caption to the 
 attachment point.

\item  \verb|obj.SetThreeDimensionalLeader (int )| -  Indicate whether the leader is 2D (no hidden line) or 3D (z-buffered).

\item  \verb|int = obj.GetThreeDimensionalLeader ()| -  Indicate whether the leader is 2D (no hidden line) or 3D (z-buffered).

\item  \verb|obj.ThreeDimensionalLeaderOn ()| -  Indicate whether the leader is 2D (no hidden line) or 3D (z-buffered).

\item  \verb|obj.ThreeDimensionalLeaderOff ()| -  Indicate whether the leader is 2D (no hidden line) or 3D (z-buffered).

\item  \verb|obj.SetLeaderGlyph (vtkPolyData )| -  Specify a glyph to be used as the leader ''head''. This could be something
 like an arrow or sphere. If not specified, no glyph is drawn. Note that
 the glyph is assumed to be aligned along the x-axis and is rotated about
 the origin.

\item  \verb|vtkPolyData = obj.GetLeaderGlyph ()| -  Specify a glyph to be used as the leader ''head''. This could be something
 like an arrow or sphere. If not specified, no glyph is drawn. Note that
 the glyph is assumed to be aligned along the x-axis and is rotated about
 the origin.

\item  \verb|obj.SetLeaderGlyphSize (double )| -  Specify the relative size of the leader head. This is expressed as a
 fraction of the size (diagonal length) of the renderer. The leader
 head is automatically scaled so that window resize, zooming or other 
 camera motion results in proportional changes in size to the leader
 glyph.

\item  \verb|double = obj.GetLeaderGlyphSizeMinValue ()| -  Specify the relative size of the leader head. This is expressed as a
 fraction of the size (diagonal length) of the renderer. The leader
 head is automatically scaled so that window resize, zooming or other 
 camera motion results in proportional changes in size to the leader
 glyph.

\item  \verb|double = obj.GetLeaderGlyphSizeMaxValue ()| -  Specify the relative size of the leader head. This is expressed as a
 fraction of the size (diagonal length) of the renderer. The leader
 head is automatically scaled so that window resize, zooming or other 
 camera motion results in proportional changes in size to the leader
 glyph.

\item  \verb|double = obj.GetLeaderGlyphSize ()| -  Specify the relative size of the leader head. This is expressed as a
 fraction of the size (diagonal length) of the renderer. The leader
 head is automatically scaled so that window resize, zooming or other 
 camera motion results in proportional changes in size to the leader
 glyph.

\item  \verb|obj.SetMaximumLeaderGlyphSize (int )| -  Specify the maximum size of the leader head (if any) in pixels. This 
 is used in conjunction with LeaderGlyphSize to cap the maximum size of
 the LeaderGlyph.

\item  \verb|int = obj.GetMaximumLeaderGlyphSizeMinValue ()| -  Specify the maximum size of the leader head (if any) in pixels. This 
 is used in conjunction with LeaderGlyphSize to cap the maximum size of
 the LeaderGlyph.

\item  \verb|int = obj.GetMaximumLeaderGlyphSizeMaxValue ()| -  Specify the maximum size of the leader head (if any) in pixels. This 
 is used in conjunction with LeaderGlyphSize to cap the maximum size of
 the LeaderGlyph.

\item  \verb|int = obj.GetMaximumLeaderGlyphSize ()| -  Specify the maximum size of the leader head (if any) in pixels. This 
 is used in conjunction with LeaderGlyphSize to cap the maximum size of
 the LeaderGlyph.

\item  \verb|obj.SetPadding (int )| -  Set/Get the padding between the caption and the border. The value
 is specified in pixels.

\item  \verb|int = obj.GetPaddingMinValue ()| -  Set/Get the padding between the caption and the border. The value
 is specified in pixels.

\item  \verb|int = obj.GetPaddingMaxValue ()| -  Set/Get the padding between the caption and the border. The value
 is specified in pixels.

\item  \verb|int = obj.GetPadding ()| -  Set/Get the padding between the caption and the border. The value
 is specified in pixels.

\item  \verb|vtkTextActor = obj.GetTextActor ()| -  Get the text actor used by the caption. This is useful if you want to control
 justification and other characteristics of the text actor.

\item  \verb|obj.SetCaptionTextProperty (vtkTextProperty p)| -  Set/Get the text property.

\item  \verb|vtkTextProperty = obj.GetCaptionTextProperty ()| -  Set/Get the text property.

\item  \verb|obj.ShallowCopy (vtkProp prop)| -  Shallow copy of this scaled text actor. Overloads the virtual
 vtkProp method.

\item  \verb|obj.SetAttachEdgeOnly (int )| -  Enable/disable whether to attach the arrow only to the edge, 
 NOT the vertices of the caption border.

\item  \verb|int = obj.GetAttachEdgeOnly ()| -  Enable/disable whether to attach the arrow only to the edge, 
 NOT the vertices of the caption border.

\item  \verb|obj.AttachEdgeOnlyOn ()| -  Enable/disable whether to attach the arrow only to the edge, 
 NOT the vertices of the caption border.

\item  \verb|obj.AttachEdgeOnlyOff ()| -  Enable/disable whether to attach the arrow only to the edge, 
 NOT the vertices of the caption border.

\end{itemize}
