\section{vtkCornerAnnotation}

\subsection{Usage}

 This is an annotation object that manages four text actors / mappers
 to provide annotation in the four corners of a viewport


To create an instance of class vtkCornerAnnotation, simply
invoke its constructor as follows
\begin{verbatim}
  obj = vtkCornerAnnotation
\end{verbatim}
\subsection{Methods}

The class vtkCornerAnnotation has several methods that can be used.
  They are listed below.
Note that the documentation is translated automatically from the VTK sources,
and may not be completely intelligible.  When in doubt, consult the VTK website.
In the methods listed below, \verb|obj| is an instance of the vtkCornerAnnotation class.
\begin{itemize}
\item  \verb|string = obj.GetClassName ()|

\item  \verb|int = obj.IsA (string name)|

\item  \verb|vtkCornerAnnotation = obj.NewInstance ()|

\item  \verb|vtkCornerAnnotation = obj.SafeDownCast (vtkObject o)|

\item  \verb|int = obj.RenderOpaqueGeometry (vtkViewport viewport)| -  Draw the scalar bar and annotation text to the screen.

\item  \verb|int = obj.RenderTranslucentPolygonalGeometry (vtkViewport )| -  Draw the scalar bar and annotation text to the screen.

\item  \verb|int = obj.RenderOverlay (vtkViewport viewport)| -  Draw the scalar bar and annotation text to the screen.

\item  \verb|int = obj.HasTranslucentPolygonalGeometry ()| -  Does this prop have some translucent polygonal geometry?

\item  \verb|obj.SetMaximumLineHeight (double )| -  Set/Get the maximum height of a line of text as a 
 percentage of the vertical area allocated to this
 scaled text actor. Defaults to 1.0

\item  \verb|double = obj.GetMaximumLineHeight ()| -  Set/Get the maximum height of a line of text as a 
 percentage of the vertical area allocated to this
 scaled text actor. Defaults to 1.0

\item  \verb|obj.SetMinimumFontSize (int )| -  Set/Get the minimum/maximum size font that will be shown.
 If the font drops below the minimum size it will not be rendered.

\item  \verb|int = obj.GetMinimumFontSize ()| -  Set/Get the minimum/maximum size font that will be shown.
 If the font drops below the minimum size it will not be rendered.

\item  \verb|obj.SetMaximumFontSize (int )| -  Set/Get the minimum/maximum size font that will be shown.
 If the font drops below the minimum size it will not be rendered.

\item  \verb|int = obj.GetMaximumFontSize ()| -  Set/Get the minimum/maximum size font that will be shown.
 If the font drops below the minimum size it will not be rendered.

\item  \verb|obj.SetLinearFontScaleFactor (double )| -  Set/Get font scaling factors
 The font size, f, is calculated as the largest possible value
 such that the annotations for the given viewport do not overlap. 
 This font size is scaled non-linearly with the viewport size,
 to maintain an acceptable readable size at larger viewport sizes, 
 without being too big.
 f' = linearScale * pow(f,nonlinearScale)

\item  \verb|double = obj.GetLinearFontScaleFactor ()| -  Set/Get font scaling factors
 The font size, f, is calculated as the largest possible value
 such that the annotations for the given viewport do not overlap. 
 This font size is scaled non-linearly with the viewport size,
 to maintain an acceptable readable size at larger viewport sizes, 
 without being too big.
 f' = linearScale * pow(f,nonlinearScale)

\item  \verb|obj.SetNonlinearFontScaleFactor (double )| -  Set/Get font scaling factors
 The font size, f, is calculated as the largest possible value
 such that the annotations for the given viewport do not overlap. 
 This font size is scaled non-linearly with the viewport size,
 to maintain an acceptable readable size at larger viewport sizes, 
 without being too big.
 f' = linearScale * pow(f,nonlinearScale)

\item  \verb|double = obj.GetNonlinearFontScaleFactor ()| -  Set/Get font scaling factors
 The font size, f, is calculated as the largest possible value
 such that the annotations for the given viewport do not overlap. 
 This font size is scaled non-linearly with the viewport size,
 to maintain an acceptable readable size at larger viewport sizes, 
 without being too big.
 f' = linearScale * pow(f,nonlinearScale)

\item  \verb|obj.ReleaseGraphicsResources (vtkWindow )| -  Release any graphics resources that are being consumed by this actor.
 The parameter window could be used to determine which graphic
 resources to release.

\item  \verb|obj.SetText (int i, string text)| -  Set/Get the text to be displayed for each corner

\item  \verb|string = obj.GetText (int i)| -  Set/Get the text to be displayed for each corner

\item  \verb|obj.ClearAllTexts ()| -  Set/Get the text to be displayed for each corner

\item  \verb|obj.CopyAllTextsFrom (vtkCornerAnnotation ca)| -  Set/Get the text to be displayed for each corner

\item  \verb|obj.SetImageActor (vtkImageActor )| -  Set an image actor to look at for slice information

\item  \verb|vtkImageActor = obj.GetImageActor ()| -  Set an image actor to look at for slice information

\item  \verb|obj.SetWindowLevel (vtkImageMapToWindowLevelColors )| -  Set an instance of vtkImageMapToWindowLevelColors to use for
 looking at window level changes

\item  \verb|vtkImageMapToWindowLevelColors = obj.GetWindowLevel ()| -  Set an instance of vtkImageMapToWindowLevelColors to use for
 looking at window level changes

\item  \verb|obj.SetLevelShift (double )| -  Set the value to shift the level by.

\item  \verb|double = obj.GetLevelShift ()| -  Set the value to shift the level by.

\item  \verb|obj.SetLevelScale (double )| -  Set the value to scale the level by.

\item  \verb|double = obj.GetLevelScale ()| -  Set the value to scale the level by.

\item  \verb|obj.SetTextProperty (vtkTextProperty p)| -  Set/Get the text property of all corners.

\item  \verb|vtkTextProperty = obj.GetTextProperty ()| -  Set/Get the text property of all corners.

\item  \verb|obj.ShowSliceAndImageOn ()| -  Even if there is an image actor, should `slice' and `image' be displayed?

\item  \verb|obj.ShowSliceAndImageOff ()| -  Even if there is an image actor, should `slice' and `image' be displayed?

\item  \verb|obj.SetShowSliceAndImage (int )| -  Even if there is an image actor, should `slice' and `image' be displayed?

\item  \verb|int = obj.GetShowSliceAndImage ()| -  Even if there is an image actor, should `slice' and `image' be displayed?

\end{itemize}
