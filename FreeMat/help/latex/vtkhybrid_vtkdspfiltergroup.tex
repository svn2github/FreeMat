\section{vtkDSPFilterGroup}

\subsection{Usage}

 vtkDSPFilterGroup is used by vtkExodusReader, vtkExodusIIReader and
 vtkPExodusReader to do temporal smoothing of data

To create an instance of class vtkDSPFilterGroup, simply
invoke its constructor as follows
\begin{verbatim}
  obj = vtkDSPFilterGroup
\end{verbatim}
\subsection{Methods}

The class vtkDSPFilterGroup has several methods that can be used.
  They are listed below.
Note that the documentation is translated automatically from the VTK sources,
and may not be completely intelligible.  When in doubt, consult the VTK website.
In the methods listed below, \verb|obj| is an instance of the vtkDSPFilterGroup class.
\begin{itemize}
\item  \verb|string = obj.GetClassName ()|

\item  \verb|int = obj.IsA (string name)|

\item  \verb|vtkDSPFilterGroup = obj.NewInstance ()|

\item  \verb|vtkDSPFilterGroup = obj.SafeDownCast (vtkObject o)|

\item  \verb|obj.AddFilter (vtkDSPFilterDefinition filter)|

\item  \verb|obj.RemoveFilter (string a\_outputVariableName)|

\item  \verb|bool = obj.IsThisInputVariableInstanceNeeded (string a\_name, int a\_timestep, int a\_outputTimestep)|

\item  \verb|bool = obj.IsThisInputVariableInstanceCached (string a\_name, int a\_timestep)|

\item  \verb|obj.AddInputVariableInstance (string a\_name, int a\_timestep, vtkFloatArray a\_data)|

\item  \verb|vtkFloatArray = obj.GetCachedInput (int a\_whichFilter, int a\_whichTimestep)|

\item  \verb|vtkFloatArray = obj.GetCachedOutput (int a\_whichFilter, int a\_whichTimestep)|

\item  \verb|string = obj.GetInputVariableName (int a\_whichFilter)|

\item  \verb|int = obj.GetNumFilters ()|

\item  \verb|obj.Copy (vtkDSPFilterGroup other)|

\item  \verb|vtkDSPFilterDefinition = obj.GetFilter (int a\_whichFilter)|

\end{itemize}
