\section{vtkExodusIIReader}

\subsection{Usage}

 vtkExodusIIReader is a unstructured grid source object that reads ExodusII
 files.  Most of the meta data associated with the file is loaded when 
 UpdateInformation is called.  This includes information like Title, number
 of blocks, number and names of arrays. This data can be retrieved from 
 methods in this reader. Separate arrays that are meant to be a single 
 vector, are combined internally for convenience.  To be combined, the array 
 names have to be identical except for a trailing X,Y and Z (or x,y,z).  By 
 default cell and point arrays are not loaded.  However, the user can flag 
 arrays to load with the methods ''SetPointArrayStatus'' and
 ''SetCellArrayStatus''.  The reader DOES NOT respond to piece requests
 

To create an instance of class vtkExodusIIReader, simply
invoke its constructor as follows
\begin{verbatim}
  obj = vtkExodusIIReader
\end{verbatim}
\subsection{Methods}

The class vtkExodusIIReader has several methods that can be used.
  They are listed below.
Note that the documentation is translated automatically from the VTK sources,
and may not be completely intelligible.  When in doubt, consult the VTK website.
In the methods listed below, \verb|obj| is an instance of the vtkExodusIIReader class.
\begin{itemize}
\item  \verb|string = obj.GetClassName ()|

\item  \verb|int = obj.IsA (string name)|

\item  \verb|vtkExodusIIReader = obj.NewInstance ()|

\item  \verb|vtkExodusIIReader = obj.SafeDownCast (vtkObject o)|

\item  \verb|int = obj.CanReadFile (string fname)| -  Determine if the file can be readed with this reader.

\item  \verb|long = obj.GetMTime ()| -  Return the object's MTime. This is overridden to include the timestamp of its internal class.

\item  \verb|long = obj.GetMetadataMTime ()| -  Return the MTime of the internal data structure.
 This is really only intended for use by vtkPExodusIIReader in order
 to determine if the filename is newer than the metadata.

\item  \verb|obj.SetFileName (string fname)| -  Specify file name of the Exodus file.

\item  \verb|string = obj.GetFileName ()| -  Specify file name of the Exodus file.

\item  \verb|obj.SetXMLFileName (string fname)| -  Specify file name of the xml file.

\item  \verb|string = obj.GetXMLFileName ()| -  Specify file name of the xml file.

\item  \verb|obj.SetTimeStep (int )| -  Which TimeStep to read.    

\item  \verb|int = obj.GetTimeStep ()| -  Which TimeStep to read.    

\item  \verb|obj.SetModeShape (int val)| -  Returns the available range of valid integer time steps.

\item  \verb|int = obj. GetTimeStepRange ()| -  Returns the available range of valid integer time steps.

\item  \verb|obj.SetTimeStepRange (int , int )| -  Returns the available range of valid integer time steps.

\item  \verb|obj.SetTimeStepRange (int  a[2])| -  Returns the available range of valid integer time steps.

\item  \verb|obj.SetGenerateObjectIdCellArray (int g)| -  Extra cell data array that can be generated.  By default, this array
 is ON.  The value of the array is the integer id found
 in the exodus file. The name of the array is returned by 
 GetBlockIdArrayName(). For cells representing elements from
 an Exodus element block, this is set to the element block ID. For
 cells representing edges from an Exodus edge block, this is the
 edge block ID. Similarly, ths is the face block ID for cells
 representing faces from an Exodus face block. The same holds
 for cells representing entries of node, edge, face, side, and element sets.

\item  \verb|int = obj.GetGenerateObjectIdCellArray ()| -  Extra cell data array that can be generated.  By default, this array
 is ON.  The value of the array is the integer id found
 in the exodus file. The name of the array is returned by 
 GetBlockIdArrayName(). For cells representing elements from
 an Exodus element block, this is set to the element block ID. For
 cells representing edges from an Exodus edge block, this is the
 edge block ID. Similarly, ths is the face block ID for cells
 representing faces from an Exodus face block. The same holds
 for cells representing entries of node, edge, face, side, and element sets.

\item  \verb|obj.GenerateObjectIdCellArrayOn ()| -  Extra cell data array that can be generated.  By default, this array
 is ON.  The value of the array is the integer id found
 in the exodus file. The name of the array is returned by 
 GetBlockIdArrayName(). For cells representing elements from
 an Exodus element block, this is set to the element block ID. For
 cells representing edges from an Exodus edge block, this is the
 edge block ID. Similarly, ths is the face block ID for cells
 representing faces from an Exodus face block. The same holds
 for cells representing entries of node, edge, face, side, and element sets.

\item  \verb|obj.GenerateObjectIdCellArrayOff ()| -  Extra cell data array that can be generated.  By default, this array
 is ON.  The value of the array is the integer id found
 in the exodus file. The name of the array is returned by 
 GetBlockIdArrayName(). For cells representing elements from
 an Exodus element block, this is set to the element block ID. For
 cells representing edges from an Exodus edge block, this is the
 edge block ID. Similarly, ths is the face block ID for cells
 representing faces from an Exodus face block. The same holds
 for cells representing entries of node, edge, face, side, and element sets.

\item  \verb|obj.SetGenerateGlobalElementIdArray (int g)|

\item  \verb|int = obj.GetGenerateGlobalElementIdArray ()|

\item  \verb|obj.GenerateGlobalElementIdArrayOn ()|

\item  \verb|obj.GenerateGlobalElementIdArrayOff ()|

\item  \verb|obj.SetGenerateGlobalNodeIdArray (int g)|

\item  \verb|int = obj.GetGenerateGlobalNodeIdArray ()|

\item  \verb|obj.GenerateGlobalNodeIdArrayOn ()|

\item  \verb|obj.GenerateGlobalNodeIdArrayOff ()|

\item  \verb|obj.SetGenerateImplicitElementIdArray (int g)|

\item  \verb|int = obj.GetGenerateImplicitElementIdArray ()|

\item  \verb|obj.GenerateImplicitElementIdArrayOn ()|

\item  \verb|obj.GenerateImplicitElementIdArrayOff ()|

\item  \verb|obj.SetGenerateImplicitNodeIdArray (int g)|

\item  \verb|int = obj.GetGenerateImplicitNodeIdArray ()|

\item  \verb|obj.GenerateImplicitNodeIdArrayOn ()|

\item  \verb|obj.GenerateImplicitNodeIdArrayOff ()|

\item  \verb|obj.SetGenerateFileIdArray (int f)|

\item  \verb|int = obj.GetGenerateFileIdArray ()|

\item  \verb|obj.GenerateFileIdArrayOn ()|

\item  \verb|obj.GenerateFileIdArrayOff ()|

\item  \verb|obj.SetFileId (int f)|

\item  \verb|int = obj.GetFileId ()|

\item  \verb|obj.SetApplyDisplacements (int d)| -  Geometric locations can include displacements.  By default, 
 this is ON.  The nodal positions are 'displaced' by the
 standard exodus displacment vector. If displacements
 are turned 'off', the user can explicitly add them by
 applying a warp filter.

\item  \verb|int = obj.GetApplyDisplacements ()| -  Geometric locations can include displacements.  By default, 
 this is ON.  The nodal positions are 'displaced' by the
 standard exodus displacment vector. If displacements
 are turned 'off', the user can explicitly add them by
 applying a warp filter.

\item  \verb|obj.ApplyDisplacementsOn ()| -  Geometric locations can include displacements.  By default, 
 this is ON.  The nodal positions are 'displaced' by the
 standard exodus displacment vector. If displacements
 are turned 'off', the user can explicitly add them by
 applying a warp filter.

\item  \verb|obj.ApplyDisplacementsOff ()| -  Geometric locations can include displacements.  By default, 
 this is ON.  The nodal positions are 'displaced' by the
 standard exodus displacment vector. If displacements
 are turned 'off', the user can explicitly add them by
 applying a warp filter.

\item  \verb|obj.SetDisplacementMagnitude (float s)| -  Geometric locations can include displacements.  By default, 
 this is ON.  The nodal positions are 'displaced' by the
 standard exodus displacment vector. If displacements
 are turned 'off', the user can explicitly add them by
 applying a warp filter.

\item  \verb|float = obj.GetDisplacementMagnitude ()| -  Geometric locations can include displacements.  By default, 
 this is ON.  The nodal positions are 'displaced' by the
 standard exodus displacment vector. If displacements
 are turned 'off', the user can explicitly add them by
 applying a warp filter.

\item  \verb|obj.SetHasModeShapes (int ms)| -  Set/Get whether the Exodus sequence number corresponds to time steps or mode shapes.
 By default, HasModeShapes is false unless two time values in the Exodus file are identical,
 in which case it is true.

\item  \verb|int = obj.GetHasModeShapes ()| -  Set/Get whether the Exodus sequence number corresponds to time steps or mode shapes.
 By default, HasModeShapes is false unless two time values in the Exodus file are identical,
 in which case it is true.

\item  \verb|obj.HasModeShapesOn ()| -  Set/Get whether the Exodus sequence number corresponds to time steps or mode shapes.
 By default, HasModeShapes is false unless two time values in the Exodus file are identical,
 in which case it is true.

\item  \verb|obj.HasModeShapesOff ()| -  Set/Get whether the Exodus sequence number corresponds to time steps or mode shapes.
 By default, HasModeShapes is false unless two time values in the Exodus file are identical,
 in which case it is true.

\item  \verb|obj.SetModeShapeTime (double phase)| -  Set/Get the time used to animate mode shapes.
 This is a number between 0 and 1 that is used to scale the  DisplacementMagnitude
 in a sinusoidal pattern. Specifically, the displacement vector for each vertex is scaled by
 $ \mathrm{DisplacementMagnitude} cos( 2i \mathrm{ModeShapeTime} ) $ before it is
 added to the vertex coordinates.

\item  \verb|double = obj.GetModeShapeTime ()| -  Set/Get the time used to animate mode shapes.
 This is a number between 0 and 1 that is used to scale the  DisplacementMagnitude
 in a sinusoidal pattern. Specifically, the displacement vector for each vertex is scaled by
 $ \mathrm{DisplacementMagnitude} cos( 2i \mathrm{ModeShapeTime} ) $ before it is
 added to the vertex coordinates.

\item  \verb|obj.SetAnimateModeShapes (int flag)| -  If this flag is on (the default) and HasModeShapes is also on, then this
 reader will report a continuous time range [0,1] and animate the
 displacements in a periodic sinusoid.  If this flag is off and
 HasModeShapes is on, this reader ignores time.  This flag has no effect if
 HasModeShapes is off.

\item  \verb|int = obj.GetAnimateModeShapes ()| -  If this flag is on (the default) and HasModeShapes is also on, then this
 reader will report a continuous time range [0,1] and animate the
 displacements in a periodic sinusoid.  If this flag is off and
 HasModeShapes is on, this reader ignores time.  This flag has no effect if
 HasModeShapes is off.

\item  \verb|obj.AnimateModeShapesOn ()| -  If this flag is on (the default) and HasModeShapes is also on, then this
 reader will report a continuous time range [0,1] and animate the
 displacements in a periodic sinusoid.  If this flag is off and
 HasModeShapes is on, this reader ignores time.  This flag has no effect if
 HasModeShapes is off.

\item  \verb|obj.AnimateModeShapesOff ()| -  If this flag is on (the default) and HasModeShapes is also on, then this
 reader will report a continuous time range [0,1] and animate the
 displacements in a periodic sinusoid.  If this flag is off and
 HasModeShapes is on, this reader ignores time.  This flag has no effect if
 HasModeShapes is off.

\item  \verb|obj.SetEdgeFieldDecorations (int d)| -  FIXME

\item  \verb|int = obj.GetEdgeFieldDecorations ()| -  FIXME

\item  \verb|obj.EdgeFieldDecorationsNone ()| -  FIXME

\item  \verb|obj.EdgeFieldDecorationsGlyphs ()| -  FIXME

\item  \verb|obj.EdgeFieldDecorationsCornerAveraged ()| -  FIXME

\item  \verb|obj.SetFaceFieldDecorations (int d)| -  FIXME

\item  \verb|int = obj.GetFaceFieldDecorations ()| -  FIXME

\item  \verb|obj.FaceFieldDecorationsNone ()| -  FIXME

\item  \verb|obj.FaceFieldDecorationsGlyphs ()| -  FIXME

\item  \verb|obj.FaceFieldDecorationsCornerAveraged ()| -  Access to meta data generated by UpdateInformation.

\item  \verb|string = obj.GetTitle ()| -  Access to meta data generated by UpdateInformation.

\item  \verb|int = obj.GetDimensionality ()| -  Access to meta data generated by UpdateInformation.

\item  \verb|int = obj.GetNumberOfTimeSteps ()| -  Access to meta data generated by UpdateInformation.

\item  \verb|int = obj.GetNumberOfNodesInFile ()|

\item  \verb|int = obj.GetNumberOfEdgesInFile ()|

\item  \verb|int = obj.GetNumberOfFacesInFile ()|

\item  \verb|int = obj.GetNumberOfElementsInFile ()|

\item  \verb|int = obj.GetObjectTypeFromName (string name)|

\item  \verb|string = obj.GetObjectTypeName (int )|

\item  \verb|int = obj.GetNumberOfNodes ()|

\item  \verb|int = obj.GetNumberOfObjects (int objectType)|

\item  \verb|int = obj.GetNumberOfEntriesInObject (int objectType, int objectIndex)|

\item  \verb|int = obj.GetObjectId (int objectType, int objectIndex)|

\item  \verb|string = obj.GetObjectName (int objectType, int objectIndex)|

\item  \verb|int = obj.GetObjectIndex (int objectType, string objectName)|

\item  \verb|int = obj.GetObjectIndex (int objectType, int id)|

\item  \verb|int = obj.GetObjectStatus (int objectType, int objectIndex)|

\item  \verb|int = obj.GetObjectStatus (int objectType, string objectName)|

\item  \verb|obj.SetObjectStatus (int objectType, int objectIndex, int status)|

\item  \verb|obj.SetObjectStatus (int objectType, string objectName, int status)|

\item  \verb|int = obj.GetNumberOfObjectArrays (int objectType)|

\item  \verb|string = obj.GetObjectArrayName (int objectType, int arrayIndex)|

\item  \verb|int = obj.GetObjectArrayIndex (int objectType, string arrayName)|

\item  \verb|int = obj.GetNumberOfObjectArrayComponents (int objectType, int arrayIndex)|

\item  \verb|int = obj.GetObjectArrayStatus (int objectType, int arrayIndex)|

\item  \verb|int = obj.GetObjectArrayStatus (int objectType, string arrayName)|

\item  \verb|obj.SetObjectArrayStatus (int objectType, int arrayIndex, int status)|

\item  \verb|obj.SetObjectArrayStatus (int objectType, string arrayName, int status)|

\item  \verb|int = obj.GetNumberOfObjectAttributes (int objectType, int objectIndex)|

\item  \verb|string = obj.GetObjectAttributeName (int objectType, int objectIndex, int attribIndex)|

\item  \verb|int = obj.GetObjectAttributeIndex (int objectType, int objectIndex, string attribName)|

\item  \verb|int = obj.GetObjectAttributeStatus (int objectType, int objectIndex, int attribIndex)|

\item  \verb|int = obj.GetObjectAttributeStatus (int objectType, int objectIndex, string attribName)|

\item  \verb|obj.SetObjectAttributeStatus (int objectType, int objectIndex, int attribIndex, int status)|

\item  \verb|obj.SetObjectAttributeStatus (int objectType, int objectIndex, string attribName, int status)|

\item  \verb|vtkIdType = obj.GetTotalNumberOfNodes ()|

\item  \verb|vtkIdType = obj.GetTotalNumberOfEdges ()|

\item  \verb|vtkIdType = obj.GetTotalNumberOfFaces ()|

\item  \verb|vtkIdType = obj.GetTotalNumberOfElements ()|

\item  \verb|int = obj.GetNumberOfPartArrays ()|

\item  \verb|string = obj.GetPartArrayName (int arrayIdx)|

\item  \verb|int = obj.GetPartArrayID (string name)|

\item  \verb|string = obj.GetPartBlockInfo (int arrayIdx)|

\item  \verb|obj.SetPartArrayStatus (int index, int flag)|

\item  \verb|obj.SetPartArrayStatus (string , int flag)|

\item  \verb|int = obj.GetPartArrayStatus (int index)|

\item  \verb|int = obj.GetPartArrayStatus (string )|

\item  \verb|int = obj.GetNumberOfMaterialArrays ()|

\item  \verb|string = obj.GetMaterialArrayName (int arrayIdx)|

\item  \verb|int = obj.GetMaterialArrayID (string name)|

\item  \verb|obj.SetMaterialArrayStatus (int index, int flag)|

\item  \verb|obj.SetMaterialArrayStatus (string , int flag)|

\item  \verb|int = obj.GetMaterialArrayStatus (int index)|

\item  \verb|int = obj.GetMaterialArrayStatus (string )|

\item  \verb|int = obj.GetNumberOfAssemblyArrays ()|

\item  \verb|string = obj.GetAssemblyArrayName (int arrayIdx)|

\item  \verb|int = obj.GetAssemblyArrayID (string name)|

\item  \verb|obj.SetAssemblyArrayStatus (int index, int flag)|

\item  \verb|obj.SetAssemblyArrayStatus (string , int flag)|

\item  \verb|int = obj.GetAssemblyArrayStatus (int index)|

\item  \verb|int = obj.GetAssemblyArrayStatus (string )|

\item  \verb|int = obj.GetNumberOfHierarchyArrays ()|

\item  \verb|string = obj.GetHierarchyArrayName (int arrayIdx)|

\item  \verb|obj.SetHierarchyArrayStatus (int index, int flag)|

\item  \verb|obj.SetHierarchyArrayStatus (string , int flag)|

\item  \verb|int = obj.GetHierarchyArrayStatus (int index)|

\item  \verb|int = obj.GetHierarchyArrayStatus (string )|

\item  \verb|int = obj.GetDisplayType ()|

\item  \verb|obj.SetDisplayType (int type)|

\item  \verb|obj.ExodusModelMetadataOn ()|

\item  \verb|obj.ExodusModelMetadataOff ()|

\item  \verb|obj.SetExodusModelMetadata (int )|

\item  \verb|int = obj.GetExodusModelMetadata ()|

\item  \verb|vtkExodusModel = obj.GetExodusModel ()| -    Returns the object which encapsulates the model metadata.

\item  \verb|obj.SetPackExodusModelOntoOutput (int )|

\item  \verb|int = obj.GetPackExodusModelOntoOutput ()|

\item  \verb|obj.PackExodusModelOntoOutputOn ()|

\item  \verb|obj.PackExodusModelOntoOutputOff ()|

\item  \verb|int = obj.IsValidVariable (string type, string name)|

\item  \verb|int = obj.GetVariableID (string type, string name)|

\item  \verb|obj.SetAllArrayStatus (int otype, int status)|

\item  \verb|int = obj.GetTimeSeriesData (int ID, string vName, string vType, vtkFloatArray result)|

\item  \verb|int = obj.GetNumberOfEdgeBlockArrays ()|

\item  \verb|string = obj.GetEdgeBlockArrayName (int index)|

\item  \verb|int = obj.GetEdgeBlockArrayStatus (string name)|

\item  \verb|obj.SetEdgeBlockArrayStatus (string name, int flag)|

\item  \verb|int = obj.GetNumberOfFaceBlockArrays ()|

\item  \verb|string = obj.GetFaceBlockArrayName (int index)|

\item  \verb|int = obj.GetFaceBlockArrayStatus (string name)|

\item  \verb|obj.SetFaceBlockArrayStatus (string name, int flag)|

\item  \verb|int = obj.GetNumberOfElementBlockArrays ()|

\item  \verb|string = obj.GetElementBlockArrayName (int index)|

\item  \verb|int = obj.GetElementBlockArrayStatus (string name)|

\item  \verb|obj.SetElementBlockArrayStatus (string name, int flag)|

\item  \verb|int = obj.GetNumberOfGlobalResultArrays ()|

\item  \verb|string = obj.GetGlobalResultArrayName (int index)|

\item  \verb|int = obj.GetGlobalResultArrayStatus (string name)|

\item  \verb|obj.SetGlobalResultArrayStatus (string name, int flag)|

\item  \verb|int = obj.GetNumberOfPointResultArrays ()|

\item  \verb|string = obj.GetPointResultArrayName (int index)|

\item  \verb|int = obj.GetPointResultArrayStatus (string name)|

\item  \verb|obj.SetPointResultArrayStatus (string name, int flag)|

\item  \verb|int = obj.GetNumberOfEdgeResultArrays ()|

\item  \verb|string = obj.GetEdgeResultArrayName (int index)|

\item  \verb|int = obj.GetEdgeResultArrayStatus (string name)|

\item  \verb|obj.SetEdgeResultArrayStatus (string name, int flag)|

\item  \verb|int = obj.GetNumberOfFaceResultArrays ()|

\item  \verb|string = obj.GetFaceResultArrayName (int index)|

\item  \verb|int = obj.GetFaceResultArrayStatus (string name)|

\item  \verb|obj.SetFaceResultArrayStatus (string name, int flag)|

\item  \verb|int = obj.GetNumberOfElementResultArrays ()|

\item  \verb|string = obj.GetElementResultArrayName (int index)|

\item  \verb|int = obj.GetElementResultArrayStatus (string name)|

\item  \verb|obj.SetElementResultArrayStatus (string name, int flag)|

\item  \verb|int = obj.GetNumberOfNodeMapArrays ()|

\item  \verb|string = obj.GetNodeMapArrayName (int index)|

\item  \verb|int = obj.GetNodeMapArrayStatus (string name)|

\item  \verb|obj.SetNodeMapArrayStatus (string name, int flag)|

\item  \verb|int = obj.GetNumberOfEdgeMapArrays ()|

\item  \verb|string = obj.GetEdgeMapArrayName (int index)|

\item  \verb|int = obj.GetEdgeMapArrayStatus (string name)|

\item  \verb|obj.SetEdgeMapArrayStatus (string name, int flag)|

\item  \verb|int = obj.GetNumberOfFaceMapArrays ()|

\item  \verb|string = obj.GetFaceMapArrayName (int index)|

\item  \verb|int = obj.GetFaceMapArrayStatus (string name)|

\item  \verb|obj.SetFaceMapArrayStatus (string name, int flag)|

\item  \verb|int = obj.GetNumberOfElementMapArrays ()|

\item  \verb|string = obj.GetElementMapArrayName (int index)|

\item  \verb|int = obj.GetElementMapArrayStatus (string name)|

\item  \verb|obj.SetElementMapArrayStatus (string name, int flag)|

\item  \verb|int = obj.GetNumberOfNodeSetArrays ()|

\item  \verb|string = obj.GetNodeSetArrayName (int index)|

\item  \verb|int = obj.GetNodeSetArrayStatus (string name)|

\item  \verb|obj.SetNodeSetArrayStatus (string name, int flag)|

\item  \verb|int = obj.GetNumberOfSideSetArrays ()|

\item  \verb|string = obj.GetSideSetArrayName (int index)|

\item  \verb|int = obj.GetSideSetArrayStatus (string name)|

\item  \verb|obj.SetSideSetArrayStatus (string name, int flag)|

\item  \verb|int = obj.GetNumberOfEdgeSetArrays ()|

\item  \verb|string = obj.GetEdgeSetArrayName (int index)|

\item  \verb|int = obj.GetEdgeSetArrayStatus (string name)|

\item  \verb|obj.SetEdgeSetArrayStatus (string name, int flag)|

\item  \verb|int = obj.GetNumberOfFaceSetArrays ()|

\item  \verb|string = obj.GetFaceSetArrayName (int index)|

\item  \verb|int = obj.GetFaceSetArrayStatus (string name)|

\item  \verb|obj.SetFaceSetArrayStatus (string name, int flag)|

\item  \verb|int = obj.GetNumberOfElementSetArrays ()|

\item  \verb|string = obj.GetElementSetArrayName (int index)|

\item  \verb|int = obj.GetElementSetArrayStatus (string name)|

\item  \verb|obj.SetElementSetArrayStatus (string name, int flag)|

\item  \verb|int = obj.GetNumberOfNodeSetResultArrays ()|

\item  \verb|string = obj.GetNodeSetResultArrayName (int index)|

\item  \verb|int = obj.GetNodeSetResultArrayStatus (string name)|

\item  \verb|obj.SetNodeSetResultArrayStatus (string name, int flag)|

\item  \verb|int = obj.GetNumberOfSideSetResultArrays ()|

\item  \verb|string = obj.GetSideSetResultArrayName (int index)|

\item  \verb|int = obj.GetSideSetResultArrayStatus (string name)|

\item  \verb|obj.SetSideSetResultArrayStatus (string name, int flag)|

\item  \verb|int = obj.GetNumberOfEdgeSetResultArrays ()|

\item  \verb|string = obj.GetEdgeSetResultArrayName (int index)|

\item  \verb|int = obj.GetEdgeSetResultArrayStatus (string name)|

\item  \verb|obj.SetEdgeSetResultArrayStatus (string name, int flag)|

\item  \verb|int = obj.GetNumberOfFaceSetResultArrays ()|

\item  \verb|string = obj.GetFaceSetResultArrayName (int index)|

\item  \verb|int = obj.GetFaceSetResultArrayStatus (string name)|

\item  \verb|obj.SetFaceSetResultArrayStatus (string name, int flag)|

\item  \verb|int = obj.GetNumberOfElementSetResultArrays ()|

\item  \verb|string = obj.GetElementSetResultArrayName (int index)|

\item  \verb|int = obj.GetElementSetResultArrayStatus (string name)|

\item  \verb|obj.SetElementSetResultArrayStatus (string name, int flag)| -  Set the fast-path keys. All three must be set for the fast-path
 option to work.
 Possible argument values: ''POINT'',''CELL'',''EDGE'',''FACE''

\item  \verb|obj.SetFastPathObjectType (string type)| -  Set the fast-path keys. All three must be set for the fast-path
 option to work.
 Possible argument values: ''POINT'',''CELL'',''EDGE'',''FACE''

\item  \verb|obj.SetFastPathIdType (string type)| -  Possible argument values: ''INDEX'',''GLOBAL''
 ''GLOBAL'' means the id refers to a global id
 ''INDEX'' means the id refers to an index into the VTK array

\item  \verb|obj.SetFastPathObjectId (vtkIdType id)| -  Possible argument values: ''INDEX'',''GLOBAL''
 ''GLOBAL'' means the id refers to a global id
 ''INDEX'' means the id refers to an index into the VTK array

\item  \verb|obj.Reset ()| -  Reset the user-specified parameters and flush internal arrays
 so that the reader state is just as it was after the reader was
 instantiated.

 It doesn't make sense to let users reset only the internal state;
 both the settings and the state are changed by this call.

\item  \verb|obj.ResetSettings ()| -  Reset the user-specified parameters to their default values.
 The only settings not affected are the filename and/or pattern
 because these have no default.

 Resetting the settings but not the state allows users to
 keep the active cache but return to initial array selections, etc.

\item  \verb|obj.ResetCache ()| -  Clears out the cache entries.

\item  \verb|obj.UpdateTimeInformation ()| -  Re-reads time information from the exodus file and updates
 TimeStepRange accordingly.

\item  \verb|obj.Dump ()|

\item  \verb|vtkGraph = obj.GetSIL ()| -  SIL describes organization of/relationships between classifications 
 eg. blocks/materials/hierarchies.

\item  \verb|int = obj.GetSILUpdateStamp ()| -  Every time the SIL is updated a this will return a different value.

\item  \verb|bool = obj.GetProducedFastPathOutput ()| -  HACK: Used by vtkPExodusIIReader to tell is the reader produced a valid
 fast path output.

\end{itemize}
