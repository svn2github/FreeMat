\section{vtkExodusReader}

\subsection{Usage}

 vtkExodusReader is a unstructured grid source object that reads ExodusII
 files.  Most of the meta data associated with the file is loaded when 
 UpdateInformation is called.  This includes information like Title, number
 of blocks, number and names of arrays. This data can be retrieved from 
 methods in this reader. Separate arrays that are meant to be a single 
 vector, are combined internally for convenience.  To be combined, the array 
 names have to be identical except for a trailing X,Y and Z (or x,y,z).  By 
 default cell and point arrays are not loaded.  However, the user can flag 
 arrays to load with the methods ''SetPointArrayStatus'' and
 ''SetCellArrayStatus''.  The reader DOES NOT respond to piece requests
 

To create an instance of class vtkExodusReader, simply
invoke its constructor as follows
\begin{verbatim}
  obj = vtkExodusReader
\end{verbatim}
\subsection{Methods}

The class vtkExodusReader has several methods that can be used.
  They are listed below.
Note that the documentation is translated automatically from the VTK sources,
and may not be completely intelligible.  When in doubt, consult the VTK website.
In the methods listed below, \verb|obj| is an instance of the vtkExodusReader class.
\begin{itemize}
\item  \verb|string = obj.GetClassName ()|

\item  \verb|int = obj.IsA (string name)|

\item  \verb|vtkExodusReader = obj.NewInstance ()|

\item  \verb|vtkExodusReader = obj.SafeDownCast (vtkObject o)|

\item  \verb|int = obj.CanReadFile (string fname)| -  Determine if the file can be readed with this reader.

\item  \verb|obj.SetFileName (string )| -  Specify file name of the Exodus file.

\item  \verb|string = obj.GetFileName ()| -  Specify file name of the Exodus file.

\item  \verb|obj.SetXMLFileName (string )| -  Specify file name of the xml file.

\item  \verb|string = obj.GetXMLFileName ()| -  Specify file name of the xml file.

\item  \verb|obj.SetTimeStep (int )| -  Which TimeStep to read.    

\item  \verb|int = obj.GetTimeStep ()| -  Which TimeStep to read.    

\item  \verb|obj.SetGenerateBlockIdCellArray (int )| -  Extra cell data array that can be generated.  By default, this array
 is ON.  The value of the array is the integer id found
 in the exodus file. The name of the array is returned by 
 GetBlockIdArrayName()

\item  \verb|int = obj.GetGenerateBlockIdCellArray ()| -  Extra cell data array that can be generated.  By default, this array
 is ON.  The value of the array is the integer id found
 in the exodus file. The name of the array is returned by 
 GetBlockIdArrayName()

\item  \verb|obj.GenerateBlockIdCellArrayOn ()| -  Extra cell data array that can be generated.  By default, this array
 is ON.  The value of the array is the integer id found
 in the exodus file. The name of the array is returned by 
 GetBlockIdArrayName()

\item  \verb|obj.GenerateBlockIdCellArrayOff ()| -  Extra cell data array that can be generated.  By default, this array
 is ON.  The value of the array is the integer id found
 in the exodus file. The name of the array is returned by 
 GetBlockIdArrayName()

\item  \verb|string = obj.GetBlockIdArrayName ()| -  Extra cell data array that can be generated.  By default, this array
 is off.  The value of the array is the integer global id of the cell.
 The name of the array is returned by GetGlobalElementIdArrayName()

\item  \verb|obj.SetGenerateGlobalElementIdArray (int )| -  Extra cell data array that can be generated.  By default, this array
 is off.  The value of the array is the integer global id of the cell.
 The name of the array is returned by GetGlobalElementIdArrayName()

\item  \verb|int = obj.GetGenerateGlobalElementIdArray ()| -  Extra cell data array that can be generated.  By default, this array
 is off.  The value of the array is the integer global id of the cell.
 The name of the array is returned by GetGlobalElementIdArrayName()

\item  \verb|obj.GenerateGlobalElementIdArrayOn ()| -  Extra cell data array that can be generated.  By default, this array
 is off.  The value of the array is the integer global id of the cell.
 The name of the array is returned by GetGlobalElementIdArrayName()

\item  \verb|obj.GenerateGlobalElementIdArrayOff ()| -  Extra cell data array that can be generated.  By default, this array
 is off.  The value of the array is the integer global id of the cell.
 The name of the array is returned by GetGlobalElementIdArrayName()

\item  \verb|obj.SetGenerateGlobalNodeIdArray (int )| -  Extra point data array that can be generated.  By default, this array
 is ON.  The value of the array is the integer id of the node.
 The id is relative to the entire data set.
 The name of the array is returned by GlobalNodeIdArrayName().

\item  \verb|int = obj.GetGenerateGlobalNodeIdArray ()| -  Extra point data array that can be generated.  By default, this array
 is ON.  The value of the array is the integer id of the node.
 The id is relative to the entire data set.
 The name of the array is returned by GlobalNodeIdArrayName().

\item  \verb|obj.GenerateGlobalNodeIdArrayOn ()| -  Extra point data array that can be generated.  By default, this array
 is ON.  The value of the array is the integer id of the node.
 The id is relative to the entire data set.
 The name of the array is returned by GlobalNodeIdArrayName().

\item  \verb|obj.GenerateGlobalNodeIdArrayOff ()| -  Extra point data array that can be generated.  By default, this array
 is ON.  The value of the array is the integer id of the node.
 The id is relative to the entire data set.
 The name of the array is returned by GlobalNodeIdArrayName().

\item  \verb|obj.SetApplyDisplacements (int )| -  Geometric locations can include displacements.  By default, 
 this is ON.  The nodal positions are 'displaced' by the
 standard exodus displacment vector. If displacements
 are turned 'off', the user can explicitly add them by
 applying a warp filter.

\item  \verb|int = obj.GetApplyDisplacements ()| -  Geometric locations can include displacements.  By default, 
 this is ON.  The nodal positions are 'displaced' by the
 standard exodus displacment vector. If displacements
 are turned 'off', the user can explicitly add them by
 applying a warp filter.

\item  \verb|obj.ApplyDisplacementsOn ()| -  Geometric locations can include displacements.  By default, 
 this is ON.  The nodal positions are 'displaced' by the
 standard exodus displacment vector. If displacements
 are turned 'off', the user can explicitly add them by
 applying a warp filter.

\item  \verb|obj.ApplyDisplacementsOff ()| -  Geometric locations can include displacements.  By default, 
 this is ON.  The nodal positions are 'displaced' by the
 standard exodus displacment vector. If displacements
 are turned 'off', the user can explicitly add them by
 applying a warp filter.

\item  \verb|obj.SetDisplacementMagnitude (float )| -  Geometric locations can include displacements.  By default, 
 this is ON.  The nodal positions are 'displaced' by the
 standard exodus displacment vector. If displacements
 are turned 'off', the user can explicitly add them by
 applying a warp filter.

\item  \verb|float = obj.GetDisplacementMagnitude ()| -  Geometric locations can include displacements.  By default, 
 this is ON.  The nodal positions are 'displaced' by the
 standard exodus displacment vector. If displacements
 are turned 'off', the user can explicitly add them by
 applying a warp filter.

\item  \verb|string = obj.GetTitle ()| -  Access to meta data generated by UpdateInformation.

\item  \verb|int = obj.GetDimensionality ()| -  Access to meta data generated by UpdateInformation.

\item  \verb|int = obj.GetNumberOfTimeSteps ()| -  Access to meta data generated by UpdateInformation.

\item  \verb|int = obj.GetNumberOfElements ()| -  Access to meta data generated by UpdateInformation.

\item  \verb|int = obj.GetNumberOfNodeSets ()| -  Access to meta data generated by UpdateInformation.

\item  \verb|int = obj.GetNumberOfSideSets ()| -  Access to meta data generated by UpdateInformation.

\item  \verb|int = obj.GetNumberOfBlocks ()| -  Access to meta data generated by UpdateInformation.

\item  \verb|int = obj. GetTimeStepRange ()| -  Access to meta data generated by UpdateInformation.

\item  \verb|obj.SetTimeStepRange (int , int )| -  Access to meta data generated by UpdateInformation.

\item  \verb|obj.SetTimeStepRange (int  a[2])| -  Access to meta data generated by UpdateInformation.

\item  \verb|int = obj.GetNumberOfNodes ()| -  Access to meta data generated by UpdateInformation.

\item  \verb|int = obj.GetNumberOfElementsInBlock (int block\_idx)| -  Access to meta data generated by UpdateInformation.

\item  \verb|int = obj.GetBlockId (int block\_idx)| -  Access to meta data generated by UpdateInformation.

\item  \verb|int = obj.GetTotalNumberOfNodes ()|

\item  \verb|int = obj.GetNumberOfPointArrays ()|

\item  \verb|string = obj.GetPointArrayName (int index)|

\item  \verb|int = obj.GetPointArrayID (string name)|

\item  \verb|int = obj.GetPointArrayNumberOfComponents (int index)|

\item  \verb|obj.SetPointArrayStatus (int index, int flag)|

\item  \verb|obj.SetPointArrayStatus (string , int flag)|

\item  \verb|int = obj.GetPointArrayStatus (int index)|

\item  \verb|int = obj.GetPointArrayStatus (string )|

\item  \verb|int = obj.GetNumberOfCellArrays ()|

\item  \verb|string = obj.GetCellArrayName (int index)|

\item  \verb|int = obj.GetCellArrayID (string name)|

\item  \verb|int = obj.GetCellArrayNumberOfComponents (int index)|

\item  \verb|obj.SetCellArrayStatus (int index, int flag)|

\item  \verb|obj.SetCellArrayStatus (string , int flag)|

\item  \verb|int = obj.GetCellArrayStatus (int index)|

\item  \verb|int = obj.GetCellArrayStatus (string )|

\item  \verb|int = obj.GetTotalNumberOfElements ()|

\item  \verb|int = obj.GetNumberOfBlockArrays ()|

\item  \verb|string = obj.GetBlockArrayName (int index)|

\item  \verb|int = obj.GetBlockArrayID (string name)|

\item  \verb|obj.SetBlockArrayStatus (int index, int flag)|

\item  \verb|obj.SetBlockArrayStatus (string , int flag)|

\item  \verb|int = obj.GetBlockArrayStatus (int index)|

\item  \verb|int = obj.GetBlockArrayStatus (string )|

\item  \verb|int = obj.GetNumberOfNodeSetArrays ()| -  By default Node/Side sets are not loaded, These methods allow the user to
 select which Node/Side sets they want to load. NumberOfNodeSets and
 NumberOfSideSets (set by vtk macros) are stored in vtkExodusReader 
 but other Node/Side set metadata are stored in vtkExodusMetaData
 Note: GetNumberOfNodeSetArrays and GetNumberOfSideSetArrays are
 just syntatic sugar for paraview server xml

\item  \verb|int = obj.GetNodeSetArrayStatus (int index)| -  By default Node/Side sets are not loaded, These methods allow the user to
 select which Node/Side sets they want to load. NumberOfNodeSets and
 NumberOfSideSets (set by vtk macros) are stored in vtkExodusReader 
 but other Node/Side set metadata are stored in vtkExodusMetaData
 Note: GetNumberOfNodeSetArrays and GetNumberOfSideSetArrays are
 just syntatic sugar for paraview server xml

\item  \verb|int = obj.GetNodeSetArrayStatus (string name)| -  By default Node/Side sets are not loaded, These methods allow the user to
 select which Node/Side sets they want to load. NumberOfNodeSets and
 NumberOfSideSets (set by vtk macros) are stored in vtkExodusReader 
 but other Node/Side set metadata are stored in vtkExodusMetaData
 Note: GetNumberOfNodeSetArrays and GetNumberOfSideSetArrays are
 just syntatic sugar for paraview server xml

\item  \verb|obj.SetNodeSetArrayStatus (int index, int flag)| -  By default Node/Side sets are not loaded, These methods allow the user to
 select which Node/Side sets they want to load. NumberOfNodeSets and
 NumberOfSideSets (set by vtk macros) are stored in vtkExodusReader 
 but other Node/Side set metadata are stored in vtkExodusMetaData
 Note: GetNumberOfNodeSetArrays and GetNumberOfSideSetArrays are
 just syntatic sugar for paraview server xml

\item  \verb|obj.SetNodeSetArrayStatus (string name, int flag)| -  By default Node/Side sets are not loaded, These methods allow the user to
 select which Node/Side sets they want to load. NumberOfNodeSets and
 NumberOfSideSets (set by vtk macros) are stored in vtkExodusReader 
 but other Node/Side set metadata are stored in vtkExodusMetaData
 Note: GetNumberOfNodeSetArrays and GetNumberOfSideSetArrays are
 just syntatic sugar for paraview server xml

\item  \verb|string = obj.GetNodeSetArrayName (int index)| -  By default Node/Side sets are not loaded, These methods allow the user to
 select which Node/Side sets they want to load. NumberOfNodeSets and
 NumberOfSideSets (set by vtk macros) are stored in vtkExodusReader 
 but other Node/Side set metadata are stored in vtkExodusMetaData
 Note: GetNumberOfNodeSetArrays and GetNumberOfSideSetArrays are
 just syntatic sugar for paraview server xml

\item  \verb|int = obj.GetNumberOfSideSetArrays ()|

\item  \verb|int = obj.GetSideSetArrayStatus (int index)|

\item  \verb|int = obj.GetSideSetArrayStatus (string name)|

\item  \verb|obj.SetSideSetArrayStatus (int index, int flag)|

\item  \verb|obj.SetSideSetArrayStatus (string name, int flag)|

\item  \verb|string = obj.GetSideSetArrayName (int index)|

\item  \verb|int = obj.GetNumberOfPartArrays ()|

\item  \verb|string = obj.GetPartArrayName (int arrayIdx)|

\item  \verb|int = obj.GetPartArrayID (string name)|

\item  \verb|string = obj.GetPartBlockInfo (int arrayIdx)|

\item  \verb|obj.SetPartArrayStatus (int index, int flag)|

\item  \verb|obj.SetPartArrayStatus (string , int flag)|

\item  \verb|int = obj.GetPartArrayStatus (int index)|

\item  \verb|int = obj.GetPartArrayStatus (string )|

\item  \verb|int = obj.GetNumberOfMaterialArrays ()|

\item  \verb|string = obj.GetMaterialArrayName (int arrayIdx)|

\item  \verb|int = obj.GetMaterialArrayID (string name)|

\item  \verb|obj.SetMaterialArrayStatus (int index, int flag)|

\item  \verb|obj.SetMaterialArrayStatus (string , int flag)|

\item  \verb|int = obj.GetMaterialArrayStatus (int index)|

\item  \verb|int = obj.GetMaterialArrayStatus (string )|

\item  \verb|int = obj.GetNumberOfAssemblyArrays ()|

\item  \verb|string = obj.GetAssemblyArrayName (int arrayIdx)|

\item  \verb|int = obj.GetAssemblyArrayID (string name)|

\item  \verb|obj.SetAssemblyArrayStatus (int index, int flag)|

\item  \verb|obj.SetAssemblyArrayStatus (string , int flag)|

\item  \verb|int = obj.GetAssemblyArrayStatus (int index)|

\item  \verb|int = obj.GetAssemblyArrayStatus (string )|

\item  \verb|int = obj.GetNumberOfHierarchyArrays ()|

\item  \verb|string = obj.GetHierarchyArrayName (int arrayIdx)|

\item  \verb|obj.SetHierarchyArrayStatus (int index, int flag)|

\item  \verb|obj.SetHierarchyArrayStatus (string , int flag)|

\item  \verb|int = obj.GetHierarchyArrayStatus (int index)|

\item  \verb|int = obj.GetHierarchyArrayStatus (string )|

\item  \verb|int = obj.GetHasModeShapes ()| -  Some simulations overload the Exodus time steps to represent mode shapes.
 In this case, it does not make sense to iterate over the ''time steps'',
 because they are not meant to be played in order.  Rather, each represents
 the vibration at a different ''mode.''  Setting this to 1 changes the
 semantics of the reader to not report the time steps to downstream filters.
 By default, this is off, which is the case for most Exodus files.

\item  \verb|obj.SetHasModeShapes (int )| -  Some simulations overload the Exodus time steps to represent mode shapes.
 In this case, it does not make sense to iterate over the ''time steps'',
 because they are not meant to be played in order.  Rather, each represents
 the vibration at a different ''mode.''  Setting this to 1 changes the
 semantics of the reader to not report the time steps to downstream filters.
 By default, this is off, which is the case for most Exodus files.

\item  \verb|obj.HasModeShapesOn ()| -  Some simulations overload the Exodus time steps to represent mode shapes.
 In this case, it does not make sense to iterate over the ''time steps'',
 because they are not meant to be played in order.  Rather, each represents
 the vibration at a different ''mode.''  Setting this to 1 changes the
 semantics of the reader to not report the time steps to downstream filters.
 By default, this is off, which is the case for most Exodus files.

\item  \verb|obj.HasModeShapesOff ()| -  Some simulations overload the Exodus time steps to represent mode shapes.
 In this case, it does not make sense to iterate over the ''time steps'',
 because they are not meant to be played in order.  Rather, each represents
 the vibration at a different ''mode.''  Setting this to 1 changes the
 semantics of the reader to not report the time steps to downstream filters.
 By default, this is off, which is the case for most Exodus files.

\item  \verb|int = obj.GetDisplayType ()|

\item  \verb|obj.SetDisplayType (int type)|

\item  \verb|obj.ExodusModelMetadataOn ()|

\item  \verb|obj.ExodusModelMetadataOff ()|

\item  \verb|obj.SetExodusModelMetadata (int )|

\item  \verb|int = obj.GetExodusModelMetadata ()|

\item  \verb|vtkExodusModel = obj.GetExodusModel ()|

\item  \verb|obj.SetPackExodusModelOntoOutput (int )|

\item  \verb|int = obj.GetPackExodusModelOntoOutput ()|

\item  \verb|obj.PackExodusModelOntoOutputOn ()|

\item  \verb|obj.PackExodusModelOntoOutputOff ()|

\item  \verb|int = obj.IsValidVariable (string type, string name)|

\item  \verb|int = obj.GetVariableID (string type, string name)|

\item  \verb|obj.SetAllAssemblyArrayStatus (int status)|

\item  \verb|obj.SetAllBlockArrayStatus (int status)|

\item  \verb|obj.SetAllCellArrayStatus (int status)|

\item  \verb|obj.SetAllHierarchyArrayStatus (int status)|

\item  \verb|obj.SetAllMaterialArrayStatus (int status)|

\item  \verb|obj.SetAllPartArrayStatus (int status)|

\item  \verb|obj.SetAllPointArrayStatus (int status)|

\item  \verb|obj.SetArrayStatus (string type, string name, int flag)|

\item  \verb|int = obj.GetArrayStatus (string type, string name)|

\item  \verb|int = obj.GetTimeSeriesData (int ID, string vName, string vType, vtkFloatArray result)|

\item  \verb|int = obj.GetNumberOfVariableArrays ()|

\item  \verb|string = obj.GetVariableArrayName (int a\_which)|

\item  \verb|obj.EnableDSPFiltering ()|

\item  \verb|obj.AddFilter (vtkDSPFilterDefinition a\_filter)|

\item  \verb|obj.StartAddingFilter ()|

\item  \verb|obj.AddFilterInputVar (string name)|

\item  \verb|obj.AddFilterOutputVar (string name)|

\item  \verb|obj.AddFilterNumeratorWeight (double weight)|

\item  \verb|obj.AddFilterForwardNumeratorWeight (double weight)|

\item  \verb|obj.AddFilterDenominatorWeight (double weight)|

\item  \verb|obj.FinishAddingFilter ()|

\item  \verb|obj.RemoveFilter (string a\_outputVariableName)|

\item  \verb|obj.GetDSPOutputArrays (int exoid, vtkUnstructuredGrid output)|

\end{itemize}
