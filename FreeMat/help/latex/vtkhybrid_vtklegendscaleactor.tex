\section{vtkLegendScaleActor}

\subsection{Usage}

 This class is used to annotate the render window. Its basic goal is to
 provide an indication of the scale of the scene. Four axes surrounding the
 render window indicate (in a variety of ways) the scale of what the camera
 is viewing. An option also exists for displaying a scale legend.

 The axes can be programmed either to display distance scales or x-y
 coordinate values. By default, the scales display a distance. However,
 if you know that the view is down the z-axis, the scales can be programmed
 to display x-y coordinate values.


To create an instance of class vtkLegendScaleActor, simply
invoke its constructor as follows
\begin{verbatim}
  obj = vtkLegendScaleActor
\end{verbatim}
\subsection{Methods}

The class vtkLegendScaleActor has several methods that can be used.
  They are listed below.
Note that the documentation is translated automatically from the VTK sources,
and may not be completely intelligible.  When in doubt, consult the VTK website.
In the methods listed below, \verb|obj| is an instance of the vtkLegendScaleActor class.
\begin{itemize}
\item  \verb|string = obj.GetClassName ()| -  Standard methods for the class.

\item  \verb|int = obj.IsA (string name)| -  Standard methods for the class.

\item  \verb|vtkLegendScaleActor = obj.NewInstance ()| -  Standard methods for the class.

\item  \verb|vtkLegendScaleActor = obj.SafeDownCast (vtkObject o)| -  Standard methods for the class.

\item  \verb|obj.SetLabelMode (int )| -  Specify the mode for labeling the scale axes. By default, the axes are
 labeled with the distance between points (centered at a distance of
 0.0). Alternatively if you know that the view is down the z-axis; the
 axes can be labeled with x-y coordinate values.

\item  \verb|int = obj.GetLabelModeMinValue ()| -  Specify the mode for labeling the scale axes. By default, the axes are
 labeled with the distance between points (centered at a distance of
 0.0). Alternatively if you know that the view is down the z-axis; the
 axes can be labeled with x-y coordinate values.

\item  \verb|int = obj.GetLabelModeMaxValue ()| -  Specify the mode for labeling the scale axes. By default, the axes are
 labeled with the distance between points (centered at a distance of
 0.0). Alternatively if you know that the view is down the z-axis; the
 axes can be labeled with x-y coordinate values.

\item  \verb|int = obj.GetLabelMode ()| -  Specify the mode for labeling the scale axes. By default, the axes are
 labeled with the distance between points (centered at a distance of
 0.0). Alternatively if you know that the view is down the z-axis; the
 axes can be labeled with x-y coordinate values.

\item  \verb|obj.SetLabelModeToDistance ()| -  Specify the mode for labeling the scale axes. By default, the axes are
 labeled with the distance between points (centered at a distance of
 0.0). Alternatively if you know that the view is down the z-axis; the
 axes can be labeled with x-y coordinate values.

\item  \verb|obj.SetLabelModeToXYCoordinates ()| -  Set/Get the flags that control which of the four axes to display (top,
 bottom, left and right). By default, all the axes are displayed.

\item  \verb|obj.SetRightAxisVisibility (int )| -  Set/Get the flags that control which of the four axes to display (top,
 bottom, left and right). By default, all the axes are displayed.

\item  \verb|int = obj.GetRightAxisVisibility ()| -  Set/Get the flags that control which of the four axes to display (top,
 bottom, left and right). By default, all the axes are displayed.

\item  \verb|obj.RightAxisVisibilityOn ()| -  Set/Get the flags that control which of the four axes to display (top,
 bottom, left and right). By default, all the axes are displayed.

\item  \verb|obj.RightAxisVisibilityOff ()| -  Set/Get the flags that control which of the four axes to display (top,
 bottom, left and right). By default, all the axes are displayed.

\item  \verb|obj.SetTopAxisVisibility (int )| -  Set/Get the flags that control which of the four axes to display (top,
 bottom, left and right). By default, all the axes are displayed.

\item  \verb|int = obj.GetTopAxisVisibility ()| -  Set/Get the flags that control which of the four axes to display (top,
 bottom, left and right). By default, all the axes are displayed.

\item  \verb|obj.TopAxisVisibilityOn ()| -  Set/Get the flags that control which of the four axes to display (top,
 bottom, left and right). By default, all the axes are displayed.

\item  \verb|obj.TopAxisVisibilityOff ()| -  Set/Get the flags that control which of the four axes to display (top,
 bottom, left and right). By default, all the axes are displayed.

\item  \verb|obj.SetLeftAxisVisibility (int )| -  Set/Get the flags that control which of the four axes to display (top,
 bottom, left and right). By default, all the axes are displayed.

\item  \verb|int = obj.GetLeftAxisVisibility ()| -  Set/Get the flags that control which of the four axes to display (top,
 bottom, left and right). By default, all the axes are displayed.

\item  \verb|obj.LeftAxisVisibilityOn ()| -  Set/Get the flags that control which of the four axes to display (top,
 bottom, left and right). By default, all the axes are displayed.

\item  \verb|obj.LeftAxisVisibilityOff ()| -  Set/Get the flags that control which of the four axes to display (top,
 bottom, left and right). By default, all the axes are displayed.

\item  \verb|obj.SetBottomAxisVisibility (int )| -  Set/Get the flags that control which of the four axes to display (top,
 bottom, left and right). By default, all the axes are displayed.

\item  \verb|int = obj.GetBottomAxisVisibility ()| -  Set/Get the flags that control which of the four axes to display (top,
 bottom, left and right). By default, all the axes are displayed.

\item  \verb|obj.BottomAxisVisibilityOn ()| -  Set/Get the flags that control which of the four axes to display (top,
 bottom, left and right). By default, all the axes are displayed.

\item  \verb|obj.BottomAxisVisibilityOff ()| -  Set/Get the flags that control which of the four axes to display (top,
 bottom, left and right). By default, all the axes are displayed.

\item  \verb|obj.SetLegendVisibility (int )| -  Indicate whether the legend scale should be displayed or not.
 The default is On.

\item  \verb|int = obj.GetLegendVisibility ()| -  Indicate whether the legend scale should be displayed or not.
 The default is On.

\item  \verb|obj.LegendVisibilityOn ()| -  Indicate whether the legend scale should be displayed or not.
 The default is On.

\item  \verb|obj.LegendVisibilityOff ()| -  Indicate whether the legend scale should be displayed or not.
 The default is On.

\item  \verb|obj.AllAxesOn ()| -  Convenience method that turns all the axes either on or off.

\item  \verb|obj.AllAxesOff ()| -  Convenience method that turns all the axes either on or off.

\item  \verb|obj.AllAnnotationsOn ()| -  Convenience method that turns all the axes and the legend scale.

\item  \verb|obj.AllAnnotationsOff ()| -  Convenience method that turns all the axes and the legend scale.

\item  \verb|obj.SetRightBorderOffset (int )| -  Set/Get the offset of the right axis from the border. This number is expressed in
 pixels, and represents the approximate distance of the axes from the sides
 of the renderer. The default is 50.

\item  \verb|int = obj.GetRightBorderOffsetMinValue ()| -  Set/Get the offset of the right axis from the border. This number is expressed in
 pixels, and represents the approximate distance of the axes from the sides
 of the renderer. The default is 50.

\item  \verb|int = obj.GetRightBorderOffsetMaxValue ()| -  Set/Get the offset of the right axis from the border. This number is expressed in
 pixels, and represents the approximate distance of the axes from the sides
 of the renderer. The default is 50.

\item  \verb|int = obj.GetRightBorderOffset ()| -  Set/Get the offset of the right axis from the border. This number is expressed in
 pixels, and represents the approximate distance of the axes from the sides
 of the renderer. The default is 50.

\item  \verb|obj.SetTopBorderOffset (int )| -  Set/Get the offset of the top axis from the border. This number is expressed in
 pixels, and represents the approximate distance of the axes from the sides
 of the renderer. The default is 30.

\item  \verb|int = obj.GetTopBorderOffsetMinValue ()| -  Set/Get the offset of the top axis from the border. This number is expressed in
 pixels, and represents the approximate distance of the axes from the sides
 of the renderer. The default is 30.

\item  \verb|int = obj.GetTopBorderOffsetMaxValue ()| -  Set/Get the offset of the top axis from the border. This number is expressed in
 pixels, and represents the approximate distance of the axes from the sides
 of the renderer. The default is 30.

\item  \verb|int = obj.GetTopBorderOffset ()| -  Set/Get the offset of the top axis from the border. This number is expressed in
 pixels, and represents the approximate distance of the axes from the sides
 of the renderer. The default is 30.

\item  \verb|obj.SetLeftBorderOffset (int )| -  Set/Get the offset of the left axis from the border. This number is expressed in
 pixels, and represents the approximate distance of the axes from the sides
 of the renderer. The default is 50.

\item  \verb|int = obj.GetLeftBorderOffsetMinValue ()| -  Set/Get the offset of the left axis from the border. This number is expressed in
 pixels, and represents the approximate distance of the axes from the sides
 of the renderer. The default is 50.

\item  \verb|int = obj.GetLeftBorderOffsetMaxValue ()| -  Set/Get the offset of the left axis from the border. This number is expressed in
 pixels, and represents the approximate distance of the axes from the sides
 of the renderer. The default is 50.

\item  \verb|int = obj.GetLeftBorderOffset ()| -  Set/Get the offset of the left axis from the border. This number is expressed in
 pixels, and represents the approximate distance of the axes from the sides
 of the renderer. The default is 50.

\item  \verb|obj.SetBottomBorderOffset (int )| -  Set/Get the offset of the bottom axis from the border. This number is expressed in
 pixels, and represents the approximate distance of the axes from the sides
 of the renderer. The default is 30.

\item  \verb|int = obj.GetBottomBorderOffsetMinValue ()| -  Set/Get the offset of the bottom axis from the border. This number is expressed in
 pixels, and represents the approximate distance of the axes from the sides
 of the renderer. The default is 30.

\item  \verb|int = obj.GetBottomBorderOffsetMaxValue ()| -  Set/Get the offset of the bottom axis from the border. This number is expressed in
 pixels, and represents the approximate distance of the axes from the sides
 of the renderer. The default is 30.

\item  \verb|int = obj.GetBottomBorderOffset ()| -  Set/Get the offset of the bottom axis from the border. This number is expressed in
 pixels, and represents the approximate distance of the axes from the sides
 of the renderer. The default is 30.

\item  \verb|obj.SetCornerOffsetFactor (double )| -  Get/Set the corner offset. This is the offset factor used to offset the
 axes at the corners. Default value is 2.0.

\item  \verb|double = obj.GetCornerOffsetFactorMinValue ()| -  Get/Set the corner offset. This is the offset factor used to offset the
 axes at the corners. Default value is 2.0.

\item  \verb|double = obj.GetCornerOffsetFactorMaxValue ()| -  Get/Set the corner offset. This is the offset factor used to offset the
 axes at the corners. Default value is 2.0.

\item  \verb|double = obj.GetCornerOffsetFactor ()| -  Get/Set the corner offset. This is the offset factor used to offset the
 axes at the corners. Default value is 2.0.

\item  \verb|vtkTextProperty = obj.GetLegendTitleProperty ()| -  Set/Get the labels text properties for the legend title and labels.

\item  \verb|vtkTextProperty = obj.GetLegendLabelProperty ()| -  Set/Get the labels text properties for the legend title and labels.

\item  \verb|vtkAxisActor2D = obj.GetRightAxis ()| -  These are methods to retrieve the vtkAxisActors used to represent
 the four axes that form this representation. Users may retrieve and
 then modify these axes to control their appearance.

\item  \verb|vtkAxisActor2D = obj.GetTopAxis ()| -  These are methods to retrieve the vtkAxisActors used to represent
 the four axes that form this representation. Users may retrieve and
 then modify these axes to control their appearance.

\item  \verb|vtkAxisActor2D = obj.GetLeftAxis ()| -  These are methods to retrieve the vtkAxisActors used to represent
 the four axes that form this representation. Users may retrieve and
 then modify these axes to control their appearance.

\item  \verb|vtkAxisActor2D = obj.GetBottomAxis ()| -  These are methods to retrieve the vtkAxisActors used to represent
 the four axes that form this representation. Users may retrieve and
 then modify these axes to control their appearance.

\item  \verb|obj.BuildRepresentation (vtkViewport viewport)|

\item  \verb|obj.GetActors2D (vtkPropCollection )|

\item  \verb|obj.ReleaseGraphicsResources (vtkWindow )|

\item  \verb|int = obj.RenderOverlay (vtkViewport )|

\item  \verb|int = obj.RenderOpaqueGeometry (vtkViewport )|

\end{itemize}
