\section{vtkPExodusReader}

\subsection{Usage}

 vtkPExodusReader is a unstructured grid source object that reads
 PExodusReaderII files. Most of the meta data associated with the
 file is loaded when UpdateInformation is called. This includes
 information like Title, number of blocks, number and names of
 arrays. This data can be retrieved from methods in this
 reader. Separate arrays that are meant to be a single vector, are
 combined internally for convenience. To be combined, the array
 names have to be identical except for a trailing X,Y and Z (or
 x,y,z). By default all cell and point arrays are loaded. However,
 the user can flag arrays not to load with the methods
 ''SetPointDataArrayLoadFlag'' and ''SetCellDataArrayLoadFlag''. The
 reader responds to piece requests by loading only a range of the
 possible blocks. Unused points are filtered out internally.

To create an instance of class vtkPExodusReader, simply
invoke its constructor as follows
\begin{verbatim}
  obj = vtkPExodusReader
\end{verbatim}
\subsection{Methods}

The class vtkPExodusReader has several methods that can be used.
  They are listed below.
Note that the documentation is translated automatically from the VTK sources,
and may not be completely intelligible.  When in doubt, consult the VTK website.
In the methods listed below, \verb|obj| is an instance of the vtkPExodusReader class.
\begin{itemize}
\item  \verb|string = obj.GetClassName ()|

\item  \verb|int = obj.IsA (string name)|

\item  \verb|vtkPExodusReader = obj.NewInstance ()|

\item  \verb|vtkPExodusReader = obj.SafeDownCast (vtkObject o)|

\item  \verb|obj.SetFilePattern (string )| -  These methods tell the reader that the data is ditributed across
 multiple files. This is for distributed execution. It this case,
 pieces are mapped to files. The pattern should have one %d to
 format the file number. FileNumberRange is used to generate file
 numbers. I was thinking of having an arbitrary list of file
 numbers. This may happen in the future. (That is why there is no
 GetFileNumberRange method.

\item  \verb|string = obj.GetFilePattern ()| -  These methods tell the reader that the data is ditributed across
 multiple files. This is for distributed execution. It this case,
 pieces are mapped to files. The pattern should have one %d to
 format the file number. FileNumberRange is used to generate file
 numbers. I was thinking of having an arbitrary list of file
 numbers. This may happen in the future. (That is why there is no
 GetFileNumberRange method.

\item  \verb|obj.SetFilePrefix (string )| -  These methods tell the reader that the data is ditributed across
 multiple files. This is for distributed execution. It this case,
 pieces are mapped to files. The pattern should have one %d to
 format the file number. FileNumberRange is used to generate file
 numbers. I was thinking of having an arbitrary list of file
 numbers. This may happen in the future. (That is why there is no
 GetFileNumberRange method.

\item  \verb|string = obj.GetFilePrefix ()| -  These methods tell the reader that the data is ditributed across
 multiple files. This is for distributed execution. It this case,
 pieces are mapped to files. The pattern should have one %d to
 format the file number. FileNumberRange is used to generate file
 numbers. I was thinking of having an arbitrary list of file
 numbers. This may happen in the future. (That is why there is no
 GetFileNumberRange method.

\item  \verb|obj.SetFileRange (int , int )| -  Set the range of files that are being loaded. The range for single
 file should add to 0.

\item  \verb|obj.SetFileRange (int r)| -  Set the range of files that are being loaded. The range for single
 file should add to 0.

\item  \verb|int = obj. GetFileRange ()| -  Set the range of files that are being loaded. The range for single
 file should add to 0.

\item  \verb|obj.SetFileName (string name)|

\item  \verb|int = obj.GetNumberOfFileNames ()| -    Return the number of files to be read.

\item  \verb|int = obj.GetNumberOfFiles ()| -    Return the number of files to be read.

\item  \verb|obj.SetGenerateFileIdArray (int flag)|

\item  \verb|int = obj.GetGenerateFileIdArray ()|

\item  \verb|obj.GenerateFileIdArrayOn ()|

\item  \verb|obj.GenerateFileIdArrayOff ()|

\item  \verb|int = obj.GetTotalNumberOfElements ()|

\item  \verb|int = obj.GetTotalNumberOfNodes ()|

\item  \verb|int = obj.GetNumberOfVariableArrays ()|

\item  \verb|string = obj.GetVariableArrayName (int a\_which)|

\item  \verb|obj.EnableDSPFiltering ()|

\item  \verb|obj.AddFilter (vtkDSPFilterDefinition a\_filter)|

\item  \verb|obj.StartAddingFilter ()|

\item  \verb|obj.AddFilterInputVar (string name)|

\item  \verb|obj.AddFilterOutputVar (string name)|

\item  \verb|obj.AddFilterNumeratorWeight (double weight)|

\item  \verb|obj.AddFilterForwardNumeratorWeight (double weight)|

\item  \verb|obj.AddFilterDenominatorWeight (double weight)|

\item  \verb|obj.FinishAddingFilter ()|

\item  \verb|obj.RemoveFilter (string a\_outputVariableName)|

\end{itemize}
