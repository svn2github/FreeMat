\section{vtkProcrustesAlignmentFilter}

\subsection{Usage}

 
 vtkProcrustesAlignmentFilter is a filter that takes a set of pointsets
 (any object derived from vtkPointSet) and aligns them in a least-squares 
 sense to their mutual mean. The algorithm is iterated until convergence, 
 as the mean must be recomputed after each alignment. 

 Call SetNumberOfInputs(n) before calling SetInput(0) ... SetInput(n-1).

 Retrieve the outputs using GetOutput(0) ... GetOutput(n-1).

 The default (in vtkLandmarkTransform) is for a similarity alignment.
 For a rigid-body alignment (to build a 'size-and-shape' model) use: 

    GetLandmarkTransform()->SetModeToRigidBody(). 

 Affine alignments are not normally used but are left in for completeness:

    GetLandmarkTransform()->SetModeToAffine(). 

 vtkProcrustesAlignmentFilter is an implementation of:

    J.C. Gower (1975) 
    Generalized Procrustes Analysis. Psychometrika, 40:33-51.


To create an instance of class vtkProcrustesAlignmentFilter, simply
invoke its constructor as follows
\begin{verbatim}
  obj = vtkProcrustesAlignmentFilter
\end{verbatim}
\subsection{Methods}

The class vtkProcrustesAlignmentFilter has several methods that can be used.
  They are listed below.
Note that the documentation is translated automatically from the VTK sources,
and may not be completely intelligible.  When in doubt, consult the VTK website.
In the methods listed below, \verb|obj| is an instance of the vtkProcrustesAlignmentFilter class.
\begin{itemize}
\item  \verb|string = obj.GetClassName ()|

\item  \verb|int = obj.IsA (string name)|

\item  \verb|vtkProcrustesAlignmentFilter = obj.NewInstance ()|

\item  \verb|vtkProcrustesAlignmentFilter = obj.SafeDownCast (vtkObject o)|

\item  \verb|vtkLandmarkTransform = obj.GetLandmarkTransform ()| -  Get the internal landmark transform. Use it to constrain the number of
 degrees of freedom of the alignment (i.e. rigid body, similarity, etc.).
 The default is a similarity alignment.

\item  \verb|vtkPoints = obj.GetMeanPoints ()| -  Get the estimated mean point cloud

\item  \verb|obj.SetNumberOfInputs (int n)| -  Specify how many pointsets are going to be given as input.

\item  \verb|obj.SetInput (int idx, vtkPointSet p)| -  Specify the input pointset with index idx.
 Call SetNumberOfInputs before calling this function.

\item  \verb|obj.SetInput (int idx, vtkDataObject input)| -  Specify the input pointset with index idx.
 Call SetNumberOfInputs before calling this function.

\item  \verb|obj.SetStartFromCentroid (bool )| -  When on, the initial alignment is to the centroid 
 of the cohort curves.  When off, the alignment is to the 
 centroid of the first input.  Default is off for
 backward compatibility.

\item  \verb|bool = obj.GetStartFromCentroid ()| -  When on, the initial alignment is to the centroid 
 of the cohort curves.  When off, the alignment is to the 
 centroid of the first input.  Default is off for
 backward compatibility.

\item  \verb|obj.StartFromCentroidOn ()| -  When on, the initial alignment is to the centroid 
 of the cohort curves.  When off, the alignment is to the 
 centroid of the first input.  Default is off for
 backward compatibility.

\item  \verb|obj.StartFromCentroidOff ()| -  When on, the initial alignment is to the centroid 
 of the cohort curves.  When off, the alignment is to the 
 centroid of the first input.  Default is off for
 backward compatibility.

\item  \verb|vtkPointSet = obj.GetInput (int idx)| -  Retrieve the input point set with index idx (usually only for pipeline
 tracing).

\end{itemize}
