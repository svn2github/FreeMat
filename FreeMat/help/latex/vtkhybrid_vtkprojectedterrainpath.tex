\section{vtkProjectedTerrainPath}

\subsection{Usage}

 vtkProjectedTerrainPath projects an input polyline onto a terrain. (The
 terrain is defined by a 2D height image and is the second input to the
 filter.) The polyline projection is controlled via several modes as
 follows. 1) Simple mode projects the polyline points onto the terrain,
 taking into account the height offset instance variable. 2) Non-occluded
 mode insures that no parts of the polyline are occluded by the terrain
 (e.g. a line passes through a mountain). This may require recursive
 subdivision of the polyline. 3) Hug mode insures that the polyine points
 remain within a constant distance from the surface. This may also require
 recursive subdivision of the polyline. Note that both non-occluded mode
 and hug mode also take into account the height offset, so it is possible 
 to create paths that hug terrain a certain distance above it. To use this
 filter, define two inputs: 1) a polyline, and 2) an image whose scalar
 values represent a height field. Then specify the mode, and the height
 offset to use.

 An description of the algorithm is as follows. The filter begins by
 projecting the polyline points to the image (offset by the specified
 height offset).  If the mode is non-occluded or hug, then the maximum
 error along each line segment is computed and placed into a priority
 queue. Each line segment is then split at the point of maximum error, and
 the two new line segments are evaluated for maximum error. This process
 continues until the line is not occluded by the terrain (non-occluded
 mode) or satisfies the error on variation from the surface (hug
 mode). (Note this process is repeated for each polyline in the
 input. Also, the maximum error is computed in two parts: a maximum
 positive error and maximum negative error. If the polyline is above the
 terrain--i.e., the height offset is positive--in non-occluded or hug mode
 all negative errors are eliminated. If the polyline is below the
 terrain--i.e., the height offset is negative--in non-occluded or hug mode
 all positive errors are eliminated.)
 

To create an instance of class vtkProjectedTerrainPath, simply
invoke its constructor as follows
\begin{verbatim}
  obj = vtkProjectedTerrainPath
\end{verbatim}
\subsection{Methods}

The class vtkProjectedTerrainPath has several methods that can be used.
  They are listed below.
Note that the documentation is translated automatically from the VTK sources,
and may not be completely intelligible.  When in doubt, consult the VTK website.
In the methods listed below, \verb|obj| is an instance of the vtkProjectedTerrainPath class.
\begin{itemize}
\item  \verb|string = obj.GetClassName ()| -  Standard methids for printing and determining type information.

\item  \verb|int = obj.IsA (string name)| -  Standard methids for printing and determining type information.

\item  \verb|vtkProjectedTerrainPath = obj.NewInstance ()| -  Standard methids for printing and determining type information.

\item  \verb|vtkProjectedTerrainPath = obj.SafeDownCast (vtkObject o)| -  Standard methids for printing and determining type information.

\item  \verb|obj.SetSource (vtkImageData source)| -  Specify the second input (the terrain) onto which the polyline(s) should
 be projected.

\item  \verb|vtkImageData = obj.GetSource ()| -  Specify the second input (the terrain) onto which the polyline(s) should
 be projected.

\item  \verb|obj.SetProjectionMode (int )| -  Determine how to control the projection process. Simple projection
 just projects the original polyline points. Non-occluded projection
 insures that the polyline does not intersect the terrain surface.
 Hug projection is similar to non-occulded projection except that
 produces a path that is nearly parallel to the terrain (within the
 user specified height tolerance).

\item  \verb|int = obj.GetProjectionModeMinValue ()| -  Determine how to control the projection process. Simple projection
 just projects the original polyline points. Non-occluded projection
 insures that the polyline does not intersect the terrain surface.
 Hug projection is similar to non-occulded projection except that
 produces a path that is nearly parallel to the terrain (within the
 user specified height tolerance).

\item  \verb|int = obj.GetProjectionModeMaxValue ()| -  Determine how to control the projection process. Simple projection
 just projects the original polyline points. Non-occluded projection
 insures that the polyline does not intersect the terrain surface.
 Hug projection is similar to non-occulded projection except that
 produces a path that is nearly parallel to the terrain (within the
 user specified height tolerance).

\item  \verb|int = obj.GetProjectionMode ()| -  Determine how to control the projection process. Simple projection
 just projects the original polyline points. Non-occluded projection
 insures that the polyline does not intersect the terrain surface.
 Hug projection is similar to non-occulded projection except that
 produces a path that is nearly parallel to the terrain (within the
 user specified height tolerance).

\item  \verb|obj.SetProjectionModeToSimple ()| -  Determine how to control the projection process. Simple projection
 just projects the original polyline points. Non-occluded projection
 insures that the polyline does not intersect the terrain surface.
 Hug projection is similar to non-occulded projection except that
 produces a path that is nearly parallel to the terrain (within the
 user specified height tolerance).

\item  \verb|obj.SetProjectionModeToNonOccluded ()| -  Determine how to control the projection process. Simple projection
 just projects the original polyline points. Non-occluded projection
 insures that the polyline does not intersect the terrain surface.
 Hug projection is similar to non-occulded projection except that
 produces a path that is nearly parallel to the terrain (within the
 user specified height tolerance).

\item  \verb|obj.SetProjectionModeToHug ()| -  This is the height above (or below) the terrain that the projected
 path should be. Positive values indicate distances above the terrain;
 negative values indicate distances below the terrain. 

\item  \verb|obj.SetHeightOffset (double )| -  This is the height above (or below) the terrain that the projected
 path should be. Positive values indicate distances above the terrain;
 negative values indicate distances below the terrain. 

\item  \verb|double = obj.GetHeightOffset ()| -  This is the height above (or below) the terrain that the projected
 path should be. Positive values indicate distances above the terrain;
 negative values indicate distances below the terrain. 

\item  \verb|obj.SetHeightTolerance (double )| -  This is the allowable variation in the altitude of the path
 with respect to the variation in the terrain. It only comes
 into play if the hug projection mode is enabled.

\item  \verb|double = obj.GetHeightToleranceMinValue ()| -  This is the allowable variation in the altitude of the path
 with respect to the variation in the terrain. It only comes
 into play if the hug projection mode is enabled.

\item  \verb|double = obj.GetHeightToleranceMaxValue ()| -  This is the allowable variation in the altitude of the path
 with respect to the variation in the terrain. It only comes
 into play if the hug projection mode is enabled.

\item  \verb|double = obj.GetHeightTolerance ()| -  This is the allowable variation in the altitude of the path
 with respect to the variation in the terrain. It only comes
 into play if the hug projection mode is enabled.

\item  \verb|obj.SetMaximumNumberOfLines (vtkIdType )| -  This instance variable can be used to limit the total number of line
 segments created during subdivision. Note that the number of input line
 segments will be the minimum number that cab be output.

\item  \verb|vtkIdType = obj.GetMaximumNumberOfLinesMinValue ()| -  This instance variable can be used to limit the total number of line
 segments created during subdivision. Note that the number of input line
 segments will be the minimum number that cab be output.

\item  \verb|vtkIdType = obj.GetMaximumNumberOfLinesMaxValue ()| -  This instance variable can be used to limit the total number of line
 segments created during subdivision. Note that the number of input line
 segments will be the minimum number that cab be output.

\item  \verb|vtkIdType = obj.GetMaximumNumberOfLines ()| -  This instance variable can be used to limit the total number of line
 segments created during subdivision. Note that the number of input line
 segments will be the minimum number that cab be output.

\end{itemize}
