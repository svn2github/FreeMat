\section{vtkTemporalShiftScale}

\subsection{Usage}

 vtkTemporalShiftScale  modify the time range or time steps of
 the data without changing the data itself. The data is not resampled
 by this filter, only the information accompanying the data is modified.

To create an instance of class vtkTemporalShiftScale, simply
invoke its constructor as follows
\begin{verbatim}
  obj = vtkTemporalShiftScale
\end{verbatim}
\subsection{Methods}

The class vtkTemporalShiftScale has several methods that can be used.
  They are listed below.
Note that the documentation is translated automatically from the VTK sources,
and may not be completely intelligible.  When in doubt, consult the VTK website.
In the methods listed below, \verb|obj| is an instance of the vtkTemporalShiftScale class.
\begin{itemize}
\item  \verb|string = obj.GetClassName ()|

\item  \verb|int = obj.IsA (string name)|

\item  \verb|vtkTemporalShiftScale = obj.NewInstance ()|

\item  \verb|vtkTemporalShiftScale = obj.SafeDownCast (vtkObject o)|

\item  \verb|obj.SetPreShift (double )| -  Apply a translation to the data before scaling.
 To convert T{5,100} to T{0,1} use Preshift=-5, Scale=1/95, PostShift=0
 To convert T{5,105} to T{5,10} use Preshift=-5, Scale=5/100, PostShift=5

\item  \verb|double = obj.GetPreShift ()| -  Apply a translation to the data before scaling.
 To convert T{5,100} to T{0,1} use Preshift=-5, Scale=1/95, PostShift=0
 To convert T{5,105} to T{5,10} use Preshift=-5, Scale=5/100, PostShift=5

\item  \verb|obj.SetPostShift (double )| -  Apply a translation to the time

\item  \verb|double = obj.GetPostShift ()| -  Apply a translation to the time

\item  \verb|obj.SetScale (double )| -  Apply a scale to the time.

\item  \verb|double = obj.GetScale ()| -  Apply a scale to the time.

\item  \verb|obj.SetPeriodic (int )| -  If Periodic is true, requests for time will be wrapped around so that 
 the source appears to be a periodic time source. If data exists for times
 {0,N-1}, setting periodic to true will cause time 0 to be produced when time
 N, 2N, 2N etc is requested. This effectively gives the source the ability to
 generate time data indefinitely in a loop.
 When combined with Shift/Scale, the time becomes periodic in the 
 shifted and scaled time frame of reference.
 Note: Since the input time may not start at zero, the wrapping of time
 from the end of one period to the start of the next, will subtract the
 initial time - a source with T{5..6} repeated periodicaly will have output
 time {5..6..7..8} etc. 

\item  \verb|int = obj.GetPeriodic ()| -  If Periodic is true, requests for time will be wrapped around so that 
 the source appears to be a periodic time source. If data exists for times
 {0,N-1}, setting periodic to true will cause time 0 to be produced when time
 N, 2N, 2N etc is requested. This effectively gives the source the ability to
 generate time data indefinitely in a loop.
 When combined with Shift/Scale, the time becomes periodic in the 
 shifted and scaled time frame of reference.
 Note: Since the input time may not start at zero, the wrapping of time
 from the end of one period to the start of the next, will subtract the
 initial time - a source with T{5..6} repeated periodicaly will have output
 time {5..6..7..8} etc. 

\item  \verb|obj.PeriodicOn ()| -  If Periodic is true, requests for time will be wrapped around so that 
 the source appears to be a periodic time source. If data exists for times
 {0,N-1}, setting periodic to true will cause time 0 to be produced when time
 N, 2N, 2N etc is requested. This effectively gives the source the ability to
 generate time data indefinitely in a loop.
 When combined with Shift/Scale, the time becomes periodic in the 
 shifted and scaled time frame of reference.
 Note: Since the input time may not start at zero, the wrapping of time
 from the end of one period to the start of the next, will subtract the
 initial time - a source with T{5..6} repeated periodicaly will have output
 time {5..6..7..8} etc. 

\item  \verb|obj.PeriodicOff ()| -  If Periodic is true, requests for time will be wrapped around so that 
 the source appears to be a periodic time source. If data exists for times
 {0,N-1}, setting periodic to true will cause time 0 to be produced when time
 N, 2N, 2N etc is requested. This effectively gives the source the ability to
 generate time data indefinitely in a loop.
 When combined with Shift/Scale, the time becomes periodic in the 
 shifted and scaled time frame of reference.
 Note: Since the input time may not start at zero, the wrapping of time
 from the end of one period to the start of the next, will subtract the
 initial time - a source with T{5..6} repeated periodicaly will have output
 time {5..6..7..8} etc. 

\item  \verb|obj.SetPeriodicEndCorrection (int )| -  if Periodic time is enabled, this flag determines if the last time step is the same
 as the first. If PeriodicEndCorrection is true, then it is assumed that the input
 data goes from 0-1 (or whatever scaled/shifted actual time) and time 1 is the
 same as time 0 so that steps will be 0,1,2,3...N,1,2,3...N,1,2,3 where step N
 is the same as 0 and step 0 is not repeated. When this flag is false
 the data is assumed to be literal and output is of the form 0,1,2,3...N,0,1,2,3...
 By default this flag is ON

\item  \verb|int = obj.GetPeriodicEndCorrection ()| -  if Periodic time is enabled, this flag determines if the last time step is the same
 as the first. If PeriodicEndCorrection is true, then it is assumed that the input
 data goes from 0-1 (or whatever scaled/shifted actual time) and time 1 is the
 same as time 0 so that steps will be 0,1,2,3...N,1,2,3...N,1,2,3 where step N
 is the same as 0 and step 0 is not repeated. When this flag is false
 the data is assumed to be literal and output is of the form 0,1,2,3...N,0,1,2,3...
 By default this flag is ON

\item  \verb|obj.PeriodicEndCorrectionOn ()| -  if Periodic time is enabled, this flag determines if the last time step is the same
 as the first. If PeriodicEndCorrection is true, then it is assumed that the input
 data goes from 0-1 (or whatever scaled/shifted actual time) and time 1 is the
 same as time 0 so that steps will be 0,1,2,3...N,1,2,3...N,1,2,3 where step N
 is the same as 0 and step 0 is not repeated. When this flag is false
 the data is assumed to be literal and output is of the form 0,1,2,3...N,0,1,2,3...
 By default this flag is ON

\item  \verb|obj.PeriodicEndCorrectionOff ()| -  if Periodic time is enabled, this flag determines if the last time step is the same
 as the first. If PeriodicEndCorrection is true, then it is assumed that the input
 data goes from 0-1 (or whatever scaled/shifted actual time) and time 1 is the
 same as time 0 so that steps will be 0,1,2,3...N,1,2,3...N,1,2,3 where step N
 is the same as 0 and step 0 is not repeated. When this flag is false
 the data is assumed to be literal and output is of the form 0,1,2,3...N,0,1,2,3...
 By default this flag is ON

\item  \verb|obj.SetMaximumNumberOfPeriods (double )| -  if Periodic time is enabled, this controls how many time periods time is reported 
 for. A filter cannot output an infinite number of time steps and therefore a finite
 number of periods is generated when reporting time.

\item  \verb|double = obj.GetMaximumNumberOfPeriods ()| -  if Periodic time is enabled, this controls how many time periods time is reported 
 for. A filter cannot output an infinite number of time steps and therefore a finite
 number of periods is generated when reporting time.

\end{itemize}
