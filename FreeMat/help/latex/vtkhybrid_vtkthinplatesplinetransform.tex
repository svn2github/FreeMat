\section{vtkThinPlateSplineTransform}

\subsection{Usage}

 vtkThinPlateSplineTransform describes a nonlinear warp transform defined
 by a set of source and target landmarks. Any point on the mesh close to a
 source landmark will be moved to a place close to the corresponding target
 landmark. The points in between are interpolated smoothly using
 Bookstein's Thin Plate Spline algorithm.

 To obtain a correct TPS warp, use the R2LogR kernel if your data is 2D, and
 the R kernel if your data is 3D. Or you can specify your own RBF. (Hence this
 class is more general than a pure TPS transform.) 

To create an instance of class vtkThinPlateSplineTransform, simply
invoke its constructor as follows
\begin{verbatim}
  obj = vtkThinPlateSplineTransform
\end{verbatim}
\subsection{Methods}

The class vtkThinPlateSplineTransform has several methods that can be used.
  They are listed below.
Note that the documentation is translated automatically from the VTK sources,
and may not be completely intelligible.  When in doubt, consult the VTK website.
In the methods listed below, \verb|obj| is an instance of the vtkThinPlateSplineTransform class.
\begin{itemize}
\item  \verb|string = obj.GetClassName ()|

\item  \verb|int = obj.IsA (string name)|

\item  \verb|vtkThinPlateSplineTransform = obj.NewInstance ()|

\item  \verb|vtkThinPlateSplineTransform = obj.SafeDownCast (vtkObject o)|

\item  \verb|double = obj.GetSigma ()| -  Specify the 'stiffness' of the spline. The default is 1.0.

\item  \verb|obj.SetSigma (double )| -  Specify the 'stiffness' of the spline. The default is 1.0.

\item  \verb|obj.SetBasis (int basis)| -  Specify the radial basis function to use.  The default is
 R2LogR which is appropriate for 2D. Use |R| (SetBasisToR) 
 if your data is 3D. Alternatively specify your own basis function, 
 however this will mean that the transform will no longer be a true 
 thin-plate spline.

\item  \verb|int = obj.GetBasis ()| -  Specify the radial basis function to use.  The default is
 R2LogR which is appropriate for 2D. Use |R| (SetBasisToR) 
 if your data is 3D. Alternatively specify your own basis function, 
 however this will mean that the transform will no longer be a true 
 thin-plate spline.

\item  \verb|obj.SetBasisToR ()| -  Specify the radial basis function to use.  The default is
 R2LogR which is appropriate for 2D. Use |R| (SetBasisToR) 
 if your data is 3D. Alternatively specify your own basis function, 
 however this will mean that the transform will no longer be a true 
 thin-plate spline.

\item  \verb|obj.SetBasisToR2LogR ()| -  Specify the radial basis function to use.  The default is
 R2LogR which is appropriate for 2D. Use |R| (SetBasisToR) 
 if your data is 3D. Alternatively specify your own basis function, 
 however this will mean that the transform will no longer be a true 
 thin-plate spline.

\item  \verb|string = obj.GetBasisAsString ()| -  Specify the radial basis function to use.  The default is
 R2LogR which is appropriate for 2D. Use |R| (SetBasisToR) 
 if your data is 3D. Alternatively specify your own basis function, 
 however this will mean that the transform will no longer be a true 
 thin-plate spline.

\item  \verb|obj.SetSourceLandmarks (vtkPoints source)| -  Set the source landmarks for the warp.  If you add or change the
 vtkPoints object, you must call Modified() on it or the transformation
 might not update.

\item  \verb|vtkPoints = obj.GetSourceLandmarks ()| -  Set the source landmarks for the warp.  If you add or change the
 vtkPoints object, you must call Modified() on it or the transformation
 might not update.

\item  \verb|obj.SetTargetLandmarks (vtkPoints target)| -  Set the target landmarks for the warp.  If you add or change the
 vtkPoints object, you must call Modified() on it or the transformation
 might not update.

\item  \verb|vtkPoints = obj.GetTargetLandmarks ()| -  Set the target landmarks for the warp.  If you add or change the
 vtkPoints object, you must call Modified() on it or the transformation
 might not update.

\item  \verb|long = obj.GetMTime ()| -  Get the MTime.

\item  \verb|vtkAbstractTransform = obj.MakeTransform ()| -  Make another transform of the same type.

\end{itemize}
