\section{vtkImageChangeInformation}

\subsection{Usage}

 vtkImageChangeInformation  modify the spacing, origin, or extent of
 the data without changing the data itself.  The data is not resampled
 by this filter, only the information accompanying the data is modified.

To create an instance of class vtkImageChangeInformation, simply
invoke its constructor as follows
\begin{verbatim}
  obj = vtkImageChangeInformation
\end{verbatim}
\subsection{Methods}

The class vtkImageChangeInformation has several methods that can be used.
  They are listed below.
Note that the documentation is translated automatically from the VTK sources,
and may not be completely intelligible.  When in doubt, consult the VTK website.
In the methods listed below, \verb|obj| is an instance of the vtkImageChangeInformation class.
\begin{itemize}
\item  \verb|string = obj.GetClassName ()|

\item  \verb|int = obj.IsA (string name)|

\item  \verb|vtkImageChangeInformation = obj.NewInstance ()|

\item  \verb|vtkImageChangeInformation = obj.SafeDownCast (vtkObject o)|

\item  \verb|obj.SetInformationInput (vtkImageData )| -  Copy the information from another data set.  By default,
 the information is copied from the input.

\item  \verb|vtkImageData = obj.GetInformationInput ()| -  Copy the information from another data set.  By default,
 the information is copied from the input.

\item  \verb|obj.SetOutputExtentStart (int , int , int )| -  Specify new starting values for the extent explicitly.
 These values are used as WholeExtent[0], WholeExtent[2] and
 WholeExtent[4] of the output.  The default is to the
 use the extent start of the Input, or of the InformationInput
 if InformationInput is set.

\item  \verb|obj.SetOutputExtentStart (int  a[3])| -  Specify new starting values for the extent explicitly.
 These values are used as WholeExtent[0], WholeExtent[2] and
 WholeExtent[4] of the output.  The default is to the
 use the extent start of the Input, or of the InformationInput
 if InformationInput is set.

\item  \verb|int = obj. GetOutputExtentStart ()| -  Specify new starting values for the extent explicitly.
 These values are used as WholeExtent[0], WholeExtent[2] and
 WholeExtent[4] of the output.  The default is to the
 use the extent start of the Input, or of the InformationInput
 if InformationInput is set.

\item  \verb|obj.SetOutputSpacing (double , double , double )| -  Specify a new data spacing explicitly.  The default is to
 use the spacing of the Input, or of the InformationInput
 if InformationInput is set.

\item  \verb|obj.SetOutputSpacing (double  a[3])| -  Specify a new data spacing explicitly.  The default is to
 use the spacing of the Input, or of the InformationInput
 if InformationInput is set.

\item  \verb|double = obj. GetOutputSpacing ()| -  Specify a new data spacing explicitly.  The default is to
 use the spacing of the Input, or of the InformationInput
 if InformationInput is set.

\item  \verb|obj.SetOutputOrigin (double , double , double )| -  Specify a new data origin explicitly.  The default is to
 use the origin of the Input, or of the InformationInput
 if InformationInput is set.

\item  \verb|obj.SetOutputOrigin (double  a[3])| -  Specify a new data origin explicitly.  The default is to
 use the origin of the Input, or of the InformationInput
 if InformationInput is set.

\item  \verb|double = obj. GetOutputOrigin ()| -  Specify a new data origin explicitly.  The default is to
 use the origin of the Input, or of the InformationInput
 if InformationInput is set.

\item  \verb|obj.SetCenterImage (int )| -  Set the Origin of the output so that image coordinate (0,0,0)
 lies at the Center of the data set.  This will override 
 SetOutputOrigin.  This is often a useful operation to apply 
 before using vtkImageReslice to apply a transformation to an image. 

\item  \verb|obj.CenterImageOn ()| -  Set the Origin of the output so that image coordinate (0,0,0)
 lies at the Center of the data set.  This will override 
 SetOutputOrigin.  This is often a useful operation to apply 
 before using vtkImageReslice to apply a transformation to an image. 

\item  \verb|obj.CenterImageOff ()| -  Set the Origin of the output so that image coordinate (0,0,0)
 lies at the Center of the data set.  This will override 
 SetOutputOrigin.  This is often a useful operation to apply 
 before using vtkImageReslice to apply a transformation to an image. 

\item  \verb|int = obj.GetCenterImage ()| -  Set the Origin of the output so that image coordinate (0,0,0)
 lies at the Center of the data set.  This will override 
 SetOutputOrigin.  This is often a useful operation to apply 
 before using vtkImageReslice to apply a transformation to an image. 

\item  \verb|obj.SetExtentTranslation (int , int , int )| -  Apply a translation to the extent.

\item  \verb|obj.SetExtentTranslation (int  a[3])| -  Apply a translation to the extent.

\item  \verb|int = obj. GetExtentTranslation ()| -  Apply a translation to the extent.

\item  \verb|obj.SetSpacingScale (double , double , double )| -  Apply a scale factor to the spacing. 

\item  \verb|obj.SetSpacingScale (double  a[3])| -  Apply a scale factor to the spacing. 

\item  \verb|double = obj. GetSpacingScale ()| -  Apply a scale factor to the spacing. 

\item  \verb|obj.SetOriginTranslation (double , double , double )| -  Apply a translation to the origin.

\item  \verb|obj.SetOriginTranslation (double  a[3])| -  Apply a translation to the origin.

\item  \verb|double = obj. GetOriginTranslation ()| -  Apply a translation to the origin.

\item  \verb|obj.SetOriginScale (double , double , double )| -  Apply a scale to the origin.  The scale is applied
 before the translation.

\item  \verb|obj.SetOriginScale (double  a[3])| -  Apply a scale to the origin.  The scale is applied
 before the translation.

\item  \verb|double = obj. GetOriginScale ()| -  Apply a scale to the origin.  The scale is applied
 before the translation.

\end{itemize}
