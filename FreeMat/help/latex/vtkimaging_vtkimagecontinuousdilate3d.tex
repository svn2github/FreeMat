\section{vtkImageContinuousDilate3D}

\subsection{Usage}

 vtkImageContinuousDilate3D replaces a pixel with the maximum over
 an ellipsoidal neighborhood.  If KernelSize of an axis is 1, no processing
 is done on that axis.

To create an instance of class vtkImageContinuousDilate3D, simply
invoke its constructor as follows
\begin{verbatim}
  obj = vtkImageContinuousDilate3D
\end{verbatim}
\subsection{Methods}

The class vtkImageContinuousDilate3D has several methods that can be used.
  They are listed below.
Note that the documentation is translated automatically from the VTK sources,
and may not be completely intelligible.  When in doubt, consult the VTK website.
In the methods listed below, \verb|obj| is an instance of the vtkImageContinuousDilate3D class.
\begin{itemize}
\item  \verb|string = obj.GetClassName ()| -  Construct an instance of vtkImageContinuousDilate3D filter.
 By default zero values are dilated.

\item  \verb|int = obj.IsA (string name)| -  Construct an instance of vtkImageContinuousDilate3D filter.
 By default zero values are dilated.

\item  \verb|vtkImageContinuousDilate3D = obj.NewInstance ()| -  Construct an instance of vtkImageContinuousDilate3D filter.
 By default zero values are dilated.

\item  \verb|vtkImageContinuousDilate3D = obj.SafeDownCast (vtkObject o)| -  Construct an instance of vtkImageContinuousDilate3D filter.
 By default zero values are dilated.

\item  \verb|obj.SetKernelSize (int size0, int size1, int size2)| -  This method sets the size of the neighborhood.  It also sets the 
 default middle of the neighborhood and computes the elliptical foot print.

\end{itemize}
