\section{vtkImageDataStreamer}

\subsection{Usage}

 To satisfy a request, this filter calls update on its input
 many times with smaller update extents.  All processing up stream
 streams smaller pieces.

To create an instance of class vtkImageDataStreamer, simply
invoke its constructor as follows
\begin{verbatim}
  obj = vtkImageDataStreamer
\end{verbatim}
\subsection{Methods}

The class vtkImageDataStreamer has several methods that can be used.
  They are listed below.
Note that the documentation is translated automatically from the VTK sources,
and may not be completely intelligible.  When in doubt, consult the VTK website.
In the methods listed below, \verb|obj| is an instance of the vtkImageDataStreamer class.
\begin{itemize}
\item  \verb|string = obj.GetClassName ()|

\item  \verb|int = obj.IsA (string name)|

\item  \verb|vtkImageDataStreamer = obj.NewInstance ()|

\item  \verb|vtkImageDataStreamer = obj.SafeDownCast (vtkObject o)|

\item  \verb|obj.SetNumberOfStreamDivisions (int )| -  Set how many pieces to divide the input into.
 void SetNumberOfStreamDivisions(int num);
 int GetNumberOfStreamDivisions();

\item  \verb|int = obj.GetNumberOfStreamDivisions ()| -  Set how many pieces to divide the input into.
 void SetNumberOfStreamDivisions(int num);
 int GetNumberOfStreamDivisions();

\item  \verb|obj.Update ()|

\item  \verb|obj.UpdateWholeExtent ()|

\item  \verb|obj.SetExtentTranslator (vtkExtentTranslator )| -  Get the extent translator that will be used to split the requests

\item  \verb|vtkExtentTranslator = obj.GetExtentTranslator ()| -  Get the extent translator that will be used to split the requests

\end{itemize}
