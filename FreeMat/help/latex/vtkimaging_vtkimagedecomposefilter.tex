\section{vtkImageDecomposeFilter}

\subsection{Usage}

 This superclass molds the vtkImageIterateFilter superclass so
 it iterates over the axes.  The filter uses dimensionality to 
 determine how many axes to execute (starting from x).  
 The filter also provides convenience methods for permuting information
 retrieved from input, output and vtkImageData.

To create an instance of class vtkImageDecomposeFilter, simply
invoke its constructor as follows
\begin{verbatim}
  obj = vtkImageDecomposeFilter
\end{verbatim}
\subsection{Methods}

The class vtkImageDecomposeFilter has several methods that can be used.
  They are listed below.
Note that the documentation is translated automatically from the VTK sources,
and may not be completely intelligible.  When in doubt, consult the VTK website.
In the methods listed below, \verb|obj| is an instance of the vtkImageDecomposeFilter class.
\begin{itemize}
\item  \verb|string = obj.GetClassName ()| -  Construct an instance of vtkImageDecomposeFilter filter with default
 dimensionality 3.

\item  \verb|int = obj.IsA (string name)| -  Construct an instance of vtkImageDecomposeFilter filter with default
 dimensionality 3.

\item  \verb|vtkImageDecomposeFilter = obj.NewInstance ()| -  Construct an instance of vtkImageDecomposeFilter filter with default
 dimensionality 3.

\item  \verb|vtkImageDecomposeFilter = obj.SafeDownCast (vtkObject o)| -  Construct an instance of vtkImageDecomposeFilter filter with default
 dimensionality 3.

\item  \verb|obj.SetDimensionality (int dim)| -  Dimensionality is the number of axes which are considered during
 execution. To process images dimensionality would be set to 2.

\item  \verb|int = obj.GetDimensionality ()| -  Dimensionality is the number of axes which are considered during
 execution. To process images dimensionality would be set to 2.

\end{itemize}
