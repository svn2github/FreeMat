\section{vtkImageDilateErode3D}

\subsection{Usage}

 vtkImageDilateErode3D will dilate one value and erode another.
 It uses an elliptical foot print, and only erodes/dilates on the
 boundary of the two values.  The filter is restricted to the 
 X, Y, and Z axes for now.  It can degenerate to a 2 or 1 dimensional
 filter by setting the kernel size to 1 for a specific axis.

To create an instance of class vtkImageDilateErode3D, simply
invoke its constructor as follows
\begin{verbatim}
  obj = vtkImageDilateErode3D
\end{verbatim}
\subsection{Methods}

The class vtkImageDilateErode3D has several methods that can be used.
  They are listed below.
Note that the documentation is translated automatically from the VTK sources,
and may not be completely intelligible.  When in doubt, consult the VTK website.
In the methods listed below, \verb|obj| is an instance of the vtkImageDilateErode3D class.
\begin{itemize}
\item  \verb|string = obj.GetClassName ()| -  Construct an instance of vtkImageDilateErode3D filter.
 By default zero values are dilated.

\item  \verb|int = obj.IsA (string name)| -  Construct an instance of vtkImageDilateErode3D filter.
 By default zero values are dilated.

\item  \verb|vtkImageDilateErode3D = obj.NewInstance ()| -  Construct an instance of vtkImageDilateErode3D filter.
 By default zero values are dilated.

\item  \verb|vtkImageDilateErode3D = obj.SafeDownCast (vtkObject o)| -  Construct an instance of vtkImageDilateErode3D filter.
 By default zero values are dilated.

\item  \verb|obj.SetKernelSize (int size0, int size1, int size2)| -  This method sets the size of the neighborhood.  It also sets the 
 default middle of the neighborhood and computes the elliptical foot print.

\item  \verb|obj.SetDilateValue (double )| -  Set/Get the Dilate and Erode values to be used by this filter.

\item  \verb|double = obj.GetDilateValue ()| -  Set/Get the Dilate and Erode values to be used by this filter.

\item  \verb|obj.SetErodeValue (double )| -  Set/Get the Dilate and Erode values to be used by this filter.

\item  \verb|double = obj.GetErodeValue ()| -  Set/Get the Dilate and Erode values to be used by this filter.

\end{itemize}
