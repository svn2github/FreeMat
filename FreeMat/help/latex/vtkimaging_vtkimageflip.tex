\section{vtkImageFlip}

\subsection{Usage}

 vtkImageFlip will reflect the data along the filtered axis.  This filter is
 actually a thin wrapper around vtkImageReslice.

To create an instance of class vtkImageFlip, simply
invoke its constructor as follows
\begin{verbatim}
  obj = vtkImageFlip
\end{verbatim}
\subsection{Methods}

The class vtkImageFlip has several methods that can be used.
  They are listed below.
Note that the documentation is translated automatically from the VTK sources,
and may not be completely intelligible.  When in doubt, consult the VTK website.
In the methods listed below, \verb|obj| is an instance of the vtkImageFlip class.
\begin{itemize}
\item  \verb|string = obj.GetClassName ()|

\item  \verb|int = obj.IsA (string name)|

\item  \verb|vtkImageFlip = obj.NewInstance ()|

\item  \verb|vtkImageFlip = obj.SafeDownCast (vtkObject o)|

\item  \verb|obj.SetFilteredAxis (int )| -  Specify which axis will be flipped.  This must be an integer
 between 0 (for x) and 2 (for z). Initial value is 0.

\item  \verb|int = obj.GetFilteredAxis ()| -  Specify which axis will be flipped.  This must be an integer
 between 0 (for x) and 2 (for z). Initial value is 0.

\item  \verb|obj.SetFlipAboutOrigin (int )| -  By default the image will be flipped about its center, and the
 Origin, Spacing and Extent of the output will be identical to
 the input.  However, if you have a coordinate system associated
 with the image and you want to use the flip to convert +ve values
 along one axis to -ve values (and vice versa) then you actually
 want to flip the image about coordinate (0,0,0) instead of about
 the center of the image.  This method will adjust the Origin of
 the output such that the flip occurs about (0,0,0).  Note that
 this method only changes the Origin (and hence the coordinate system)
 the output data: the actual pixel values are the same whether or not
 this method is used.  Also note that the Origin in this method name
 refers to (0,0,0) in the coordinate system associated with the image,
 it does not refer to the Origin ivar that is associated with a
 vtkImageData.

\item  \verb|int = obj.GetFlipAboutOrigin ()| -  By default the image will be flipped about its center, and the
 Origin, Spacing and Extent of the output will be identical to
 the input.  However, if you have a coordinate system associated
 with the image and you want to use the flip to convert +ve values
 along one axis to -ve values (and vice versa) then you actually
 want to flip the image about coordinate (0,0,0) instead of about
 the center of the image.  This method will adjust the Origin of
 the output such that the flip occurs about (0,0,0).  Note that
 this method only changes the Origin (and hence the coordinate system)
 the output data: the actual pixel values are the same whether or not
 this method is used.  Also note that the Origin in this method name
 refers to (0,0,0) in the coordinate system associated with the image,
 it does not refer to the Origin ivar that is associated with a
 vtkImageData.

\item  \verb|obj.FlipAboutOriginOn ()| -  By default the image will be flipped about its center, and the
 Origin, Spacing and Extent of the output will be identical to
 the input.  However, if you have a coordinate system associated
 with the image and you want to use the flip to convert +ve values
 along one axis to -ve values (and vice versa) then you actually
 want to flip the image about coordinate (0,0,0) instead of about
 the center of the image.  This method will adjust the Origin of
 the output such that the flip occurs about (0,0,0).  Note that
 this method only changes the Origin (and hence the coordinate system)
 the output data: the actual pixel values are the same whether or not
 this method is used.  Also note that the Origin in this method name
 refers to (0,0,0) in the coordinate system associated with the image,
 it does not refer to the Origin ivar that is associated with a
 vtkImageData.

\item  \verb|obj.FlipAboutOriginOff ()| -  By default the image will be flipped about its center, and the
 Origin, Spacing and Extent of the output will be identical to
 the input.  However, if you have a coordinate system associated
 with the image and you want to use the flip to convert +ve values
 along one axis to -ve values (and vice versa) then you actually
 want to flip the image about coordinate (0,0,0) instead of about
 the center of the image.  This method will adjust the Origin of
 the output such that the flip occurs about (0,0,0).  Note that
 this method only changes the Origin (and hence the coordinate system)
 the output data: the actual pixel values are the same whether or not
 this method is used.  Also note that the Origin in this method name
 refers to (0,0,0) in the coordinate system associated with the image,
 it does not refer to the Origin ivar that is associated with a
 vtkImageData.

\item  \verb|obj.SetFilteredAxes (int axis)| -  Keep the mis-named Axes variations around for compatibility with old
 scripts. Axis is singular, not plural...

\item  \verb|int = obj.GetFilteredAxes ()| -  PreserveImageExtentOff wasn't covered by test scripts and its
 implementation was broken.  It is deprecated now and it has
 no effect (i.e. the ImageExtent is always preserved).

\item  \verb|obj.SetPreserveImageExtent (int )| -  PreserveImageExtentOff wasn't covered by test scripts and its
 implementation was broken.  It is deprecated now and it has
 no effect (i.e. the ImageExtent is always preserved).

\item  \verb|int = obj.GetPreserveImageExtent ()| -  PreserveImageExtentOff wasn't covered by test scripts and its
 implementation was broken.  It is deprecated now and it has
 no effect (i.e. the ImageExtent is always preserved).

\item  \verb|obj.PreserveImageExtentOn ()| -  PreserveImageExtentOff wasn't covered by test scripts and its
 implementation was broken.  It is deprecated now and it has
 no effect (i.e. the ImageExtent is always preserved).

\item  \verb|obj.PreserveImageExtentOff ()| -  PreserveImageExtentOff wasn't covered by test scripts and its
 implementation was broken.  It is deprecated now and it has
 no effect (i.e. the ImageExtent is always preserved).

\end{itemize}
