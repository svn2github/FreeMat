\section{vtkImageIterateFilter}

\subsection{Usage}

 vtkImageIterateFilter is a filter superclass that supports calling execute
 multiple times per update.  The largest hack/open issue is that the input
 and output caches are temporarily changed to ''fool'' the subclasses.  I
 believe the correct solution is to pass the in and out cache to the
 subclasses methods as arguments.  Now the data is passes.  Can the caches
 be passed, and data retrieved from the cache? 

To create an instance of class vtkImageIterateFilter, simply
invoke its constructor as follows
\begin{verbatim}
  obj = vtkImageIterateFilter
\end{verbatim}
\subsection{Methods}

The class vtkImageIterateFilter has several methods that can be used.
  They are listed below.
Note that the documentation is translated automatically from the VTK sources,
and may not be completely intelligible.  When in doubt, consult the VTK website.
In the methods listed below, \verb|obj| is an instance of the vtkImageIterateFilter class.
\begin{itemize}
\item  \verb|string = obj.GetClassName ()|

\item  \verb|int = obj.IsA (string name)|

\item  \verb|vtkImageIterateFilter = obj.NewInstance ()|

\item  \verb|vtkImageIterateFilter = obj.SafeDownCast (vtkObject o)|

\item  \verb|int = obj.GetIteration ()| -  Get which iteration is current being performed. Normally the
 user will not access this method.

\item  \verb|int = obj.GetNumberOfIterations ()| -  Get which iteration is current being performed. Normally the
 user will not access this method.

\end{itemize}
