\section{vtkImageLaplacian}

\subsection{Usage}

 vtkImageLaplacian computes the Laplacian (like a second derivative)
 of a scalar image.  The operation is the same as taking the
 divergence after a gradient.  Boundaries are handled, so the input
 is the same as the output.
 Dimensionality determines how the input regions are interpreted.
 (images, or volumes). The Dimensionality defaults to two.

To create an instance of class vtkImageLaplacian, simply
invoke its constructor as follows
\begin{verbatim}
  obj = vtkImageLaplacian
\end{verbatim}
\subsection{Methods}

The class vtkImageLaplacian has several methods that can be used.
  They are listed below.
Note that the documentation is translated automatically from the VTK sources,
and may not be completely intelligible.  When in doubt, consult the VTK website.
In the methods listed below, \verb|obj| is an instance of the vtkImageLaplacian class.
\begin{itemize}
\item  \verb|string = obj.GetClassName ()|

\item  \verb|int = obj.IsA (string name)|

\item  \verb|vtkImageLaplacian = obj.NewInstance ()|

\item  \verb|vtkImageLaplacian = obj.SafeDownCast (vtkObject o)|

\item  \verb|obj.SetDimensionality (int )| -  Determines how the input is interpreted (set of 2d slices ...)

\item  \verb|int = obj.GetDimensionalityMinValue ()| -  Determines how the input is interpreted (set of 2d slices ...)

\item  \verb|int = obj.GetDimensionalityMaxValue ()| -  Determines how the input is interpreted (set of 2d slices ...)

\item  \verb|int = obj.GetDimensionality ()| -  Determines how the input is interpreted (set of 2d slices ...)

\end{itemize}
