\section{vtkImageNonMaximumSuppression}

\subsection{Usage}

 vtkImageNonMaximumSuppression Sets to zero any pixel that is not a peak.
 If a pixel has a neighbor along the vector that has larger magnitude, the
 smaller pixel is set to zero.  The filter takes two inputs: a magnitude
 and a vector.  Output is magnitude information and is always in doubles.
 Typically this filter is used with vtkImageGradient and
 vtkImageGradientMagnitude as inputs.

To create an instance of class vtkImageNonMaximumSuppression, simply
invoke its constructor as follows
\begin{verbatim}
  obj = vtkImageNonMaximumSuppression
\end{verbatim}
\subsection{Methods}

The class vtkImageNonMaximumSuppression has several methods that can be used.
  They are listed below.
Note that the documentation is translated automatically from the VTK sources,
and may not be completely intelligible.  When in doubt, consult the VTK website.
In the methods listed below, \verb|obj| is an instance of the vtkImageNonMaximumSuppression class.
\begin{itemize}
\item  \verb|string = obj.GetClassName ()|

\item  \verb|int = obj.IsA (string name)|

\item  \verb|vtkImageNonMaximumSuppression = obj.NewInstance ()|

\item  \verb|vtkImageNonMaximumSuppression = obj.SafeDownCast (vtkObject o)|

\item  \verb|obj.SetMagnitudeInput (vtkImageData input)| -  Set the magnitude and vector inputs.

\item  \verb|obj.SetVectorInput (vtkImageData input)| -  Set the magnitude and vector inputs.

\item  \verb|obj.SetHandleBoundaries (int )| -  If ''HandleBoundariesOn'' then boundary pixels are duplicated
 So central differences can get values.

\item  \verb|int = obj.GetHandleBoundaries ()| -  If ''HandleBoundariesOn'' then boundary pixels are duplicated
 So central differences can get values.

\item  \verb|obj.HandleBoundariesOn ()| -  If ''HandleBoundariesOn'' then boundary pixels are duplicated
 So central differences can get values.

\item  \verb|obj.HandleBoundariesOff ()| -  If ''HandleBoundariesOn'' then boundary pixels are duplicated
 So central differences can get values.

\item  \verb|obj.SetDimensionality (int )| -  Determines how the input is interpreted (set of 2d slices or a 3D volume)

\item  \verb|int = obj.GetDimensionalityMinValue ()| -  Determines how the input is interpreted (set of 2d slices or a 3D volume)

\item  \verb|int = obj.GetDimensionalityMaxValue ()| -  Determines how the input is interpreted (set of 2d slices or a 3D volume)

\item  \verb|int = obj.GetDimensionality ()| -  Determines how the input is interpreted (set of 2d slices or a 3D volume)

\end{itemize}
