\section{vtkImagePermute}

\subsection{Usage}

 vtkImagePermute reorders the axes of the input. Filtered axes specify
 the input axes which become X, Y, Z.  The input has to have the
 same scalar type of the output. The filter does copy the 
 data when it executes. This filter is actually a very thin wrapper
 around vtkImageReslice.

To create an instance of class vtkImagePermute, simply
invoke its constructor as follows
\begin{verbatim}
  obj = vtkImagePermute
\end{verbatim}
\subsection{Methods}

The class vtkImagePermute has several methods that can be used.
  They are listed below.
Note that the documentation is translated automatically from the VTK sources,
and may not be completely intelligible.  When in doubt, consult the VTK website.
In the methods listed below, \verb|obj| is an instance of the vtkImagePermute class.
\begin{itemize}
\item  \verb|string = obj.GetClassName ()|

\item  \verb|int = obj.IsA (string name)|

\item  \verb|vtkImagePermute = obj.NewInstance ()|

\item  \verb|vtkImagePermute = obj.SafeDownCast (vtkObject o)|

\item  \verb|obj.SetFilteredAxes (int x, int y, int z)| -  The filtered axes are the input axes that get relabeled to X,Y,Z.

\item  \verb|obj.SetFilteredAxes (int xyz[3])| -  The filtered axes are the input axes that get relabeled to X,Y,Z.

\item  \verb|int = obj. GetFilteredAxes ()| -  The filtered axes are the input axes that get relabeled to X,Y,Z.

\end{itemize}
