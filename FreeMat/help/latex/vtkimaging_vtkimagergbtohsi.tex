\section{vtkImageRGBToHSI}

\subsection{Usage}

 For each pixel with red, blue, and green components this
 filter output the color coded as hue, saturation and intensity.
 Output type must be the same as input type.

To create an instance of class vtkImageRGBToHSI, simply
invoke its constructor as follows
\begin{verbatim}
  obj = vtkImageRGBToHSI
\end{verbatim}
\subsection{Methods}

The class vtkImageRGBToHSI has several methods that can be used.
  They are listed below.
Note that the documentation is translated automatically from the VTK sources,
and may not be completely intelligible.  When in doubt, consult the VTK website.
In the methods listed below, \verb|obj| is an instance of the vtkImageRGBToHSI class.
\begin{itemize}
\item  \verb|string = obj.GetClassName ()|

\item  \verb|int = obj.IsA (string name)|

\item  \verb|vtkImageRGBToHSI = obj.NewInstance ()|

\item  \verb|vtkImageRGBToHSI = obj.SafeDownCast (vtkObject o)|

\item  \verb|obj.SetMaximum (double )| -  Hue is an angle. Maximum specifies when it maps back to 0.  HueMaximum
 defaults to 255 instead of 2PI, because unsigned char is expected as
 input.  Maximum also specifies the maximum of the Saturation.

\item  \verb|double = obj.GetMaximum ()| -  Hue is an angle. Maximum specifies when it maps back to 0.  HueMaximum
 defaults to 255 instead of 2PI, because unsigned char is expected as
 input.  Maximum also specifies the maximum of the Saturation.

\end{itemize}
