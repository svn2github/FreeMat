\section{vtkRTAnalyticSource}

\subsection{Usage}

 vtkRTAnalyticSource just produces images with pixel values determined 
 by a Maximum*Gaussian*XMag*sin(XFreq*x)*sin(YFreq*y)*cos(ZFreq*z)
 Values are float scalars on point data with name ''RTData''.

To create an instance of class vtkRTAnalyticSource, simply
invoke its constructor as follows
\begin{verbatim}
  obj = vtkRTAnalyticSource
\end{verbatim}
\subsection{Methods}

The class vtkRTAnalyticSource has several methods that can be used.
  They are listed below.
Note that the documentation is translated automatically from the VTK sources,
and may not be completely intelligible.  When in doubt, consult the VTK website.
In the methods listed below, \verb|obj| is an instance of the vtkRTAnalyticSource class.
\begin{itemize}
\item  \verb|string = obj.GetClassName ()|

\item  \verb|int = obj.IsA (string name)|

\item  \verb|vtkRTAnalyticSource = obj.NewInstance ()|

\item  \verb|vtkRTAnalyticSource = obj.SafeDownCast (vtkObject o)|

\item  \verb|obj.SetWholeExtent (int xMinx, int xMax, int yMin, int yMax, int zMin, int zMax)| -  Set/Get the extent of the whole output image. Initial value is
 {-10,10,-10,10,-10,10}

\item  \verb|int = obj. GetWholeExtent ()| -  Set/Get the extent of the whole output image. Initial value is
 {-10,10,-10,10,-10,10}

\item  \verb|obj.SetCenter (double , double , double )| -  Set/Get the center of function. Initial value is {0.0,0.0,0.0}

\item  \verb|obj.SetCenter (double  a[3])| -  Set/Get the center of function. Initial value is {0.0,0.0,0.0}

\item  \verb|double = obj. GetCenter ()| -  Set/Get the center of function. Initial value is {0.0,0.0,0.0}

\item  \verb|obj.SetMaximum (double )| -  Set/Get the Maximum value of the function. Initial value is 255.0.

\item  \verb|double = obj.GetMaximum ()| -  Set/Get the Maximum value of the function. Initial value is 255.0.

\item  \verb|obj.SetStandardDeviation (double )| -  Set/Get the standard deviation of the function. Initial value is 0.5.

\item  \verb|double = obj.GetStandardDeviation ()| -  Set/Get the standard deviation of the function. Initial value is 0.5.

\item  \verb|obj.SetXFreq (double )| -  Set/Get the natural frequency in x. Initial value is 60.

\item  \verb|double = obj.GetXFreq ()| -  Set/Get the natural frequency in x. Initial value is 60.

\item  \verb|obj.SetYFreq (double )| -  Set/Get the natural frequency in y. Initial value is 30.

\item  \verb|double = obj.GetYFreq ()| -  Set/Get the natural frequency in y. Initial value is 30.

\item  \verb|obj.SetZFreq (double )| -  Set/Get the natural frequency in z. Initial value is 40.

\item  \verb|double = obj.GetZFreq ()| -  Set/Get the natural frequency in z. Initial value is 40.

\item  \verb|obj.SetXMag (double )| -  Set/Get the magnitude in x. Initial value is 10.

\item  \verb|double = obj.GetXMag ()| -  Set/Get the magnitude in x. Initial value is 10.

\item  \verb|obj.SetYMag (double )| -  Set/Get the magnitude in y. Initial value is 18.

\item  \verb|double = obj.GetYMag ()| -  Set/Get the magnitude in y. Initial value is 18.

\item  \verb|obj.SetZMag (double )| -  Set/Get the magnitude in z. Initial value is 5.

\item  \verb|double = obj.GetZMag ()| -  Set/Get the magnitude in z. Initial value is 5.

\item  \verb|obj.SetSubsampleRate (int )| -  Set/Get the sub-sample rate. Initial value is 1.

\item  \verb|int = obj.GetSubsampleRate ()| -  Set/Get the sub-sample rate. Initial value is 1.

\end{itemize}
