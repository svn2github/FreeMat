\section{vtkVoxelModeller}

\subsection{Usage}

 vtkVoxelModeller is a filter that converts an arbitrary data set to a
 structured point (i.e., voxel) representation. It is very similar to 
 vtkImplicitModeller, except that it doesn't record distance; instead it
 records occupancy. By default it supports a compact output of 0/1
 VTK\_BIT. Other vtk scalar types can be specified. The Foreground and
 Background values of the output can also be specified.
 NOTE: Not all vtk filters/readers/writers support the VTK\_BIT
 scalar type. You may want to use VTK\_CHAR as an alternative.

To create an instance of class vtkVoxelModeller, simply
invoke its constructor as follows
\begin{verbatim}
  obj = vtkVoxelModeller
\end{verbatim}
\subsection{Methods}

The class vtkVoxelModeller has several methods that can be used.
  They are listed below.
Note that the documentation is translated automatically from the VTK sources,
and may not be completely intelligible.  When in doubt, consult the VTK website.
In the methods listed below, \verb|obj| is an instance of the vtkVoxelModeller class.
\begin{itemize}
\item  \verb|string = obj.GetClassName ()|

\item  \verb|int = obj.IsA (string name)|

\item  \verb|vtkVoxelModeller = obj.NewInstance ()|

\item  \verb|vtkVoxelModeller = obj.SafeDownCast (vtkObject o)|

\item  \verb|double = obj.ComputeModelBounds (double origin[3], double ar[3])| -  Compute the ModelBounds based on the input geometry.

\item  \verb|obj.SetSampleDimensions (int i, int j, int k)| -  Set the i-j-k dimensions on which to sample the distance function.
 Default is (50, 50, 50)

\item  \verb|obj.SetSampleDimensions (int dim[3])| -  Set the i-j-k dimensions on which to sample the distance function.
 Default is (50, 50, 50)

\item  \verb|int = obj. GetSampleDimensions ()| -  Set the i-j-k dimensions on which to sample the distance function.
 Default is (50, 50, 50)

\item  \verb|obj.SetMaximumDistance (double )| -  Specify distance away from surface of input geometry to sample. Smaller
 values make large increases in performance. Default is 1.0.

\item  \verb|double = obj.GetMaximumDistanceMinValue ()| -  Specify distance away from surface of input geometry to sample. Smaller
 values make large increases in performance. Default is 1.0.

\item  \verb|double = obj.GetMaximumDistanceMaxValue ()| -  Specify distance away from surface of input geometry to sample. Smaller
 values make large increases in performance. Default is 1.0.

\item  \verb|double = obj.GetMaximumDistance ()| -  Specify distance away from surface of input geometry to sample. Smaller
 values make large increases in performance. Default is 1.0.

\item  \verb|obj.SetModelBounds (double bounds[6])| -  Specify the position in space to perform the voxelization.
 Default is (0, 0, 0, 0, 0, 0)

\item  \verb|obj.SetModelBounds (double xmin, double xmax, double ymin, double ymax, double zmin, double zmax)| -  Specify the position in space to perform the voxelization.
 Default is (0, 0, 0, 0, 0, 0)

\item  \verb|double = obj. GetModelBounds ()| -  Specify the position in space to perform the voxelization.
 Default is (0, 0, 0, 0, 0, 0)

\item  \verb|obj.SetScalarType (int )| -  Control the scalar type of the output image. The default is
 VTK\_BIT.
 NOTE: Not all filters/readers/writers support the VTK\_BIT
 scalar type. You may want to use VTK\_CHAR as an alternative.

\item  \verb|obj.SetScalarTypeToFloat ()| -  Control the scalar type of the output image. The default is
 VTK\_BIT.
 NOTE: Not all filters/readers/writers support the VTK\_BIT
 scalar type. You may want to use VTK\_CHAR as an alternative.

\item  \verb|obj.SetScalarTypeToDouble ()| -  Control the scalar type of the output image. The default is
 VTK\_BIT.
 NOTE: Not all filters/readers/writers support the VTK\_BIT
 scalar type. You may want to use VTK\_CHAR as an alternative.

\item  \verb|obj.SetScalarTypeToInt ()| -  Control the scalar type of the output image. The default is
 VTK\_BIT.
 NOTE: Not all filters/readers/writers support the VTK\_BIT
 scalar type. You may want to use VTK\_CHAR as an alternative.

\item  \verb|obj.SetScalarTypeToUnsignedInt ()| -  Control the scalar type of the output image. The default is
 VTK\_BIT.
 NOTE: Not all filters/readers/writers support the VTK\_BIT
 scalar type. You may want to use VTK\_CHAR as an alternative.

\item  \verb|obj.SetScalarTypeToLong ()| -  Control the scalar type of the output image. The default is
 VTK\_BIT.
 NOTE: Not all filters/readers/writers support the VTK\_BIT
 scalar type. You may want to use VTK\_CHAR as an alternative.

\item  \verb|obj.SetScalarTypeToUnsignedLong ()| -  Control the scalar type of the output image. The default is
 VTK\_BIT.
 NOTE: Not all filters/readers/writers support the VTK\_BIT
 scalar type. You may want to use VTK\_CHAR as an alternative.

\item  \verb|obj.SetScalarTypeToShort ()| -  Control the scalar type of the output image. The default is
 VTK\_BIT.
 NOTE: Not all filters/readers/writers support the VTK\_BIT
 scalar type. You may want to use VTK\_CHAR as an alternative.

\item  \verb|obj.SetScalarTypeToUnsignedShort ()| -  Control the scalar type of the output image. The default is
 VTK\_BIT.
 NOTE: Not all filters/readers/writers support the VTK\_BIT
 scalar type. You may want to use VTK\_CHAR as an alternative.

\item  \verb|obj.SetScalarTypeToUnsignedChar ()| -  Control the scalar type of the output image. The default is
 VTK\_BIT.
 NOTE: Not all filters/readers/writers support the VTK\_BIT
 scalar type. You may want to use VTK\_CHAR as an alternative.

\item  \verb|obj.SetScalarTypeToChar ()| -  Control the scalar type of the output image. The default is
 VTK\_BIT.
 NOTE: Not all filters/readers/writers support the VTK\_BIT
 scalar type. You may want to use VTK\_CHAR as an alternative.

\item  \verb|obj.SetScalarTypeToBit ()| -  Control the scalar type of the output image. The default is
 VTK\_BIT.
 NOTE: Not all filters/readers/writers support the VTK\_BIT
 scalar type. You may want to use VTK\_CHAR as an alternative.

\item  \verb|int = obj.GetScalarType ()| -  Control the scalar type of the output image. The default is
 VTK\_BIT.
 NOTE: Not all filters/readers/writers support the VTK\_BIT
 scalar type. You may want to use VTK\_CHAR as an alternative.

\item  \verb|obj.SetForegroundValue (double )| -  Set the Foreground/Background values of the output. The
 Foreground value is set when a voxel is occupied. The Background
 value is set when a voxel is not occupied.
 The default ForegroundValue is 1. The default BackgroundValue is
 0.

\item  \verb|double = obj.GetForegroundValue ()| -  Set the Foreground/Background values of the output. The
 Foreground value is set when a voxel is occupied. The Background
 value is set when a voxel is not occupied.
 The default ForegroundValue is 1. The default BackgroundValue is
 0.

\item  \verb|obj.SetBackgroundValue (double )| -  Set the Foreground/Background values of the output. The
 Foreground value is set when a voxel is occupied. The Background
 value is set when a voxel is not occupied.
 The default ForegroundValue is 1. The default BackgroundValue is
 0.

\item  \verb|double = obj.GetBackgroundValue ()| -  Set the Foreground/Background values of the output. The
 Foreground value is set when a voxel is occupied. The Background
 value is set when a voxel is not occupied.
 The default ForegroundValue is 1. The default BackgroundValue is
 0.

\end{itemize}
