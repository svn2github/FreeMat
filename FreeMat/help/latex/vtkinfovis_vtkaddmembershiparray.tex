\section{vtkAddMembershipArray}

\subsection{Usage}

 This filter takes an input selection, vtkDataSetAttribute
 information, and data object and adds a bit array to the output 
 vtkDataSetAttributes indicating whether each index was selected or not.

To create an instance of class vtkAddMembershipArray, simply
invoke its constructor as follows
\begin{verbatim}
  obj = vtkAddMembershipArray
\end{verbatim}
\subsection{Methods}

The class vtkAddMembershipArray has several methods that can be used.
  They are listed below.
Note that the documentation is translated automatically from the VTK sources,
and may not be completely intelligible.  When in doubt, consult the VTK website.
In the methods listed below, \verb|obj| is an instance of the vtkAddMembershipArray class.
\begin{itemize}
\item  \verb|string = obj.GetClassName ()|

\item  \verb|int = obj.IsA (string name)|

\item  \verb|vtkAddMembershipArray = obj.NewInstance ()|

\item  \verb|vtkAddMembershipArray = obj.SafeDownCast (vtkObject o)|

\item  \verb|int = obj.GetFieldType ()| -  The field type to add the membership array to.

\item  \verb|obj.SetFieldType (int )| -  The field type to add the membership array to.

\item  \verb|int = obj.GetFieldTypeMinValue ()| -  The field type to add the membership array to.

\item  \verb|int = obj.GetFieldTypeMaxValue ()| -  The field type to add the membership array to.

\item  \verb|obj.SetOutputArrayName (string )| -  The name of the array added to the output vtkDataSetAttributes 
 indicating membership. Defaults to ''membership''.

\item  \verb|string = obj.GetOutputArrayName ()| -  The name of the array added to the output vtkDataSetAttributes 
 indicating membership. Defaults to ''membership''.

\item  \verb|obj.SetInputArrayName (string )|

\item  \verb|string = obj.GetInputArrayName ()|

\item  \verb|obj.SetInputValues (vtkAbstractArray )|

\item  \verb|vtkAbstractArray = obj.GetInputValues ()|

\end{itemize}
