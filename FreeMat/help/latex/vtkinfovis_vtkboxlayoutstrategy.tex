\section{vtkBoxLayoutStrategy}

\subsection{Usage}

 vtkBoxLayoutStrategy recursively partitions the space for children vertices
 in a tree-map into square regions (or regions very close to a square).

 .SECTION Thanks
 Thanks to Brian Wylie from Sandia National Laboratories for creating this class.

To create an instance of class vtkBoxLayoutStrategy, simply
invoke its constructor as follows
\begin{verbatim}
  obj = vtkBoxLayoutStrategy
\end{verbatim}
\subsection{Methods}

The class vtkBoxLayoutStrategy has several methods that can be used.
  They are listed below.
Note that the documentation is translated automatically from the VTK sources,
and may not be completely intelligible.  When in doubt, consult the VTK website.
In the methods listed below, \verb|obj| is an instance of the vtkBoxLayoutStrategy class.
\begin{itemize}
\item  \verb|string = obj.GetClassName ()|

\item  \verb|int = obj.IsA (string name)|

\item  \verb|vtkBoxLayoutStrategy = obj.NewInstance ()|

\item  \verb|vtkBoxLayoutStrategy = obj.SafeDownCast (vtkObject o)|

\item  \verb|obj.Layout (vtkTree inputTree, vtkDataArray coordsArray, vtkDataArray sizeArray)| -  Perform the layout of a tree and place the results as 4-tuples in
 coordsArray (Xmin, Xmax, Ymin, Ymax).

\end{itemize}
