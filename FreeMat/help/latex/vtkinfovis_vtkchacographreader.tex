\section{vtkChacoGraphReader}

\subsection{Usage}

 vtkChacoGraphReader reads in files in the Chaco format into a vtkGraph.
 An example is the following
 <code>
 10 13
 2 6 10
 1 3
 2 4 8
 3 5
 4 6 10
 1 5 7
 6 8
 3 7 9
 8 10
 1 5 9
 </code>
 The first line specifies the number of vertices and edges
 in the graph. Each additional line contains the vertices adjacent
 to a particular vertex.  In this example, vertex 1 is adjacent to
 2, 6 and 10, vertex 2 is adjacent to 1 and 3, etc.  Since Chaco ids
 start at 1 and VTK ids start at 0, the vertex ids in the vtkGraph
 will be 1 less than the Chaco ids.

To create an instance of class vtkChacoGraphReader, simply
invoke its constructor as follows
\begin{verbatim}
  obj = vtkChacoGraphReader
\end{verbatim}
\subsection{Methods}

The class vtkChacoGraphReader has several methods that can be used.
  They are listed below.
Note that the documentation is translated automatically from the VTK sources,
and may not be completely intelligible.  When in doubt, consult the VTK website.
In the methods listed below, \verb|obj| is an instance of the vtkChacoGraphReader class.
\begin{itemize}
\item  \verb|string = obj.GetClassName ()|

\item  \verb|int = obj.IsA (string name)|

\item  \verb|vtkChacoGraphReader = obj.NewInstance ()|

\item  \verb|vtkChacoGraphReader = obj.SafeDownCast (vtkObject o)|

\item  \verb|string = obj.GetFileName ()| -  The Chaco file name.

\item  \verb|obj.SetFileName (string )| -  The Chaco file name.

\end{itemize}
