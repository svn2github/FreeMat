\section{vtkDescriptiveStatistics}

\subsection{Usage}

 Given a selection of columns of interest in an input data table, this 
 class provides the following functionalities, depending on the
 execution mode it is executed in:
 * Learn: calculate extremal values, arithmetic mean, unbiased variance 
   estimator, skewness estimator, and both sample and G2 estimation of the 
   kurtosis excess. More precisely, Learn calculates the sums; if
    finalize is set to true (default), the final statistics are calculated
   with CalculateFromSums. Otherwise, only raw sums are output; this 
   option is made for efficient parallel calculations.
   Note that CalculateFromSums is a static function, so that it can be used
   directly with no need to instantiate a vtkDescriptiveStatistics object.
 * Assess: given an input data set in port INPUT\_DATA, and a reference value x along
   with an acceptable deviation d>0, assess all entries in the data set which
   are outside of [x-d,x+d].

 .SECTION Thanks
 Thanks to Philippe Pebay and David Thompson from Sandia National Laboratories 
 for implementing this class.

To create an instance of class vtkDescriptiveStatistics, simply
invoke its constructor as follows
\begin{verbatim}
  obj = vtkDescriptiveStatistics
\end{verbatim}
\subsection{Methods}

The class vtkDescriptiveStatistics has several methods that can be used.
  They are listed below.
Note that the documentation is translated automatically from the VTK sources,
and may not be completely intelligible.  When in doubt, consult the VTK website.
In the methods listed below, \verb|obj| is an instance of the vtkDescriptiveStatistics class.
\begin{itemize}
\item  \verb|string = obj.GetClassName ()|

\item  \verb|int = obj.IsA (string name)|

\item  \verb|vtkDescriptiveStatistics = obj.NewInstance ()|

\item  \verb|vtkDescriptiveStatistics = obj.SafeDownCast (vtkObject o)|

\item  \verb|obj.SetUnbiasedVariance (int )| -  Set/get whether the unbiased estimator for the variance should be used, or if
 the population variance will be calculated.
 The default is that the unbiased estimator will be used.

\item  \verb|int = obj.GetUnbiasedVariance ()| -  Set/get whether the unbiased estimator for the variance should be used, or if
 the population variance will be calculated.
 The default is that the unbiased estimator will be used.

\item  \verb|obj.UnbiasedVarianceOn ()| -  Set/get whether the unbiased estimator for the variance should be used, or if
 the population variance will be calculated.
 The default is that the unbiased estimator will be used.

\item  \verb|obj.UnbiasedVarianceOff ()| -  Set/get whether the unbiased estimator for the variance should be used, or if
 the population variance will be calculated.
 The default is that the unbiased estimator will be used.

\item  \verb|obj.SetSignedDeviations (int )| -  Set/get whether the deviations returned should be signed, or should
 only have their magnitude reported.
 The default is that signed deviations will be computed.

\item  \verb|int = obj.GetSignedDeviations ()| -  Set/get whether the deviations returned should be signed, or should
 only have their magnitude reported.
 The default is that signed deviations will be computed.

\item  \verb|obj.SignedDeviationsOn ()| -  Set/get whether the deviations returned should be signed, or should
 only have their magnitude reported.
 The default is that signed deviations will be computed.

\item  \verb|obj.SignedDeviationsOff ()| -  Set/get whether the deviations returned should be signed, or should
 only have their magnitude reported.
 The default is that signed deviations will be computed.

\item  \verb|obj.SetNominalParameter (string name)| -  A convenience method (in particular for UI wrapping) to set the name of the
 column that contains the nominal value for the Assess option.

\item  \verb|obj.SetDeviationParameter (string name)| -  A convenience method (in particular for UI wrapping) to set the name of the
 column that contains the deviation for the Assess option.

\item  \verb|obj.Aggregate (vtkDataObjectCollection , vtkDataObject )| -  Given a collection of models, calculate aggregate model

\end{itemize}
