\section{vtkForceDirectedLayoutStrategy}

\subsection{Usage}

 Lays out a graph in 2D or 3D using a force-directed algorithm.
 The user may specify whether to layout the graph randomly initially,
 the bounds, the number of dimensions (2 or 3), and the cool-down rate.

 .SECTION Thanks
 Thanks to Brian Wylie for adding functionality for allowing this layout
 to be incremental.

To create an instance of class vtkForceDirectedLayoutStrategy, simply
invoke its constructor as follows
\begin{verbatim}
  obj = vtkForceDirectedLayoutStrategy
\end{verbatim}
\subsection{Methods}

The class vtkForceDirectedLayoutStrategy has several methods that can be used.
  They are listed below.
Note that the documentation is translated automatically from the VTK sources,
and may not be completely intelligible.  When in doubt, consult the VTK website.
In the methods listed below, \verb|obj| is an instance of the vtkForceDirectedLayoutStrategy class.
\begin{itemize}
\item  \verb|string = obj.GetClassName ()|

\item  \verb|int = obj.IsA (string name)|

\item  \verb|vtkForceDirectedLayoutStrategy = obj.NewInstance ()|

\item  \verb|vtkForceDirectedLayoutStrategy = obj.SafeDownCast (vtkObject o)|

\item  \verb|obj.SetRandomSeed (int )| -  Seed the random number generator used to jitter point positions.
 This has a significant effect on their final positions when
 the layout is complete.

\item  \verb|int = obj.GetRandomSeedMinValue ()| -  Seed the random number generator used to jitter point positions.
 This has a significant effect on their final positions when
 the layout is complete.

\item  \verb|int = obj.GetRandomSeedMaxValue ()| -  Seed the random number generator used to jitter point positions.
 This has a significant effect on their final positions when
 the layout is complete.

\item  \verb|int = obj.GetRandomSeed ()| -  Seed the random number generator used to jitter point positions.
 This has a significant effect on their final positions when
 the layout is complete.

\item  \verb|obj.SetGraphBounds (double , double , double , double , double , double )| -  Set / get the region in space in which to place the final graph.
 The GraphBounds only affects the results if AutomaticBoundsComputation
 is off.

\item  \verb|obj.SetGraphBounds (double  a[6])| -  Set / get the region in space in which to place the final graph.
 The GraphBounds only affects the results if AutomaticBoundsComputation
 is off.

\item  \verb|double = obj. GetGraphBounds ()| -  Set / get the region in space in which to place the final graph.
 The GraphBounds only affects the results if AutomaticBoundsComputation
 is off.

\item  \verb|obj.SetAutomaticBoundsComputation (int )| -  Turn on/off automatic graph bounds calculation. If this
 boolean is off, then the manually specified GraphBounds is used.
 If on, then the input's bounds us used as the graph bounds.

\item  \verb|int = obj.GetAutomaticBoundsComputation ()| -  Turn on/off automatic graph bounds calculation. If this
 boolean is off, then the manually specified GraphBounds is used.
 If on, then the input's bounds us used as the graph bounds.

\item  \verb|obj.AutomaticBoundsComputationOn ()| -  Turn on/off automatic graph bounds calculation. If this
 boolean is off, then the manually specified GraphBounds is used.
 If on, then the input's bounds us used as the graph bounds.

\item  \verb|obj.AutomaticBoundsComputationOff ()| -  Turn on/off automatic graph bounds calculation. If this
 boolean is off, then the manually specified GraphBounds is used.
 If on, then the input's bounds us used as the graph bounds.

\item  \verb|obj.SetMaxNumberOfIterations (int )| -  Set/Get the maximum number of iterations to be used.
 The higher this number, the more iterations through the algorithm
 is possible, and thus, the more the graph gets modified.
 The default is '50' for no particular reason

\item  \verb|int = obj.GetMaxNumberOfIterationsMinValue ()| -  Set/Get the maximum number of iterations to be used.
 The higher this number, the more iterations through the algorithm
 is possible, and thus, the more the graph gets modified.
 The default is '50' for no particular reason

\item  \verb|int = obj.GetMaxNumberOfIterationsMaxValue ()| -  Set/Get the maximum number of iterations to be used.
 The higher this number, the more iterations through the algorithm
 is possible, and thus, the more the graph gets modified.
 The default is '50' for no particular reason

\item  \verb|int = obj.GetMaxNumberOfIterations ()| -  Set/Get the maximum number of iterations to be used.
 The higher this number, the more iterations through the algorithm
 is possible, and thus, the more the graph gets modified.
 The default is '50' for no particular reason

\item  \verb|obj.SetIterationsPerLayout (int )| -  Set/Get the number of iterations per layout.
 The only use for this ivar is for the application
 to do visualizations of the layout before it's complete.
 The default is '50' to match the default 'MaxNumberOfIterations'

\item  \verb|int = obj.GetIterationsPerLayoutMinValue ()| -  Set/Get the number of iterations per layout.
 The only use for this ivar is for the application
 to do visualizations of the layout before it's complete.
 The default is '50' to match the default 'MaxNumberOfIterations'

\item  \verb|int = obj.GetIterationsPerLayoutMaxValue ()| -  Set/Get the number of iterations per layout.
 The only use for this ivar is for the application
 to do visualizations of the layout before it's complete.
 The default is '50' to match the default 'MaxNumberOfIterations'

\item  \verb|int = obj.GetIterationsPerLayout ()| -  Set/Get the number of iterations per layout.
 The only use for this ivar is for the application
 to do visualizations of the layout before it's complete.
 The default is '50' to match the default 'MaxNumberOfIterations'

\item  \verb|obj.SetCoolDownRate (double )| -  Set/Get the Cool-down rate.
 The higher this number is, the longer it will take to ''cool-down'',
 and thus, the more the graph will be modified.

\item  \verb|double = obj.GetCoolDownRateMinValue ()| -  Set/Get the Cool-down rate.
 The higher this number is, the longer it will take to ''cool-down'',
 and thus, the more the graph will be modified.

\item  \verb|double = obj.GetCoolDownRateMaxValue ()| -  Set/Get the Cool-down rate.
 The higher this number is, the longer it will take to ''cool-down'',
 and thus, the more the graph will be modified.

\item  \verb|double = obj.GetCoolDownRate ()| -  Set/Get the Cool-down rate.
 The higher this number is, the longer it will take to ''cool-down'',
 and thus, the more the graph will be modified.

\item  \verb|obj.SetThreeDimensionalLayout (int )| -  Turn on/off layout of graph in three dimensions. If off, graph
 layout occurs in two dimensions. By default, three dimensional
 layout is off.

\item  \verb|int = obj.GetThreeDimensionalLayout ()| -  Turn on/off layout of graph in three dimensions. If off, graph
 layout occurs in two dimensions. By default, three dimensional
 layout is off.

\item  \verb|obj.ThreeDimensionalLayoutOn ()| -  Turn on/off layout of graph in three dimensions. If off, graph
 layout occurs in two dimensions. By default, three dimensional
 layout is off.

\item  \verb|obj.ThreeDimensionalLayoutOff ()| -  Turn on/off layout of graph in three dimensions. If off, graph
 layout occurs in two dimensions. By default, three dimensional
 layout is off.

\item  \verb|obj.SetRandomInitialPoints (int )| -  Turn on/off use of random positions within the graph bounds as initial points.

\item  \verb|int = obj.GetRandomInitialPoints ()| -  Turn on/off use of random positions within the graph bounds as initial points.

\item  \verb|obj.RandomInitialPointsOn ()| -  Turn on/off use of random positions within the graph bounds as initial points.

\item  \verb|obj.RandomInitialPointsOff ()| -  Turn on/off use of random positions within the graph bounds as initial points.

\item  \verb|obj.SetInitialTemperature (float )| -  Set the initial temperature.  If zero (the default) , the initial temperature
 will be computed automatically.

\item  \verb|float = obj.GetInitialTemperatureMinValue ()| -  Set the initial temperature.  If zero (the default) , the initial temperature
 will be computed automatically.

\item  \verb|float = obj.GetInitialTemperatureMaxValue ()| -  Set the initial temperature.  If zero (the default) , the initial temperature
 will be computed automatically.

\item  \verb|float = obj.GetInitialTemperature ()| -  Set the initial temperature.  If zero (the default) , the initial temperature
 will be computed automatically.

\item  \verb|obj.Initialize ()| -  This strategy sets up some data structures
 for faster processing of each Layout() call

\item  \verb|obj.Layout ()| -  This is the layout method where the graph that was
 set in SetGraph() is laid out. The method can either
 entirely layout the graph or iteratively lay out the
 graph. If you have an iterative layout please implement
 the IsLayoutComplete() method.

\item  \verb|int = obj.IsLayoutComplete ()|

\end{itemize}
