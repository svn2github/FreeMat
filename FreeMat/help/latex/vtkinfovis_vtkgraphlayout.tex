\section{vtkGraphLayout}

\subsection{Usage}

 This class is a shell for many graph layout strategies which may be set
 using the SetLayoutStrategy() function.  The layout strategies do the
 actual work.

 .SECION Thanks
 Thanks to Brian Wylie from Sandia National Laboratories for adding incremental
 layout capabilities.

To create an instance of class vtkGraphLayout, simply
invoke its constructor as follows
\begin{verbatim}
  obj = vtkGraphLayout
\end{verbatim}
\subsection{Methods}

The class vtkGraphLayout has several methods that can be used.
  They are listed below.
Note that the documentation is translated automatically from the VTK sources,
and may not be completely intelligible.  When in doubt, consult the VTK website.
In the methods listed below, \verb|obj| is an instance of the vtkGraphLayout class.
\begin{itemize}
\item  \verb|string = obj.GetClassName ()|

\item  \verb|int = obj.IsA (string name)|

\item  \verb|vtkGraphLayout = obj.NewInstance ()|

\item  \verb|vtkGraphLayout = obj.SafeDownCast (vtkObject o)|

\item  \verb|obj.SetLayoutStrategy (vtkGraphLayoutStrategy strategy)| -  The layout strategy to use during graph layout.

\item  \verb|vtkGraphLayoutStrategy = obj.GetLayoutStrategy ()| -  The layout strategy to use during graph layout.

\item  \verb|int = obj.IsLayoutComplete ()| -  Ask the layout algorithm if the layout is complete

\item  \verb|long = obj.GetMTime ()| -  Get the modification time of the layout algorithm.

\item  \verb|double = obj.GetZRange ()| -  Set the ZRange for the output data.
 If the initial layout is planar (i.e. all z coordinates are zero),
 the coordinates will be evenly spaced from 0.0 to ZRange.
 The default is zero, which has no effect.

\item  \verb|obj.SetZRange (double )| -  Set the ZRange for the output data.
 If the initial layout is planar (i.e. all z coordinates are zero),
 the coordinates will be evenly spaced from 0.0 to ZRange.
 The default is zero, which has no effect.

\item  \verb|vtkAbstractTransform = obj.GetTransform ()| -  Transform the graph vertices after the layout.

\item  \verb|obj.SetTransform (vtkAbstractTransform t)| -  Transform the graph vertices after the layout.

\item  \verb|obj.SetUseTransform (bool )| -  Whether to use the specified transform after layout.

\item  \verb|bool = obj.GetUseTransform ()| -  Whether to use the specified transform after layout.

\item  \verb|obj.UseTransformOn ()| -  Whether to use the specified transform after layout.

\item  \verb|obj.UseTransformOff ()| -  Whether to use the specified transform after layout.

\end{itemize}
