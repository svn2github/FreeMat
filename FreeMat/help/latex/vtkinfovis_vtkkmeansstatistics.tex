\section{vtkKMeansStatistics}

\subsection{Usage}

 This class takes as input an optional vtkTable on port LEARN\_PARAMETERS 
 specifying initial  set(s) of cluster values of the following form:  
 \begin{verbatim}
           K     | Col1            |  ...    | ColN    
      -----------+-----------------+---------+---------------
           M     |clustCoord(1, 1) |  ...    | clustCoord(1, N)
           M     |clustCoord(2, 1) |  ...    | clustCoord(2, N)
           .     |       .         |   .     |        .
           .     |       .         |   .     |        .
           .     |       .         |   .     |        .
           M     |clustCoord(M, 1) |  ...    | clustCoord(M, N)
           L     |clustCoord(1, 1) |  ...    | clustCoord(1, N)
           L     |clustCoord(2, 1) |  ...    | clustCoord(2, N)
           .     |       .         |   .     |        .
           .     |       .         |   .     |        .
           .     |       .         |   .     |        .
           L     |clustCoord(L, 1) |  ...    | clustCoord(L, N)
 \end{verbatim}

 Because the desired value of K is often not known in advance and the 
 results of the algorithm are dependent on the initial cluster centers, 
 we provide a mechanism for the user to test multiple runs or sets of cluster centers
 within a single call to the Learn phase.  The first column of the table identifies 
 the number of clusters K in the particular run (the entries in this column should be 
 of type vtkIdType), while the remaining columns are a 
 subset of the columns contained in the table on port INPUT\_DATA.  We require that 
 all user specified clusters be of the same dimension N and consequently, that the 
 LEARN\_PARAMETERS table have N+1 columns. Due to this restriction, only one request
 can be processed for each call to the Learn phase and subsequent requests are 
 silently ignored. Note that, if the first column of the LEARN\_PARAMETERS table is not 
 of type vtkIdType, then the table will be ignored and a single run will be performed using
 the first DefaultNumberOfClusters input data observations as initial cluster centers.

 When the user does not supply an initial set of clusters, then the first 
 DefaultNumberOfClusters input data observations are used as initial cluster 
 centers and a single run is performed.


 This class provides the following functionalities, depending on the 
 mode it is executed in:
 * Learn: calculates new cluster centers for each run.  The output metadata on 
   port OUTPUT\_MODEL is a multiblock dataset containing at a minimum
   one vtkTable with columns specifying the following for each run:
   the run ID, number of clusters, number of iterations required for convergence, 
   total error associated with the cluster (sum of squared Euclidean distance from each observation
   to its nearest cluster center), the cardinality of the cluster, and the new
   cluster coordinates.

 *Derive:  An additional vtkTable is stored in the multiblock dataset output on port OUTPUT\_MODEL.
   This table contains columns that store for each run: the runID, number of clusters, 
   total error for all clusters in the run, local rank, and global rank.
   The local rank is computed by comparing squared Euclidean errors of all runs with
   the same number of clusters.  The global rank is computed analagously across all runs.

 * Assess: This requires a multiblock dataset (as computed from Learn and Derive) on input port INPUT\_MODEL
   and tabular data on input port INPUT\_DATA that contains column names matching those
   of the tables on input port INPUT\_MODEL. The assess mode reports the closest cluster center
   and associated squared Euclidean distance of each observation in port INPUT\_DATA's table to the cluster centers for
   each run in the multiblock dataset provided on port INPUT\_MODEL.
  
 The code can handle a wide variety of data types as it operates on vtkAbstractArrays 
 and is not limited to vtkDataArrays.  A default distance functor that
 computes the sum of the squares of the Euclidean distance between two objects is provided 
 (vtkKMeansDistanceFunctor). The default distance functor can be overridden to use alternative distance metrics.

 .SECTION Thanks
 Thanks to Janine Bennett, David Thompson, and Philippe Pebay of
 Sandia National Laboratories for implementing this class.

To create an instance of class vtkKMeansStatistics, simply
invoke its constructor as follows
\begin{verbatim}
  obj = vtkKMeansStatistics
\end{verbatim}
\subsection{Methods}

The class vtkKMeansStatistics has several methods that can be used.
  They are listed below.
Note that the documentation is translated automatically from the VTK sources,
and may not be completely intelligible.  When in doubt, consult the VTK website.
In the methods listed below, \verb|obj| is an instance of the vtkKMeansStatistics class.
\begin{itemize}
\item  \verb|string = obj.GetClassName ()|

\item  \verb|int = obj.IsA (string name)|

\item  \verb|vtkKMeansStatistics = obj.NewInstance ()|

\item  \verb|vtkKMeansStatistics = obj.SafeDownCast (vtkObject o)|

\item  \verb|obj.SetDistanceFunctor (vtkKMeansDistanceFunctor )| -  Set the DistanceFunctor.

\item  \verb|vtkKMeansDistanceFunctor = obj.GetDistanceFunctor ()| -  Set the DistanceFunctor.

\item  \verb|obj.SetDefaultNumberOfClusters (int )| -  Set/get the  DefaultNumberOfClusters, used when no initial cluster coordinates are specified.

\item  \verb|int = obj.GetDefaultNumberOfClusters ()| -  Set/get the  DefaultNumberOfClusters, used when no initial cluster coordinates are specified.

\item  \verb|obj.SetKValuesArrayName (string )| -  Set/get the KValuesArrayName.

\item  \verb|string = obj.GetKValuesArrayName ()| -  Set/get the KValuesArrayName.

\item  \verb|obj.SetMaxNumIterations (int )| -  Set/get the MaxNumIterations used to terminate iterations on
 cluster center coordinates when the relative tolerance can not be met.

\item  \verb|int = obj.GetMaxNumIterations ()| -  Set/get the MaxNumIterations used to terminate iterations on
 cluster center coordinates when the relative tolerance can not be met.

\item  \verb|obj.SetTolerance (double )| -  Set/get the relative  Tolerance used to terminate iterations on
 cluster center coordinates.

\item  \verb|double = obj.GetTolerance ()| -  Set/get the relative  Tolerance used to terminate iterations on
 cluster center coordinates.

\item  \verb|obj.Aggregate (vtkDataObjectCollection , vtkDataObject )| -  Given a collection of models, calculate aggregate model
 NB: not implemented

\end{itemize}
