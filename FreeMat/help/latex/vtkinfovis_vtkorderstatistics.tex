\section{vtkOrderStatistics}

\subsection{Usage}

 Given a selection of columns of interest in an input data table, this 
 class provides the following functionalities, depending on the
 execution mode it is executed in:
 * Learn: calculate 5-point statistics (minimum, 1st quartile, median, third
   quartile, maximum) and all other deciles (1,2,3,4,6,7,8,9).
 * Assess: given an input data set in port INPUT\_DATA, and two percentiles p1 < p2,
   assess all entries in the data set which are outside of [p1,p2].

 .SECTION Thanks
 Thanks to Philippe Pebay and David Thompson from Sandia National Laboratories 
 for implementing this class.

To create an instance of class vtkOrderStatistics, simply
invoke its constructor as follows
\begin{verbatim}
  obj = vtkOrderStatistics
\end{verbatim}
\subsection{Methods}

The class vtkOrderStatistics has several methods that can be used.
  They are listed below.
Note that the documentation is translated automatically from the VTK sources,
and may not be completely intelligible.  When in doubt, consult the VTK website.
In the methods listed below, \verb|obj| is an instance of the vtkOrderStatistics class.
\begin{itemize}
\item  \verb|string = obj.GetClassName ()|

\item  \verb|int = obj.IsA (string name)|

\item  \verb|vtkOrderStatistics = obj.NewInstance ()|

\item  \verb|vtkOrderStatistics = obj.SafeDownCast (vtkObject o)|

\item  \verb|obj.SetNumberOfIntervals (vtkIdType )| -  Set the number of quantiles (with uniform spacing).

\item  \verb|vtkIdType = obj.GetNumberOfIntervals ()| -  Get the number of quantiles (with uniform spacing).

\item  \verb|obj.SetQuantileDefinition (int )| -  Set the quantile definition.

\item  \verb|vtkIdType = obj.GetQuantileDefinition ()| -  Given a collection of models, calculate aggregate model
 NB: not implemented

\item  \verb|obj.Aggregate (vtkDataObjectCollection , vtkDataObject )| -  Given a collection of models, calculate aggregate model
 NB: not implemented

\end{itemize}
