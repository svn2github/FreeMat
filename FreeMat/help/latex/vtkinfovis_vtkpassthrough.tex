\section{vtkPassThrough}

\subsection{Usage}

 The output type is always the same as the input object type.

To create an instance of class vtkPassThrough, simply
invoke its constructor as follows
\begin{verbatim}
  obj = vtkPassThrough
\end{verbatim}
\subsection{Methods}

The class vtkPassThrough has several methods that can be used.
  They are listed below.
Note that the documentation is translated automatically from the VTK sources,
and may not be completely intelligible.  When in doubt, consult the VTK website.
In the methods listed below, \verb|obj| is an instance of the vtkPassThrough class.
\begin{itemize}
\item  \verb|string = obj.GetClassName ()|

\item  \verb|int = obj.IsA (string name)|

\item  \verb|vtkPassThrough = obj.NewInstance ()|

\item  \verb|vtkPassThrough = obj.SafeDownCast (vtkObject o)|

\item  \verb|int = obj.FillInputPortInformation (int port, vtkInformation info)| -  Specify the first input port as optional

\item  \verb|obj.SetDeepCopyInput (int )| -  Whether or not to deep copy the input. This can be useful if you
 want to create a copy of a data object. You can then disconnect
 this filter's input connections and it will act like a source.
 Defaults to OFF.

\item  \verb|int = obj.GetDeepCopyInput ()| -  Whether or not to deep copy the input. This can be useful if you
 want to create a copy of a data object. You can then disconnect
 this filter's input connections and it will act like a source.
 Defaults to OFF.

\item  \verb|obj.DeepCopyInputOn ()| -  Whether or not to deep copy the input. This can be useful if you
 want to create a copy of a data object. You can then disconnect
 this filter's input connections and it will act like a source.
 Defaults to OFF.

\item  \verb|obj.DeepCopyInputOff ()| -  Whether or not to deep copy the input. This can be useful if you
 want to create a copy of a data object. You can then disconnect
 this filter's input connections and it will act like a source.
 Defaults to OFF.

\end{itemize}
