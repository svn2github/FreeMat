\section{vtkPassThroughLayoutStrategy}

\subsection{Usage}

 Yes, this incredible strategy does absoluted nothing to the data
 so in affect passes through the graph untouched. This strategy
 is useful in the cases where the graph is already laid out.

To create an instance of class vtkPassThroughLayoutStrategy, simply
invoke its constructor as follows
\begin{verbatim}
  obj = vtkPassThroughLayoutStrategy
\end{verbatim}
\subsection{Methods}

The class vtkPassThroughLayoutStrategy has several methods that can be used.
  They are listed below.
Note that the documentation is translated automatically from the VTK sources,
and may not be completely intelligible.  When in doubt, consult the VTK website.
In the methods listed below, \verb|obj| is an instance of the vtkPassThroughLayoutStrategy class.
\begin{itemize}
\item  \verb|string = obj.GetClassName ()|

\item  \verb|int = obj.IsA (string name)|

\item  \verb|vtkPassThroughLayoutStrategy = obj.NewInstance ()|

\item  \verb|vtkPassThroughLayoutStrategy = obj.SafeDownCast (vtkObject o)|

\item  \verb|obj.Initialize ()| -  This strategy sets up some data structures
 for faster processing of each Layout() call

\item  \verb|obj.Layout ()| -  This is the layout method where the graph that was
 set in SetGraph() is laid out. The method can either
 entirely layout the graph or iteratively lay out the
 graph. If you have an iterative layout please implement
 the IsLayoutComplete() method.

\item  \verb|int = obj.IsLayoutComplete ()|

\end{itemize}
