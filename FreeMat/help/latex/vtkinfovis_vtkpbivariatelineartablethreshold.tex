\section{vtkPBivariateLinearTableThreshold}

\subsection{Usage}

 Perform the table filtering operations provided by 
 vtkBivariateLinearTableThreshold in parallel.

To create an instance of class vtkPBivariateLinearTableThreshold, simply
invoke its constructor as follows
\begin{verbatim}
  obj = vtkPBivariateLinearTableThreshold
\end{verbatim}
\subsection{Methods}

The class vtkPBivariateLinearTableThreshold has several methods that can be used.
  They are listed below.
Note that the documentation is translated automatically from the VTK sources,
and may not be completely intelligible.  When in doubt, consult the VTK website.
In the methods listed below, \verb|obj| is an instance of the vtkPBivariateLinearTableThreshold class.
\begin{itemize}
\item  \verb|string = obj.GetClassName ()|

\item  \verb|int = obj.IsA (string name)|

\item  \verb|vtkPBivariateLinearTableThreshold = obj.NewInstance ()|

\item  \verb|vtkPBivariateLinearTableThreshold = obj.SafeDownCast (vtkObject o)|

\item  \verb|obj.SetController (vtkMultiProcessController )| -  Set the vtkMultiProcessController to be used for combining filter
 results from the individual nodes.

\item  \verb|vtkMultiProcessController = obj.GetController ()| -  Set the vtkMultiProcessController to be used for combining filter
 results from the individual nodes.

\end{itemize}
