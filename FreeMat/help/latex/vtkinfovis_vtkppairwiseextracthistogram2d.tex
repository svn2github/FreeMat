\section{vtkPPairwiseExtractHistogram2D}

\subsection{Usage}

  This class does exactly the same this as vtkPairwiseExtractHistogram2D,
  but does it in a multi-process environment.  After each node
  computes their own local histograms, this class does an AllReduce
  that distributes the sum of all local histograms onto each node.

  Because vtkPairwiseExtractHistogram2D is a light wrapper around a series
  of vtkExtractHistogram2D classes, this class just overrides the function
  that instantiates new histogram filters and returns the parallel version
  (vtkPExtractHistogram2D).


To create an instance of class vtkPPairwiseExtractHistogram2D, simply
invoke its constructor as follows
\begin{verbatim}
  obj = vtkPPairwiseExtractHistogram2D
\end{verbatim}
\subsection{Methods}

The class vtkPPairwiseExtractHistogram2D has several methods that can be used.
  They are listed below.
Note that the documentation is translated automatically from the VTK sources,
and may not be completely intelligible.  When in doubt, consult the VTK website.
In the methods listed below, \verb|obj| is an instance of the vtkPPairwiseExtractHistogram2D class.
\begin{itemize}
\item  \verb|string = obj.GetClassName ()|

\item  \verb|int = obj.IsA (string name)|

\item  \verb|vtkPPairwiseExtractHistogram2D = obj.NewInstance ()|

\item  \verb|vtkPPairwiseExtractHistogram2D = obj.SafeDownCast (vtkObject o)|

\item  \verb|obj.SetController (vtkMultiProcessController )|

\item  \verb|vtkMultiProcessController = obj.GetController ()|

\end{itemize}
