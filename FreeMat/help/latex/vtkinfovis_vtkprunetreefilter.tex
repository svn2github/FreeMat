\section{vtkPruneTreeFilter}

\subsection{Usage}

 Removes a subtree rooted at a particular vertex in a vtkTree.


To create an instance of class vtkPruneTreeFilter, simply
invoke its constructor as follows
\begin{verbatim}
  obj = vtkPruneTreeFilter
\end{verbatim}
\subsection{Methods}

The class vtkPruneTreeFilter has several methods that can be used.
  They are listed below.
Note that the documentation is translated automatically from the VTK sources,
and may not be completely intelligible.  When in doubt, consult the VTK website.
In the methods listed below, \verb|obj| is an instance of the vtkPruneTreeFilter class.
\begin{itemize}
\item  \verb|string = obj.GetClassName ()|

\item  \verb|int = obj.IsA (string name)|

\item  \verb|vtkPruneTreeFilter = obj.NewInstance ()|

\item  \verb|vtkPruneTreeFilter = obj.SafeDownCast (vtkObject o)|

\item  \verb|vtkIdType = obj.GetParentVertex ()| -  Set the parent vertex of the subtree to remove.

\item  \verb|obj.SetParentVertex (vtkIdType )| -  Set the parent vertex of the subtree to remove.

\end{itemize}
