\section{vtkRemoveHiddenData}

\subsection{Usage}

 Output only those rows/vertices/edges of the input vtkDataObject that 
 are visible, as defined by the vtkAnnotation::HIDE() flag of the input 
 vtkAnnotationLayers.
 Inputs:
    Port 0 - vtkDataObject
    Port 1 - vtkAnnotationLayers (optional)


To create an instance of class vtkRemoveHiddenData, simply
invoke its constructor as follows
\begin{verbatim}
  obj = vtkRemoveHiddenData
\end{verbatim}
\subsection{Methods}

The class vtkRemoveHiddenData has several methods that can be used.
  They are listed below.
Note that the documentation is translated automatically from the VTK sources,
and may not be completely intelligible.  When in doubt, consult the VTK website.
In the methods listed below, \verb|obj| is an instance of the vtkRemoveHiddenData class.
\begin{itemize}
\item  \verb|string = obj.GetClassName ()|

\item  \verb|int = obj.IsA (string name)|

\item  \verb|vtkRemoveHiddenData = obj.NewInstance ()|

\item  \verb|vtkRemoveHiddenData = obj.SafeDownCast (vtkObject o)|

\end{itemize}
