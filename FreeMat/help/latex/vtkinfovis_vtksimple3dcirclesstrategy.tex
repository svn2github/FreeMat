\section{vtkSimple3DCirclesStrategy}

\subsection{Usage}

 Places vertices on circles depending on the graph vertices hierarchy level.
 The source graph could be vtkDirectedAcyclicGraph or vtkDirectedGraph if MarkedStartPoints array was added.
 The algorithm collects the standalone points, too and take them to a separated circle. If method is FixedRadiusMethod,
 the radius of the circles will be equal. If method is FixedDistanceMethod, the distance beetwen the points on circles will
 be equal.

 In first step initial points are searched. A point is initial, if its in degree equal zero and out degree is greater than zero (or
 marked by MarkedStartVertices and out degree is greater than zero). Independent vertices (in and out degree equal zero) are collected
 separatelly. In second step the hierarchical level is generated for every vertex. In third step the hierarchical order is generated.
 If a vertex has no hierarchical level and it is not independent, the graph has loop so the algorithm exit with error message. Finally
 the vertices positions are calculated by the hierarchical order and by the vertices hierarchy levels.

 .SECTION Thanks
 Ferenc Nasztanovics, naszta@naszta.hu, Budapest University of Technology and Economics, Department of Structural Mechanics

 .SECTION References
 in 3D rotation was used: http://en.citizendium.org/wiki/Rotation\_matrix

To create an instance of class vtkSimple3DCirclesStrategy, simply
invoke its constructor as follows
\begin{verbatim}
  obj = vtkSimple3DCirclesStrategy
\end{verbatim}
\subsection{Methods}

The class vtkSimple3DCirclesStrategy has several methods that can be used.
  They are listed below.
Note that the documentation is translated automatically from the VTK sources,
and may not be completely intelligible.  When in doubt, consult the VTK website.
In the methods listed below, \verb|obj| is an instance of the vtkSimple3DCirclesStrategy class.
\begin{itemize}
\item  \verb|string = obj.GetClassName ()|

\item  \verb|int = obj.IsA (string name)|

\item  \verb|vtkSimple3DCirclesStrategy = obj.NewInstance ()|

\item  \verb|vtkSimple3DCirclesStrategy = obj.SafeDownCast (vtkObject o)|

\item  \verb|obj.SetMethod (int )| -  Set or get cicrle generating method (FixedRadiusMethod/FixedDistanceMethod). Default is FixedRadiusMethod.

\item  \verb|int = obj.GetMethod ()| -  Set or get cicrle generating method (FixedRadiusMethod/FixedDistanceMethod). Default is FixedRadiusMethod.

\item  \verb|obj.SetRadius (double )| -  If Method is FixedRadiusMethod: Set or get the radius of the circles.
 If Method is FixedDistanceMethod: Set or get the distance of the points in the circle.

\item  \verb|double = obj.GetRadius ()| -  If Method is FixedRadiusMethod: Set or get the radius of the circles.
 If Method is FixedDistanceMethod: Set or get the distance of the points in the circle.

\item  \verb|obj.SetHeight (double )| -  Set or get the vertical (local z) distance between the circles. If AutoHeight is on, this is the minimal height between
 the circle layers

\item  \verb|double = obj.GetHeight ()| -  Set or get the vertical (local z) distance between the circles. If AutoHeight is on, this is the minimal height between
 the circle layers

\item  \verb|obj.SetOrign (double , double , double )| -  Set or get the orign of the geometry. This is the center of the first circle. SetOrign(x,y,z)

\item  \verb|obj.SetOrign (double  a[3])| -  Set or get the orign of the geometry. This is the center of the first circle. SetOrign(x,y,z)

\item  \verb|double = obj. GetOrign ()| -  Set or get the orign of the geometry. This is the center of the first circle. SetOrign(x,y,z)

\item  \verb|obj.SetDirection (double dx, double dy, double dz)| -  Set or get the normal vector of the circles plain. The height is growing in this direction. The direction must not be zero vector.
 The default vector is (0.0,0.0,1.0)

\item  \verb|obj.SetDirection (double d[3])| -  Set or get the normal vector of the circles plain. The height is growing in this direction. The direction must not be zero vector.
 The default vector is (0.0,0.0,1.0)

\item  \verb|double = obj. GetDirection ()| -  Set or get the normal vector of the circles plain. The height is growing in this direction. The direction must not be zero vector.
 The default vector is (0.0,0.0,1.0)

\item  \verb|obj.SetMarkedStartVertices (vtkIntArray \_arg)| -  Set or get initial vertices. If MarkedStartVertices is added, loop is accepted in the graph. (If all of the loop start vertices are
 marked in MarkedStartVertices array.) MarkedStartVertices size must be equal with the number of the vertices in the graph. Start
 vertices must be marked by MarkedValue. (E.g.: if MarkedValue=3 and MarkedStartPoints is { 0, 3, 5, 3 }, the start points ids will
 be {1,3}.) )

\item  \verb|vtkIntArray = obj.GetMarkedStartVertices ()| -  Set or get initial vertices. If MarkedStartVertices is added, loop is accepted in the graph. (If all of the loop start vertices are
 marked in MarkedStartVertices array.) MarkedStartVertices size must be equal with the number of the vertices in the graph. Start
 vertices must be marked by MarkedValue. (E.g.: if MarkedValue=3 and MarkedStartPoints is { 0, 3, 5, 3 }, the start points ids will
 be {1,3}.) )

\item  \verb|obj.SetMarkedValue (int )| -  Set or get MarkedValue. See: MarkedStartVertices.

\item  \verb|int = obj.GetMarkedValue ()| -  Set or get MarkedValue. See: MarkedStartVertices.

\item  \verb|obj.SetForceToUseUniversalStartPointsFinder (int )| -  Set or get ForceToUseUniversalStartPointsFinder. If ForceToUseUniversalStartPointsFinder is true, MarkedStartVertices won't be used.
 In this case the input graph must be vtkDirectedAcyclicGraph (Defualt: false).

\item  \verb|int = obj.GetForceToUseUniversalStartPointsFinder ()| -  Set or get ForceToUseUniversalStartPointsFinder. If ForceToUseUniversalStartPointsFinder is true, MarkedStartVertices won't be used.
 In this case the input graph must be vtkDirectedAcyclicGraph (Defualt: false).

\item  \verb|obj.ForceToUseUniversalStartPointsFinderOn ()| -  Set or get ForceToUseUniversalStartPointsFinder. If ForceToUseUniversalStartPointsFinder is true, MarkedStartVertices won't be used.
 In this case the input graph must be vtkDirectedAcyclicGraph (Defualt: false).

\item  \verb|obj.ForceToUseUniversalStartPointsFinderOff ()| -  Set or get ForceToUseUniversalStartPointsFinder. If ForceToUseUniversalStartPointsFinder is true, MarkedStartVertices won't be used.
 In this case the input graph must be vtkDirectedAcyclicGraph (Defualt: false).

\item  \verb|obj.SetAutoHeight (int )| -  Set or get auto height (Default: false). If AutoHeight is true, (r(i+1) - r(i-1))/Height will be smaller than tan(MinimumRadian).
 If you want equal distances and parallel circles, you should turn off AutoHeight.

\item  \verb|int = obj.GetAutoHeight ()| -  Set or get auto height (Default: false). If AutoHeight is true, (r(i+1) - r(i-1))/Height will be smaller than tan(MinimumRadian).
 If you want equal distances and parallel circles, you should turn off AutoHeight.

\item  \verb|obj.AutoHeightOn ()| -  Set or get auto height (Default: false). If AutoHeight is true, (r(i+1) - r(i-1))/Height will be smaller than tan(MinimumRadian).
 If you want equal distances and parallel circles, you should turn off AutoHeight.

\item  \verb|obj.AutoHeightOff ()| -  Set or get auto height (Default: false). If AutoHeight is true, (r(i+1) - r(i-1))/Height will be smaller than tan(MinimumRadian).
 If you want equal distances and parallel circles, you should turn off AutoHeight.

\item  \verb|obj.SetMinimumRadian (double )| -  Set or get minimum radian (used by auto height).

\item  \verb|double = obj.GetMinimumRadian ()| -  Set or get minimum radian (used by auto height).

\item  \verb|obj.SetMinimumDegree (double degree)| -  Set or get minimum degree (used by auto height). There is no separated minimum degree, so minimum radian will be changed.

\item  \verb|double = obj.GetMinimumDegree (void )| -  Set or get minimum degree (used by auto height). There is no separated minimum degree, so minimum radian will be changed.

\item  \verb|obj.SetHierarchicalLayers (vtkIntArray \_arg)| -  Set or get hierarchical layers id by vertices (An usual vertex's layer id is greater or equal to zero. If a vertex is standalone, its
 layer id is -2.) If no HierarchicalLayers array is defined, vtkSimple3DCirclesStrategy will generate it automatically (default).

\item  \verb|vtkIntArray = obj.GetHierarchicalLayers ()| -  Set or get hierarchical layers id by vertices (An usual vertex's layer id is greater or equal to zero. If a vertex is standalone, its
 layer id is -2.) If no HierarchicalLayers array is defined, vtkSimple3DCirclesStrategy will generate it automatically (default).

\item  \verb|obj.SetHierarchicalOrder (vtkIdTypeArray \_arg)| -  Set or get hierarchical ordering of vertices (The array starts from the first vertex's id. All id must be greater or equal to zero!)
 If no HierarchicalOrder is defined, vtkSimple3DCirclesStrategy will generate it automatically (default).

\item  \verb|vtkIdTypeArray = obj.GetHierarchicalOrder ()| -  Set or get hierarchical ordering of vertices (The array starts from the first vertex's id. All id must be greater or equal to zero!)
 If no HierarchicalOrder is defined, vtkSimple3DCirclesStrategy will generate it automatically (default).

\item  \verb|obj.Layout (void )| -  Standard layout method

\item  \verb|obj.SetGraph (vtkGraph graph)| -  Set graph (warning: HierarchicalOrder and HierarchicalLayers will set to zero. These reference counts will be decreased!)

\end{itemize}
