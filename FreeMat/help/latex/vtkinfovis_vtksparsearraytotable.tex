\section{vtkSparseArrayToTable}

\subsection{Usage}

 Converts any sparse array to a vtkTable containing one row for each value
 stored in the array.  The table will contain one column of coordinates for each
 dimension in the source array, plus one column of array values.  A common use-case
 for vtkSparseArrayToTable would be converting a sparse array into a table
 suitable for use as an input to vtkTableToGraph.

 The coordinate columns in the output table will be named using the dimension labels
 from the source array,  The value column name can be explicitly set using
 SetValueColumn().

 .SECTION Thanks
 Developed by Timothy M. Shead (tshead@sandia.gov) at Sandia National Laboratories.

To create an instance of class vtkSparseArrayToTable, simply
invoke its constructor as follows
\begin{verbatim}
  obj = vtkSparseArrayToTable
\end{verbatim}
\subsection{Methods}

The class vtkSparseArrayToTable has several methods that can be used.
  They are listed below.
Note that the documentation is translated automatically from the VTK sources,
and may not be completely intelligible.  When in doubt, consult the VTK website.
In the methods listed below, \verb|obj| is an instance of the vtkSparseArrayToTable class.
\begin{itemize}
\item  \verb|string = obj.GetClassName ()|

\item  \verb|int = obj.IsA (string name)|

\item  \verb|vtkSparseArrayToTable = obj.NewInstance ()|

\item  \verb|vtkSparseArrayToTable = obj.SafeDownCast (vtkObject o)|

\item  \verb|string = obj.GetValueColumn ()| -  Specify the name of the output table column that contains array values.
 Default: ''value''

\item  \verb|obj.SetValueColumn (string )| -  Specify the name of the output table column that contains array values.
 Default: ''value''

\end{itemize}
