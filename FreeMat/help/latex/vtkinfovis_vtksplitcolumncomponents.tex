\section{vtkSplitColumnComponents}

\subsection{Usage}

 Splits any columns in a table that have more than one component into
 individual columns. Single component columns are passed through without
 any data duplication. So if column names ''Points'' had three components
 this column would be split into ''Points (0)'', ''Points (1)'' and Points (2)''.

To create an instance of class vtkSplitColumnComponents, simply
invoke its constructor as follows
\begin{verbatim}
  obj = vtkSplitColumnComponents
\end{verbatim}
\subsection{Methods}

The class vtkSplitColumnComponents has several methods that can be used.
  They are listed below.
Note that the documentation is translated automatically from the VTK sources,
and may not be completely intelligible.  When in doubt, consult the VTK website.
In the methods listed below, \verb|obj| is an instance of the vtkSplitColumnComponents class.
\begin{itemize}
\item  \verb|string = obj.GetClassName ()|

\item  \verb|int = obj.IsA (string name)|

\item  \verb|vtkSplitColumnComponents = obj.NewInstance ()|

\item  \verb|vtkSplitColumnComponents = obj.SafeDownCast (vtkObject o)|

\item  \verb|obj.SetCalculateMagnitudes (bool )| -  If on this filter will calculate an additional magnitude column for all
 columns it splits with two or more components.
 Default is on.

\item  \verb|bool = obj.GetCalculateMagnitudes ()| -  If on this filter will calculate an additional magnitude column for all
 columns it splits with two or more components.
 Default is on.

\end{itemize}
