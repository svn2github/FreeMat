\section{vtkTreeMapToPolyData}

\subsection{Usage}

 This algorithm requires that the vtkTreeMapLayout filter has already applied to the
 data in order to create the quadruple array (min x, max x, min y, max y) of
 bounds for each vertex of the tree.

To create an instance of class vtkTreeMapToPolyData, simply
invoke its constructor as follows
\begin{verbatim}
  obj = vtkTreeMapToPolyData
\end{verbatim}
\subsection{Methods}

The class vtkTreeMapToPolyData has several methods that can be used.
  They are listed below.
Note that the documentation is translated automatically from the VTK sources,
and may not be completely intelligible.  When in doubt, consult the VTK website.
In the methods listed below, \verb|obj| is an instance of the vtkTreeMapToPolyData class.
\begin{itemize}
\item  \verb|string = obj.GetClassName ()|

\item  \verb|int = obj.IsA (string name)|

\item  \verb|vtkTreeMapToPolyData = obj.NewInstance ()|

\item  \verb|vtkTreeMapToPolyData = obj.SafeDownCast (vtkObject o)|

\item  \verb|obj.SetRectanglesArrayName (string name)| -  The field containing the level of each tree node.
 This can be added using vtkTreeLevelsFilter before this filter.
 If this is not present, the filter simply calls tree->GetLevel(v) for
 each vertex, which will produce the same result, but
 may not be as efficient.

\item  \verb|obj.SetLevelArrayName (string name)| -  The spacing along the z-axis between tree map levels.

\item  \verb|double = obj.GetLevelDeltaZ ()| -  The spacing along the z-axis between tree map levels.

\item  \verb|obj.SetLevelDeltaZ (double )| -  The spacing along the z-axis between tree map levels.

\item  \verb|bool = obj.GetAddNormals ()| -  The spacing along the z-axis between tree map levels.

\item  \verb|obj.SetAddNormals (bool )| -  The spacing along the z-axis between tree map levels.

\item  \verb|int = obj.FillInputPortInformation (int port, vtkInformation info)|

\end{itemize}
