\section{vtkUnivariateStatisticsAlgorithm}

\subsection{Usage}

 This class specializes statistics algorithms to the univariate case, where
 a number of columns of interest can be selected in the input data set.
 This is done by the means of the following functions:

 ResetColumns() - reset the list of columns of interest.
 Add/RemoveColum( namCol ) - try to add/remove column with name namCol to/from
 the list.
 SetColumnStatus ( namCol, status ) - mostly for UI wrapping purposes, try to 
 add/remove (depending on status) namCol from the list of columns of interest.
 The verb ''try'' is used in the sense that neither attempting to 
 repeat an existing entry nor to remove a non-existent entry will work.
 
 .SECTION Thanks
 Thanks to Philippe Pebay and David Thompson from Sandia National Laboratories 
 for implementing this class.

To create an instance of class vtkUnivariateStatisticsAlgorithm, simply
invoke its constructor as follows
\begin{verbatim}
  obj = vtkUnivariateStatisticsAlgorithm
\end{verbatim}
\subsection{Methods}

The class vtkUnivariateStatisticsAlgorithm has several methods that can be used.
  They are listed below.
Note that the documentation is translated automatically from the VTK sources,
and may not be completely intelligible.  When in doubt, consult the VTK website.
In the methods listed below, \verb|obj| is an instance of the vtkUnivariateStatisticsAlgorithm class.
\begin{itemize}
\item  \verb|string = obj.GetClassName ()|

\item  \verb|int = obj.IsA (string name)|

\item  \verb|vtkUnivariateStatisticsAlgorithm = obj.NewInstance ()|

\item  \verb|vtkUnivariateStatisticsAlgorithm = obj.SafeDownCast (vtkObject o)|

\item  \verb|obj.AddColumn (string namCol)| -  Convenience method to create a request with a single column name  namCol in a single
 call; this is the preferred method to select columns, ensuring selection consistency
 (a single column per request).
 Warning: no name checking is performed on  namCol; it is the user's
 responsibility to use valid column names.

\item  \verb|int = obj.RequestSelectedColumns ()| -  Use the current column status values to produce a new request for statistics
 to be produced when RequestData() is called.
 Unlike the superclass implementation, this version adds a new request for each selected column
 instead of a single request containing all the columns.

\end{itemize}
