\section{vtkDataObjectReader}

\subsection{Usage}

 vtkDataObjectReader is a source object that reads ASCII or binary field
 data files in vtk format. Fields are general matrix structures used
 represent complex data. (See text for format details).  The output of this
 reader is a single vtkDataObject.  The superclass of this class,
 vtkDataReader, provides many methods for controlling the reading of the
 data file, see vtkDataReader for more information.

To create an instance of class vtkDataObjectReader, simply
invoke its constructor as follows
\begin{verbatim}
  obj = vtkDataObjectReader
\end{verbatim}
\subsection{Methods}

The class vtkDataObjectReader has several methods that can be used.
  They are listed below.
Note that the documentation is translated automatically from the VTK sources,
and may not be completely intelligible.  When in doubt, consult the VTK website.
In the methods listed below, \verb|obj| is an instance of the vtkDataObjectReader class.
\begin{itemize}
\item  \verb|string = obj.GetClassName ()|

\item  \verb|int = obj.IsA (string name)|

\item  \verb|vtkDataObjectReader = obj.NewInstance ()|

\item  \verb|vtkDataObjectReader = obj.SafeDownCast (vtkObject o)|

\item  \verb|vtkDataObject = obj.GetOutput ()| -  Get the output field of this reader.

\item  \verb|vtkDataObject = obj.GetOutput (int idx)| -  Get the output field of this reader.

\item  \verb|obj.SetOutput (vtkDataObject )| -  Get the output field of this reader.

\end{itemize}
