\section{vtkDEMReader}

\subsection{Usage}

 vtkDEMReader reads digital elevation files and creates image data.
 Digital elevation files are produced by the
 <A HREF=''http://www.usgs.gov''>US Geological Survey</A>. 
 A complete description of the DEM file is located at the USGS site.
 The reader reads the entire dem file and create a vtkImageData that
 contains a single scalar component that is the elevation in meters.
 The spacing is also expressed in meters. A number of get methods
 provide access to fields on the header.

To create an instance of class vtkDEMReader, simply
invoke its constructor as follows
\begin{verbatim}
  obj = vtkDEMReader
\end{verbatim}
\subsection{Methods}

The class vtkDEMReader has several methods that can be used.
  They are listed below.
Note that the documentation is translated automatically from the VTK sources,
and may not be completely intelligible.  When in doubt, consult the VTK website.
In the methods listed below, \verb|obj| is an instance of the vtkDEMReader class.
\begin{itemize}
\item  \verb|string = obj.GetClassName ()|

\item  \verb|int = obj.IsA (string name)|

\item  \verb|vtkDEMReader = obj.NewInstance ()|

\item  \verb|vtkDEMReader = obj.SafeDownCast (vtkObject o)|

\item  \verb|obj.SetFileName (string )| -  Specify file name of Digital Elevation Model (DEM) file

\item  \verb|string = obj.GetFileName ()| -  Specify file name of Digital Elevation Model (DEM) file

\item  \verb|obj.SetElevationReference (int )| -  Specify the elevation origin to use. By default, the elevation origin
 is equal to ElevationBounds[0]. A more convenient origin is to use sea
 level (i.e., a value of 0.0).

\item  \verb|int = obj.GetElevationReferenceMinValue ()| -  Specify the elevation origin to use. By default, the elevation origin
 is equal to ElevationBounds[0]. A more convenient origin is to use sea
 level (i.e., a value of 0.0).

\item  \verb|int = obj.GetElevationReferenceMaxValue ()| -  Specify the elevation origin to use. By default, the elevation origin
 is equal to ElevationBounds[0]. A more convenient origin is to use sea
 level (i.e., a value of 0.0).

\item  \verb|int = obj.GetElevationReference ()| -  Specify the elevation origin to use. By default, the elevation origin
 is equal to ElevationBounds[0]. A more convenient origin is to use sea
 level (i.e., a value of 0.0).

\item  \verb|obj.SetElevationReferenceToSeaLevel ()| -  Specify the elevation origin to use. By default, the elevation origin
 is equal to ElevationBounds[0]. A more convenient origin is to use sea
 level (i.e., a value of 0.0).

\item  \verb|obj.SetElevationReferenceToElevationBounds ()| -  Specify the elevation origin to use. By default, the elevation origin
 is equal to ElevationBounds[0]. A more convenient origin is to use sea
 level (i.e., a value of 0.0).

\item  \verb|string = obj.GetElevationReferenceAsString (void )| -  Specify the elevation origin to use. By default, the elevation origin
 is equal to ElevationBounds[0]. A more convenient origin is to use sea
 level (i.e., a value of 0.0).

\item  \verb|string = obj.GetMapLabel ()| -  An ASCII description of the map

\item  \verb|int = obj.GetDEMLevel ()| -  Code 1=DEM-1, 2=DEM\_2, ...

\item  \verb|int = obj.GetElevationPattern ()| -  Code 1=regular, 2=random, reserved for future use

\item  \verb|int = obj.GetGroundSystem ()| -  Ground planimetric reference system

\item  \verb|int = obj.GetGroundZone ()| -  Zone in ground planimetric reference system

\item  \verb|float = obj. GetProjectionParameters ()| -  Map Projection parameters. All are zero.

\item  \verb|int = obj.GetPlaneUnitOfMeasure ()| -  Defining unit of measure for ground planimetric coordinates throughout
 the file. 0 = radians, 1 = feet, 2 = meters, 3 = arc-seconds.

\item  \verb|int = obj.GetElevationUnitOfMeasure ()| -  Defining unit of measure for elevation coordinates throughout
 the file. 1 = feet, 2 = meters

\item  \verb|int = obj.GetPolygonSize ()| -  Number of sides in the polygon which defines the coverage of
 the DEM file. Set to 4.

\item  \verb|float = obj. GetElevationBounds ()| -  Minimum and maximum elevation for the DEM. The units in the file
 are in ElevationUnitOfMeasure. This class converts them to meters.

\item  \verb|float = obj.GetLocalRotation ()| -  Counterclockwise angle (in radians) from the primary axis of the planimetric
 reference to the primary axis of the DEM local reference system.
 IGNORED BY THIS IMPLEMENTATION.

\item  \verb|int = obj.GetAccuracyCode ()| -  Accuracy code for elevations. 0=unknown accuracy

\item  \verb|float = obj. GetSpatialResolution ()| -  DEM spatial resolution for x,y,z. Values are expressed in units of resolution.
 Since elevations are read as integers, this permits fractional elevations.

\item  \verb|int = obj. GetProfileDimension ()| -  The number of rows and columns in the DEM.

\end{itemize}
