\section{vtkDICOMImageReader}

\subsection{Usage}

 DICOM (stands for Digital Imaging in COmmunications and Medicine)
 is a medical image file format widely used to exchange data, provided
 by various modalities.
 .SECTION Warnings
 This reader might eventually handle ACR-NEMA file (predecessor of the DICOM
 format for medical images).
 This reader does not handle encapsulated format, only plain raw file are
 handled. This reader also does not handle multi-frames DICOM datasets.
 .SECTION Warnings
 Internally DICOMParser assumes the x,y pixel spacing is stored in 0028,0030 and
 that z spacing is stored in Slice Thickness (correct only when slice were acquired
 contiguous): 0018,0050. Which means this is only valid for some rare 
 MR Image Storage
 

To create an instance of class vtkDICOMImageReader, simply
invoke its constructor as follows
\begin{verbatim}
  obj = vtkDICOMImageReader
\end{verbatim}
\subsection{Methods}

The class vtkDICOMImageReader has several methods that can be used.
  They are listed below.
Note that the documentation is translated automatically from the VTK sources,
and may not be completely intelligible.  When in doubt, consult the VTK website.
In the methods listed below, \verb|obj| is an instance of the vtkDICOMImageReader class.
\begin{itemize}
\item  \verb|string = obj.GetClassName ()| -  Static method for construction.

\item  \verb|int = obj.IsA (string name)| -  Static method for construction.

\item  \verb|vtkDICOMImageReader = obj.NewInstance ()| -  Static method for construction.

\item  \verb|vtkDICOMImageReader = obj.SafeDownCast (vtkObject o)| -  Static method for construction.

\item  \verb|obj.SetFileName (string fn)| -  Set the directory name for the reader to look in for DICOM
 files. If this method is used, the reader will try to find
 all the DICOM files in a directory. It will select the subset
 corresponding to the first series UID it stumbles across and
 it will try to build an ordered volume from them based on
 the slice number. The volume building will be upgraded to
 something more sophisticated in the future.

\item  \verb|obj.SetDirectoryName (string dn)| -  Set the directory name for the reader to look in for DICOM
 files. If this method is used, the reader will try to find
 all the DICOM files in a directory. It will select the subset
 corresponding to the first series UID it stumbles across and
 it will try to build an ordered volume from them based on
 the slice number. The volume building will be upgraded to
 something more sophisticated in the future.

\item  \verb|string = obj.GetDirectoryName ()| -  Returns the directory name.

\item  \verb|double = obj.GetPixelSpacing ()| -  Returns the pixel spacing (in X, Y, Z).
 Note: if there is only one slice, the Z spacing is set to the slice
 thickness. If there is more than one slice, it is set to the distance
 between the first two slices.

\item  \verb|int = obj.GetWidth ()| -  Returns the image width.

\item  \verb|int = obj.GetHeight ()| -  Returns the image height.

\item  \verb|float = obj.GetImagePositionPatient ()| -  Get the (DICOM) x,y,z coordinates of the first pixel in the
 image (upper left hand corner) of the last image processed by the
 DICOMParser

\item  \verb|float = obj.GetImageOrientationPatient ()| -  Get the (DICOM) directions cosines. It consist of the components
 of the first two vectors. The third vector needs to be computed
 to form an orthonormal basis.

\item  \verb|int = obj.GetBitsAllocated ()| -  Get the number of bits allocated for each pixel in the file.

\item  \verb|int = obj.GetPixelRepresentation ()| -  Get the pixel representation of the last image processed by the
 DICOMParser. A zero is a unsigned quantity.  A one indicates a
 signed quantity

\item  \verb|int = obj.GetNumberOfComponents ()| -  Get the number of components of the image data for the last
 image processed.

\item  \verb|string = obj.GetTransferSyntaxUID ()| -  Get the transfer syntax UID for the last image processed.

\item  \verb|float = obj.GetRescaleSlope ()| -  Get the rescale slope for the pixel data.

\item  \verb|float = obj.GetRescaleOffset ()| -  Get the rescale offset for the pixel data.

\item  \verb|string = obj.GetPatientName ()| -  Get the patient name for the last image processed.

\item  \verb|string = obj.GetStudyUID ()| -  Get the study uid for the last image processed.

\item  \verb|string = obj.GetStudyID ()| -  Get the Study ID for the last image processed.

\item  \verb|float = obj.GetGantryAngle ()| -  Get the gantry angle for the last image processed.

\item  \verb|int = obj.CanReadFile (string fname)|

\item  \verb|string = obj.GetFileExtensions ()| -  Return a descriptive name for the file format that might be useful in a GUI.

\item  \verb|string = obj.GetDescriptiveName ()|

\end{itemize}
