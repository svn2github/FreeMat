\section{vtkGenericEnSightReader}

\subsection{Usage}

 The class vtkGenericEnSightReader allows the user to read an EnSight data
 set without a priori knowledge of what type of EnSight data set it is.

To create an instance of class vtkGenericEnSightReader, simply
invoke its constructor as follows
\begin{verbatim}
  obj = vtkGenericEnSightReader
\end{verbatim}
\subsection{Methods}

The class vtkGenericEnSightReader has several methods that can be used.
  They are listed below.
Note that the documentation is translated automatically from the VTK sources,
and may not be completely intelligible.  When in doubt, consult the VTK website.
In the methods listed below, \verb|obj| is an instance of the vtkGenericEnSightReader class.
\begin{itemize}
\item  \verb|string = obj.GetClassName ()|

\item  \verb|int = obj.IsA (string name)|

\item  \verb|vtkGenericEnSightReader = obj.NewInstance ()|

\item  \verb|vtkGenericEnSightReader = obj.SafeDownCast (vtkObject o)|

\item  \verb|obj.SetCaseFileName (string fileName)| -  Set/Get the Case file name.

\item  \verb|string = obj.GetCaseFileName ()| -  Set/Get the Case file name.

\item  \verb|obj.SetFilePath (string )| -  Set/Get the file path.

\item  \verb|string = obj.GetFilePath ()| -  Set/Get the file path.

\item  \verb|int = obj.GetNumberOfVariables ()| -  Get the number of variables listed in the case file.

\item  \verb|int = obj.GetNumberOfComplexVariables ()| -  Get the number of variables listed in the case file.

\item  \verb|int = obj.GetNumberOfVariables (int type)| -  Get the number of variables of a particular type.

\item  \verb|int = obj.GetNumberOfScalarsPerNode ()| -  Get the number of variables of a particular type.

\item  \verb|int = obj.GetNumberOfVectorsPerNode ()| -  Get the number of variables of a particular type.

\item  \verb|int = obj.GetNumberOfTensorsSymmPerNode ()| -  Get the number of variables of a particular type.

\item  \verb|int = obj.GetNumberOfScalarsPerElement ()| -  Get the number of variables of a particular type.

\item  \verb|int = obj.GetNumberOfVectorsPerElement ()| -  Get the number of variables of a particular type.

\item  \verb|int = obj.GetNumberOfTensorsSymmPerElement ()| -  Get the number of variables of a particular type.

\item  \verb|int = obj.GetNumberOfScalarsPerMeasuredNode ()| -  Get the number of variables of a particular type.

\item  \verb|int = obj.GetNumberOfVectorsPerMeasuredNode ()| -  Get the number of variables of a particular type.

\item  \verb|int = obj.GetNumberOfComplexScalarsPerNode ()| -  Get the number of variables of a particular type.

\item  \verb|int = obj.GetNumberOfComplexVectorsPerNode ()| -  Get the number of variables of a particular type.

\item  \verb|int = obj.GetNumberOfComplexScalarsPerElement ()| -  Get the number of variables of a particular type.

\item  \verb|int = obj.GetNumberOfComplexVectorsPerElement ()| -  Get the number of variables of a particular type.

\item  \verb|string = obj.GetDescription (int n)| -  Get the nth description for a non-complex variable.

\item  \verb|string = obj.GetComplexDescription (int n)| -  Get the nth description for a complex variable.

\item  \verb|string = obj.GetDescription (int n, int type)| -  Get the nth description of a particular variable type.  Returns NULL if no
 variable of this type exists in this data set.
 SCALAR\_PER\_NODE = 0; VECTOR\_PER\_NODE = 1;
 TENSOR\_SYMM\_PER\_NODE = 2; SCALAR\_PER\_ELEMENT = 3;
 VECTOR\_PER\_ELEMENT = 4; TENSOR\_SYMM\_PER\_ELEMENT = 5;
 SCALAR\_PER\_MEASURED\_NODE = 6; VECTOR\_PER\_MEASURED\_NODE = 7;
 COMPLEX\_SCALAR\_PER\_NODE = 8; COMPLEX\_VECTOR\_PER\_NODE 9;
 COMPLEX\_SCALAR\_PER\_ELEMENT  = 10; COMPLEX\_VECTOR\_PER\_ELEMENT = 11

\item  \verb|int = obj.GetVariableType (int n)| -  Get the variable type of variable n.

\item  \verb|int = obj.GetComplexVariableType (int n)| -  Get the variable type of variable n.

\item  \verb|obj.SetTimeValue (float value)| -  Set/Get the time value at which to get the value.

\item  \verb|float = obj.GetTimeValue ()| -  Set/Get the time value at which to get the value.

\item  \verb|float = obj.GetMinimumTimeValue ()| -  Get the minimum or maximum time value for this data set.

\item  \verb|float = obj.GetMaximumTimeValue ()| -  Get the minimum or maximum time value for this data set.

\item  \verb|vtkDataArrayCollection = obj.GetTimeSets ()| -  Get the time values per time set

\item  \verb|int = obj.DetermineEnSightVersion (int quiet)| -  Reads the FORMAT part of the case file to determine whether this is an
 EnSight6 or EnSightGold data set.  Returns an identifier listed in
 the FileTypes enum or -1 if an error occurred or the file could not
 be indentified as any EnSight type.

\item  \verb|obj.ReadAllVariablesOn ()| -  Set/get the flag for whether to read all the variables

\item  \verb|obj.ReadAllVariablesOff ()| -  Set/get the flag for whether to read all the variables

\item  \verb|obj.SetReadAllVariables (int )| -  Set/get the flag for whether to read all the variables

\item  \verb|int = obj.GetReadAllVariables ()| -  Set/get the flag for whether to read all the variables

\item  \verb|vtkDataArraySelection = obj.GetPointDataArraySelection ()| -  Get the data array selection tables used to configure which data
 arrays are loaded by the reader.

\item  \verb|vtkDataArraySelection = obj.GetCellDataArraySelection ()| -  Get the data array selection tables used to configure which data
 arrays are loaded by the reader.

\item  \verb|int = obj.GetNumberOfPointArrays ()| -  Get the number of point or cell arrays available in the input.

\item  \verb|int = obj.GetNumberOfCellArrays ()| -  Get the number of point or cell arrays available in the input.

\item  \verb|string = obj.GetPointArrayName (int index)| -  Get the name of the point or cell array with the given index in
 the input.

\item  \verb|string = obj.GetCellArrayName (int index)| -  Get the name of the point or cell array with the given index in
 the input.

\item  \verb|int = obj.GetPointArrayStatus (string name)| -  Get/Set whether the point or cell array with the given name is to
 be read.

\item  \verb|int = obj.GetCellArrayStatus (string name)| -  Get/Set whether the point or cell array with the given name is to
 be read.

\item  \verb|obj.SetPointArrayStatus (string name, int status)| -  Get/Set whether the point or cell array with the given name is to
 be read.

\item  \verb|obj.SetCellArrayStatus (string name, int status)| -  Get/Set whether the point or cell array with the given name is to
 be read.

\item  \verb|obj.SetByteOrderToBigEndian ()| -  Set the byte order of the file (remember, more Unix workstations
 write big endian whereas PCs write little endian). Default is
 big endian (since most older PLOT3D files were written by
 workstations).

\item  \verb|obj.SetByteOrderToLittleEndian ()| -  Set the byte order of the file (remember, more Unix workstations
 write big endian whereas PCs write little endian). Default is
 big endian (since most older PLOT3D files were written by
 workstations).

\item  \verb|obj.SetByteOrder (int )| -  Set the byte order of the file (remember, more Unix workstations
 write big endian whereas PCs write little endian). Default is
 big endian (since most older PLOT3D files were written by
 workstations).

\item  \verb|int = obj.GetByteOrder ()| -  Set the byte order of the file (remember, more Unix workstations
 write big endian whereas PCs write little endian). Default is
 big endian (since most older PLOT3D files were written by
 workstations).

\item  \verb|string = obj.GetByteOrderAsString ()| -  Set the byte order of the file (remember, more Unix workstations
 write big endian whereas PCs write little endian). Default is
 big endian (since most older PLOT3D files were written by
 workstations).

\item  \verb|string = obj.GetGeometryFileName ()| -  Get the Geometry file name. Made public to allow access from 
 apps requiring detailed info about the Data contents

\item  \verb|obj.SetParticleCoordinatesByIndex (int )| -  The MeasuredGeometryFile should list particle coordinates
 from 0->N-1.
 If a file is loaded where point Ids are listed from 1-N
 the Id to points reference will be wrong and the data
 will be generated incorrectly.
 Setting ParticleCoordinatesByIndex to true will force
 all Id's to increment from 0->N-1 (relative to their order
 in the file) and regardless of the actual Id of of the point.
 Warning, if the Points are listed in non sequential order
 then setting this flag will reorder them.

\item  \verb|int = obj.GetParticleCoordinatesByIndex ()| -  The MeasuredGeometryFile should list particle coordinates
 from 0->N-1.
 If a file is loaded where point Ids are listed from 1-N
 the Id to points reference will be wrong and the data
 will be generated incorrectly.
 Setting ParticleCoordinatesByIndex to true will force
 all Id's to increment from 0->N-1 (relative to their order
 in the file) and regardless of the actual Id of of the point.
 Warning, if the Points are listed in non sequential order
 then setting this flag will reorder them.

\item  \verb|obj.ParticleCoordinatesByIndexOn ()| -  The MeasuredGeometryFile should list particle coordinates
 from 0->N-1.
 If a file is loaded where point Ids are listed from 1-N
 the Id to points reference will be wrong and the data
 will be generated incorrectly.
 Setting ParticleCoordinatesByIndex to true will force
 all Id's to increment from 0->N-1 (relative to their order
 in the file) and regardless of the actual Id of of the point.
 Warning, if the Points are listed in non sequential order
 then setting this flag will reorder them.

\item  \verb|obj.ParticleCoordinatesByIndexOff ()| -  The MeasuredGeometryFile should list particle coordinates
 from 0->N-1.
 If a file is loaded where point Ids are listed from 1-N
 the Id to points reference will be wrong and the data
 will be generated incorrectly.
 Setting ParticleCoordinatesByIndex to true will force
 all Id's to increment from 0->N-1 (relative to their order
 in the file) and regardless of the actual Id of of the point.
 Warning, if the Points are listed in non sequential order
 then setting this flag will reorder them.

\end{itemize}
