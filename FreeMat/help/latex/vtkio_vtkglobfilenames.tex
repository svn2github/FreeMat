\section{vtkGlobFileNames}

\subsection{Usage}

 vtkGlobFileNames is a utility for finding files and directories
 that match a given wildcard pattern.  Allowed wildcards are
 *, ?, [...], [!...]. The ''*'' wildcard matches any substring,
 the ''?'' matches any single character, the [...] matches any one of
 the enclosed characters, e.g. [abc] will match one of a, b, or c,
 while [0-9] will match any digit, and [!...] will match any single
 character except for the ones within the brackets.  Special
 treatment is given to ''/'' (or ''\'' on Windows) because these are
 path separators.  These are never matched by a wildcard, they are
 only matched with another file separator.

To create an instance of class vtkGlobFileNames, simply
invoke its constructor as follows
\begin{verbatim}
  obj = vtkGlobFileNames
\end{verbatim}
\subsection{Methods}

The class vtkGlobFileNames has several methods that can be used.
  They are listed below.
Note that the documentation is translated automatically from the VTK sources,
and may not be completely intelligible.  When in doubt, consult the VTK website.
In the methods listed below, \verb|obj| is an instance of the vtkGlobFileNames class.
\begin{itemize}
\item  \verb|string = obj.GetClassName ()| -  Return the class name as a string.

\item  \verb|int = obj.IsA (string name)| -  Return the class name as a string.

\item  \verb|vtkGlobFileNames = obj.NewInstance ()| -  Return the class name as a string.

\item  \verb|vtkGlobFileNames = obj.SafeDownCast (vtkObject o)| -  Return the class name as a string.

\item  \verb|obj.Reset ()| -  Reset the glob by clearing the list of output filenames.

\item  \verb|obj.SetDirectory (string )| -  Set the directory in which to perform the glob.  If this is
 not set, then the current directory will be used.  Also, if
 you use a glob pattern that contains absolute path (one that
 starts with ''/'' or a drive letter) then that absolute path
 will be used and Directory will be ignored.

\item  \verb|string = obj.GetDirectory ()| -  Set the directory in which to perform the glob.  If this is
 not set, then the current directory will be used.  Also, if
 you use a glob pattern that contains absolute path (one that
 starts with ''/'' or a drive letter) then that absolute path
 will be used and Directory will be ignored.

\item  \verb|int = obj.AddFileNames (string pattern)| -  Search for all files that match the given expression,
 sort them, and add them to the output.  This method can
 be called repeatedly to add files matching additional patterns.
 Returns 1 if successful, otherwise returns zero.

\item  \verb|obj.SetRecurse (int )| -  Recurse into subdirectories.

\item  \verb|obj.RecurseOn ()| -  Recurse into subdirectories.

\item  \verb|obj.RecurseOff ()| -  Recurse into subdirectories.

\item  \verb|int = obj.GetRecurse ()| -  Recurse into subdirectories.

\item  \verb|int = obj.GetNumberOfFileNames ()| -  Return the number of files found.

\item  \verb|string = obj.GetNthFileName (int index)| -  Return the file at the given index, the indexing is 0 based.

\item  \verb|vtkStringArray = obj.GetFileNames ()| -  Get an array that contains all the file names.

\end{itemize}
