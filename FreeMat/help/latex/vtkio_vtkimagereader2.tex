\section{vtkImageReader2}

\subsection{Usage}

 vtkImageReader2 is the parent class for vtkImageReader.  It 
 is a good super class for streaming readers that do not require
 a mask or transform on the data.  vtkImageReader was implemented
 before vtkImageReader2, vtkImageReader2 is intended to have 
 a simpler interface.

To create an instance of class vtkImageReader2, simply
invoke its constructor as follows
\begin{verbatim}
  obj = vtkImageReader2
\end{verbatim}
\subsection{Methods}

The class vtkImageReader2 has several methods that can be used.
  They are listed below.
Note that the documentation is translated automatically from the VTK sources,
and may not be completely intelligible.  When in doubt, consult the VTK website.
In the methods listed below, \verb|obj| is an instance of the vtkImageReader2 class.
\begin{itemize}
\item  \verb|string = obj.GetClassName ()|

\item  \verb|int = obj.IsA (string name)|

\item  \verb|vtkImageReader2 = obj.NewInstance ()|

\item  \verb|vtkImageReader2 = obj.SafeDownCast (vtkObject o)|

\item  \verb|obj.SetFileName (string )| -  Specify file name for the image file. If the data is stored in
 multiple files, then use SetFileNames or SetFilePrefix instead.

\item  \verb|string = obj.GetFileName ()| -  Specify file name for the image file. If the data is stored in
 multiple files, then use SetFileNames or SetFilePrefix instead.

\item  \verb|obj.SetFileNames (vtkStringArray )| -  Specify a list of file names.  Each file must be a single slice,
 and each slice must be of the same size. The files must be in the
 correct order.
 Use SetFileName when reading a volume (multiple slice), since
 DataExtent will be modified after a SetFileNames call.

\item  \verb|vtkStringArray = obj.GetFileNames ()| -  Specify a list of file names.  Each file must be a single slice,
 and each slice must be of the same size. The files must be in the
 correct order.
 Use SetFileName when reading a volume (multiple slice), since
 DataExtent will be modified after a SetFileNames call.

\item  \verb|obj.SetFilePrefix (string )| -  Specify file prefix for the image file or files.  This can be
 used in place of SetFileName or SetFileNames if the filenames
 follow a specific naming pattern, but you must explicitly set
 the DataExtent so that the reader will know what range of slices
 to load.

\item  \verb|string = obj.GetFilePrefix ()| -  Specify file prefix for the image file or files.  This can be
 used in place of SetFileName or SetFileNames if the filenames
 follow a specific naming pattern, but you must explicitly set
 the DataExtent so that the reader will know what range of slices
 to load.

\item  \verb|obj.SetFilePattern (string )| -  The sprintf-style format string used to build filename from
 FilePrefix and slice number.

\item  \verb|string = obj.GetFilePattern ()| -  The sprintf-style format string used to build filename from
 FilePrefix and slice number.

\item  \verb|obj.SetDataScalarType (int type)| -  Set the data type of pixels in the file.  
 If you want the output scalar type to have a different value, set it
 after this method is called.

\item  \verb|obj.SetDataScalarTypeToFloat ()| -  Set the data type of pixels in the file.  
 If you want the output scalar type to have a different value, set it
 after this method is called.

\item  \verb|obj.SetDataScalarTypeToDouble ()| -  Set the data type of pixels in the file.  
 If you want the output scalar type to have a different value, set it
 after this method is called.

\item  \verb|obj.SetDataScalarTypeToInt ()| -  Set the data type of pixels in the file.  
 If you want the output scalar type to have a different value, set it
 after this method is called.

\item  \verb|obj.SetDataScalarTypeToUnsignedInt ()| -  Set the data type of pixels in the file.  
 If you want the output scalar type to have a different value, set it
 after this method is called.

\item  \verb|obj.SetDataScalarTypeToShort ()| -  Set the data type of pixels in the file.  
 If you want the output scalar type to have a different value, set it
 after this method is called.

\item  \verb|obj.SetDataScalarTypeToUnsignedShort ()| -  Set the data type of pixels in the file.  
 If you want the output scalar type to have a different value, set it
 after this method is called.

\item  \verb|obj.SetDataScalarTypeToChar ()| -  Set the data type of pixels in the file.  
 If you want the output scalar type to have a different value, set it
 after this method is called.

\item  \verb|obj.SetDataScalarTypeToSignedChar ()| -  Set the data type of pixels in the file.  
 If you want the output scalar type to have a different value, set it
 after this method is called.

\item  \verb|obj.SetDataScalarTypeToUnsignedChar ()| -  Get the file format.  Pixels are this type in the file.

\item  \verb|int = obj.GetDataScalarType ()| -  Get the file format.  Pixels are this type in the file.

\item  \verb|obj.SetNumberOfScalarComponents (int )| -  Set/Get the number of scalar components

\item  \verb|int = obj.GetNumberOfScalarComponents ()| -  Set/Get the number of scalar components

\item  \verb|obj.SetDataExtent (int , int , int , int , int , int )| -  Get/Set the extent of the data on disk.  

\item  \verb|obj.SetDataExtent (int  a[6])| -  Get/Set the extent of the data on disk.  

\item  \verb|int = obj. GetDataExtent ()| -  Get/Set the extent of the data on disk.  

\item  \verb|obj.SetFileDimensionality (int )| -  The number of dimensions stored in a file. This defaults to two.

\item  \verb|int = obj.GetFileDimensionality ()| -  Set/Get the spacing of the data in the file.

\item  \verb|obj.SetDataSpacing (double , double , double )| -  Set/Get the spacing of the data in the file.

\item  \verb|obj.SetDataSpacing (double  a[3])| -  Set/Get the spacing of the data in the file.

\item  \verb|double = obj. GetDataSpacing ()| -  Set/Get the spacing of the data in the file.

\item  \verb|obj.SetDataOrigin (double , double , double )| -  Set/Get the origin of the data (location of first pixel in the file).

\item  \verb|obj.SetDataOrigin (double  a[3])| -  Set/Get the origin of the data (location of first pixel in the file).

\item  \verb|double = obj. GetDataOrigin ()| -  Set/Get the origin of the data (location of first pixel in the file).

\item  \verb|long = obj.GetHeaderSize ()| -  Get the size of the header computed by this object.

\item  \verb|long = obj.GetHeaderSize (long slice)| -  Get the size of the header computed by this object.

\item  \verb|obj.SetHeaderSize (long size)| -  If there is a tail on the file, you want to explicitly set the
 header size.

\item  \verb|obj.SetDataByteOrderToBigEndian ()| -  These methods should be used instead of the SwapBytes methods.
 They indicate the byte ordering of the file you are trying
 to read in. These methods will then either swap or not swap
 the bytes depending on the byte ordering of the machine it is
 being run on. For example, reading in a BigEndian file on a
 BigEndian machine will result in no swapping. Trying to read
 the same file on a LittleEndian machine will result in swapping.
 As a quick note most UNIX machines are BigEndian while PC's
 and VAX tend to be LittleEndian. So if the file you are reading
 in was generated on a VAX or PC, SetDataByteOrderToLittleEndian 
 otherwise SetDataByteOrderToBigEndian. 

\item  \verb|obj.SetDataByteOrderToLittleEndian ()| -  These methods should be used instead of the SwapBytes methods.
 They indicate the byte ordering of the file you are trying
 to read in. These methods will then either swap or not swap
 the bytes depending on the byte ordering of the machine it is
 being run on. For example, reading in a BigEndian file on a
 BigEndian machine will result in no swapping. Trying to read
 the same file on a LittleEndian machine will result in swapping.
 As a quick note most UNIX machines are BigEndian while PC's
 and VAX tend to be LittleEndian. So if the file you are reading
 in was generated on a VAX or PC, SetDataByteOrderToLittleEndian 
 otherwise SetDataByteOrderToBigEndian. 

\item  \verb|int = obj.GetDataByteOrder ()| -  These methods should be used instead of the SwapBytes methods.
 They indicate the byte ordering of the file you are trying
 to read in. These methods will then either swap or not swap
 the bytes depending on the byte ordering of the machine it is
 being run on. For example, reading in a BigEndian file on a
 BigEndian machine will result in no swapping. Trying to read
 the same file on a LittleEndian machine will result in swapping.
 As a quick note most UNIX machines are BigEndian while PC's
 and VAX tend to be LittleEndian. So if the file you are reading
 in was generated on a VAX or PC, SetDataByteOrderToLittleEndian 
 otherwise SetDataByteOrderToBigEndian. 

\item  \verb|obj.SetDataByteOrder (int )| -  These methods should be used instead of the SwapBytes methods.
 They indicate the byte ordering of the file you are trying
 to read in. These methods will then either swap or not swap
 the bytes depending on the byte ordering of the machine it is
 being run on. For example, reading in a BigEndian file on a
 BigEndian machine will result in no swapping. Trying to read
 the same file on a LittleEndian machine will result in swapping.
 As a quick note most UNIX machines are BigEndian while PC's
 and VAX tend to be LittleEndian. So if the file you are reading
 in was generated on a VAX or PC, SetDataByteOrderToLittleEndian 
 otherwise SetDataByteOrderToBigEndian. 

\item  \verb|string = obj.GetDataByteOrderAsString ()| -  These methods should be used instead of the SwapBytes methods.
 They indicate the byte ordering of the file you are trying
 to read in. These methods will then either swap or not swap
 the bytes depending on the byte ordering of the machine it is
 being run on. For example, reading in a BigEndian file on a
 BigEndian machine will result in no swapping. Trying to read
 the same file on a LittleEndian machine will result in swapping.
 As a quick note most UNIX machines are BigEndian while PC's
 and VAX tend to be LittleEndian. So if the file you are reading
 in was generated on a VAX or PC, SetDataByteOrderToLittleEndian 
 otherwise SetDataByteOrderToBigEndian. 

\item  \verb|obj.SetFileNameSliceOffset (int )| -  When reading files which start at an unusual index, this can be added
 to the slice number when generating the file name (default = 0)

\item  \verb|int = obj.GetFileNameSliceOffset ()| -  When reading files which start at an unusual index, this can be added
 to the slice number when generating the file name (default = 0)

\item  \verb|obj.SetFileNameSliceSpacing (int )| -  When reading files which have regular, but non contiguous slices
 (eg filename.1,filename.3,filename.5)
 a spacing can be specified to skip missing files (default = 1)

\item  \verb|int = obj.GetFileNameSliceSpacing ()| -  When reading files which have regular, but non contiguous slices
 (eg filename.1,filename.3,filename.5)
 a spacing can be specified to skip missing files (default = 1)

\item  \verb|obj.SetSwapBytes (int )| -  Set/Get the byte swapping to explicitly swap the bytes of a file.

\item  \verb|int = obj.GetSwapBytes ()| -  Set/Get the byte swapping to explicitly swap the bytes of a file.

\item  \verb|obj.SwapBytesOn ()| -  Set/Get the byte swapping to explicitly swap the bytes of a file.

\item  \verb|obj.SwapBytesOff ()| -  Set/Get the byte swapping to explicitly swap the bytes of a file.

\item  \verb|int = obj.OpenFile ()|

\item  \verb|obj.SeekFile (int i, int j, int k)|

\item  \verb|obj.FileLowerLeftOn ()| -  Set/Get whether the data comes from the file starting in the lower left
 corner or upper left corner.

\item  \verb|obj.FileLowerLeftOff ()| -  Set/Get whether the data comes from the file starting in the lower left
 corner or upper left corner.

\item  \verb|int = obj.GetFileLowerLeft ()| -  Set/Get whether the data comes from the file starting in the lower left
 corner or upper left corner.

\item  \verb|obj.SetFileLowerLeft (int )| -  Set/Get whether the data comes from the file starting in the lower left
 corner or upper left corner.

\item  \verb|obj.ComputeInternalFileName (int slice)| -  Set/Get the internal file name

\item  \verb|string = obj.GetInternalFileName ()| -  Set/Get the internal file name

\item  \verb|int = obj.CanReadFile (string )| -  Get the file extensions for this format.
 Returns a string with a space separated list of extensions in 
 the format .extension

\item  \verb|string = obj.GetFileExtensions ()| -  Return a descriptive name for the file format that might be useful in a GUI.

\item  \verb|string = obj.GetDescriptiveName ()| -  Return a descriptive name for the file format that might be useful in a GUI.

\end{itemize}
