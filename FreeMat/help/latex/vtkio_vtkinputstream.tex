\section{vtkInputStream}

\subsection{Usage}

 vtkInputStream provides a VTK-style interface wrapping around a
 standard input stream.  The access methods are virtual so that
 subclasses can transparently provide decoding of an encoded stream.
 Data lengths for Seek and Read calls refer to the length of the
 input data.  The actual length in the stream may differ for
 subclasses that implement an encoding scheme.

To create an instance of class vtkInputStream, simply
invoke its constructor as follows
\begin{verbatim}
  obj = vtkInputStream
\end{verbatim}
\subsection{Methods}

The class vtkInputStream has several methods that can be used.
  They are listed below.
Note that the documentation is translated automatically from the VTK sources,
and may not be completely intelligible.  When in doubt, consult the VTK website.
In the methods listed below, \verb|obj| is an instance of the vtkInputStream class.
\begin{itemize}
\item  \verb|string = obj.GetClassName ()|

\item  \verb|int = obj.IsA (string name)|

\item  \verb|vtkInputStream = obj.NewInstance ()|

\item  \verb|vtkInputStream = obj.SafeDownCast (vtkObject o)|

\item  \verb|obj.StartReading ()| -  Called after the stream position has been set by the caller, but
 before any Seek or Read calls.  The stream position should not be
 adjusted by the caller until after an EndReading call.

\item  \verb|int = obj.Seek (long offset)| -  Seek to the given offset in the input data.  Returns 1 for
 success, 0 for failure.

\item  \verb|long = obj.Read (string data, long length)| -  Read input data of the given length.  Returns amount actually
 read.

\item  \verb|long = obj.Read (string data, long length)| -  Read input data of the given length.  Returns amount actually
 read.

\item  \verb|obj.EndReading ()| -  Called after all desired calls to Seek and Read have been made.
 After this call, the caller is free to change the position of the
 stream.  Additional reads should not be done until after another
 call to StartReading.

\end{itemize}
