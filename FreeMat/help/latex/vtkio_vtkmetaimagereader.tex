\section{vtkMetaImageReader}

\subsection{Usage}

 One of the formats for which a reader is already available in the toolkit is
 the MetaImage file format. This is a fairly simple yet powerful format
 consisting of a text header and a binary data section. The following
 instructions describe how you can write a MetaImage header for the data that
 you download from the BrainWeb page.

 The minimal structure of the MetaImage header is the following:

    NDims = 3
    DimSize = 181 217 181
    ElementType = MET\_UCHAR
    ElementSpacing = 1.0 1.0 1.0
    ElementByteOrderMSB = False
    ElementDataFile = brainweb1.raw

    * NDims indicate that this is a 3D image. ITK can handle images of
      arbitrary dimension.
    * DimSize indicates the size of the volume in pixels along each
      direction.
    * ElementType indicate the primitive type used for pixels. In this case
      is ''unsigned char'', implying that the data is digitized in 8 bits /
      pixel.
    * ElementSpacing indicates the physical separation between the center of
      one pixel and the center of the next pixel along each direction in space.
      The units used are millimeters.
    * ElementByteOrderMSB indicates is the data is encoded in little or big
      endian order. You might want to play with this value when moving data
      between different computer platforms.
    * ElementDataFile is the name of the file containing the raw binary data
      of the image. This file must be in the same directory as the header. 

 MetaImage headers are expected to have extension: ''.mha'' or ''.mhd''

 Once you write this header text file, it should be possible to read the
 image into your ITK based application using the itk::FileIOToImageFilter
 class.

To create an instance of class vtkMetaImageReader, simply
invoke its constructor as follows
\begin{verbatim}
  obj = vtkMetaImageReader
\end{verbatim}
\subsection{Methods}

The class vtkMetaImageReader has several methods that can be used.
  They are listed below.
Note that the documentation is translated automatically from the VTK sources,
and may not be completely intelligible.  When in doubt, consult the VTK website.
In the methods listed below, \verb|obj| is an instance of the vtkMetaImageReader class.
\begin{itemize}
\item  \verb|string = obj.GetClassName ()|

\item  \verb|int = obj.IsA (string name)|

\item  \verb|vtkMetaImageReader = obj.NewInstance ()|

\item  \verb|vtkMetaImageReader = obj.SafeDownCast (vtkObject o)|

\item  \verb|string = obj.GetFileExtensions ()|

\item  \verb|string = obj.GetDescriptiveName ()|

\item  \verb|int = obj.GetWidth ()|

\item  \verb|int = obj.GetHeight ()|

\item  \verb|int = obj.GetNumberOfComponents ()|

\item  \verb|int = obj.GetPixelRepresentation ()|

\item  \verb|int = obj.GetDataByteOrder (void )|

\item  \verb|double = obj.GetRescaleSlope ()|

\item  \verb|double = obj.GetRescaleOffset ()|

\item  \verb|int = obj.GetBitsAllocated ()|

\item  \verb|string = obj.GetDistanceUnits ()|

\item  \verb|string = obj.GetAnatomicalOrientation ()|

\item  \verb|double = obj.GetGantryAngle ()|

\item  \verb|string = obj.GetPatientName ()|

\item  \verb|string = obj.GetPatientID ()|

\item  \verb|string = obj.GetDate ()|

\item  \verb|string = obj.GetSeries ()|

\item  \verb|string = obj.GetImageNumber ()|

\item  \verb|string = obj.GetModality ()|

\item  \verb|string = obj.GetStudyID ()|

\item  \verb|string = obj.GetStudyUID ()|

\item  \verb|string = obj.GetTransferSyntaxUID ()|

\item  \verb|int = obj.CanReadFile (string name)| -  Test whether the file with the given name can be read by this
 reader.

\end{itemize}
