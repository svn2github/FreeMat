\section{vtkMINCImageWriter}

\subsection{Usage}

 MINC is a NetCDF-based medical image file format that was developed
 at the Montreal Neurological Institute in 1992. 
 The data is written slice-by-slice, and this writer is therefore
 suitable for streaming MINC data that is larger than the memory
 size through VTK.  This writer can also produce files with up to
 4 dimensions, where the fourth dimension is provided by using
 AddInput() to specify multiple input data sets.  If you want to
 set header information for the file, you must supply a
 vtkMINCImageAttributes 

To create an instance of class vtkMINCImageWriter, simply
invoke its constructor as follows
\begin{verbatim}
  obj = vtkMINCImageWriter
\end{verbatim}
\subsection{Methods}

The class vtkMINCImageWriter has several methods that can be used.
  They are listed below.
Note that the documentation is translated automatically from the VTK sources,
and may not be completely intelligible.  When in doubt, consult the VTK website.
In the methods listed below, \verb|obj| is an instance of the vtkMINCImageWriter class.
\begin{itemize}
\item  \verb|string = obj.GetClassName ()|

\item  \verb|int = obj.IsA (string name)|

\item  \verb|vtkMINCImageWriter = obj.NewInstance ()|

\item  \verb|vtkMINCImageWriter = obj.SafeDownCast (vtkObject o)|

\item  \verb|string = obj.GetFileExtensions ()| -  Get the name of this file format.

\item  \verb|string = obj.GetDescriptiveName ()| -  Set the file name.

\item  \verb|obj.SetFileName (string name)| -  Set the file name.

\item  \verb|obj.Write ()| -  Write the data.  This will attempt to stream the data
 slice-by-slice through the pipeline and out to the file,
 unless the whole extent of the input has already been
 updated.

\item  \verb|obj.SetDirectionCosines (vtkMatrix4x4 matrix)| -  Set a matrix that describes the orientation of the data.  The
 three columns of this matrix should give the unit-vector
 directions for the VTK x, y and z dimensions respectively.  
 The writer will use this information to determine how to map
 the VTK dimensions to the canonical MINC dimensions, and if
 necessary, the writer will re-order one or more dimensions
 back-to-front to ensure that no MINC dimension ends up with
 a direction cosines vector whose dot product with the canonical
 unit vector for that dimension is negative.

\item  \verb|vtkMatrix4x4 = obj.GetDirectionCosines ()| -  Set a matrix that describes the orientation of the data.  The
 three columns of this matrix should give the unit-vector
 directions for the VTK x, y and z dimensions respectively.  
 The writer will use this information to determine how to map
 the VTK dimensions to the canonical MINC dimensions, and if
 necessary, the writer will re-order one or more dimensions
 back-to-front to ensure that no MINC dimension ends up with
 a direction cosines vector whose dot product with the canonical
 unit vector for that dimension is negative.

\item  \verb|obj.SetRescaleSlope (double )| -  Set the slope and intercept for rescaling the intensities.  The
 default values are zero, which indicates to the reader that no
 rescaling is to be performed.

\item  \verb|double = obj.GetRescaleSlope ()| -  Set the slope and intercept for rescaling the intensities.  The
 default values are zero, which indicates to the reader that no
 rescaling is to be performed.

\item  \verb|obj.SetRescaleIntercept (double )| -  Set the slope and intercept for rescaling the intensities.  The
 default values are zero, which indicates to the reader that no
 rescaling is to be performed.

\item  \verb|double = obj.GetRescaleIntercept ()| -  Set the slope and intercept for rescaling the intensities.  The
 default values are zero, which indicates to the reader that no
 rescaling is to be performed.

\item  \verb|obj.SetImageAttributes (vtkMINCImageAttributes attributes)| -  Set the image attributes, which contain patient information and
 other useful metadata.

\item  \verb|vtkMINCImageAttributes = obj.GetImageAttributes ()| -  Set the image attributes, which contain patient information and
 other useful metadata.

\item  \verb|obj.SetStrictValidation (int )| -  Set whether to validate that all variable attributes that
 have been set are ones that are listed in the MINC standard.

\item  \verb|obj.StrictValidationOn ()| -  Set whether to validate that all variable attributes that
 have been set are ones that are listed in the MINC standard.

\item  \verb|obj.StrictValidationOff ()| -  Set whether to validate that all variable attributes that
 have been set are ones that are listed in the MINC standard.

\item  \verb|int = obj.GetStrictValidation ()| -  Set whether to validate that all variable attributes that
 have been set are ones that are listed in the MINC standard.

\item  \verb|obj.SetHistoryAddition (string )| -  Set a string value to append to the history of the file.  This
 string should describe, briefly, how the file was processed.

\item  \verb|string = obj.GetHistoryAddition ()| -  Set a string value to append to the history of the file.  This
 string should describe, briefly, how the file was processed.

\end{itemize}
