\section{vtkNetCDFReader}

\subsection{Usage}


 A superclass for reading netCDF files.  Subclass add conventions to the
 reader.  This class just outputs data into a multi block data set with a
 vtkImageData at each block.  A block is created for each variable except that
 variables with matching dimensions will be placed in the same block.

To create an instance of class vtkNetCDFReader, simply
invoke its constructor as follows
\begin{verbatim}
  obj = vtkNetCDFReader
\end{verbatim}
\subsection{Methods}

The class vtkNetCDFReader has several methods that can be used.
  They are listed below.
Note that the documentation is translated automatically from the VTK sources,
and may not be completely intelligible.  When in doubt, consult the VTK website.
In the methods listed below, \verb|obj| is an instance of the vtkNetCDFReader class.
\begin{itemize}
\item  \verb|string = obj.GetClassName ()|

\item  \verb|int = obj.IsA (string name)|

\item  \verb|vtkNetCDFReader = obj.NewInstance ()|

\item  \verb|vtkNetCDFReader = obj.SafeDownCast (vtkObject o)|

\item  \verb|obj.SetFileName (string filename)|

\item  \verb|string = obj.GetFileName ()|

\item  \verb|int = obj.UpdateMetaData ()| -  Update the meta data from the current file.  Automatically called
 during the RequestInformation pipeline update stage.

\item  \verb|int = obj.GetNumberOfVariableArrays ()| -  Variable array selection.

\item  \verb|string = obj.GetVariableArrayName (int idx)| -  Variable array selection.

\item  \verb|int = obj.GetVariableArrayStatus (string name)| -  Variable array selection.

\item  \verb|obj.SetVariableArrayStatus (string name, int status)| -  Variable array selection.

\item  \verb|vtkStringArray = obj.GetVariableDimensions ()| -  Returns an array with string encodings for the dimensions used in each of
 the variables.  The indices in the returned array correspond to those used
 in the GetVariableArrayName method.  Two arrays with the same dimensions
 will have the same encoded string returned by this method.

\item  \verb|obj.SetDimensions (string dimensions)| -  Loads the grid with the given dimensions.  The dimensions are encoded in a
 string that conforms to the same format as returned by
 GetVariableDimensions and GetAllDimensions.  This method is really a
 convenience method for SetVariableArrayStatus.  It turns on all variables
 that have the given dimensions and turns off all other variables.

\item  \verb|vtkStringArray = obj.GetAllDimensions ()| -  Returns an array with string encodings for the dimension combinations used
 in the variables.  The result is the same as GetVariableDimensions except
 that each entry in the array is unique (a set of dimensions is only given
 once even if it occurs for multiple variables) and the order is
 meaningless.

\item  \verb|int = obj.GetReplaceFillValueWithNan ()| -  If on, any float or double variable read that has a \_FillValue attribute
 will have that fill value replaced with a not-a-number (NaN) value.  The
 advantage of setting these to NaN values is that, if implemented properly
 by the system and careful math operations are used, they can implicitly be
 ignored by calculations like finding the range of the values.  That said,
 this option should be used with caution as VTK does not fully support NaN
 values and therefore odd calculations may occur.  By default this is off.

\item  \verb|obj.SetReplaceFillValueWithNan (int )| -  If on, any float or double variable read that has a \_FillValue attribute
 will have that fill value replaced with a not-a-number (NaN) value.  The
 advantage of setting these to NaN values is that, if implemented properly
 by the system and careful math operations are used, they can implicitly be
 ignored by calculations like finding the range of the values.  That said,
 this option should be used with caution as VTK does not fully support NaN
 values and therefore odd calculations may occur.  By default this is off.

\item  \verb|obj.ReplaceFillValueWithNanOn ()| -  If on, any float or double variable read that has a \_FillValue attribute
 will have that fill value replaced with a not-a-number (NaN) value.  The
 advantage of setting these to NaN values is that, if implemented properly
 by the system and careful math operations are used, they can implicitly be
 ignored by calculations like finding the range of the values.  That said,
 this option should be used with caution as VTK does not fully support NaN
 values and therefore odd calculations may occur.  By default this is off.

\item  \verb|obj.ReplaceFillValueWithNanOff ()| -  If on, any float or double variable read that has a \_FillValue attribute
 will have that fill value replaced with a not-a-number (NaN) value.  The
 advantage of setting these to NaN values is that, if implemented properly
 by the system and careful math operations are used, they can implicitly be
 ignored by calculations like finding the range of the values.  That said,
 this option should be used with caution as VTK does not fully support NaN
 values and therefore odd calculations may occur.  By default this is off.

\end{itemize}
