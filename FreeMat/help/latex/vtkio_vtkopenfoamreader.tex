\section{vtkOpenFOAMReader}

\subsection{Usage}

 vtkOpenFOAMReader creates a multiblock dataset. It reads mesh
 information and time dependent data.  The polyMesh folders contain
 mesh information. The time folders contain transient data for the
 cells. Each folder can contain any number of data files.

To create an instance of class vtkOpenFOAMReader, simply
invoke its constructor as follows
\begin{verbatim}
  obj = vtkOpenFOAMReader
\end{verbatim}
\subsection{Methods}

The class vtkOpenFOAMReader has several methods that can be used.
  They are listed below.
Note that the documentation is translated automatically from the VTK sources,
and may not be completely intelligible.  When in doubt, consult the VTK website.
In the methods listed below, \verb|obj| is an instance of the vtkOpenFOAMReader class.
\begin{itemize}
\item  \verb|string = obj.GetClassName ()|

\item  \verb|int = obj.IsA (string name)|

\item  \verb|vtkOpenFOAMReader = obj.NewInstance ()|

\item  \verb|vtkOpenFOAMReader = obj.SafeDownCast (vtkObject o)|

\item  \verb|int = obj.CanReadFile (string )| -  Determine if the file can be readed with this reader.

\item  \verb|obj.SetFileName (string )| -  Set/Get the filename.

\item  \verb|string = obj.GetFileName ()| -  Set/Get the filename.

\item  \verb|int = obj.GetNumberOfCellArrays (void )| -  Get/Set whether the cell array with the given name is to
 be read.

\item  \verb|int = obj.GetCellArrayStatus (string name)| -  Get/Set whether the cell array with the given name is to
 be read.

\item  \verb|obj.SetCellArrayStatus (string name, int status)| -  Get the name of the  cell array with the given index in
 the input.

\item  \verb|string = obj.GetCellArrayName (int index)| -  Turn on/off all cell arrays.

\item  \verb|obj.DisableAllCellArrays ()| -  Turn on/off all cell arrays.

\item  \verb|obj.EnableAllCellArrays ()| -  Get the number of point arrays available in the input.

\item  \verb|int = obj.GetNumberOfPointArrays (void )| -  Get/Set whether the point array with the given name is to
 be read.

\item  \verb|int = obj.GetPointArrayStatus (string name)| -  Get/Set whether the point array with the given name is to
 be read.

\item  \verb|obj.SetPointArrayStatus (string name, int status)| -  Get the name of the  point array with the given index in
 the input.

\item  \verb|string = obj.GetPointArrayName (int index)| -  Turn on/off all point arrays.

\item  \verb|obj.DisableAllPointArrays ()| -  Turn on/off all point arrays.

\item  \verb|obj.EnableAllPointArrays ()| -  Get the number of Lagrangian arrays available in the input.

\item  \verb|int = obj.GetNumberOfLagrangianArrays (void )| -  Get/Set whether the Lagrangian array with the given name is to
 be read.

\item  \verb|int = obj.GetLagrangianArrayStatus (string name)| -  Get/Set whether the Lagrangian array with the given name is to
 be read.

\item  \verb|obj.SetLagrangianArrayStatus (string name, int status)| -  Get the name of the  Lagrangian array with the given index in
 the input.

\item  \verb|string = obj.GetLagrangianArrayName (int index)| -  Turn on/off all Lagrangian arrays.

\item  \verb|obj.DisableAllLagrangianArrays ()| -  Turn on/off all Lagrangian arrays.

\item  \verb|obj.EnableAllLagrangianArrays ()| -  Get the number of Patches (inlcuding Internal Mesh) available in the input.

\item  \verb|int = obj.GetNumberOfPatchArrays (void )| -  Get/Set whether the Patch with the given name is to
 be read.

\item  \verb|int = obj.GetPatchArrayStatus (string name)| -  Get/Set whether the Patch with the given name is to
 be read.

\item  \verb|obj.SetPatchArrayStatus (string name, int status)| -  Get the name of the Patch with the given index in
 the input.

\item  \verb|string = obj.GetPatchArrayName (int index)| -  Turn on/off all Patches including the Internal Mesh.

\item  \verb|obj.DisableAllPatchArrays ()| -  Turn on/off all Patches including the Internal Mesh.

\item  \verb|obj.EnableAllPatchArrays ()| -  Set/Get whether to create cell-to-point translated data for cell-type data

\item  \verb|obj.SetCreateCellToPoint (int )| -  Set/Get whether to create cell-to-point translated data for cell-type data

\item  \verb|int = obj.GetCreateCellToPoint ()| -  Set/Get whether to create cell-to-point translated data for cell-type data

\item  \verb|obj.CreateCellToPointOn ()| -  Set/Get whether to create cell-to-point translated data for cell-type data

\item  \verb|obj.CreateCellToPointOff ()| -  Set/Get whether to create cell-to-point translated data for cell-type data

\item  \verb|obj.SetCacheMesh (int )| -  Set/Get whether mesh is to be cached.

\item  \verb|int = obj.GetCacheMesh ()| -  Set/Get whether mesh is to be cached.

\item  \verb|obj.CacheMeshOn ()| -  Set/Get whether mesh is to be cached.

\item  \verb|obj.CacheMeshOff ()| -  Set/Get whether mesh is to be cached.

\item  \verb|obj.SetDecomposePolyhedra (int )| -  Set/Get whether polyhedra are to be decomposed.

\item  \verb|int = obj.GetDecomposePolyhedra ()| -  Set/Get whether polyhedra are to be decomposed.

\item  \verb|obj.DecomposePolyhedraOn ()| -  Set/Get whether polyhedra are to be decomposed.

\item  \verb|obj.DecomposePolyhedraOff ()| -  Set/Get whether polyhedra are to be decomposed.

\item  \verb|obj.SetPositionsIsIn13Format (int )| -  Set/Get whether the lagrangian/positions is in OF 1.3 format

\item  \verb|int = obj.GetPositionsIsIn13Format ()| -  Set/Get whether the lagrangian/positions is in OF 1.3 format

\item  \verb|obj.PositionsIsIn13FormatOn ()| -  Set/Get whether the lagrangian/positions is in OF 1.3 format

\item  \verb|obj.PositionsIsIn13FormatOff ()| -  Set/Get whether the lagrangian/positions is in OF 1.3 format

\item  \verb|obj.SetListTimeStepsByControlDict (int )| -  Determine if time directories are to be listed according to controlDict

\item  \verb|int = obj.GetListTimeStepsByControlDict ()| -  Determine if time directories are to be listed according to controlDict

\item  \verb|obj.ListTimeStepsByControlDictOn ()| -  Determine if time directories are to be listed according to controlDict

\item  \verb|obj.ListTimeStepsByControlDictOff ()| -  Determine if time directories are to be listed according to controlDict

\item  \verb|obj.SetAddDimensionsToArrayNames (int )| -  Add dimensions to array names

\item  \verb|int = obj.GetAddDimensionsToArrayNames ()| -  Add dimensions to array names

\item  \verb|obj.AddDimensionsToArrayNamesOn ()| -  Add dimensions to array names

\item  \verb|obj.AddDimensionsToArrayNamesOff ()| -  Add dimensions to array names

\item  \verb|obj.SetReadZones (int )| -  Set/Get whether zones will be read.

\item  \verb|int = obj.GetReadZones ()| -  Set/Get whether zones will be read.

\item  \verb|obj.ReadZonesOn ()| -  Set/Get whether zones will be read.

\item  \verb|obj.ReadZonesOff ()| -  Set/Get whether zones will be read.

\item  \verb|obj.SetRefresh ()|

\item  \verb|obj.SetParent (vtkOpenFOAMReader parent)|

\item  \verb|bool = obj.SetTimeValue (double )|

\item  \verb|vtkDoubleArray = obj.GetTimeValues ()|

\item  \verb|int = obj.MakeMetaDataAtTimeStep (bool )|

\end{itemize}
