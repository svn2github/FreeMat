\section{vtkParticleReader}

\subsection{Usage}

 vtkParticleReader reads either a binary or a text file of 
  particles. Each particle can have associated with it an optional
  scalar value. So the format is: x, y, z, scalar 
  (all floats or doubles). The text file can consist of a comma 
  delimited set of values. In most cases vtkParticleReader can 
  automatically determine whether the file is text or binary. 
  The data can be either float or double. 
  Progress updates are provided. 
  With respect to binary files, random access into the file to read 
  pieces is supported.
  

To create an instance of class vtkParticleReader, simply
invoke its constructor as follows
\begin{verbatim}
  obj = vtkParticleReader
\end{verbatim}
\subsection{Methods}

The class vtkParticleReader has several methods that can be used.
  They are listed below.
Note that the documentation is translated automatically from the VTK sources,
and may not be completely intelligible.  When in doubt, consult the VTK website.
In the methods listed below, \verb|obj| is an instance of the vtkParticleReader class.
\begin{itemize}
\item  \verb|string = obj.GetClassName ()|

\item  \verb|int = obj.IsA (string name)|

\item  \verb|vtkParticleReader = obj.NewInstance ()|

\item  \verb|vtkParticleReader = obj.SafeDownCast (vtkObject o)|

\item  \verb|obj.SetFileName (string )| -  Specify file name.

\item  \verb|string = obj.GetFileName ()| -  Specify file name.

\item  \verb|obj.SetDataByteOrderToBigEndian ()| -  These methods should be used instead of the SwapBytes methods.
 They indicate the byte ordering of the file you are trying
 to read in. These methods will then either swap or not swap
 the bytes depending on the byte ordering of the machine it is
 being run on. For example, reading in a BigEndian file on a
 BigEndian machine will result in no swapping. Trying to read
 the same file on a LittleEndian machine will result in swapping.
 As a quick note most UNIX machines are BigEndian while PC's
 and VAX tend to be LittleEndian. So if the file you are reading
 in was generated on a VAX or PC, SetDataByteOrderToLittleEndian 
 otherwise SetDataByteOrderToBigEndian. Not used when reading
 text files. 

\item  \verb|obj.SetDataByteOrderToLittleEndian ()| -  These methods should be used instead of the SwapBytes methods.
 They indicate the byte ordering of the file you are trying
 to read in. These methods will then either swap or not swap
 the bytes depending on the byte ordering of the machine it is
 being run on. For example, reading in a BigEndian file on a
 BigEndian machine will result in no swapping. Trying to read
 the same file on a LittleEndian machine will result in swapping.
 As a quick note most UNIX machines are BigEndian while PC's
 and VAX tend to be LittleEndian. So if the file you are reading
 in was generated on a VAX or PC, SetDataByteOrderToLittleEndian 
 otherwise SetDataByteOrderToBigEndian. Not used when reading
 text files. 

\item  \verb|int = obj.GetDataByteOrder ()| -  These methods should be used instead of the SwapBytes methods.
 They indicate the byte ordering of the file you are trying
 to read in. These methods will then either swap or not swap
 the bytes depending on the byte ordering of the machine it is
 being run on. For example, reading in a BigEndian file on a
 BigEndian machine will result in no swapping. Trying to read
 the same file on a LittleEndian machine will result in swapping.
 As a quick note most UNIX machines are BigEndian while PC's
 and VAX tend to be LittleEndian. So if the file you are reading
 in was generated on a VAX or PC, SetDataByteOrderToLittleEndian 
 otherwise SetDataByteOrderToBigEndian. Not used when reading
 text files. 

\item  \verb|obj.SetDataByteOrder (int )| -  These methods should be used instead of the SwapBytes methods.
 They indicate the byte ordering of the file you are trying
 to read in. These methods will then either swap or not swap
 the bytes depending on the byte ordering of the machine it is
 being run on. For example, reading in a BigEndian file on a
 BigEndian machine will result in no swapping. Trying to read
 the same file on a LittleEndian machine will result in swapping.
 As a quick note most UNIX machines are BigEndian while PC's
 and VAX tend to be LittleEndian. So if the file you are reading
 in was generated on a VAX or PC, SetDataByteOrderToLittleEndian 
 otherwise SetDataByteOrderToBigEndian. Not used when reading
 text files. 

\item  \verb|string = obj.GetDataByteOrderAsString ()| -  These methods should be used instead of the SwapBytes methods.
 They indicate the byte ordering of the file you are trying
 to read in. These methods will then either swap or not swap
 the bytes depending on the byte ordering of the machine it is
 being run on. For example, reading in a BigEndian file on a
 BigEndian machine will result in no swapping. Trying to read
 the same file on a LittleEndian machine will result in swapping.
 As a quick note most UNIX machines are BigEndian while PC's
 and VAX tend to be LittleEndian. So if the file you are reading
 in was generated on a VAX or PC, SetDataByteOrderToLittleEndian 
 otherwise SetDataByteOrderToBigEndian. Not used when reading
 text files. 

\item  \verb|obj.SetSwapBytes (int )| -  Set/Get the byte swapping to explicitly swap the bytes of a file.
 Not used when reading text files.

\item  \verb|int = obj.GetSwapBytes ()| -  Set/Get the byte swapping to explicitly swap the bytes of a file.
 Not used when reading text files.

\item  \verb|obj.SwapBytesOn ()| -  Set/Get the byte swapping to explicitly swap the bytes of a file.
 Not used when reading text files.

\item  \verb|obj.SwapBytesOff ()| -  Set/Get the byte swapping to explicitly swap the bytes of a file.
 Not used when reading text files.

\item  \verb|obj.SetHasScalar (int )| -  Default: 1. If 1 then each particle has a value associated with it.

\item  \verb|int = obj.GetHasScalar ()| -  Default: 1. If 1 then each particle has a value associated with it.

\item  \verb|obj.HasScalarOn ()| -  Default: 1. If 1 then each particle has a value associated with it.

\item  \verb|obj.HasScalarOff ()| -  Default: 1. If 1 then each particle has a value associated with it.

\item  \verb|obj.SetFileType (int )| -  Get/Set the file type.  The options are:
 - FILE\_TYPE\_IS\_UNKNOWN (default) the class 
     will attempt to determine the file type.
     If this fails then you should set the file type
     yourself.
 - FILE\_TYPE\_IS\_TEXT the file type is text.
 - FILE\_TYPE\_IS\_BINARY the file type is binary.

\item  \verb|int = obj.GetFileTypeMinValue ()| -  Get/Set the file type.  The options are:
 - FILE\_TYPE\_IS\_UNKNOWN (default) the class 
     will attempt to determine the file type.
     If this fails then you should set the file type
     yourself.
 - FILE\_TYPE\_IS\_TEXT the file type is text.
 - FILE\_TYPE\_IS\_BINARY the file type is binary.

\item  \verb|int = obj.GetFileTypeMaxValue ()| -  Get/Set the file type.  The options are:
 - FILE\_TYPE\_IS\_UNKNOWN (default) the class 
     will attempt to determine the file type.
     If this fails then you should set the file type
     yourself.
 - FILE\_TYPE\_IS\_TEXT the file type is text.
 - FILE\_TYPE\_IS\_BINARY the file type is binary.

\item  \verb|int = obj.GetFileType ()| -  Get/Set the file type.  The options are:
 - FILE\_TYPE\_IS\_UNKNOWN (default) the class 
     will attempt to determine the file type.
     If this fails then you should set the file type
     yourself.
 - FILE\_TYPE\_IS\_TEXT the file type is text.
 - FILE\_TYPE\_IS\_BINARY the file type is binary.

\item  \verb|obj.SetFileTypeToUnknown ()| -  Get/Set the file type.  The options are:
 - FILE\_TYPE\_IS\_UNKNOWN (default) the class 
     will attempt to determine the file type.
     If this fails then you should set the file type
     yourself.
 - FILE\_TYPE\_IS\_TEXT the file type is text.
 - FILE\_TYPE\_IS\_BINARY the file type is binary.

\item  \verb|obj.SetFileTypeToText ()| -  Get/Set the file type.  The options are:
 - FILE\_TYPE\_IS\_UNKNOWN (default) the class 
     will attempt to determine the file type.
     If this fails then you should set the file type
     yourself.
 - FILE\_TYPE\_IS\_TEXT the file type is text.
 - FILE\_TYPE\_IS\_BINARY the file type is binary.

\item  \verb|obj.SetFileTypeToBinary ()| -  Get/Set the data type.  The options are:
 - VTK\_FLOAT (default) single precision floating point.
 - VTK\_DOUBLE double precision floating point.

\item  \verb|obj.SetDataType (int )| -  Get/Set the data type.  The options are:
 - VTK\_FLOAT (default) single precision floating point.
 - VTK\_DOUBLE double precision floating point.

\item  \verb|int = obj.GetDataTypeMinValue ()| -  Get/Set the data type.  The options are:
 - VTK\_FLOAT (default) single precision floating point.
 - VTK\_DOUBLE double precision floating point.

\item  \verb|int = obj.GetDataTypeMaxValue ()| -  Get/Set the data type.  The options are:
 - VTK\_FLOAT (default) single precision floating point.
 - VTK\_DOUBLE double precision floating point.

\item  \verb|int = obj.GetDataType ()| -  Get/Set the data type.  The options are:
 - VTK\_FLOAT (default) single precision floating point.
 - VTK\_DOUBLE double precision floating point.

\item  \verb|obj.SetDataTypeToFloat ()| -  Get/Set the data type.  The options are:
 - VTK\_FLOAT (default) single precision floating point.
 - VTK\_DOUBLE double precision floating point.

\item  \verb|obj.SetDataTypeToDouble ()|

\end{itemize}
