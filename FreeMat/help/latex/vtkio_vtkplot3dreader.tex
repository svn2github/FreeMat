\section{vtkPLOT3DReader}

\subsection{Usage}

 vtkPLOT3DReader is a reader object that reads PLOT3D formatted files and 
 generates structured grid(s) on output. PLOT3D is a computer graphics 
 program designed to visualize the grids and solutions of computational 
 fluid dynamics. Please see the ''PLOT3D User's Manual'' available from 
 NASA Ames Research Center, Moffett Field CA.

 PLOT3D files consist of a grid file (also known as XYZ file), an 
 optional solution file (also known as a Q file), and an optional function 
 file that contains user created data (currently unsupported). The Q file 
 contains solution  information as follows: the four parameters free stream 
 mach number (Fsmach), angle of attack (Alpha), Reynolds number (Re), and 
 total integration time (Time). This information is stored in an array
 called Properties in the FieldData of each output (tuple 0: fsmach, tuple 1:
 alpha, tuple 2: re, tuple 3: time). In addition, the solution file contains 
 the flow density (scalar), flow momentum (vector), and flow energy (scalar).

 The reader can generate additional scalars and vectors (or ''functions'')
 from this information. To use vtkPLOT3DReader, you must specify the 
 particular function number for the scalar and vector you want to visualize.
 This implementation of the reader provides the following functions. The
 scalar functions are:
    -1  - don't read or compute any scalars
    100 - density
    110 - pressure
    120 - temperature
    130 - enthalpy
    140 - internal energy
    144 - kinetic energy
    153 - velocity magnitude
    163 - stagnation energy
    170 - entropy
    184 - swirl.

 The vector functions are:
    -1  - don't read or compute any vectors
    200 - velocity
    201 - vorticity
    202 - momentum
    210 - pressure gradient.

 (Other functions are described in the PLOT3D spec, but only those listed are
 implemented here.) Note that by default, this reader creates the density 
 scalar (100) and momentum vector (202) as output. (These are just read in
 from the solution file.) Please note that the validity of computation is
 a function of this class's gas constants (R, Gamma) and the equations used.
 They may not be suitable for your computational domain.

 Additionally, you can read other data and associate it as a vtkDataArray
 into the output's point attribute data. Use the method AddFunction()
 to list all the functions that you'd like to read. AddFunction() accepts
 an integer parameter that defines the function number.


To create an instance of class vtkPLOT3DReader, simply
invoke its constructor as follows
\begin{verbatim}
  obj = vtkPLOT3DReader
\end{verbatim}
\subsection{Methods}

The class vtkPLOT3DReader has several methods that can be used.
  They are listed below.
Note that the documentation is translated automatically from the VTK sources,
and may not be completely intelligible.  When in doubt, consult the VTK website.
In the methods listed below, \verb|obj| is an instance of the vtkPLOT3DReader class.
\begin{itemize}
\item  \verb|string = obj.GetClassName ()|

\item  \verb|int = obj.IsA (string name)|

\item  \verb|vtkPLOT3DReader = obj.NewInstance ()|

\item  \verb|vtkPLOT3DReader = obj.SafeDownCast (vtkObject o)|

\item  \verb|obj.SetFileName (string name)| -  Set/Get the PLOT3D geometry filename.

\item  \verb|string = obj.GetFileName ()| -  Set/Get the PLOT3D geometry filename.

\item  \verb|obj.SetXYZFileName (string )| -  Set/Get the PLOT3D geometry filename.

\item  \verb|string = obj.GetXYZFileName ()| -  Set/Get the PLOT3D geometry filename.

\item  \verb|obj.SetQFileName (string )| -  Set/Get the PLOT3D solution filename.

\item  \verb|string = obj.GetQFileName ()| -  Set/Get the PLOT3D solution filename.

\item  \verb|obj.SetFunctionFileName (string )| -  Set/Get the PLOT3D Function Filename (optional)

\item  \verb|string = obj.GetFunctionFileName ()| -  Set/Get the PLOT3D Function Filename (optional)

\item  \verb|int = obj.GetNumberOfOutputs ()| -  This returns the number of outputs this reader will produce.
 This number is equal to the number of grids in the current 
 file. This method has to be called before getting any output
 if the number of outputs will be greater than 1 (the first
 output is always the same). Note that every time this method
 is invoked, the header file is opened and part of the header 
 is read.

\item  \verb|int = obj.GetNumberOfGrids ()| -  Replace an output.

\item  \verb|obj.SetOutput (int idx, vtkStructuredGrid output)| -  Replace an output.

\item  \verb|obj.SetBinaryFile (int )| -  Is the file to be read written in binary format (as opposed
 to ascii).

\item  \verb|int = obj.GetBinaryFile ()| -  Is the file to be read written in binary format (as opposed
 to ascii).

\item  \verb|obj.BinaryFileOn ()| -  Is the file to be read written in binary format (as opposed
 to ascii).

\item  \verb|obj.BinaryFileOff ()| -  Is the file to be read written in binary format (as opposed
 to ascii).

\item  \verb|obj.SetMultiGrid (int )| -  Does the file to be read contain information about number of
 grids. In some PLOT3D files, the first value contains the number 
 of grids (even if there is only 1). If reading such a file,
 set this to true.

\item  \verb|int = obj.GetMultiGrid ()| -  Does the file to be read contain information about number of
 grids. In some PLOT3D files, the first value contains the number 
 of grids (even if there is only 1). If reading such a file,
 set this to true.

\item  \verb|obj.MultiGridOn ()| -  Does the file to be read contain information about number of
 grids. In some PLOT3D files, the first value contains the number 
 of grids (even if there is only 1). If reading such a file,
 set this to true.

\item  \verb|obj.MultiGridOff ()| -  Does the file to be read contain information about number of
 grids. In some PLOT3D files, the first value contains the number 
 of grids (even if there is only 1). If reading such a file,
 set this to true.

\item  \verb|obj.SetHasByteCount (int )| -  Were the arrays written with leading and trailing byte counts ?
 Usually, files written by a fortran program will contain these
 byte counts whereas the ones written by C/C++ won't.

\item  \verb|int = obj.GetHasByteCount ()| -  Were the arrays written with leading and trailing byte counts ?
 Usually, files written by a fortran program will contain these
 byte counts whereas the ones written by C/C++ won't.

\item  \verb|obj.HasByteCountOn ()| -  Were the arrays written with leading and trailing byte counts ?
 Usually, files written by a fortran program will contain these
 byte counts whereas the ones written by C/C++ won't.

\item  \verb|obj.HasByteCountOff ()| -  Were the arrays written with leading and trailing byte counts ?
 Usually, files written by a fortran program will contain these
 byte counts whereas the ones written by C/C++ won't.

\item  \verb|obj.SetIBlanking (int )| -  Is there iblanking (point visibility) information in the file.
 If there is iblanking arrays, these will be read and assigned
 to the PointVisibility array of the output.

\item  \verb|int = obj.GetIBlanking ()| -  Is there iblanking (point visibility) information in the file.
 If there is iblanking arrays, these will be read and assigned
 to the PointVisibility array of the output.

\item  \verb|obj.IBlankingOn ()| -  Is there iblanking (point visibility) information in the file.
 If there is iblanking arrays, these will be read and assigned
 to the PointVisibility array of the output.

\item  \verb|obj.IBlankingOff ()| -  Is there iblanking (point visibility) information in the file.
 If there is iblanking arrays, these will be read and assigned
 to the PointVisibility array of the output.

\item  \verb|obj.SetTwoDimensionalGeometry (int )| -  If only two-dimensional data was written to the file,
 turn this on.

\item  \verb|int = obj.GetTwoDimensionalGeometry ()| -  If only two-dimensional data was written to the file,
 turn this on.

\item  \verb|obj.TwoDimensionalGeometryOn ()| -  If only two-dimensional data was written to the file,
 turn this on.

\item  \verb|obj.TwoDimensionalGeometryOff ()| -  If only two-dimensional data was written to the file,
 turn this on.

\item  \verb|obj.SetForceRead (int )| -  Try to read a binary file even if the file length seems to be
 inconsistent with the header information. Use this with caution,
 if the file length is not the same as calculated from the header.
 either the file is corrupt or the settings are wrong. 

\item  \verb|int = obj.GetForceRead ()| -  Try to read a binary file even if the file length seems to be
 inconsistent with the header information. Use this with caution,
 if the file length is not the same as calculated from the header.
 either the file is corrupt or the settings are wrong. 

\item  \verb|obj.ForceReadOn ()| -  Try to read a binary file even if the file length seems to be
 inconsistent with the header information. Use this with caution,
 if the file length is not the same as calculated from the header.
 either the file is corrupt or the settings are wrong. 

\item  \verb|obj.ForceReadOff ()| -  Try to read a binary file even if the file length seems to be
 inconsistent with the header information. Use this with caution,
 if the file length is not the same as calculated from the header.
 either the file is corrupt or the settings are wrong. 

\item  \verb|obj.SetDoNotReduceNumberOfOutputs (int )| -  If this is on, the reader will never reduce the number of outputs  
 after reading a file with n grids and producing n outputs. If the
 file read afterwards contains fewer grids, the extra outputs will
 be empty. This option can be used by application which rely on
 the initial number of outputs not shrinking.

\item  \verb|int = obj.GetDoNotReduceNumberOfOutputs ()| -  If this is on, the reader will never reduce the number of outputs  
 after reading a file with n grids and producing n outputs. If the
 file read afterwards contains fewer grids, the extra outputs will
 be empty. This option can be used by application which rely on
 the initial number of outputs not shrinking.

\item  \verb|obj.DoNotReduceNumberOfOutputsOn ()| -  If this is on, the reader will never reduce the number of outputs  
 after reading a file with n grids and producing n outputs. If the
 file read afterwards contains fewer grids, the extra outputs will
 be empty. This option can be used by application which rely on
 the initial number of outputs not shrinking.

\item  \verb|obj.DoNotReduceNumberOfOutputsOff ()| -  If this is on, the reader will never reduce the number of outputs  
 after reading a file with n grids and producing n outputs. If the
 file read afterwards contains fewer grids, the extra outputs will
 be empty. This option can be used by application which rely on
 the initial number of outputs not shrinking.

\item  \verb|obj.SetByteOrderToBigEndian ()| -  Set the byte order of the file (remember, more Unix workstations
 write big endian whereas PCs write little endian). Default is
 big endian (since most older PLOT3D files were written by
 workstations).

\item  \verb|obj.SetByteOrderToLittleEndian ()| -  Set the byte order of the file (remember, more Unix workstations
 write big endian whereas PCs write little endian). Default is
 big endian (since most older PLOT3D files were written by
 workstations).

\item  \verb|obj.SetByteOrder (int )| -  Set the byte order of the file (remember, more Unix workstations
 write big endian whereas PCs write little endian). Default is
 big endian (since most older PLOT3D files were written by
 workstations).

\item  \verb|int = obj.GetByteOrder ()| -  Set the byte order of the file (remember, more Unix workstations
 write big endian whereas PCs write little endian). Default is
 big endian (since most older PLOT3D files were written by
 workstations).

\item  \verb|string = obj.GetByteOrderAsString ()| -  Set the byte order of the file (remember, more Unix workstations
 write big endian whereas PCs write little endian). Default is
 big endian (since most older PLOT3D files were written by
 workstations).

\item  \verb|obj.SetR (double )| -  Set/Get the gas constant. Default is 1.0.

\item  \verb|double = obj.GetR ()| -  Set/Get the gas constant. Default is 1.0.

\item  \verb|obj.SetGamma (double )| -  Set/Get the ratio of specific heats. Default is 1.4.

\item  \verb|double = obj.GetGamma ()| -  Set/Get the ratio of specific heats. Default is 1.4.

\item  \verb|obj.SetUvinf (double )| -  Set/Get the x-component of the free-stream velocity. Default is 1.0.

\item  \verb|double = obj.GetUvinf ()| -  Set/Get the x-component of the free-stream velocity. Default is 1.0.

\item  \verb|obj.SetVvinf (double )| -  Set/Get the y-component of the free-stream velocity. Default is 1.0.

\item  \verb|double = obj.GetVvinf ()| -  Set/Get the y-component of the free-stream velocity. Default is 1.0.

\item  \verb|obj.SetWvinf (double )| -  Set/Get the z-component of the free-stream velocity. Default is 1.0.

\item  \verb|double = obj.GetWvinf ()| -  Set/Get the z-component of the free-stream velocity. Default is 1.0.

\item  \verb|obj.SetScalarFunctionNumber (int num)| -  Specify the scalar function to extract. If ==(-1), then no scalar 
 function is extracted.

\item  \verb|int = obj.GetScalarFunctionNumber ()| -  Specify the scalar function to extract. If ==(-1), then no scalar 
 function is extracted.

\item  \verb|obj.SetVectorFunctionNumber (int num)| -  Specify the vector function to extract. If ==(-1), then no vector
 function is extracted.

\item  \verb|int = obj.GetVectorFunctionNumber ()| -  Specify the vector function to extract. If ==(-1), then no vector
 function is extracted.

\item  \verb|obj.AddFunction (int functionNumber)| -  Specify additional functions to read. These are placed into the
 point data as data arrays. Later on they can be used by labeling
 them as scalars, etc.

\item  \verb|obj.RemoveFunction (int )| -  Specify additional functions to read. These are placed into the
 point data as data arrays. Later on they can be used by labeling
 them as scalars, etc.

\item  \verb|obj.RemoveAllFunctions ()| -  Specify additional functions to read. These are placed into the
 point data as data arrays. Later on they can be used by labeling
 them as scalars, etc.

\item  \verb|int = obj.CanReadBinaryFile (string fname)| -  Return 1 if the reader can read the given file name. Only meaningful
 for binary files.

\end{itemize}
