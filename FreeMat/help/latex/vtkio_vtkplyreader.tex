\section{vtkPLYReader}

\subsection{Usage}

 vtkPLYReader is a source object that reads polygonal data in
 Stanford University PLY file format (see 
 http://graphics.stanford.edu/data/3Dscanrep). It requires that
 the elements ''vertex'' and ''face'' are defined. The ''vertex'' element
 must have the properties ''x'', ''y'', and ''z''. The ''face'' element must
 have the property ''vertex\_indices'' defined. Optionally, if the ''face''
 element has the properties ''intensity'' and/or the triplet ''red'',
 ''green'', and ''blue''; these are read and added as scalars to the
 output data.

To create an instance of class vtkPLYReader, simply
invoke its constructor as follows
\begin{verbatim}
  obj = vtkPLYReader
\end{verbatim}
\subsection{Methods}

The class vtkPLYReader has several methods that can be used.
  They are listed below.
Note that the documentation is translated automatically from the VTK sources,
and may not be completely intelligible.  When in doubt, consult the VTK website.
In the methods listed below, \verb|obj| is an instance of the vtkPLYReader class.
\begin{itemize}
\item  \verb|string = obj.GetClassName ()|

\item  \verb|int = obj.IsA (string name)|

\item  \verb|vtkPLYReader = obj.NewInstance ()|

\item  \verb|vtkPLYReader = obj.SafeDownCast (vtkObject o)|

\item  \verb|obj.SetFileName (string )| -  Specify file name of stereo lithography file.

\item  \verb|string = obj.GetFileName ()| -  Specify file name of stereo lithography file.

\end{itemize}
