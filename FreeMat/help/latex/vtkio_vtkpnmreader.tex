\section{vtkPNMReader}

\subsection{Usage}

 vtkPNMReader is a source object that reads pnm (portable anymap) files.
 This includes .pbm (bitmap), .pgm (grayscale), and .ppm (pixmap) files.
 (Currently this object only reads binary versions of these files.)

 PNMReader creates structured point datasets. The dimension of the 
 dataset depends upon the number of files read. Reading a single file 
 results in a 2D image, while reading more than one file results in a 
 3D volume.

 To read a volume, files must be of the form ''FileName.<number>'' (e.g.,
 foo.ppm.0, foo.ppm.1, ...). You must also specify the DataExtent.  The
 fifth and sixth values of the DataExtent specify the beginning and ending
 files to read.

To create an instance of class vtkPNMReader, simply
invoke its constructor as follows
\begin{verbatim}
  obj = vtkPNMReader
\end{verbatim}
\subsection{Methods}

The class vtkPNMReader has several methods that can be used.
  They are listed below.
Note that the documentation is translated automatically from the VTK sources,
and may not be completely intelligible.  When in doubt, consult the VTK website.
In the methods listed below, \verb|obj| is an instance of the vtkPNMReader class.
\begin{itemize}
\item  \verb|string = obj.GetClassName ()|

\item  \verb|int = obj.IsA (string name)|

\item  \verb|vtkPNMReader = obj.NewInstance ()|

\item  \verb|vtkPNMReader = obj.SafeDownCast (vtkObject o)|

\item  \verb|int = obj.CanReadFile (string fname)|

\item  \verb|string = obj.GetFileExtensions ()| -  PNM 

\item  \verb|string = obj.GetDescriptiveName ()|

\end{itemize}
