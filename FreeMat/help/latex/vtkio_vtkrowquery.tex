\section{vtkRowQuery}

\subsection{Usage}

 The abstract superclass of query classes that return row-oriented (table)
 results.  A subclass will provide database-specific query parameters and
 implement the vtkRowQuery API to return query results:

 Execute() - Execute the query.  No results need to be retrieved at this
             point, unless you are performing caching.

 GetNumberOfFields() - After Execute() is performed, returns the number
                       of fields in the query results.

 GetFieldName() - The name of the field at an index.

 GetFieldType() - The data type of the field at an index.

 NextRow() - Advances the query results by one row, and returns whether
             there are more rows left in the query.

 DataValue() - Extract a single data value from the current row.

 .SECTION Thanks
 Thanks to Andrew Wilson from Sandia National Laboratories for his work
 on the database classes.


To create an instance of class vtkRowQuery, simply
invoke its constructor as follows
\begin{verbatim}
  obj = vtkRowQuery
\end{verbatim}
\subsection{Methods}

The class vtkRowQuery has several methods that can be used.
  They are listed below.
Note that the documentation is translated automatically from the VTK sources,
and may not be completely intelligible.  When in doubt, consult the VTK website.
In the methods listed below, \verb|obj| is an instance of the vtkRowQuery class.
\begin{itemize}
\item  \verb|string = obj.GetClassName ()|

\item  \verb|int = obj.IsA (string name)|

\item  \verb|vtkRowQuery = obj.NewInstance ()|

\item  \verb|vtkRowQuery = obj.SafeDownCast (vtkObject o)|

\item  \verb|bool = obj.Execute ()| -  Execute the query.  This must be performed
 before any field name or data access functions
 are used.

\item  \verb|int = obj.GetNumberOfFields ()| -  The number of fields in the query result.

\item  \verb|string = obj.GetFieldName (int i)| -  Return the name of the specified query field.

\item  \verb|int = obj.GetFieldType (int i)| -  Return the type of the field, using the constants defined in vtkType.h.

\item  \verb|int = obj.GetFieldIndex (string name)| -  Return the index of the specified query field.
 Uses GetNumberOfFields() and GetFieldName()
 to match field name.

\item  \verb|bool = obj.NextRow ()| -  Advance row, return false if past end.

\item  \verb|bool = obj.IsActive ()| -  Return true if the query is active (i.e. execution was successful
 and results are ready to be fetched).  Returns false on error or
 inactive query.

\item  \verb|bool = obj.HasError ()| -  Returns true if an error is set, otherwise false.

\item  \verb|string = obj.GetLastErrorText ()| -  Get the last error text from the query

\item  \verb|obj.SetCaseSensitiveFieldNames (bool )| -  Many databases do not preserve case in field names.  This can
 cause GetFieldIndex to fail if you search for a field named
 someFieldName when the database actually stores it as
 SOMEFIELDNAME.  This ivar controls whether GetFieldIndex()
 expects field names to be case-sensitive.  The default is OFF,
 i.e. case is not preserved.

\item  \verb|bool = obj.GetCaseSensitiveFieldNames ()| -  Many databases do not preserve case in field names.  This can
 cause GetFieldIndex to fail if you search for a field named
 someFieldName when the database actually stores it as
 SOMEFIELDNAME.  This ivar controls whether GetFieldIndex()
 expects field names to be case-sensitive.  The default is OFF,
 i.e. case is not preserved.

\item  \verb|obj.CaseSensitiveFieldNamesOn ()| -  Many databases do not preserve case in field names.  This can
 cause GetFieldIndex to fail if you search for a field named
 someFieldName when the database actually stores it as
 SOMEFIELDNAME.  This ivar controls whether GetFieldIndex()
 expects field names to be case-sensitive.  The default is OFF,
 i.e. case is not preserved.

\item  \verb|obj.CaseSensitiveFieldNamesOff ()| -  Many databases do not preserve case in field names.  This can
 cause GetFieldIndex to fail if you search for a field named
 someFieldName when the database actually stores it as
 SOMEFIELDNAME.  This ivar controls whether GetFieldIndex()
 expects field names to be case-sensitive.  The default is OFF,
 i.e. case is not preserved.

\end{itemize}
