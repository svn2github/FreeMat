\section{vtkSimplePointsReader}

\subsection{Usage}

 vtkSimplePointsReader is a source object that reads a list of
 points from a file.  Each point is specified by three
 floating-point values in ASCII format.  There is one point per line
 of the file.  A vertex cell is created for each point in the
 output.  This reader is meant as an example of how to write a
 reader in VTK.

To create an instance of class vtkSimplePointsReader, simply
invoke its constructor as follows
\begin{verbatim}
  obj = vtkSimplePointsReader
\end{verbatim}
\subsection{Methods}

The class vtkSimplePointsReader has several methods that can be used.
  They are listed below.
Note that the documentation is translated automatically from the VTK sources,
and may not be completely intelligible.  When in doubt, consult the VTK website.
In the methods listed below, \verb|obj| is an instance of the vtkSimplePointsReader class.
\begin{itemize}
\item  \verb|string = obj.GetClassName ()|

\item  \verb|int = obj.IsA (string name)|

\item  \verb|vtkSimplePointsReader = obj.NewInstance ()|

\item  \verb|vtkSimplePointsReader = obj.SafeDownCast (vtkObject o)|

\item  \verb|obj.SetFileName (string )| -  Set/Get the name of the file from which to read points.

\item  \verb|string = obj.GetFileName ()| -  Set/Get the name of the file from which to read points.

\end{itemize}
