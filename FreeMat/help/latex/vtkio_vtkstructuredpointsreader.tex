\section{vtkStructuredPointsReader}

\subsection{Usage}

 vtkStructuredPointsReader is a source object that reads ASCII or binary 
 structured points data files in vtk format (see text for format details).
 The output of this reader is a single vtkStructuredPoints data object.
 The superclass of this class, vtkDataReader, provides many methods for
 controlling the reading of the data file, see vtkDataReader for more
 information.

To create an instance of class vtkStructuredPointsReader, simply
invoke its constructor as follows
\begin{verbatim}
  obj = vtkStructuredPointsReader
\end{verbatim}
\subsection{Methods}

The class vtkStructuredPointsReader has several methods that can be used.
  They are listed below.
Note that the documentation is translated automatically from the VTK sources,
and may not be completely intelligible.  When in doubt, consult the VTK website.
In the methods listed below, \verb|obj| is an instance of the vtkStructuredPointsReader class.
\begin{itemize}
\item  \verb|string = obj.GetClassName ()|

\item  \verb|int = obj.IsA (string name)|

\item  \verb|vtkStructuredPointsReader = obj.NewInstance ()|

\item  \verb|vtkStructuredPointsReader = obj.SafeDownCast (vtkObject o)|

\item  \verb|obj.SetOutput (vtkStructuredPoints output)| -  Set/Get the output of this reader.

\item  \verb|vtkStructuredPoints = obj.GetOutput (int idx)| -  Set/Get the output of this reader.

\item  \verb|vtkStructuredPoints = obj.GetOutput ()| -  Set/Get the output of this reader.

\item  \verb|int = obj.ReadMetaData (vtkInformation outInfo)| -  Read the meta information from the file.  This needs to be public to it
 can be accessed by vtkDataSetReader.

\end{itemize}
