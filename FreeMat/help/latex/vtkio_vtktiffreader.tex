\section{vtkTIFFReader}

\subsection{Usage}

 vtkTIFFReader is a source object that reads TIFF files.
 It should be able to read almost any TIFF file


To create an instance of class vtkTIFFReader, simply
invoke its constructor as follows
\begin{verbatim}
  obj = vtkTIFFReader
\end{verbatim}
\subsection{Methods}

The class vtkTIFFReader has several methods that can be used.
  They are listed below.
Note that the documentation is translated automatically from the VTK sources,
and may not be completely intelligible.  When in doubt, consult the VTK website.
In the methods listed below, \verb|obj| is an instance of the vtkTIFFReader class.
\begin{itemize}
\item  \verb|string = obj.GetClassName ()|

\item  \verb|int = obj.IsA (string name)|

\item  \verb|vtkTIFFReader = obj.NewInstance ()|

\item  \verb|vtkTIFFReader = obj.SafeDownCast (vtkObject o)|

\item  \verb|int = obj.CanReadFile (string fname)| -  Is the given file name a tiff file file?

\item  \verb|string = obj.GetFileExtensions ()| -  Return a descriptive name for the file format that might be useful
 in a GUI.

\item  \verb|string = obj.GetDescriptiveName ()| -  Auxilary methods used by the reader internally.

\item  \verb|obj.InitializeColors ()| -  Auxilary methods used by the reader internally.

\item  \verb|obj.SetOrientationType (int orientationType)| -  Set orientation type 
 ORIENTATION\_TOPLEFT         1       (row 0 top, col 0 lhs)
 ORIENTATION\_TOPRIGHT        2       (row 0 top, col 0 rhs)
 ORIENTATION\_BOTRIGHT        3       (row 0 bottom, col 0 rhs)
 ORIENTATION\_BOTLEFT         4       (row 0 bottom, col 0 lhs)
 ORIENTATION\_LEFTTOP         5       (row 0 lhs, col 0 top)
 ORIENTATION\_RIGHTTOP        6       (row 0 rhs, col 0 top)
 ORIENTATION\_RIGHTBOT        7       (row 0 rhs, col 0 bottom)
 ORIENTATION\_LEFTBOT         8       (row 0 lhs, col 0 bottom)
 User need to explicitely include vtk\_tiff.h header to have access to those \#define

\item  \verb|int = obj.GetOrientationType ()| -  Set orientation type 
 ORIENTATION\_TOPLEFT         1       (row 0 top, col 0 lhs)
 ORIENTATION\_TOPRIGHT        2       (row 0 top, col 0 rhs)
 ORIENTATION\_BOTRIGHT        3       (row 0 bottom, col 0 rhs)
 ORIENTATION\_BOTLEFT         4       (row 0 bottom, col 0 lhs)
 ORIENTATION\_LEFTTOP         5       (row 0 lhs, col 0 top)
 ORIENTATION\_RIGHTTOP        6       (row 0 rhs, col 0 top)
 ORIENTATION\_RIGHTBOT        7       (row 0 rhs, col 0 bottom)
 ORIENTATION\_LEFTBOT         8       (row 0 lhs, col 0 bottom)
 User need to explicitely include vtk\_tiff.h header to have access to those \#define

\item  \verb|bool = obj.GetOrientationTypeSpecifiedFlag ()| -  Get method to check if orientation type is specified

\item  \verb|obj.SetOriginSpecifiedFlag (bool )| -  Set/get methods to see if manual Origin/Spacing have
 been set.

\item  \verb|bool = obj.GetOriginSpecifiedFlag ()| -  Set/get methods to see if manual Origin/Spacing have
 been set.

\item  \verb|obj.OriginSpecifiedFlagOn ()| -  Set/get methods to see if manual Origin/Spacing have
 been set.

\item  \verb|obj.OriginSpecifiedFlagOff ()| -  Set/get methods to see if manual Origin/Spacing have
 been set.

\item  \verb|obj.SetSpacingSpecifiedFlag (bool )| -  

\item  \verb|bool = obj.GetSpacingSpecifiedFlag ()| -  

\item  \verb|obj.SpacingSpecifiedFlagOn ()| -  

\item  \verb|obj.SpacingSpecifiedFlagOff ()| -  

\end{itemize}
