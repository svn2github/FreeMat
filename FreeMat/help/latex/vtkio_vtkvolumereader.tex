\section{vtkVolumeReader}

\subsection{Usage}

 vtkVolumeReader is a source object that reads image files.

 VolumeReader creates structured point datasets. The dimension of the 
 dataset depends upon the number of files read. Reading a single file 
 results in a 2D image, while reading more than one file results in a 
 3D volume.

 File names are created using FilePattern and FilePrefix as follows:
 sprintf (filename, FilePattern, FilePrefix, number);
 where number is in the range ImageRange[0] to ImageRange[1]. If
 ImageRange[1] <= ImageRange[0], then slice number ImageRange[0] is
 read. Thus to read an image set ImageRange[0] = ImageRange[1] = slice 
 number. The default behavior is to read a single file (i.e., image slice 1).

 The DataMask instance variable is used to read data files with imbedded
 connectivity or segmentation information. For example, some data has
 the high order bit set to indicate connected surface. The DataMask allows
 you to select this data. Other important ivars include HeaderSize, which
 allows you to skip over initial info, and SwapBytes, which turns on/off
 byte swapping. Consider using vtkImageReader as a replacement.

To create an instance of class vtkVolumeReader, simply
invoke its constructor as follows
\begin{verbatim}
  obj = vtkVolumeReader
\end{verbatim}
\subsection{Methods}

The class vtkVolumeReader has several methods that can be used.
  They are listed below.
Note that the documentation is translated automatically from the VTK sources,
and may not be completely intelligible.  When in doubt, consult the VTK website.
In the methods listed below, \verb|obj| is an instance of the vtkVolumeReader class.
\begin{itemize}
\item  \verb|string = obj.GetClassName ()|

\item  \verb|int = obj.IsA (string name)|

\item  \verb|vtkVolumeReader = obj.NewInstance ()|

\item  \verb|vtkVolumeReader = obj.SafeDownCast (vtkObject o)|

\item  \verb|obj.SetFilePrefix (string )| -  Specify file prefix for the image file(s).

\item  \verb|string = obj.GetFilePrefix ()| -  Specify file prefix for the image file(s).

\item  \verb|obj.SetFilePattern (string )| -  The sprintf format used to build filename from FilePrefix and number.

\item  \verb|string = obj.GetFilePattern ()| -  The sprintf format used to build filename from FilePrefix and number.

\item  \verb|obj.SetImageRange (int , int )| -  Set the range of files to read.

\item  \verb|obj.SetImageRange (int  a[2])| -  Set the range of files to read.

\item  \verb|int = obj. GetImageRange ()| -  Set the range of files to read.

\item  \verb|obj.SetDataSpacing (double , double , double )| -  Specify the spacing for the data.

\item  \verb|obj.SetDataSpacing (double  a[3])| -  Specify the spacing for the data.

\item  \verb|double = obj. GetDataSpacing ()| -  Specify the spacing for the data.

\item  \verb|obj.SetDataOrigin (double , double , double )| -  Specify the origin for the data.

\item  \verb|obj.SetDataOrigin (double  a[3])| -  Specify the origin for the data.

\item  \verb|double = obj. GetDataOrigin ()| -  Specify the origin for the data.

\item  \verb|vtkImageData = obj.GetImage (int ImageNumber)| -  Other objects make use of this method.

\end{itemize}
