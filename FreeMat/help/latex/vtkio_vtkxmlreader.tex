\section{vtkXMLReader}

\subsection{Usage}

 vtkXMLReader uses vtkXMLDataParser to parse a VTK XML input file.
 Concrete subclasses then traverse the parsed file structure and
 extract data.

To create an instance of class vtkXMLReader, simply
invoke its constructor as follows
\begin{verbatim}
  obj = vtkXMLReader
\end{verbatim}
\subsection{Methods}

The class vtkXMLReader has several methods that can be used.
  They are listed below.
Note that the documentation is translated automatically from the VTK sources,
and may not be completely intelligible.  When in doubt, consult the VTK website.
In the methods listed below, \verb|obj| is an instance of the vtkXMLReader class.
\begin{itemize}
\item  \verb|string = obj.GetClassName ()|

\item  \verb|int = obj.IsA (string name)|

\item  \verb|vtkXMLReader = obj.NewInstance ()|

\item  \verb|vtkXMLReader = obj.SafeDownCast (vtkObject o)|

\item  \verb|obj.SetFileName (string )| -  Get/Set the name of the input file.

\item  \verb|string = obj.GetFileName ()| -  Get/Set the name of the input file.

\item  \verb|int = obj.CanReadFile (string name)| -  Test whether the file with the given name can be read by this
 reader.

\item  \verb|vtkDataSet = obj.GetOutputAsDataSet ()| -  Get the output as a vtkDataSet pointer.

\item  \verb|vtkDataSet = obj.GetOutputAsDataSet (int index)| -  Get the output as a vtkDataSet pointer.

\item  \verb|vtkDataArraySelection = obj.GetPointDataArraySelection ()| -  Get the data array selection tables used to configure which data
 arrays are loaded by the reader.

\item  \verb|vtkDataArraySelection = obj.GetCellDataArraySelection ()| -  Get the data array selection tables used to configure which data
 arrays are loaded by the reader.

\item  \verb|int = obj.GetNumberOfPointArrays ()| -  Get the number of point or cell arrays available in the input.

\item  \verb|int = obj.GetNumberOfCellArrays ()| -  Get the number of point or cell arrays available in the input.

\item  \verb|string = obj.GetPointArrayName (int index)| -  Get the name of the point or cell array with the given index in
 the input.

\item  \verb|string = obj.GetCellArrayName (int index)| -  Get the name of the point or cell array with the given index in
 the input.

\item  \verb|int = obj.GetPointArrayStatus (string name)| -  Get/Set whether the point or cell array with the given name is to
 be read.

\item  \verb|int = obj.GetCellArrayStatus (string name)| -  Get/Set whether the point or cell array with the given name is to
 be read.

\item  \verb|obj.SetPointArrayStatus (string name, int status)| -  Get/Set whether the point or cell array with the given name is to
 be read.

\item  \verb|obj.SetCellArrayStatus (string name, int status)| -  Get/Set whether the point or cell array with the given name is to
 be read.

\item  \verb|obj.CopyOutputInformation (vtkInformation , int )| -  Which TimeStep to read.    

\item  \verb|obj.SetTimeStep (int )| -  Which TimeStep to read.    

\item  \verb|int = obj.GetTimeStep ()| -  Which TimeStep to read.    

\item  \verb|int = obj.GetNumberOfTimeSteps ()|

\item  \verb|int = obj. GetTimeStepRange ()| -  Which TimeStepRange to read

\item  \verb|obj.SetTimeStepRange (int , int )| -  Which TimeStepRange to read

\item  \verb|obj.SetTimeStepRange (int  a[2])| -  Which TimeStepRange to read

\end{itemize}
