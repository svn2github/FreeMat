\section{vtkZLibDataCompressor}

\subsection{Usage}

 vtkZLibDataCompressor provides a concrete vtkDataCompressor class
 using zlib for compressing and uncompressing data.

To create an instance of class vtkZLibDataCompressor, simply
invoke its constructor as follows
\begin{verbatim}
  obj = vtkZLibDataCompressor
\end{verbatim}
\subsection{Methods}

The class vtkZLibDataCompressor has several methods that can be used.
  They are listed below.
Note that the documentation is translated automatically from the VTK sources,
and may not be completely intelligible.  When in doubt, consult the VTK website.
In the methods listed below, \verb|obj| is an instance of the vtkZLibDataCompressor class.
\begin{itemize}
\item  \verb|string = obj.GetClassName ()|

\item  \verb|int = obj.IsA (string name)|

\item  \verb|vtkZLibDataCompressor = obj.NewInstance ()|

\item  \verb|vtkZLibDataCompressor = obj.SafeDownCast (vtkObject o)|

\item  \verb|long = obj.GetMaximumCompressionSpace (long size)| -  Get the maximum space that may be needed to store data of the
 given uncompressed size after compression.  This is the minimum
 size of the output buffer that can be passed to the four-argument
 Compress method.

\item  \verb|obj.SetCompressionLevel (int )| -  Get/Set the compression level.

\item  \verb|int = obj.GetCompressionLevelMinValue ()| -  Get/Set the compression level.

\item  \verb|int = obj.GetCompressionLevelMaxValue ()| -  Get/Set the compression level.

\item  \verb|int = obj.GetCompressionLevel ()| -  Get/Set the compression level.

\end{itemize}
