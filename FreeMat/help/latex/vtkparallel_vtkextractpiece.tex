\section{vtkExtractPiece}

\subsection{Usage}

 vtkExtractPiece returns the appropriate piece of each
 sub-dataset in the vtkCompositeDataSet.
 This filter can handle sub-datasets of type vtkImageData, vtkPolyData,
 vtkRectilinearGrid, vtkStructuredGrid, and vtkUnstructuredGrid; it does
 not handle sub-grids of type vtkCompositeDataSet.

To create an instance of class vtkExtractPiece, simply
invoke its constructor as follows
\begin{verbatim}
  obj = vtkExtractPiece
\end{verbatim}
\subsection{Methods}

The class vtkExtractPiece has several methods that can be used.
  They are listed below.
Note that the documentation is translated automatically from the VTK sources,
and may not be completely intelligible.  When in doubt, consult the VTK website.
In the methods listed below, \verb|obj| is an instance of the vtkExtractPiece class.
\begin{itemize}
\item  \verb|string = obj.GetClassName ()|

\item  \verb|int = obj.IsA (string name)|

\item  \verb|vtkExtractPiece = obj.NewInstance ()|

\item  \verb|vtkExtractPiece = obj.SafeDownCast (vtkObject o)|

\end{itemize}
