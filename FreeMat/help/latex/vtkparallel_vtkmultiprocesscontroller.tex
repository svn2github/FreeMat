\section{vtkMultiProcessController}

\subsection{Usage}

 vtkMultiProcessController is used to control multiple processes
 in a distributed computing environment. It has
 methods for executing single/multiple method(s) on multiple processors,
 triggering registered callbacks (Remote Methods) (AddRMI(), TriggerRMI())
 and communication. Please note that the communication is done using
 the communicator which is accessible to the user. Therefore it is
 possible to get the communicator with GetCommunicator() and use
 it to send and receive data. This is the encouraged communication method.
 The internal (RMI) communications are done using a second internal
 communicator (called RMICommunicator).


To create an instance of class vtkMultiProcessController, simply
invoke its constructor as follows
\begin{verbatim}
  obj = vtkMultiProcessController
\end{verbatim}
\subsection{Methods}

The class vtkMultiProcessController has several methods that can be used.
  They are listed below.
Note that the documentation is translated automatically from the VTK sources,
and may not be completely intelligible.  When in doubt, consult the VTK website.
In the methods listed below, \verb|obj| is an instance of the vtkMultiProcessController class.
\begin{itemize}
\item  \verb|string = obj.GetClassName ()|

\item  \verb|int = obj.IsA (string name)|

\item  \verb|vtkMultiProcessController = obj.NewInstance ()|

\item  \verb|vtkMultiProcessController = obj.SafeDownCast (vtkObject o)|

\item  \verb|obj.Finalize ()| -  This method is for cleaning up.
 If a subclass needs to clean up process communication (i.e. MPI)
 it would over ride this method.

\item  \verb|obj.Finalize (int finalizedExternally)| -  This method is for cleaning up.
 If a subclass needs to clean up process communication (i.e. MPI)
 it would over ride this method.  Provided for finalization outside vtk.

\item  \verb|obj.SetNumberOfProcesses (int num)| -  Set the number of processes you will be using.  This defaults
 to the maximum number available.  If you set this to a value
 higher than the default, you will get an error.

\item  \verb|int = obj.GetNumberOfProcesses ()| -  Set the number of processes you will be using.  This defaults
 to the maximum number available.  If you set this to a value
 higher than the default, you will get an error.

\item  \verb|obj.SingleMethodExecute ()| -  Execute the SingleMethod (as define by SetSingleMethod) using
 this->NumberOfProcesses processes.  This will only return when
 all the processes finish executing their methods.

\item  \verb|obj.MultipleMethodExecute ()| -  Execute the MultipleMethods (as define by calling SetMultipleMethod
 for each of the required this->NumberOfProcesses methods) using
 this->NumberOfProcesses processes.

\item  \verb|int = obj.GetLocalProcessId ()| -  Tells you which process [0, NumProcess) you are in.

\item  \verb|obj.CreateOutputWindow ()| -  This method can be used to tell the controller to create
 a special output window in which all messages are preceded
 by the process id.

\item  \verb|vtkMultiProcessController = obj.CreateSubController (vtkProcessGroup group)| -  Creates a new controller with the processes specified by the given group.
 The new controller will already be initialized for you.  You are
 responsible for deleting the controller once you are done.  It is invalid
 to pass this method a group with a different communicator than is used by
 this controller.  This operation is collective accross all processes
 defined in the group.  It is undefined what will happen if the group is not
 the same on all processes.  This method must be called by all processes in
 the controller regardless of whether they are in the group.  NULL is
 returned on all process not in the group.

\item  \verb|vtkMultiProcessController = obj.PartitionController (int localColor, int localKey)| -  Partitions this controller based on a coloring.  That is, each process
 passes in a color.  All processes with the same color are grouped into the
 same partition.  The processes are ordered by their self-assigned key.
 Lower keys have lower process ids.  Ties are broken by the current process
 ids.  (For example, if all the keys are 0, then the resulting processes
 will be ordered in the same way.)  This method returns a new controller to
 each process that represents the local partition.  This is basically the
 same operation as MPI\_Comm\_split.

\item  \verb|obj.TriggerBreakRMIs ()| -  A conveniance method.  Called on process 0 to break ''ProcessRMIs'' loop
 on all other processes.

\item  \verb|obj.TriggerRMI (int remoteProcessId, string arg, int tag)| -  Convenience method when there is no argument.

\item  \verb|obj.TriggerRMI (int remoteProcessId, int tag)| -  This is a convenicence method to trigger an RMI call on all the ''children''
 of the current node. The children of the current node can be determined by
 drawing a binary tree starting at node 0 and then assigned nodes ids
 incrementally in a breadth-first fashion from left to right. This is
 designed to be used when trigger an RMI call on all satellites from the
 root node.

\item  \verb|obj.TriggerRMIOnAllChildren (string arg, int tag)| -  This is a convenicence method to trigger an RMI call on all the ''children''
 of the current node. The children of the current node can be determined by
 drawing a binary tree starting at node 0 and then assigned nodes ids
 incrementally in a breadth-first fashion from left to right. This is
 designed to be used when trigger an RMI call on all satellites from the
 root node.

\item  \verb|obj.TriggerRMIOnAllChildren (int tag)| -  Calling this method gives control to the controller to start
 processing RMIs. Possible return values are:
 RMI\_NO\_ERROR,
 RMI\_TAG\_ERROR : rmi tag could not be received,
 RMI\_ARG\_ERROR : rmi arg could not be received.
 If reportErrors is false, no vtkErrorMacro is called.
 ProcessRMIs() calls ProcessRMIs(int) with reportErrors = 0.
 If dont\_loop is 1, this call just process one RMI message
 and exits.

\item  \verb|int = obj.ProcessRMIs (int reportErrors, int dont\_loop)| -  Calling this method gives control to the controller to start
 processing RMIs. Possible return values are:
 RMI\_NO\_ERROR,
 RMI\_TAG\_ERROR : rmi tag could not be received,
 RMI\_ARG\_ERROR : rmi arg could not be received.
 If reportErrors is false, no vtkErrorMacro is called.
 ProcessRMIs() calls ProcessRMIs(int) with reportErrors = 0.
 If dont\_loop is 1, this call just process one RMI message
 and exits.

\item  \verb|int = obj.ProcessRMIs ()| -  Calling this method gives control to the controller to start
 processing RMIs. Possible return values are:
 RMI\_NO\_ERROR,
 RMI\_TAG\_ERROR : rmi tag could not be received,
 RMI\_ARG\_ERROR : rmi arg could not be received.
 If reportErrors is false, no vtkErrorMacro is called.
 ProcessRMIs() calls ProcessRMIs(int) with reportErrors = 0.
 If dont\_loop is 1, this call just process one RMI message
 and exits.

\item  \verb|obj.SetBreakFlag (int )| -  Setting this flag to 1 will cause the ProcessRMIs loop to return.
 This also causes vtkUpStreamPorts to return from
 their WaitForUpdate loops.

\item  \verb|int = obj.GetBreakFlag ()| -  Setting this flag to 1 will cause the ProcessRMIs loop to return.
 This also causes vtkUpStreamPorts to return from
 their WaitForUpdate loops.

\item  \verb|vtkCommunicator = obj.GetCommunicator ()| -  Returns the communicator associated with this controller.
 A default communicator is created in constructor.

\item  \verb|obj.Barrier ()| -  This method can be used to synchronize processes.

\item  \verb|int = obj.Send (int data, vtkIdType length, int remoteProcessId, int tag)| -  This method sends data to another process.  Tag eliminates ambiguity
 when multiple sends or receives exist in the same process.
 It is recommended to use custom tag number over 100.
 vtkMultiProcessController has reserved tags between 1 and 4.
 vtkCommunicator has reserved tags between 10 and 16.

\item  \verb|int = obj.Send (int data, vtkIdType length, int remoteProcessId, int tag)| -  This method sends data to another process.  Tag eliminates ambiguity
 when multiple sends or receives exist in the same process.
 It is recommended to use custom tag number over 100.
 vtkMultiProcessController has reserved tags between 1 and 4.
 vtkCommunicator has reserved tags between 10 and 16.

\item  \verb|int = obj.Send (long data, vtkIdType length, int remoteProcessId, int tag)| -  This method sends data to another process.  Tag eliminates ambiguity
 when multiple sends or receives exist in the same process.
 It is recommended to use custom tag number over 100.
 vtkMultiProcessController has reserved tags between 1 and 4.
 vtkCommunicator has reserved tags between 10 and 16.

\item  \verb|int = obj.Send (string data, vtkIdType length, int remoteProcessId, int tag)| -  This method sends data to another process.  Tag eliminates ambiguity
 when multiple sends or receives exist in the same process.
 It is recommended to use custom tag number over 100.
 vtkMultiProcessController has reserved tags between 1 and 4.
 vtkCommunicator has reserved tags between 10 and 16.

\item  \verb|int = obj.Send (string data, vtkIdType length, int remoteProcessId, int tag)| -  This method sends data to another process.  Tag eliminates ambiguity
 when multiple sends or receives exist in the same process.
 It is recommended to use custom tag number over 100.
 vtkMultiProcessController has reserved tags between 1 and 4.
 vtkCommunicator has reserved tags between 10 and 16.

\item  \verb|int = obj.Send (float data, vtkIdType length, int remoteProcessId, int tag)| -  This method sends data to another process.  Tag eliminates ambiguity
 when multiple sends or receives exist in the same process.
 It is recommended to use custom tag number over 100.
 vtkMultiProcessController has reserved tags between 1 and 4.
 vtkCommunicator has reserved tags between 10 and 16.

\item  \verb|int = obj.Send (double data, vtkIdType length, int remoteProcessId, int tag)| -  This method sends data to another process.  Tag eliminates ambiguity
 when multiple sends or receives exist in the same process.
 It is recommended to use custom tag number over 100.
 vtkMultiProcessController has reserved tags between 1 and 4.
 vtkCommunicator has reserved tags between 10 and 16.

\item  \verb|int = obj.Send (vtkDataObject data, int remoteId, int tag)| -  This method sends data to another process.  Tag eliminates ambiguity
 when multiple sends or receives exist in the same process.
 It is recommended to use custom tag number over 100.
 vtkMultiProcessController has reserved tags between 1 and 4.
 vtkCommunicator has reserved tags between 10 and 16.

\item  \verb|int = obj.Send (vtkDataArray data, int remoteId, int tag)| -  This method sends data to another process.  Tag eliminates ambiguity
 when multiple sends or receives exist in the same process.
 It is recommended to use custom tag number over 100.
 vtkMultiProcessController has reserved tags between 1 and 4.
 vtkCommunicator has reserved tags between 10 and 16.

\item  \verb|int = obj.Receive (int data, vtkIdType maxlength, int remoteProcessId, int tag)| -  This method receives data from a corresponding send. It blocks
 until the receive is finished.  It calls methods in ''data''
 to communicate the sending data. In the overrloads that take in a \c
 maxlength argument, this length is the maximum length of the message to
 receive. If the maxlength is less than the length of the message sent by
 the sender, an error will be flagged. Once a message is received, use the
 GetCount() method to determine the actual size of the data received.

\item  \verb|int = obj.Receive (int data, vtkIdType maxlength, int remoteProcessId, int tag)| -  This method receives data from a corresponding send. It blocks
 until the receive is finished.  It calls methods in ''data''
 to communicate the sending data. In the overrloads that take in a \c
 maxlength argument, this length is the maximum length of the message to
 receive. If the maxlength is less than the length of the message sent by
 the sender, an error will be flagged. Once a message is received, use the
 GetCount() method to determine the actual size of the data received.

\item  \verb|int = obj.Receive (long data, vtkIdType maxlength, int remoteProcessId, int tag)| -  This method receives data from a corresponding send. It blocks
 until the receive is finished.  It calls methods in ''data''
 to communicate the sending data. In the overrloads that take in a \c
 maxlength argument, this length is the maximum length of the message to
 receive. If the maxlength is less than the length of the message sent by
 the sender, an error will be flagged. Once a message is received, use the
 GetCount() method to determine the actual size of the data received.

\item  \verb|int = obj.Receive (string data, vtkIdType maxlength, int remoteProcessId, int tag)| -  This method receives data from a corresponding send. It blocks
 until the receive is finished.  It calls methods in ''data''
 to communicate the sending data. In the overrloads that take in a \c
 maxlength argument, this length is the maximum length of the message to
 receive. If the maxlength is less than the length of the message sent by
 the sender, an error will be flagged. Once a message is received, use the
 GetCount() method to determine the actual size of the data received.

\item  \verb|int = obj.Receive (string data, vtkIdType maxlength, int remoteProcessId, int tag)| -  This method receives data from a corresponding send. It blocks
 until the receive is finished.  It calls methods in ''data''
 to communicate the sending data. In the overrloads that take in a \c
 maxlength argument, this length is the maximum length of the message to
 receive. If the maxlength is less than the length of the message sent by
 the sender, an error will be flagged. Once a message is received, use the
 GetCount() method to determine the actual size of the data received.

\item  \verb|int = obj.Receive (float data, vtkIdType maxlength, int remoteProcessId, int tag)| -  This method receives data from a corresponding send. It blocks
 until the receive is finished.  It calls methods in ''data''
 to communicate the sending data. In the overrloads that take in a \c
 maxlength argument, this length is the maximum length of the message to
 receive. If the maxlength is less than the length of the message sent by
 the sender, an error will be flagged. Once a message is received, use the
 GetCount() method to determine the actual size of the data received.

\item  \verb|int = obj.Receive (double data, vtkIdType maxlength, int remoteProcessId, int tag)| -  This method receives data from a corresponding send. It blocks
 until the receive is finished.  It calls methods in ''data''
 to communicate the sending data. In the overrloads that take in a \c
 maxlength argument, this length is the maximum length of the message to
 receive. If the maxlength is less than the length of the message sent by
 the sender, an error will be flagged. Once a message is received, use the
 GetCount() method to determine the actual size of the data received.

\item  \verb|int = obj.Receive (vtkDataObject data, int remoteId, int tag)| -  This method receives data from a corresponding send. It blocks
 until the receive is finished.  It calls methods in ''data''
 to communicate the sending data. In the overrloads that take in a \c
 maxlength argument, this length is the maximum length of the message to
 receive. If the maxlength is less than the length of the message sent by
 the sender, an error will be flagged. Once a message is received, use the
 GetCount() method to determine the actual size of the data received.

\item  \verb|int = obj.Receive (vtkDataArray data, int remoteId, int tag)| -  This method receives data from a corresponding send. It blocks
 until the receive is finished.  It calls methods in ''data''
 to communicate the sending data. In the overrloads that take in a \c
 maxlength argument, this length is the maximum length of the message to
 receive. If the maxlength is less than the length of the message sent by
 the sender, an error will be flagged. Once a message is received, use the
 GetCount() method to determine the actual size of the data received.

\item  \verb|vtkDataObject = obj.ReceiveDataObject (int remoteId, int tag)| -  This method receives data from a corresponding send. It blocks
 until the receive is finished.  It calls methods in ''data''
 to communicate the sending data. In the overrloads that take in a \c
 maxlength argument, this length is the maximum length of the message to
 receive. If the maxlength is less than the length of the message sent by
 the sender, an error will be flagged. Once a message is received, use the
 GetCount() method to determine the actual size of the data received.

\item  \verb|vtkIdType = obj.GetCount ()| -  Returns the number of words received by the most recent Receive().
 Note that this is not the number of bytes received, but the number of items
 of the data-type received by the most recent Receive() eg. if
 Receive(int*,..) was used, then this returns the number of ints received;
 if Receive(double*,..) was used, then this returns the number of doubles
 received etc. The return value is valid only after a successful Receive().

\item  \verb|int = obj.Broadcast (int data, vtkIdType length, int srcProcessId)| -  Broadcast sends the array in the process with id \c srcProcessId to all of
 the other processes.  All processes must call these method with the same
 arguments in order for it to complete.

\item  \verb|int = obj.Broadcast (long data, vtkIdType length, int srcProcessId)| -  Broadcast sends the array in the process with id \c srcProcessId to all of
 the other processes.  All processes must call these method with the same
 arguments in order for it to complete.

\item  \verb|int = obj.Broadcast (string data, vtkIdType length, int srcProcessId)| -  Broadcast sends the array in the process with id \c srcProcessId to all of
 the other processes.  All processes must call these method with the same
 arguments in order for it to complete.

\item  \verb|int = obj.Broadcast (string data, vtkIdType length, int srcProcessId)| -  Broadcast sends the array in the process with id \c srcProcessId to all of
 the other processes.  All processes must call these method with the same
 arguments in order for it to complete.

\item  \verb|int = obj.Broadcast (float data, vtkIdType length, int srcProcessId)| -  Broadcast sends the array in the process with id \c srcProcessId to all of
 the other processes.  All processes must call these method with the same
 arguments in order for it to complete.

\item  \verb|int = obj.Broadcast (double data, vtkIdType length, int srcProcessId)| -  Broadcast sends the array in the process with id \c srcProcessId to all of
 the other processes.  All processes must call these method with the same
 arguments in order for it to complete.

\item  \verb|int = obj.Broadcast (vtkDataObject data, int srcProcessId)| -  Broadcast sends the array in the process with id \c srcProcessId to all of
 the other processes.  All processes must call these method with the same
 arguments in order for it to complete.

\item  \verb|int = obj.Broadcast (vtkDataArray data, int srcProcessId)| -  Gather collects arrays in the process with id \c destProcessId.  Each
 process (including the destination) sends the contents of its send buffer
 to the destination process.  The destination process receives the
 messages and stores them in rank order.  The \c length argument
 (which must be the same on all processes) is the length of the
 sendBuffers.  The \c recvBuffer (on te destination process) must be of
 length length*numProcesses.  Gather is the inverse operation of Scatter.

\item  \verb|int = obj.Gather (int sendBuffer, int recvBuffer, vtkIdType length, int destProcessId)| -  Gather collects arrays in the process with id \c destProcessId.  Each
 process (including the destination) sends the contents of its send buffer
 to the destination process.  The destination process receives the
 messages and stores them in rank order.  The \c length argument
 (which must be the same on all processes) is the length of the
 sendBuffers.  The \c recvBuffer (on te destination process) must be of
 length length*numProcesses.  Gather is the inverse operation of Scatter.

\item  \verb|int = obj.Gather (long sendBuffer, long recvBuffer, vtkIdType length, int destProcessId)| -  Gather collects arrays in the process with id \c destProcessId.  Each
 process (including the destination) sends the contents of its send buffer
 to the destination process.  The destination process receives the
 messages and stores them in rank order.  The \c length argument
 (which must be the same on all processes) is the length of the
 sendBuffers.  The \c recvBuffer (on te destination process) must be of
 length length*numProcesses.  Gather is the inverse operation of Scatter.

\item  \verb|int = obj.Gather (string sendBuffer, string recvBuffer, vtkIdType length, int destProcessId)| -  Gather collects arrays in the process with id \c destProcessId.  Each
 process (including the destination) sends the contents of its send buffer
 to the destination process.  The destination process receives the
 messages and stores them in rank order.  The \c length argument
 (which must be the same on all processes) is the length of the
 sendBuffers.  The \c recvBuffer (on te destination process) must be of
 length length*numProcesses.  Gather is the inverse operation of Scatter.

\item  \verb|int = obj.Gather (string sendBuffer, string recvBuffer, vtkIdType length, int destProcessId)| -  Gather collects arrays in the process with id \c destProcessId.  Each
 process (including the destination) sends the contents of its send buffer
 to the destination process.  The destination process receives the
 messages and stores them in rank order.  The \c length argument
 (which must be the same on all processes) is the length of the
 sendBuffers.  The \c recvBuffer (on te destination process) must be of
 length length*numProcesses.  Gather is the inverse operation of Scatter.

\item  \verb|int = obj.Gather (float sendBuffer, float recvBuffer, vtkIdType length, int destProcessId)| -  Gather collects arrays in the process with id \c destProcessId.  Each
 process (including the destination) sends the contents of its send buffer
 to the destination process.  The destination process receives the
 messages and stores them in rank order.  The \c length argument
 (which must be the same on all processes) is the length of the
 sendBuffers.  The \c recvBuffer (on te destination process) must be of
 length length*numProcesses.  Gather is the inverse operation of Scatter.

\item  \verb|int = obj.Gather (double sendBuffer, double recvBuffer, vtkIdType length, int destProcessId)| -  Gather collects arrays in the process with id \c destProcessId.  Each
 process (including the destination) sends the contents of its send buffer
 to the destination process.  The destination process receives the
 messages and stores them in rank order.  The \c length argument
 (which must be the same on all processes) is the length of the
 sendBuffers.  The \c recvBuffer (on te destination process) must be of
 length length*numProcesses.  Gather is the inverse operation of Scatter.

\item  \verb|int = obj.Gather (vtkDataArray sendBuffer, vtkDataArray recvBuffer, int destProcessId)| -  GatherV is the vector variant of Gather.  It extends the functionality of
 Gather by allowing a varying count of data from each process.
 GatherV collects arrays in the process with id \c destProcessId.  Each
 process (including the destination) sends the contents of its send buffer
 to the destination process.  The destination process receives the
 messages and stores them in rank order.  The \c sendLength argument
 defines how much the local process sends to \c destProcessId and
 \c recvLengths is an array containing the amount \c destProcessId
 receives from each process, in rank order.

\item  \verb|int = obj.Scatter (int sendBuffer, int recvBuffer, vtkIdType length, int srcProcessId)| -  Scatter takes an array in the process with id \c srcProcessId and
 distributes it.  Each process (including the source) receives a portion of
 the send buffer.  Process 0 receives the first \c length values, process 1
 receives the second \c length values, and so on.  Scatter is the inverse
 operation of Gather.

\item  \verb|int = obj.Scatter (long sendBuffer, long recvBuffer, vtkIdType length, int srcProcessId)| -  Scatter takes an array in the process with id \c srcProcessId and
 distributes it.  Each process (including the source) receives a portion of
 the send buffer.  Process 0 receives the first \c length values, process 1
 receives the second \c length values, and so on.  Scatter is the inverse
 operation of Gather.

\item  \verb|int = obj.Scatter (string sendBuffer, string recvBuffer, vtkIdType length, int srcProcessId)| -  Scatter takes an array in the process with id \c srcProcessId and
 distributes it.  Each process (including the source) receives a portion of
 the send buffer.  Process 0 receives the first \c length values, process 1
 receives the second \c length values, and so on.  Scatter is the inverse
 operation of Gather.

\item  \verb|int = obj.Scatter (string sendBuffer, string recvBuffer, vtkIdType length, int srcProcessId)| -  Scatter takes an array in the process with id \c srcProcessId and
 distributes it.  Each process (including the source) receives a portion of
 the send buffer.  Process 0 receives the first \c length values, process 1
 receives the second \c length values, and so on.  Scatter is the inverse
 operation of Gather.

\item  \verb|int = obj.Scatter (float sendBuffer, float recvBuffer, vtkIdType length, int srcProcessId)| -  Scatter takes an array in the process with id \c srcProcessId and
 distributes it.  Each process (including the source) receives a portion of
 the send buffer.  Process 0 receives the first \c length values, process 1
 receives the second \c length values, and so on.  Scatter is the inverse
 operation of Gather.

\item  \verb|int = obj.Scatter (double sendBuffer, double recvBuffer, vtkIdType length, int srcProcessId)| -  Scatter takes an array in the process with id \c srcProcessId and
 distributes it.  Each process (including the source) receives a portion of
 the send buffer.  Process 0 receives the first \c length values, process 1
 receives the second \c length values, and so on.  Scatter is the inverse
 operation of Gather.

\item  \verb|int = obj.Scatter (vtkDataArray sendBuffer, vtkDataArray recvBuffer, int srcProcessId)| -  ScatterV is the vector variant of Scatter.  It extends the functionality of
 Scatter by allowing a varying count of data to each process.
 ScatterV takes an array in the process with id \c srcProcessId and
 distributes it.  Each process (including the source) receives a portion of
 the send buffer defined by the \c sendLengths and \c offsets arrays.

\item  \verb|int = obj.AllGather (int sendBuffer, int recvBuffer, vtkIdType length)| -  Same as gather except that the result ends up on all processes.

\item  \verb|int = obj.AllGather (long sendBuffer, long recvBuffer, vtkIdType length)| -  Same as gather except that the result ends up on all processes.

\item  \verb|int = obj.AllGather (string sendBuffer, string recvBuffer, vtkIdType length)| -  Same as gather except that the result ends up on all processes.

\item  \verb|int = obj.AllGather (string sendBuffer, string recvBuffer, vtkIdType length)| -  Same as gather except that the result ends up on all processes.

\item  \verb|int = obj.AllGather (float sendBuffer, float recvBuffer, vtkIdType length)| -  Same as gather except that the result ends up on all processes.

\item  \verb|int = obj.AllGather (double sendBuffer, double recvBuffer, vtkIdType length)| -  Same as gather except that the result ends up on all processes.

\item  \verb|int = obj.AllGather (vtkDataArray sendBuffer, vtkDataArray recvBuffer)| -  Same as GatherV except that the result is placed in all processes.

\item  \verb|int = obj.Reduce (int sendBuffer, int recvBuffer, vtkIdType length, int operation, int destProcessId)| -  Reduce an array to the given destination process.  This version of Reduce
 takes an identifier defined in the
 vtkCommunicator::StandardOperations enum to define the operation.

\item  \verb|int = obj.Reduce (long sendBuffer, long recvBuffer, vtkIdType length, int operation, int destProcessId)| -  Reduce an array to the given destination process.  This version of Reduce
 takes an identifier defined in the
 vtkCommunicator::StandardOperations enum to define the operation.

\item  \verb|int = obj.Reduce (string sendBuffer, string recvBuffer, vtkIdType length, int operation, int destProcessId)| -  Reduce an array to the given destination process.  This version of Reduce
 takes an identifier defined in the
 vtkCommunicator::StandardOperations enum to define the operation.

\item  \verb|int = obj.Reduce (string sendBuffer, string recvBuffer, vtkIdType length, int operation, int destProcessId)| -  Reduce an array to the given destination process.  This version of Reduce
 takes an identifier defined in the
 vtkCommunicator::StandardOperations enum to define the operation.

\item  \verb|int = obj.Reduce (float sendBuffer, float recvBuffer, vtkIdType length, int operation, int destProcessId)| -  Reduce an array to the given destination process.  This version of Reduce
 takes an identifier defined in the
 vtkCommunicator::StandardOperations enum to define the operation.

\item  \verb|int = obj.Reduce (double sendBuffer, double recvBuffer, vtkIdType length, int operation, int destProcessId)| -  Reduce an array to the given destination process.  This version of Reduce
 takes an identifier defined in the
 vtkCommunicator::StandardOperations enum to define the operation.

\item  \verb|int = obj.Reduce (vtkDataArray sendBuffer, vtkDataArray recvBuffer, int operation, int destProcessId)| -  Same as Reduce except that the result is placed in all of the processes.

\item  \verb|int = obj.AllReduce (int sendBuffer, int recvBuffer, vtkIdType length, int operation)| -  Same as Reduce except that the result is placed in all of the processes.

\item  \verb|int = obj.AllReduce (long sendBuffer, long recvBuffer, vtkIdType length, int operation)| -  Same as Reduce except that the result is placed in all of the processes.

\item  \verb|int = obj.AllReduce (string sendBuffer, string recvBuffer, vtkIdType length, int operation)| -  Same as Reduce except that the result is placed in all of the processes.

\item  \verb|int = obj.AllReduce (string sendBuffer, string recvBuffer, vtkIdType length, int operation)| -  Same as Reduce except that the result is placed in all of the processes.

\item  \verb|int = obj.AllReduce (float sendBuffer, float recvBuffer, vtkIdType length, int operation)| -  Same as Reduce except that the result is placed in all of the processes.

\item  \verb|int = obj.AllReduce (double sendBuffer, double recvBuffer, vtkIdType length, int operation)| -  Same as Reduce except that the result is placed in all of the processes.

\item  \verb|int = obj.AllReduce (vtkDataArray sendBuffer, vtkDataArray recvBuffer, int operation)|

\end{itemize}
