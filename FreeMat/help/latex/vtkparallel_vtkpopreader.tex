\section{vtkPOPReader}

\subsection{Usage}

 vtkPOPReader Just converts from images to a structured grid for now.

To create an instance of class vtkPOPReader, simply
invoke its constructor as follows
\begin{verbatim}
  obj = vtkPOPReader
\end{verbatim}
\subsection{Methods}

The class vtkPOPReader has several methods that can be used.
  They are listed below.
Note that the documentation is translated automatically from the VTK sources,
and may not be completely intelligible.  When in doubt, consult the VTK website.
In the methods listed below, \verb|obj| is an instance of the vtkPOPReader class.
\begin{itemize}
\item  \verb|string = obj.GetClassName ()|

\item  \verb|int = obj.IsA (string name)|

\item  \verb|vtkPOPReader = obj.NewInstance ()|

\item  \verb|vtkPOPReader = obj.SafeDownCast (vtkObject o)|

\item  \verb|int = obj. GetDimensions ()| -  This is the longitude and latitude dimensions of the structured grid.

\item  \verb|string = obj.GetGridFileName ()| -  This file contains the latitude and longitude of the grid.  
 It must be double with no header.

\item  \verb|string = obj.GetUFlowFileName ()| -  These files contains the u and v components of the flow.

\item  \verb|string = obj.GetVFlowFileName ()| -  These files contains the u and v components of the flow.

\item  \verb|obj.SetFileName (string )| -  This file contains information about all the files.

\item  \verb|string = obj.GetFileName ()| -  This file contains information about all the files.

\item  \verb|obj.SetRadius (double )| -  Radius of the earth.

\item  \verb|double = obj.GetRadius ()| -  Radius of the earth.

\item  \verb|obj.SetClipExtent (int , int , int , int , int , int )| -  Because the data can be so large, here is an option to clip
 while reading.

\item  \verb|obj.SetClipExtent (int  a[6])| -  Because the data can be so large, here is an option to clip
 while reading.

\item  \verb|int = obj. GetClipExtent ()| -  Because the data can be so large, here is an option to clip
 while reading.

\item  \verb|obj.SetNumberOfGhostLevels (int )| -  Set the number of ghost levels to include in the data

\item  \verb|int = obj.GetNumberOfGhostLevels ()| -  Set the number of ghost levels to include in the data

\end{itemize}
