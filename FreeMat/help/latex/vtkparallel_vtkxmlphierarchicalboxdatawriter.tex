\section{vtkXMLPHierarchicalBoxDataWriter}

\subsection{Usage}

 vtkXMLPCompositeDataWriter writes (in parallel or serially) the VTK XML
 multi-group, multi-block hierarchical and hierarchical box files. XML
 multi-group data files are meta-files that point to a list of serial VTK
 XML files.

To create an instance of class vtkXMLPHierarchicalBoxDataWriter, simply
invoke its constructor as follows
\begin{verbatim}
  obj = vtkXMLPHierarchicalBoxDataWriter
\end{verbatim}
\subsection{Methods}

The class vtkXMLPHierarchicalBoxDataWriter has several methods that can be used.
  They are listed below.
Note that the documentation is translated automatically from the VTK sources,
and may not be completely intelligible.  When in doubt, consult the VTK website.
In the methods listed below, \verb|obj| is an instance of the vtkXMLPHierarchicalBoxDataWriter class.
\begin{itemize}
\item  \verb|string = obj.GetClassName ()|

\item  \verb|int = obj.IsA (string name)|

\item  \verb|vtkXMLPHierarchicalBoxDataWriter = obj.NewInstance ()|

\item  \verb|vtkXMLPHierarchicalBoxDataWriter = obj.SafeDownCast (vtkObject o)|

\item  \verb|obj.SetController (vtkMultiProcessController )| -  Controller used to communicate data type of blocks.
 By default, the global controller is used. If you want another
 controller to be used, set it with this.
 If no controller is set, only the local blocks will be written
 to the meta-file.

\item  \verb|vtkMultiProcessController = obj.GetController ()| -  Controller used to communicate data type of blocks.
 By default, the global controller is used. If you want another
 controller to be used, set it with this.
 If no controller is set, only the local blocks will be written
 to the meta-file.

\item  \verb|obj.SetWriteMetaFile (int flag)| -  Set whether this instance will write the meta-file. WriteMetaFile
 is set to flag only on process 0 and all other processes have
 WriteMetaFile set to 0 by default.

\end{itemize}
