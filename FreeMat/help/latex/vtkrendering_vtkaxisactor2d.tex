\section{vtkAxisActor2D}

\subsection{Usage}

 vtkAxisActor2D creates an axis with tick marks, labels, and/or a title,
 depending on the particular instance variable settings. vtkAxisActor2D is
 a 2D actor; that is, it is drawn on the overlay plane and is not
 occluded by 3D geometry. To use this class, you typically specify two
 points defining the start and end points of the line (x-y definition using
 vtkCoordinate class), the number of labels, and the data range
 (min,max). You can also control what parts of the axis are visible
 including the line, the tick marks, the labels, and the title.  You can 
 also specify the label format (a printf style format).

 This class decides what font size to use and how to locate the labels. It
 also decides how to create reasonable tick marks and labels. The number
 of labels and the range of values may not match the number specified, but
 should be close.

 Labels are drawn on the ''right'' side of the axis. The ''right'' side is
 the side of the axis on the right as you move from Position to Position2. 
 The way the labels and title line up with the axis and tick marks depends on
 whether the line is considered horizontal or vertical.

 The vtkActor2D instance variables Position and Position2 are instances of 
 vtkCoordinate. Note that the Position2 is an absolute position in that 
 class (it was by default relative to Position in vtkActor2D).

 What this means is that you can specify the axis in a variety of coordinate
 systems. Also, the axis does not have to be either horizontal or vertical.
 The tick marks are created so that they are perpendicular to the axis.

 Set the text property/attributes of the title and the labels through the 
 vtkTextProperty objects associated to this actor.


To create an instance of class vtkAxisActor2D, simply
invoke its constructor as follows
\begin{verbatim}
  obj = vtkAxisActor2D
\end{verbatim}
\subsection{Methods}

The class vtkAxisActor2D has several methods that can be used.
  They are listed below.
Note that the documentation is translated automatically from the VTK sources,
and may not be completely intelligible.  When in doubt, consult the VTK website.
In the methods listed below, \verb|obj| is an instance of the vtkAxisActor2D class.
\begin{itemize}
\item  \verb|string = obj.GetClassName ()|

\item  \verb|int = obj.IsA (string name)|

\item  \verb|vtkAxisActor2D = obj.NewInstance ()|

\item  \verb|vtkAxisActor2D = obj.SafeDownCast (vtkObject o)|

\item  \verb|vtkCoordinate = obj.GetPoint1Coordinate ()| -  Specify the position of the first point defining the axis.
 Note: backward compatibility only, use vtkActor2D's Position instead.

\item  \verb|obj.SetPoint1 (double x[2])| -  Specify the position of the first point defining the axis.
 Note: backward compatibility only, use vtkActor2D's Position instead.

\item  \verb|obj.SetPoint1 (double x, double y)| -  Specify the position of the first point defining the axis.
 Note: backward compatibility only, use vtkActor2D's Position instead.

\item  \verb|vtkCoordinate = obj.GetPoint2Coordinate ()| -  Specify the position of the second point defining the axis. Note that
 the order from Point1 to Point2 controls which side the tick marks
 are drawn on (ticks are drawn on the right, if visible).
 Note: backward compatibility only, use vtkActor2D's Position2 instead.

\item  \verb|obj.SetPoint2 (double x[2])| -  Specify the position of the second point defining the axis. Note that
 the order from Point1 to Point2 controls which side the tick marks
 are drawn on (ticks are drawn on the right, if visible).
 Note: backward compatibility only, use vtkActor2D's Position2 instead.

\item  \verb|obj.SetPoint2 (double x, double y)| -  Specify the position of the second point defining the axis. Note that
 the order from Point1 to Point2 controls which side the tick marks
 are drawn on (ticks are drawn on the right, if visible).
 Note: backward compatibility only, use vtkActor2D's Position2 instead.

\item  \verb|obj.SetRange (double , double )| -  Specify the (min,max) axis range. This will be used in the generation
 of labels, if labels are visible.

\item  \verb|obj.SetRange (double  a[2])| -  Specify the (min,max) axis range. This will be used in the generation
 of labels, if labels are visible.

\item  \verb|double = obj. GetRange ()| -  Specify the (min,max) axis range. This will be used in the generation
 of labels, if labels are visible.

\item  \verb|obj.SetNumberOfLabels (int )| -  Set/Get the number of annotation labels to show.

\item  \verb|int = obj.GetNumberOfLabelsMinValue ()| -  Set/Get the number of annotation labels to show.

\item  \verb|int = obj.GetNumberOfLabelsMaxValue ()| -  Set/Get the number of annotation labels to show.

\item  \verb|int = obj.GetNumberOfLabels ()| -  Set/Get the number of annotation labels to show.

\item  \verb|obj.SetLabelFormat (string )| -  Set/Get the format with which to print the labels on the scalar
 bar.

\item  \verb|string = obj.GetLabelFormat ()| -  Set/Get the format with which to print the labels on the scalar
 bar.

\item  \verb|obj.SetAdjustLabels (int )| -  Set/Get the flag that controls whether the labels and ticks are
 adjusted for ''nice'' numerical values to make it easier to read 
 the labels. The adjustment is based in the Range instance variable.
 Call GetAdjustedRange and GetAdjustedNumberOfLabels to get the adjusted
 range and number of labels.

\item  \verb|int = obj.GetAdjustLabels ()| -  Set/Get the flag that controls whether the labels and ticks are
 adjusted for ''nice'' numerical values to make it easier to read 
 the labels. The adjustment is based in the Range instance variable.
 Call GetAdjustedRange and GetAdjustedNumberOfLabels to get the adjusted
 range and number of labels.

\item  \verb|obj.AdjustLabelsOn ()| -  Set/Get the flag that controls whether the labels and ticks are
 adjusted for ''nice'' numerical values to make it easier to read 
 the labels. The adjustment is based in the Range instance variable.
 Call GetAdjustedRange and GetAdjustedNumberOfLabels to get the adjusted
 range and number of labels.

\item  \verb|obj.AdjustLabelsOff ()| -  Set/Get the flag that controls whether the labels and ticks are
 adjusted for ''nice'' numerical values to make it easier to read 
 the labels. The adjustment is based in the Range instance variable.
 Call GetAdjustedRange and GetAdjustedNumberOfLabels to get the adjusted
 range and number of labels.

\item  \verb|obj.GetAdjustedRange (double \_arg[2])| -  Set/Get the flag that controls whether the labels and ticks are
 adjusted for ''nice'' numerical values to make it easier to read 
 the labels. The adjustment is based in the Range instance variable.
 Call GetAdjustedRange and GetAdjustedNumberOfLabels to get the adjusted
 range and number of labels.

\item  \verb|int = obj.GetAdjustedNumberOfLabels ()| -  Set/Get the title of the scalar bar actor,

\item  \verb|obj.SetTitle (string )| -  Set/Get the title of the scalar bar actor,

\item  \verb|string = obj.GetTitle ()| -  Set/Get the title of the scalar bar actor,

\item  \verb|obj.SetTitleTextProperty (vtkTextProperty p)| -  Set/Get the title text property.

\item  \verb|vtkTextProperty = obj.GetTitleTextProperty ()| -  Set/Get the title text property.

\item  \verb|obj.SetLabelTextProperty (vtkTextProperty p)| -  Set/Get the labels text property.

\item  \verb|vtkTextProperty = obj.GetLabelTextProperty ()| -  Set/Get the labels text property.

\item  \verb|obj.SetTickLength (int )| -  Set/Get the length of the tick marks (expressed in pixels or display
 coordinates). 

\item  \verb|int = obj.GetTickLengthMinValue ()| -  Set/Get the length of the tick marks (expressed in pixels or display
 coordinates). 

\item  \verb|int = obj.GetTickLengthMaxValue ()| -  Set/Get the length of the tick marks (expressed in pixels or display
 coordinates). 

\item  \verb|int = obj.GetTickLength ()| -  Set/Get the length of the tick marks (expressed in pixels or display
 coordinates). 

\item  \verb|obj.SetNumberOfMinorTicks (int )| -  Number of minor ticks to be displayed between each tick. Default
 is 0.

\item  \verb|int = obj.GetNumberOfMinorTicksMinValue ()| -  Number of minor ticks to be displayed between each tick. Default
 is 0.

\item  \verb|int = obj.GetNumberOfMinorTicksMaxValue ()| -  Number of minor ticks to be displayed between each tick. Default
 is 0.

\item  \verb|int = obj.GetNumberOfMinorTicks ()| -  Number of minor ticks to be displayed between each tick. Default
 is 0.

\item  \verb|obj.SetMinorTickLength (int )| -  Set/Get the length of the minor tick marks (expressed in pixels or
 display coordinates).

\item  \verb|int = obj.GetMinorTickLengthMinValue ()| -  Set/Get the length of the minor tick marks (expressed in pixels or
 display coordinates).

\item  \verb|int = obj.GetMinorTickLengthMaxValue ()| -  Set/Get the length of the minor tick marks (expressed in pixels or
 display coordinates).

\item  \verb|int = obj.GetMinorTickLength ()| -  Set/Get the length of the minor tick marks (expressed in pixels or
 display coordinates).

\item  \verb|obj.SetTickOffset (int )| -  Set/Get the offset of the labels (expressed in pixels or display
 coordinates). The offset is the distance of labels from tick marks
 or other objects.

\item  \verb|int = obj.GetTickOffsetMinValue ()| -  Set/Get the offset of the labels (expressed in pixels or display
 coordinates). The offset is the distance of labels from tick marks
 or other objects.

\item  \verb|int = obj.GetTickOffsetMaxValue ()| -  Set/Get the offset of the labels (expressed in pixels or display
 coordinates). The offset is the distance of labels from tick marks
 or other objects.

\item  \verb|int = obj.GetTickOffset ()| -  Set/Get the offset of the labels (expressed in pixels or display
 coordinates). The offset is the distance of labels from tick marks
 or other objects.

\item  \verb|obj.SetAxisVisibility (int )| -  Set/Get visibility of the axis line.

\item  \verb|int = obj.GetAxisVisibility ()| -  Set/Get visibility of the axis line.

\item  \verb|obj.AxisVisibilityOn ()| -  Set/Get visibility of the axis line.

\item  \verb|obj.AxisVisibilityOff ()| -  Set/Get visibility of the axis line.

\item  \verb|obj.SetTickVisibility (int )| -  Set/Get visibility of the axis tick marks.

\item  \verb|int = obj.GetTickVisibility ()| -  Set/Get visibility of the axis tick marks.

\item  \verb|obj.TickVisibilityOn ()| -  Set/Get visibility of the axis tick marks.

\item  \verb|obj.TickVisibilityOff ()| -  Set/Get visibility of the axis tick marks.

\item  \verb|obj.SetLabelVisibility (int )| -  Set/Get visibility of the axis labels.

\item  \verb|int = obj.GetLabelVisibility ()| -  Set/Get visibility of the axis labels.

\item  \verb|obj.LabelVisibilityOn ()| -  Set/Get visibility of the axis labels.

\item  \verb|obj.LabelVisibilityOff ()| -  Set/Get visibility of the axis labels.

\item  \verb|obj.SetTitleVisibility (int )| -  Set/Get visibility of the axis title.

\item  \verb|int = obj.GetTitleVisibility ()| -  Set/Get visibility of the axis title.

\item  \verb|obj.TitleVisibilityOn ()| -  Set/Get visibility of the axis title.

\item  \verb|obj.TitleVisibilityOff ()| -  Set/Get visibility of the axis title.

\item  \verb|obj.SetTitlePosition (double )| -  Set/Get position of the axis title. 0 is at the start of the
 axis whereas 1 is at the end.

\item  \verb|double = obj.GetTitlePosition ()| -  Set/Get position of the axis title. 0 is at the start of the
 axis whereas 1 is at the end.

\item  \verb|obj.SetFontFactor (double )| -  Set/Get the factor that controls the overall size of the fonts used
 to label and title the axes. This ivar used in conjunction with
 the LabelFactor can be used to control font sizes.

\item  \verb|double = obj.GetFontFactorMinValue ()| -  Set/Get the factor that controls the overall size of the fonts used
 to label and title the axes. This ivar used in conjunction with
 the LabelFactor can be used to control font sizes.

\item  \verb|double = obj.GetFontFactorMaxValue ()| -  Set/Get the factor that controls the overall size of the fonts used
 to label and title the axes. This ivar used in conjunction with
 the LabelFactor can be used to control font sizes.

\item  \verb|double = obj.GetFontFactor ()| -  Set/Get the factor that controls the overall size of the fonts used
 to label and title the axes. This ivar used in conjunction with
 the LabelFactor can be used to control font sizes.

\item  \verb|obj.SetLabelFactor (double )| -  Set/Get the factor that controls the relative size of the axis labels
 to the axis title.

\item  \verb|double = obj.GetLabelFactorMinValue ()| -  Set/Get the factor that controls the relative size of the axis labels
 to the axis title.

\item  \verb|double = obj.GetLabelFactorMaxValue ()| -  Set/Get the factor that controls the relative size of the axis labels
 to the axis title.

\item  \verb|double = obj.GetLabelFactor ()| -  Set/Get the factor that controls the relative size of the axis labels
 to the axis title.

\item  \verb|int = obj.RenderOverlay (vtkViewport viewport)| -  Draw the axis. 

\item  \verb|int = obj.RenderOpaqueGeometry (vtkViewport viewport)| -  Draw the axis. 

\item  \verb|int = obj.RenderTranslucentPolygonalGeometry (vtkViewport )| -  Does this prop have some translucent polygonal geometry?

\item  \verb|int = obj.HasTranslucentPolygonalGeometry ()| -  Does this prop have some translucent polygonal geometry?

\item  \verb|obj.ReleaseGraphicsResources (vtkWindow )| -  Release any graphics resources that are being consumed by this actor.
 The parameter window could be used to determine which graphic
 resources to release.

\item  \verb|obj.SetSizeFontRelativeToAxis (int )| -  Specify whether to size the fonts relative to the viewport or relative to
 length of the axis. By default, fonts are resized relative to the axis.

\item  \verb|int = obj.GetSizeFontRelativeToAxis ()| -  Specify whether to size the fonts relative to the viewport or relative to
 length of the axis. By default, fonts are resized relative to the axis.

\item  \verb|obj.SizeFontRelativeToAxisOn ()| -  Specify whether to size the fonts relative to the viewport or relative to
 length of the axis. By default, fonts are resized relative to the axis.

\item  \verb|obj.SizeFontRelativeToAxisOff ()| -  Specify whether to size the fonts relative to the viewport or relative to
 length of the axis. By default, fonts are resized relative to the axis.

\item  \verb|obj.ShallowCopy (vtkProp prop)| -  Shallow copy of an axis actor. Overloads the virtual vtkProp method.

\end{itemize}
