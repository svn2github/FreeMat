\section{vtkCellPicker}

\subsection{Usage}

 vtkCellPicker will shoot a ray into a 3D scene and return information
 about the first object that the ray hits.  It works for all Prop3Ds.
 For vtkVolume objects, it shoots a ray into the volume and returns
 the point where the ray intersects an isosurface of a chosen opacity.
 For vtkImageActor objects, it intersects the ray with the displayed
 slice. For vtkActor objects, it intersects the actor's polygons.
 If the object's mapper has ClippingPlanes, then it takes the clipping
 into account, and will return the Id of the clipping plane that was
 intersected.
 For all prop types, it returns point and cell information, plus the
 normal of the surface that was intersected at the pick position.  For
 volumes and images, it also returns (i,j,k) coordinates for the point
 and the cell that were picked.  


To create an instance of class vtkCellPicker, simply
invoke its constructor as follows
\begin{verbatim}
  obj = vtkCellPicker
\end{verbatim}
\subsection{Methods}

The class vtkCellPicker has several methods that can be used.
  They are listed below.
Note that the documentation is translated automatically from the VTK sources,
and may not be completely intelligible.  When in doubt, consult the VTK website.
In the methods listed below, \verb|obj| is an instance of the vtkCellPicker class.
\begin{itemize}
\item  \verb|string = obj.GetClassName ()|

\item  \verb|int = obj.IsA (string name)|

\item  \verb|vtkCellPicker = obj.NewInstance ()|

\item  \verb|vtkCellPicker = obj.SafeDownCast (vtkObject o)|

\item  \verb|int = obj.Pick (double selectionX, double selectionY, double selectionZ, vtkRenderer renderer)| -  Perform pick operation with selection point provided. Normally the 
 first two values are the (x,y) pixel coordinates for the pick, and
 the third value is z=0. The return value will be non-zero if
 something was successfully picked.

\item  \verb|obj.AddLocator (vtkAbstractCellLocator locator)| -  Add a locator for one of the data sets that will be included in the
 scene.  You must set up the locator with exactly the same data set
 that was input to the mapper of one or more of the actors in the
 scene.  As well, you must either build the locator before doing the
 pick, or you must turn on LazyEvaluation in the locator to make it
 build itself on the first pick.  Note that if you try to add the
 same locator to the picker twice, the second addition will be ignored.

\item  \verb|obj.RemoveLocator (vtkAbstractCellLocator locator)| -  Remove a locator that was previously added.  If you try to remove a
 nonexistent locator, then nothing will happen and no errors will be
 raised.

\item  \verb|obj.RemoveAllLocators ()| -  Remove all locators associated with this picker.

\item  \verb|obj.SetVolumeOpacityIsovalue (double )| -  Set the opacity isovalue to use for defining volume surfaces.  The
 pick will occur at the location along the pick ray where the 
 opacity of the volume is equal to this isovalue.  If you want to do
 the pick based on an actual data isovalue rather than the opacity,
 then pass the data value through the scalar opacity function before
 using this method.

\item  \verb|double = obj.GetVolumeOpacityIsovalue ()| -  Set the opacity isovalue to use for defining volume surfaces.  The
 pick will occur at the location along the pick ray where the 
 opacity of the volume is equal to this isovalue.  If you want to do
 the pick based on an actual data isovalue rather than the opacity,
 then pass the data value through the scalar opacity function before
 using this method.

\item  \verb|obj.SetUseVolumeGradientOpacity (int )| -  Use the product of the scalar and gradient opacity functions when
 computing the opacity isovalue, instead of just using the scalar
 opacity. This parameter is only relevant to volume picking and
 is off by default.

\item  \verb|obj.UseVolumeGradientOpacityOn ()| -  Use the product of the scalar and gradient opacity functions when
 computing the opacity isovalue, instead of just using the scalar
 opacity. This parameter is only relevant to volume picking and
 is off by default.

\item  \verb|obj.UseVolumeGradientOpacityOff ()| -  Use the product of the scalar and gradient opacity functions when
 computing the opacity isovalue, instead of just using the scalar
 opacity. This parameter is only relevant to volume picking and
 is off by default.

\item  \verb|int = obj.GetUseVolumeGradientOpacity ()| -  Use the product of the scalar and gradient opacity functions when
 computing the opacity isovalue, instead of just using the scalar
 opacity. This parameter is only relevant to volume picking and
 is off by default.

\item  \verb|obj.SetPickClippingPlanes (int )| -  The PickClippingPlanes setting controls how clipping planes are
 handled by the pick.  If it is On, then the clipping planes become
 pickable objects, even though they are usually invisible.  This
 means that if the pick ray intersects a clipping plane before it
 hits anything else, the pick will stop at that clipping plane.
 The GetProp3D() and GetMapper() methods will return the Prop3D
 and Mapper that the clipping plane belongs to.  The
 GetClippingPlaneId() method will return the index of the clipping
 plane so that you can retrieve it from the mapper, or -1 if no
 clipping plane was picked. The picking of vtkImageActors is not
 influenced by this setting, since they have no clipping planes.

\item  \verb|obj.PickClippingPlanesOn ()| -  The PickClippingPlanes setting controls how clipping planes are
 handled by the pick.  If it is On, then the clipping planes become
 pickable objects, even though they are usually invisible.  This
 means that if the pick ray intersects a clipping plane before it
 hits anything else, the pick will stop at that clipping plane.
 The GetProp3D() and GetMapper() methods will return the Prop3D
 and Mapper that the clipping plane belongs to.  The
 GetClippingPlaneId() method will return the index of the clipping
 plane so that you can retrieve it from the mapper, or -1 if no
 clipping plane was picked. The picking of vtkImageActors is not
 influenced by this setting, since they have no clipping planes.

\item  \verb|obj.PickClippingPlanesOff ()| -  The PickClippingPlanes setting controls how clipping planes are
 handled by the pick.  If it is On, then the clipping planes become
 pickable objects, even though they are usually invisible.  This
 means that if the pick ray intersects a clipping plane before it
 hits anything else, the pick will stop at that clipping plane.
 The GetProp3D() and GetMapper() methods will return the Prop3D
 and Mapper that the clipping plane belongs to.  The
 GetClippingPlaneId() method will return the index of the clipping
 plane so that you can retrieve it from the mapper, or -1 if no
 clipping plane was picked. The picking of vtkImageActors is not
 influenced by this setting, since they have no clipping planes.

\item  \verb|int = obj.GetPickClippingPlanes ()| -  The PickClippingPlanes setting controls how clipping planes are
 handled by the pick.  If it is On, then the clipping planes become
 pickable objects, even though they are usually invisible.  This
 means that if the pick ray intersects a clipping plane before it
 hits anything else, the pick will stop at that clipping plane.
 The GetProp3D() and GetMapper() methods will return the Prop3D
 and Mapper that the clipping plane belongs to.  The
 GetClippingPlaneId() method will return the index of the clipping
 plane so that you can retrieve it from the mapper, or -1 if no
 clipping plane was picked. The picking of vtkImageActors is not
 influenced by this setting, since they have no clipping planes.

\item  \verb|int = obj.GetClippingPlaneId ()| -  Get the index of the clipping plane that was intersected during
 the pick.  This will be set regardless of whether PickClippingPlanes
 is On, all that is required is that the pick intersected a clipping
 plane of the Prop3D that was picked.  The result will be -1 if the
 Prop3D that was picked has no clipping planes, or if the ray didn't
 intersect the planes.

\item  \verb|double = obj. GetPickNormal ()| -  Return the normal of the picked surface at the PickPosition.  If no
 surface was picked, then a vector pointing back at the camera is
 returned.

\item  \verb|double = obj. GetMapperNormal ()| -  Return the normal of the surface at the PickPosition in mapper
 coordinates.  The result is undefined if no prop was picked.

\item  \verb|int = obj. GetPointIJK ()| -  Get the structured coordinates of the point at the PickPosition.
 Only valid for image actors and volumes with vtkImageData.

\item  \verb|int = obj. GetCellIJK ()| -  Get the structured coordinates of the cell at the PickPosition.
 Only valid for image actors and volumes with vtkImageData.
 Combine this with the PCoords to get the position within the cell.

\item  \verb|vtkIdType = obj.GetPointId ()| -  Get the id of the picked point. If PointId = -1, nothing was picked.
 This point will be a member of any cell that is picked.

\item  \verb|vtkIdType = obj.GetCellId ()| -  Get the id of the picked cell. If CellId = -1, nothing was picked.

\item  \verb|int = obj.GetSubId ()| -  Get the subId of the picked cell. This is useful, for example, if
 the data is made of triangle strips. If SubId = -1, nothing was picked.

\item  \verb|double = obj. GetPCoords ()| -  Get the parametric coordinates of the picked cell. Only valid if
 a prop was picked.  The PCoords can be used to compute the weights
 that are needed to interpolate data values within the cell.

\item  \verb|vtkTexture = obj.GetTexture ()| -  Get the texture that was picked.  This will always be set if the
 picked prop has a texture, and will always be null otherwise.

\item  \verb|obj.SetPickTextureData (int )| -  If this is ''On'' and if the picked prop has a texture, then the data
 returned by GetDataSet() will be the texture's data instead of the
 mapper's data.  The GetPointId(), GetCellId(), GetPCoords() etc. will
 all return information for use with the texture's data.  If the picked
 prop does not have any texture, then GetDataSet() will return the
 mapper's data instead and GetPointId() etc. will return information
 related to the mapper's data.  The default value of PickTextureData
 is ''Off''.

\item  \verb|obj.PickTextureDataOn ()| -  If this is ''On'' and if the picked prop has a texture, then the data
 returned by GetDataSet() will be the texture's data instead of the
 mapper's data.  The GetPointId(), GetCellId(), GetPCoords() etc. will
 all return information for use with the texture's data.  If the picked
 prop does not have any texture, then GetDataSet() will return the
 mapper's data instead and GetPointId() etc. will return information
 related to the mapper's data.  The default value of PickTextureData
 is ''Off''.

\item  \verb|obj.PickTextureDataOff ()| -  If this is ''On'' and if the picked prop has a texture, then the data
 returned by GetDataSet() will be the texture's data instead of the
 mapper's data.  The GetPointId(), GetCellId(), GetPCoords() etc. will
 all return information for use with the texture's data.  If the picked
 prop does not have any texture, then GetDataSet() will return the
 mapper's data instead and GetPointId() etc. will return information
 related to the mapper's data.  The default value of PickTextureData
 is ''Off''.

\item  \verb|int = obj.GetPickTextureData ()| -  If this is ''On'' and if the picked prop has a texture, then the data
 returned by GetDataSet() will be the texture's data instead of the
 mapper's data.  The GetPointId(), GetCellId(), GetPCoords() etc. will
 all return information for use with the texture's data.  If the picked
 prop does not have any texture, then GetDataSet() will return the
 mapper's data instead and GetPointId() etc. will return information
 related to the mapper's data.  The default value of PickTextureData
 is ''Off''.

\end{itemize}
