\section{vtkColorMaterialHelper}

\subsection{Usage}

  vtkColorMaterialHelper is a helper to assist in simulating the
  ColorMaterial behaviour of the default OpenGL pipeline. Look at
  vtkColorMaterialHelper\_s for available GLSL functions.


To create an instance of class vtkColorMaterialHelper, simply
invoke its constructor as follows
\begin{verbatim}
  obj = vtkColorMaterialHelper
\end{verbatim}
\subsection{Methods}

The class vtkColorMaterialHelper has several methods that can be used.
  They are listed below.
Note that the documentation is translated automatically from the VTK sources,
and may not be completely intelligible.  When in doubt, consult the VTK website.
In the methods listed below, \verb|obj| is an instance of the vtkColorMaterialHelper class.
\begin{itemize}
\item  \verb|string = obj.GetClassName ()|

\item  \verb|int = obj.IsA (string name)|

\item  \verb|vtkColorMaterialHelper = obj.NewInstance ()|

\item  \verb|vtkColorMaterialHelper = obj.SafeDownCast (vtkObject o)|

\item  \verb|obj.PrepareForRendering ()| -  Prepares the shader i.e. reads color material paramters state from OpenGL. 
 This must be called before the shader is bound. 

\item  \verb|obj.Render ()| -  Uploads any uniforms needed. This must be called only
 after the shader has been bound, but before rendering the geometry.

\end{itemize}
