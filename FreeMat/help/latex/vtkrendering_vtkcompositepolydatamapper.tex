\section{vtkCompositePolyDataMapper}

\subsection{Usage}

 This class uses a set of vtkPolyDataMappers to render input data
 which may be hierarchical. The input to this mapper may be
 either vtkPolyData or a vtkCompositeDataSet built from 
 polydata. If something other than vtkPolyData is encountered,
 an error message will be produced.

To create an instance of class vtkCompositePolyDataMapper, simply
invoke its constructor as follows
\begin{verbatim}
  obj = vtkCompositePolyDataMapper
\end{verbatim}
\subsection{Methods}

The class vtkCompositePolyDataMapper has several methods that can be used.
  They are listed below.
Note that the documentation is translated automatically from the VTK sources,
and may not be completely intelligible.  When in doubt, consult the VTK website.
In the methods listed below, \verb|obj| is an instance of the vtkCompositePolyDataMapper class.
\begin{itemize}
\item  \verb|string = obj.GetClassName ()|

\item  \verb|int = obj.IsA (string name)|

\item  \verb|vtkCompositePolyDataMapper = obj.NewInstance ()|

\item  \verb|vtkCompositePolyDataMapper = obj.SafeDownCast (vtkObject o)|

\item  \verb|obj.Render (vtkRenderer ren, vtkActor a)| -  Standard method for rendering a mapper. This method will be 
 called by the actor.

\item  \verb|double = obj.GetBounds ()| -  Standard vtkProp method to get 3D bounds of a 3D prop

\item  \verb|obj.GetBounds (double bounds[6])| -  Standard vtkProp method to get 3D bounds of a 3D prop

\item  \verb|obj.ReleaseGraphicsResources (vtkWindow )| -  Release the underlying resources associated with this mapper  

\end{itemize}
