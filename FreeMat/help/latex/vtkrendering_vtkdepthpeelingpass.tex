\section{vtkDepthPeelingPass}

\subsection{Usage}

 Render the translucent polygonal geometry of a scene without sorting
 polygons in the view direction.

 This pass expects an initialized depth buffer and color buffer.
 Initialized buffers means they have been cleared with farest z-value and
 background color/gradient/transparent color.
 An opaque pass may have been performed right after the initialization.

 The depth peeling algorithm works by rendering the translucent polygonal
 geometry multiple times (once for each peel). The actually rendering of
 the translucent polygonal geometry is peformed by its delegate
 TranslucentPass. This delegate is therefore used multiple times.

 Its delegate is usually set to a vtkTranslucentPass.
 

To create an instance of class vtkDepthPeelingPass, simply
invoke its constructor as follows
\begin{verbatim}
  obj = vtkDepthPeelingPass
\end{verbatim}
\subsection{Methods}

The class vtkDepthPeelingPass has several methods that can be used.
  They are listed below.
Note that the documentation is translated automatically from the VTK sources,
and may not be completely intelligible.  When in doubt, consult the VTK website.
In the methods listed below, \verb|obj| is an instance of the vtkDepthPeelingPass class.
\begin{itemize}
\item  \verb|string = obj.GetClassName ()|

\item  \verb|int = obj.IsA (string name)|

\item  \verb|vtkDepthPeelingPass = obj.NewInstance ()|

\item  \verb|vtkDepthPeelingPass = obj.SafeDownCast (vtkObject o)|

\item  \verb|obj.ReleaseGraphicsResources (vtkWindow w)| -  Release graphics resources and ask components to release their own
 resources.
 

\item  \verb|vtkRenderPass = obj.GetTranslucentPass ()| -  Delegate for rendering the translucent polygonal geometry.
 If it is NULL, nothing will be rendered and a warning will be emitted.
 It is usually set to a vtkTranslucentPass.
 Initial value is a NULL pointer.

\item  \verb|obj.SetTranslucentPass (vtkRenderPass translucentPass)| -  Delegate for rendering the translucent polygonal geometry.
 If it is NULL, nothing will be rendered and a warning will be emitted.
 It is usually set to a vtkTranslucentPass.
 Initial value is a NULL pointer.

\item  \verb|obj.SetOcclusionRatio (double )| -  In case of use of depth peeling technique for rendering translucent
 material, define the threshold under which the algorithm stops to
 iterate over peel layers. This is the ratio of the number of pixels
 that have been touched by the last layer over the total number of pixels
 of the viewport area.
 Initial value is 0.0, meaning rendering have to be exact. Greater values
 may speed-up the rendering with small impact on the quality.

\item  \verb|double = obj.GetOcclusionRatioMinValue ()| -  In case of use of depth peeling technique for rendering translucent
 material, define the threshold under which the algorithm stops to
 iterate over peel layers. This is the ratio of the number of pixels
 that have been touched by the last layer over the total number of pixels
 of the viewport area.
 Initial value is 0.0, meaning rendering have to be exact. Greater values
 may speed-up the rendering with small impact on the quality.

\item  \verb|double = obj.GetOcclusionRatioMaxValue ()| -  In case of use of depth peeling technique for rendering translucent
 material, define the threshold under which the algorithm stops to
 iterate over peel layers. This is the ratio of the number of pixels
 that have been touched by the last layer over the total number of pixels
 of the viewport area.
 Initial value is 0.0, meaning rendering have to be exact. Greater values
 may speed-up the rendering with small impact on the quality.

\item  \verb|double = obj.GetOcclusionRatio ()| -  In case of use of depth peeling technique for rendering translucent
 material, define the threshold under which the algorithm stops to
 iterate over peel layers. This is the ratio of the number of pixels
 that have been touched by the last layer over the total number of pixels
 of the viewport area.
 Initial value is 0.0, meaning rendering have to be exact. Greater values
 may speed-up the rendering with small impact on the quality.

\item  \verb|obj.SetMaximumNumberOfPeels (int )| -  In case of depth peeling, define the maximum number of peeling layers.
 Initial value is 4. A special value of 0 means no maximum limit.
 It has to be a positive value.

\item  \verb|int = obj.GetMaximumNumberOfPeels ()| -  In case of depth peeling, define the maximum number of peeling layers.
 Initial value is 4. A special value of 0 means no maximum limit.
 It has to be a positive value.

\item  \verb|bool = obj.GetLastRenderingUsedDepthPeeling ()| -  Tells if the last time this pass was executed, the depth peeling
 algorithm was actually used. Initial value is false.

\end{itemize}
