\section{vtkGenericVertexAttributeMapping}

\subsection{Usage}

 vtkGenericVertexAttributeMapping stores mapping between data arrays and
 generic vertex attributes. It is used by vtkPainterPolyDataMapper to pass the
 mappings to the painter which rendering the attributes.
 .SECTION Thanks
 Support for generic vertex attributes in VTK was contributed in
 collaboration with Stephane Ploix at EDF.

To create an instance of class vtkGenericVertexAttributeMapping, simply
invoke its constructor as follows
\begin{verbatim}
  obj = vtkGenericVertexAttributeMapping
\end{verbatim}
\subsection{Methods}

The class vtkGenericVertexAttributeMapping has several methods that can be used.
  They are listed below.
Note that the documentation is translated automatically from the VTK sources,
and may not be completely intelligible.  When in doubt, consult the VTK website.
In the methods listed below, \verb|obj| is an instance of the vtkGenericVertexAttributeMapping class.
\begin{itemize}
\item  \verb|string = obj.GetClassName ()|

\item  \verb|int = obj.IsA (string name)|

\item  \verb|vtkGenericVertexAttributeMapping = obj.NewInstance ()|

\item  \verb|vtkGenericVertexAttributeMapping = obj.SafeDownCast (vtkObject o)|

\item  \verb|obj.AddMapping (string attributeName, string arrayName, int fieldAssociation, int component)| -  Select a data array from the point/cell data
 and map it to a generic vertex attribute.
 Note that indices change when a mapping is added/removed.

\item  \verb|obj.AddMapping (int unit, string arrayName, int fieldAssociation, int component)| -  Select a data array and use it as multitexture texture
 coordinates.
 Note the texture unit parameter should correspond to the texture
 unit set on the texture.

\item  \verb|bool = obj.RemoveMapping (string attributeName)| -  Remove a vertex attribute mapping.

\item  \verb|obj.RemoveAllMappings ()| -  Remove all mappings.

\item  \verb|int = obj.GetNumberOfMappings ()| -  Get number of mapppings.

\item  \verb|string = obj.GetAttributeName (int index)| -  Get the attribute name at the given index.

\item  \verb|string = obj.GetArrayName (int index)| -  Get the array name at the given index.

\item  \verb|int = obj.GetFieldAssociation (int index)| -  Get the field association at the given index.

\item  \verb|int = obj.GetComponent (int index)| -  Get the component no. at the given index.

\item  \verb|int = obj.GetTextureUnit (int index)| -  Get the component no. at the given index.

\end{itemize}
