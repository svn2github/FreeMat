\section{vtkGraphToGlyphs}

\subsection{Usage}

 Converts a vtkGraph to a vtkPolyData containing a glyph for each vertex.
 This assumes that the points
 of the graph have already been filled (perhaps by vtkGraphLayout).
 The glyphs will automatically be scaled to be the same size in screen
 coordinates. To do this the filter requires a pointer to the renderer
 into which the glyphs will be rendered.

To create an instance of class vtkGraphToGlyphs, simply
invoke its constructor as follows
\begin{verbatim}
  obj = vtkGraphToGlyphs
\end{verbatim}
\subsection{Methods}

The class vtkGraphToGlyphs has several methods that can be used.
  They are listed below.
Note that the documentation is translated automatically from the VTK sources,
and may not be completely intelligible.  When in doubt, consult the VTK website.
In the methods listed below, \verb|obj| is an instance of the vtkGraphToGlyphs class.
\begin{itemize}
\item  \verb|string = obj.GetClassName ()|

\item  \verb|int = obj.IsA (string name)|

\item  \verb|vtkGraphToGlyphs = obj.NewInstance ()|

\item  \verb|vtkGraphToGlyphs = obj.SafeDownCast (vtkObject o)|

\item  \verb|obj.SetGlyphType (int )| -  The glyph type, specified as one of the enumerated values in this
 class. VERTEX is a special glyph that cannot be scaled, but instead
 is rendered as an OpenGL vertex primitive. This may appear as a box
 or circle depending on the hardware.

\item  \verb|int = obj.GetGlyphType ()| -  The glyph type, specified as one of the enumerated values in this
 class. VERTEX is a special glyph that cannot be scaled, but instead
 is rendered as an OpenGL vertex primitive. This may appear as a box
 or circle depending on the hardware.

\item  \verb|obj.SetFilled (bool )| -  Whether to fill the glyph, or to just render the outline.

\item  \verb|bool = obj.GetFilled ()| -  Whether to fill the glyph, or to just render the outline.

\item  \verb|obj.FilledOn ()| -  Whether to fill the glyph, or to just render the outline.

\item  \verb|obj.FilledOff ()| -  Whether to fill the glyph, or to just render the outline.

\item  \verb|obj.SetScreenSize (double )| -  Set the desired screen size of each glyph. If you are using scaling,
 this will be the size of the glyph when rendering an object with
 scaling value 1.0.

\item  \verb|double = obj.GetScreenSize ()| -  Set the desired screen size of each glyph. If you are using scaling,
 this will be the size of the glyph when rendering an object with
 scaling value 1.0.

\item  \verb|obj.SetRenderer (vtkRenderer ren)| -  The renderer in which the glyphs will be placed.

\item  \verb|vtkRenderer = obj.GetRenderer ()| -  The renderer in which the glyphs will be placed.

\item  \verb|obj.SetScaling (bool b)| -  Whether to use the input array to process in order to scale the
 vertices.

\item  \verb|bool = obj.GetScaling ()| -  Whether to use the input array to process in order to scale the
 vertices.

\item  \verb|long = obj.GetMTime ()| -  The modified time of this filter.

\end{itemize}
