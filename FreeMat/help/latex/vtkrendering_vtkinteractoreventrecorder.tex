\section{vtkInteractorEventRecorder}

\subsection{Usage}

 vtkInteractorEventRecorder records all VTK events invoked from a
 vtkRenderWindowInteractor. The events are recorded to a
 file. vtkInteractorEventRecorder can also be used to play those events
 back and invoke them on an vtkRenderWindowInteractor. (Note: the events
 can also be played back from a file or string.)

 The format of the event file is simple. It is:
  EventName X Y ctrl shift keycode repeatCount keySym
 The format also allows ''\#'' comments.

To create an instance of class vtkInteractorEventRecorder, simply
invoke its constructor as follows
\begin{verbatim}
  obj = vtkInteractorEventRecorder
\end{verbatim}
\subsection{Methods}

The class vtkInteractorEventRecorder has several methods that can be used.
  They are listed below.
Note that the documentation is translated automatically from the VTK sources,
and may not be completely intelligible.  When in doubt, consult the VTK website.
In the methods listed below, \verb|obj| is an instance of the vtkInteractorEventRecorder class.
\begin{itemize}
\item  \verb|string = obj.GetClassName ()|

\item  \verb|int = obj.IsA (string name)|

\item  \verb|vtkInteractorEventRecorder = obj.NewInstance ()|

\item  \verb|vtkInteractorEventRecorder = obj.SafeDownCast (vtkObject o)|

\item  \verb|obj.SetEnabled (int )|

\item  \verb|obj.SetInteractor (vtkRenderWindowInteractor iren)|

\item  \verb|obj.SetFileName (string )| -  Set/Get the name of a file events should be written to/from.

\item  \verb|string = obj.GetFileName ()| -  Set/Get the name of a file events should be written to/from.

\item  \verb|obj.Record ()| -  Invoke this method to begin recording events. The events will be
 recorded to the filename indicated.

\item  \verb|obj.Play ()| -  Invoke this method to begin playing events from the current position.
 The events will be played back from the filename indicated.

\item  \verb|obj.Stop ()| -  Invoke this method to stop recording/playing events.

\item  \verb|obj.Rewind ()| -  Rewind to the beginning of the file.

\item  \verb|obj.SetReadFromInputString (int )| -  Enable reading from an InputString as compared to the default
 behavior, which is to read from a file.

\item  \verb|int = obj.GetReadFromInputString ()| -  Enable reading from an InputString as compared to the default
 behavior, which is to read from a file.

\item  \verb|obj.ReadFromInputStringOn ()| -  Enable reading from an InputString as compared to the default
 behavior, which is to read from a file.

\item  \verb|obj.ReadFromInputStringOff ()| -  Enable reading from an InputString as compared to the default
 behavior, which is to read from a file.

\item  \verb|obj.SetInputString (string )| -  Set/Get the string to read from.

\item  \verb|string = obj.GetInputString ()| -  Set/Get the string to read from.

\end{itemize}
