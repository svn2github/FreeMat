\section{vtkOpenGLExtensionManager}

\subsection{Usage}


 vtkOpenGLExtensionManager acts as an interface to OpenGL extensions.  It
 provides methods to query OpenGL extensions on the current or a given
 render window and to load extension function pointers.  Currently does
 not support GLU extensions since the GLU library is not linked to VTK.

 Before using vtkOpenGLExtensionManager, an OpenGL context must be created.
 This is generally done with a vtkRenderWindow.  Note that simply creating
 the vtkRenderWindow is not sufficient.  Usually you have to call Render
 before the actual OpenGL context is created.  You can specify the
 RenderWindow with the SetRenderWindow method.
 \begin{verbatim}
 vtkOpenGLExtensionManager *extensions = vtkOpenGLExtensionManager::New();
 extensions->SetRenderWindow(renwin);
 \end{verbatim}
 If no vtkRenderWindow is specified, the current OpenGL context (if any)
 is used.

 Generally speaking, when using OpenGL extensions, you will need an
 vtkOpenGLExtensionManager and the prototypes defined in vtkgl.h.
 \begin{verbatim}
 \#include ''vtkOpenGLExtensionManager.h''
 \#include ''vtkgl.h''
 \end{verbatim}
 The vtkgl.h include file contains all the constants and function
 pointers required for using OpenGL extensions in a portable and
 namespace safe way.  vtkgl.h is built from parsed glext.h, glxext.h, and
 wglext.h files.  Snapshots of these files are distributed with VTK,
 but you can also set CMake options to use other files.

 To use an OpenGL extension, you first need to make an instance of
 vtkOpenGLExtensionManager and give it a vtkRenderWindow.  You can then
 query the vtkOpenGLExtensionManager to see if the extension is supported
 with the ExtensionSupported method.  Valid names for extensions are
 given in the OpenGL extension registry at
 http://www.opengl.org/registry/ .
 You can also grep vtkgl.h (which will be in the binary build directory
 if VTK is not installed) for appropriate names.  There are also
 special extensions GL\_VERSION\_X\_X (where X\_X is replaced with a major
 and minor version, respectively) which contain all the constants and
 functions for OpenGL versions for which the gl.h header file is of an
 older version than the driver.

 \begin{verbatim}
 if (   !extensions->ExtensionSupported(''GL\_VERSION\_1\_2'')
     || !extensions->ExtensionSupported(''GL\_ARB\_multitexture'') ) {
   {
   vtkErrorMacro(''Required extensions not supported!'');
   }
 \end{verbatim}

 Once you have verified that the extensions you want exist, before you
 use them you have to load them with the LoadExtension method.

 \begin{verbatim}
 extensions->LoadExtension(''GL\_VERSION\_1\_2'');
 extensions->LoadExtension(''GL\_ARB\_multitexture'');
 \end{verbatim}

 Alternatively, you can use the LoadSupportedExtension method, which checks
 whether the requested extension is supported and, if so, loads it. The
 LoadSupportedExtension method will not raise any errors or warnings if it
 fails, so it is important for callers to pay attention to the return value.

 \begin{verbatim}
 if (   extensions->LoadSupportedExtension(''GL\_VERSION\_1\_2'')
     \&\& extensions->LoadSupportedExtension(''GL\_ARB\_multitexture'') ) {
   {
   vtkgl::ActiveTexture(vtkgl::TEXTURE0\_ARB);
   }
 else
   {
   vtkErrorMacro(''Required extensions could not be loaded!'');
   }
 \end{verbatim}

 Once you have queried and loaded all of the extensions you need, you can
 delete the vtkOpenGLExtensionManager.  To use a constant of an extension,
 simply replace the ''GL\_'' prefix with ''vtkgl::''.  Likewise, replace the
 ''gl'' prefix of functions with ''vtkgl::''.  In rare cases, an extension will
 add a type. In this case, add vtkgl:: to the type (i.e. vtkgl::GLchar).

 \begin{verbatim}
 extensions->Delete();
 ...
 vtkgl::ActiveTexture(vtkgl::TEXTURE0\_ARB);
 \end{verbatim}

 For wgl extensions, replace the ''WGL\_'' and ''wgl'' prefixes with
 ''vtkwgl::''.  For glX extensions, replace the ''GLX\_'' and ''glX'' prefixes
 with ''vtkglX::''.


To create an instance of class vtkOpenGLExtensionManager, simply
invoke its constructor as follows
\begin{verbatim}
  obj = vtkOpenGLExtensionManager
\end{verbatim}
\subsection{Methods}

The class vtkOpenGLExtensionManager has several methods that can be used.
  They are listed below.
Note that the documentation is translated automatically from the VTK sources,
and may not be completely intelligible.  When in doubt, consult the VTK website.
In the methods listed below, \verb|obj| is an instance of the vtkOpenGLExtensionManager class.
\begin{itemize}
\item  \verb|string = obj.GetClassName ()|

\item  \verb|int = obj.IsA (string name)|

\item  \verb|vtkOpenGLExtensionManager = obj.NewInstance ()|

\item  \verb|vtkOpenGLExtensionManager = obj.SafeDownCast (vtkObject o)|

\item  \verb|vtkRenderWindow = obj.GetRenderWindow ()| -  Set/Get the render window to query extensions on.  If set to null,
 justs queries the current render window.

\item  \verb|obj.SetRenderWindow (vtkRenderWindow renwin)| -  Set/Get the render window to query extensions on.  If set to null,
 justs queries the current render window.

\item  \verb|obj.Update ()| -  Updates the extensions string.

\item  \verb|string = obj.GetExtensionsString ()| -  Returns a string listing all available extensions.  Call Update first
 to validate this string.

\item  \verb|int = obj.ExtensionSupported (string name)| -  Returns true if the extension is supported, false otherwise.

\item  \verb|obj.LoadExtension (string name)| -  Loads all the functions associated with the given extension into the
 appropriate static members of vtkgl. This method emits a warning if the
 requested extension is not supported. It emits an error if the extension
 does not load successfully.

\item  \verb|int = obj.LoadSupportedExtension (string name)| -  Returns true if the extension is supported and loaded successfully,
 false otherwise. This method will ''fail silently/gracefully'' if the
 extension is not supported or does not load properly. It emits neither
 warnings nor errors. It is up to the caller to determine if the
 extension loaded properly by paying attention to the return value.

\item  \verb|obj.LoadCorePromotedExtension (string name)| -  Loads all the functions associated with the given core-promoted extension
 into the appropriate static members of vtkgl associated with the OpenGL
 version that promoted the extension as a core feature. This method emits a
 warning if the requested extension is not supported. It emits an error if
 the extension does not load successfully.

 For instance, extension GL\_ARB\_multitexture was promoted as a core
 feature into OpenGL 1.3. An implementation that uses this
 feature has to (IN THIS ORDER), check if OpenGL 1.3 is supported
 with ExtensionSupported(''GL\_VERSION\_1\_3''), if true, load the extension
 with LoadExtension(''GL\_VERSION\_1\_3''). If false, test for the extension
 with ExtensionSupported(''GL\_ARB\_multitexture''),if true load the extension
 with this method LoadCorePromotedExtension(''GL\_ARB\_multitexture'').
 If any of those loading stage succeeded, use vtgl::ActiveTexture() in
 any case, NOT vtgl::ActiveTextureARB().
 This method avoids the use of if statements everywhere in implementations
 using core-promoted extensions.
 Without this method, the implementation code should look like:
 \begin{verbatim}
 int opengl\_1\_3=extensions->ExtensionSupported(''GL\_VERSION\_1\_3'');
 if(opengl\_1\_3)
 {
   extensions->LoadExtension(''GL\_VERSION\_1\_3'');
 }
 else
 {
  if(extensions->ExtensionSupported(''GL\_ARB\_multitexture''))
  {
   extensions->LoadCorePromotedExtension(''GL\_ARB\_multitexture'');
  }
  else
  {
   vtkErrorMacro(''Required multitexture feature is not supported!'');
  }
 }
 ...
 if(opengl\_1\_3)
 {
  vtkgl::ActiveTexture(vtkgl::TEXTURE0)
 }
 else
 {
  vtkgl::ActiveTextureARB(vtkgl::TEXTURE0\_ARB)
 }
 \end{verbatim}
 Thanks to this method, the code looks like:
 \begin{verbatim}
 int opengl\_1\_3=extensions->ExtensionSupported(''GL\_VERSION\_1\_3'');
 if(opengl\_1\_3)
 {
   extensions->LoadExtension(''GL\_VERSION\_1\_3'');
 }
 else
 {
  if(extensions->ExtensionSupported(''GL\_ARB\_multitexture''))
  {
   extensions->LoadCorePromotedExtension(''GL\_ARB\_multitexture'');
  }
  else
  {
   vtkErrorMacro(''Required multitexture feature is not supported!'');
  }
 }
 ...
 vtkgl::ActiveTexture(vtkgl::TEXTURE0);
 \end{verbatim}

\end{itemize}
