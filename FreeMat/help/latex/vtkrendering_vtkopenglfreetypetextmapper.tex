\section{vtkOpenGLFreeTypeTextMapper}

\subsection{Usage}

 vtkOpenGLFreeTypeTextMapper provides 2D text annotation support for VTK
 using the FreeType and FTGL libraries. Normally the user should use 
 vtktextMapper which in turn will use this class.

To create an instance of class vtkOpenGLFreeTypeTextMapper, simply
invoke its constructor as follows
\begin{verbatim}
  obj = vtkOpenGLFreeTypeTextMapper
\end{verbatim}
\subsection{Methods}

The class vtkOpenGLFreeTypeTextMapper has several methods that can be used.
  They are listed below.
Note that the documentation is translated automatically from the VTK sources,
and may not be completely intelligible.  When in doubt, consult the VTK website.
In the methods listed below, \verb|obj| is an instance of the vtkOpenGLFreeTypeTextMapper class.
\begin{itemize}
\item  \verb|string = obj.GetClassName ()|

\item  \verb|int = obj.IsA (string name)|

\item  \verb|vtkOpenGLFreeTypeTextMapper = obj.NewInstance ()|

\item  \verb|vtkOpenGLFreeTypeTextMapper = obj.SafeDownCast (vtkObject o)|

\item  \verb|obj.RenderOverlay (vtkViewport viewport, vtkActor2D actor)| -  Actally draw the text.

\item  \verb|obj.ReleaseGraphicsResources (vtkWindow )| -  Release any graphics resources that are being consumed by this actor.
 The parameter window could be used to determine which graphic
 resources to release.

\item  \verb|obj.GetSize (vtkViewport viewport, int size[2])| -  What is the size of the rectangle required to draw this
 mapper ?

\end{itemize}
