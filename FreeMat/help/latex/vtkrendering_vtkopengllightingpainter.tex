\section{vtkOpenGLLightingPainter}

\subsection{Usage}

 This painter manages lighting.
 Ligting is disabled when rendering points/lines and no normals are present
 or rendering Polygons/TStrips and representation is points and no normals 
 are present.

To create an instance of class vtkOpenGLLightingPainter, simply
invoke its constructor as follows
\begin{verbatim}
  obj = vtkOpenGLLightingPainter
\end{verbatim}
\subsection{Methods}

The class vtkOpenGLLightingPainter has several methods that can be used.
  They are listed below.
Note that the documentation is translated automatically from the VTK sources,
and may not be completely intelligible.  When in doubt, consult the VTK website.
In the methods listed below, \verb|obj| is an instance of the vtkOpenGLLightingPainter class.
\begin{itemize}
\item  \verb|string = obj.GetClassName ()|

\item  \verb|int = obj.IsA (string name)|

\item  \verb|vtkOpenGLLightingPainter = obj.NewInstance ()|

\item  \verb|vtkOpenGLLightingPainter = obj.SafeDownCast (vtkObject o)|

\item  \verb|double = obj.GetTimeToDraw ()| -  This painter overrides GetTimeToDraw() to never pass the request to the
 delegate. This is done since this class may propagate a single render
 request multiple times to the delegate. In that case the time accumulation
 responsibility is borne by the painter causing the multiple rendering
 requests i.e. this painter itself.

\end{itemize}
