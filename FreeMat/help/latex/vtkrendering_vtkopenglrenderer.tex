\section{vtkOpenGLRenderer}

\subsection{Usage}

 vtkOpenGLRenderer is a concrete implementation of the abstract class
 vtkRenderer. vtkOpenGLRenderer interfaces to the OpenGL graphics library.

To create an instance of class vtkOpenGLRenderer, simply
invoke its constructor as follows
\begin{verbatim}
  obj = vtkOpenGLRenderer
\end{verbatim}
\subsection{Methods}

The class vtkOpenGLRenderer has several methods that can be used.
  They are listed below.
Note that the documentation is translated automatically from the VTK sources,
and may not be completely intelligible.  When in doubt, consult the VTK website.
In the methods listed below, \verb|obj| is an instance of the vtkOpenGLRenderer class.
\begin{itemize}
\item  \verb|string = obj.GetClassName ()|

\item  \verb|int = obj.IsA (string name)|

\item  \verb|vtkOpenGLRenderer = obj.NewInstance ()|

\item  \verb|vtkOpenGLRenderer = obj.SafeDownCast (vtkObject o)|

\item  \verb|obj.DeviceRender (void )| -  Concrete open gl render method.

\item  \verb|obj.DeviceRenderTranslucentPolygonalGeometry ()| -  Render translucent polygonal geometry. Default implementation just call
 UpdateTranslucentPolygonalGeometry().
 Subclasses of vtkRenderer that can deal with depth peeling must
 override this method.

\item  \verb|obj.ClearLights (void )| -  Internal method temporarily removes lights before reloading them
 into graphics pipeline.

\item  \verb|obj.Clear (void )|

\item  \verb|int = obj.UpdateLights (void )| -  Ask lights to load themselves into graphics pipeline.

\item  \verb|int = obj.GetDepthPeelingHigherLayer ()| -  Is rendering at translucent geometry stage using depth peeling and
 rendering a layer other than the first one? (Boolean value)
 If so, the uniform variables UseTexture and Texture can be set.
 (Used by vtkOpenGLProperty or vtkOpenGLTexture)

\end{itemize}
