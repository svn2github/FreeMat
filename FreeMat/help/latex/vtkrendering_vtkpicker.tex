\section{vtkPicker}

\subsection{Usage}

 vtkPicker is used to select instances of vtkProp3D by shooting a ray 
 into a graphics window and intersecting with the actor's bounding box. 
 The ray is defined from a point defined in window (or pixel) coordinates, 
 and a point located from the camera's position.

 vtkPicker may return more than one vtkProp3D, since more than one bounding 
 box may be intersected. vtkPicker returns an unsorted list of props that
 were hit, and a list of the corresponding world points of the hits.
 For the vtkProp3D that is closest to the camera, vtkPicker returns the
 pick coordinates in world and untransformed mapper space, the prop itself,
 the data set, and the mapper.  For vtkPicker the closest prop is the one
 whose center point (i.e., center of bounding box) projected on the view
 ray is closest to the camera.  Subclasses of vtkPicker use other methods
 for computing the pick point.

To create an instance of class vtkPicker, simply
invoke its constructor as follows
\begin{verbatim}
  obj = vtkPicker
\end{verbatim}
\subsection{Methods}

The class vtkPicker has several methods that can be used.
  They are listed below.
Note that the documentation is translated automatically from the VTK sources,
and may not be completely intelligible.  When in doubt, consult the VTK website.
In the methods listed below, \verb|obj| is an instance of the vtkPicker class.
\begin{itemize}
\item  \verb|string = obj.GetClassName ()|

\item  \verb|int = obj.IsA (string name)|

\item  \verb|vtkPicker = obj.NewInstance ()|

\item  \verb|vtkPicker = obj.SafeDownCast (vtkObject o)|

\item  \verb|obj.SetTolerance (double )| -  Specify tolerance for performing pick operation. Tolerance is specified
 as fraction of rendering window size. (Rendering window size is measured
 across diagonal.)

\item  \verb|double = obj.GetTolerance ()| -  Specify tolerance for performing pick operation. Tolerance is specified
 as fraction of rendering window size. (Rendering window size is measured
 across diagonal.)

\item  \verb|double = obj. GetMapperPosition ()| -  Return position in mapper (i.e., non-transformed) coordinates of 
 pick point.

\item  \verb|vtkAbstractMapper3D = obj.GetMapper ()| -  Return mapper that was picked (if any).

\item  \verb|vtkDataSet = obj.GetDataSet ()| -  Get a pointer to the dataset that was picked (if any). If nothing 
 was picked then NULL is returned.

\item  \verb|vtkProp3DCollection = obj.GetProp3Ds ()| -  Return a collection of all the prop 3D's that were intersected
 by the pick ray. This collection is not sorted.

\item  \verb|vtkActorCollection = obj.GetActors ()| -  Return a collection of all the actors that were intersected.
 This collection is not sorted. (This is a convenience method
 to maintain backward compatibility.)

\item  \verb|vtkPoints = obj.GetPickedPositions ()| -  Return a list of the points the the actors returned by GetProp3Ds
 were intersected at. The order of this list will match the order of
 GetProp3Ds.

\item  \verb|int = obj.Pick (double selectionX, double selectionY, double selectionZ, vtkRenderer renderer)| -  Perform pick operation with selection point provided. Normally the 
 first two values for the selection point are x-y pixel coordinate, and
 the third value is =0. Return non-zero if something was successfully 
 picked.

\item  \verb|int = obj.Pick (double selectionPt[3], vtkRenderer ren)| -  Perform pick operation with selection point provided. Normally the first
 two values for the selection point are x-y pixel coordinate, and the
 third value is =0. Return non-zero if something was successfully picked.

\end{itemize}
