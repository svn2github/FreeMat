\section{vtkPolyDataMapper2D}

\subsection{Usage}

 vtkPolyDataMapper2D is a mapper that renders 3D polygonal data 
 (vtkPolyData) onto the 2D image plane (i.e., the renderer's viewport).
 By default, the 3D data is transformed into 2D data by ignoring the 
 z-coordinate of the 3D points in vtkPolyData, and taking the x-y values 
 as local display values (i.e., pixel coordinates). Alternatively, you
 can provide a vtkCoordinate object that will transform the data into
 local display coordinates (use the vtkCoordinate::SetCoordinateSystem()
 methods to indicate which coordinate system you are transforming the
 data from).

To create an instance of class vtkPolyDataMapper2D, simply
invoke its constructor as follows
\begin{verbatim}
  obj = vtkPolyDataMapper2D
\end{verbatim}
\subsection{Methods}

The class vtkPolyDataMapper2D has several methods that can be used.
  They are listed below.
Note that the documentation is translated automatically from the VTK sources,
and may not be completely intelligible.  When in doubt, consult the VTK website.
In the methods listed below, \verb|obj| is an instance of the vtkPolyDataMapper2D class.
\begin{itemize}
\item  \verb|string = obj.GetClassName ()|

\item  \verb|int = obj.IsA (string name)|

\item  \verb|vtkPolyDataMapper2D = obj.NewInstance ()|

\item  \verb|vtkPolyDataMapper2D = obj.SafeDownCast (vtkObject o)|

\item  \verb|obj.SetInput (vtkPolyData in)| -  Set the input to the mapper.  

\item  \verb|vtkPolyData = obj.GetInput ()| -  Set the input to the mapper.  

\item  \verb|obj.SetLookupTable (vtkScalarsToColors lut)| -  Specify a lookup table for the mapper to use.

\item  \verb|vtkScalarsToColors = obj.GetLookupTable ()| -  Specify a lookup table for the mapper to use.

\item  \verb|obj.CreateDefaultLookupTable ()| -  Create default lookup table. Generally used to create one when none
 is available with the scalar data.

\item  \verb|obj.SetScalarVisibility (int )| -  Turn on/off flag to control whether scalar data is used to color objects.

\item  \verb|int = obj.GetScalarVisibility ()| -  Turn on/off flag to control whether scalar data is used to color objects.

\item  \verb|obj.ScalarVisibilityOn ()| -  Turn on/off flag to control whether scalar data is used to color objects.

\item  \verb|obj.ScalarVisibilityOff ()| -  Turn on/off flag to control whether scalar data is used to color objects.

\item  \verb|obj.SetColorMode (int )| -  Control how the scalar data is mapped to colors.  By default
 (ColorModeToDefault), unsigned char scalars are treated as colors, and
 NOT mapped through the lookup table, while everything else is. Setting
 ColorModeToMapScalars means that all scalar data will be mapped through
 the lookup table.  (Note that for multi-component scalars, the
 particular component to use for mapping can be specified using the
 ColorByArrayComponent() method.)

\item  \verb|int = obj.GetColorMode ()| -  Control how the scalar data is mapped to colors.  By default
 (ColorModeToDefault), unsigned char scalars are treated as colors, and
 NOT mapped through the lookup table, while everything else is. Setting
 ColorModeToMapScalars means that all scalar data will be mapped through
 the lookup table.  (Note that for multi-component scalars, the
 particular component to use for mapping can be specified using the
 ColorByArrayComponent() method.)

\item  \verb|obj.SetColorModeToDefault ()| -  Control how the scalar data is mapped to colors.  By default
 (ColorModeToDefault), unsigned char scalars are treated as colors, and
 NOT mapped through the lookup table, while everything else is. Setting
 ColorModeToMapScalars means that all scalar data will be mapped through
 the lookup table.  (Note that for multi-component scalars, the
 particular component to use for mapping can be specified using the
 ColorByArrayComponent() method.)

\item  \verb|obj.SetColorModeToMapScalars ()| -  Control how the scalar data is mapped to colors.  By default
 (ColorModeToDefault), unsigned char scalars are treated as colors, and
 NOT mapped through the lookup table, while everything else is. Setting
 ColorModeToMapScalars means that all scalar data will be mapped through
 the lookup table.  (Note that for multi-component scalars, the
 particular component to use for mapping can be specified using the
 ColorByArrayComponent() method.)

\item  \verb|string = obj.GetColorModeAsString ()| -  Return the method of coloring scalar data.

\item  \verb|obj.SetUseLookupTableScalarRange (int )| -  Control whether the mapper sets the lookuptable range based on its
 own ScalarRange, or whether it will use the LookupTable ScalarRange
 regardless of it's own setting. By default the Mapper is allowed to set
 the LookupTable range, but users who are sharing LookupTables between
 mappers/actors will probably wish to force the mapper to use the
 LookupTable unchanged.

\item  \verb|int = obj.GetUseLookupTableScalarRange ()| -  Control whether the mapper sets the lookuptable range based on its
 own ScalarRange, or whether it will use the LookupTable ScalarRange
 regardless of it's own setting. By default the Mapper is allowed to set
 the LookupTable range, but users who are sharing LookupTables between
 mappers/actors will probably wish to force the mapper to use the
 LookupTable unchanged.

\item  \verb|obj.UseLookupTableScalarRangeOn ()| -  Control whether the mapper sets the lookuptable range based on its
 own ScalarRange, or whether it will use the LookupTable ScalarRange
 regardless of it's own setting. By default the Mapper is allowed to set
 the LookupTable range, but users who are sharing LookupTables between
 mappers/actors will probably wish to force the mapper to use the
 LookupTable unchanged.

\item  \verb|obj.UseLookupTableScalarRangeOff ()| -  Control whether the mapper sets the lookuptable range based on its
 own ScalarRange, or whether it will use the LookupTable ScalarRange
 regardless of it's own setting. By default the Mapper is allowed to set
 the LookupTable range, but users who are sharing LookupTables between
 mappers/actors will probably wish to force the mapper to use the
 LookupTable unchanged.

\item  \verb|obj.SetScalarRange (double , double )| -  Specify range in terms of scalar minimum and maximum (smin,smax). These
 values are used to map scalars into lookup table. Has no effect when
 UseLookupTableScalarRange is true.

\item  \verb|obj.SetScalarRange (double  a[2])| -  Specify range in terms of scalar minimum and maximum (smin,smax). These
 values are used to map scalars into lookup table. Has no effect when
 UseLookupTableScalarRange is true.

\item  \verb|double = obj. GetScalarRange ()| -  Specify range in terms of scalar minimum and maximum (smin,smax). These
 values are used to map scalars into lookup table. Has no effect when
 UseLookupTableScalarRange is true.

\item  \verb|obj.SetScalarMode (int )| -  Control how the filter works with scalar point data and cell attribute
 data.  By default (ScalarModeToDefault), the filter will use point data,
 and if no point data is available, then cell data is used. Alternatively
 you can explicitly set the filter to use point data
 (ScalarModeToUsePointData) or cell data (ScalarModeToUseCellData).
 You can also choose to get the scalars from an array in point field
 data (ScalarModeToUsePointFieldData) or cell field data
 (ScalarModeToUseCellFieldData).  If scalars are coming from a field
 data array, you must call ColorByArrayComponent before you call
 GetColors.

\item  \verb|int = obj.GetScalarMode ()| -  Control how the filter works with scalar point data and cell attribute
 data.  By default (ScalarModeToDefault), the filter will use point data,
 and if no point data is available, then cell data is used. Alternatively
 you can explicitly set the filter to use point data
 (ScalarModeToUsePointData) or cell data (ScalarModeToUseCellData).
 You can also choose to get the scalars from an array in point field
 data (ScalarModeToUsePointFieldData) or cell field data
 (ScalarModeToUseCellFieldData).  If scalars are coming from a field
 data array, you must call ColorByArrayComponent before you call
 GetColors.

\item  \verb|obj.SetScalarModeToDefault ()| -  Control how the filter works with scalar point data and cell attribute
 data.  By default (ScalarModeToDefault), the filter will use point data,
 and if no point data is available, then cell data is used. Alternatively
 you can explicitly set the filter to use point data
 (ScalarModeToUsePointData) or cell data (ScalarModeToUseCellData).
 You can also choose to get the scalars from an array in point field
 data (ScalarModeToUsePointFieldData) or cell field data
 (ScalarModeToUseCellFieldData).  If scalars are coming from a field
 data array, you must call ColorByArrayComponent before you call
 GetColors.

\item  \verb|obj.SetScalarModeToUsePointData ()| -  Control how the filter works with scalar point data and cell attribute
 data.  By default (ScalarModeToDefault), the filter will use point data,
 and if no point data is available, then cell data is used. Alternatively
 you can explicitly set the filter to use point data
 (ScalarModeToUsePointData) or cell data (ScalarModeToUseCellData).
 You can also choose to get the scalars from an array in point field
 data (ScalarModeToUsePointFieldData) or cell field data
 (ScalarModeToUseCellFieldData).  If scalars are coming from a field
 data array, you must call ColorByArrayComponent before you call
 GetColors.

\item  \verb|obj.SetScalarModeToUseCellData ()| -  Control how the filter works with scalar point data and cell attribute
 data.  By default (ScalarModeToDefault), the filter will use point data,
 and if no point data is available, then cell data is used. Alternatively
 you can explicitly set the filter to use point data
 (ScalarModeToUsePointData) or cell data (ScalarModeToUseCellData).
 You can also choose to get the scalars from an array in point field
 data (ScalarModeToUsePointFieldData) or cell field data
 (ScalarModeToUseCellFieldData).  If scalars are coming from a field
 data array, you must call ColorByArrayComponent before you call
 GetColors.

\item  \verb|obj.SetScalarModeToUsePointFieldData ()| -  Control how the filter works with scalar point data and cell attribute
 data.  By default (ScalarModeToDefault), the filter will use point data,
 and if no point data is available, then cell data is used. Alternatively
 you can explicitly set the filter to use point data
 (ScalarModeToUsePointData) or cell data (ScalarModeToUseCellData).
 You can also choose to get the scalars from an array in point field
 data (ScalarModeToUsePointFieldData) or cell field data
 (ScalarModeToUseCellFieldData).  If scalars are coming from a field
 data array, you must call ColorByArrayComponent before you call
 GetColors.

\item  \verb|obj.SetScalarModeToUseCellFieldData ()| -  Control how the filter works with scalar point data and cell attribute
 data.  By default (ScalarModeToDefault), the filter will use point data,
 and if no point data is available, then cell data is used. Alternatively
 you can explicitly set the filter to use point data
 (ScalarModeToUsePointData) or cell data (ScalarModeToUseCellData).
 You can also choose to get the scalars from an array in point field
 data (ScalarModeToUsePointFieldData) or cell field data
 (ScalarModeToUseCellFieldData).  If scalars are coming from a field
 data array, you must call ColorByArrayComponent before you call
 GetColors.

\item  \verb|obj.ColorByArrayComponent (int arrayNum, int component)| -  Choose which component of which field data array to color by.

\item  \verb|obj.ColorByArrayComponent (string arrayName, int component)| -  Choose which component of which field data array to color by.

\item  \verb|string = obj.GetArrayName ()| -  Get the array name or number and component to color by.

\item  \verb|int = obj.GetArrayId ()| -  Get the array name or number and component to color by.

\item  \verb|int = obj.GetArrayAccessMode ()| -  Get the array name or number and component to color by.

\item  \verb|int = obj.GetArrayComponent ()| -  Overload standard modified time function. If lookup table is modified,
 then this object is modified as well.

\item  \verb|long = obj.GetMTime ()| -  Overload standard modified time function. If lookup table is modified,
 then this object is modified as well.

\item  \verb|obj.SetTransformCoordinate (vtkCoordinate )| -  Specify a vtkCoordinate object to be used to transform the vtkPolyData
 point coordinates. By default (no vtkCoordinate specified), the point 
 coordinates are taken as local display coordinates.

\item  \verb|vtkCoordinate = obj.GetTransformCoordinate ()| -  Specify a vtkCoordinate object to be used to transform the vtkPolyData
 point coordinates. By default (no vtkCoordinate specified), the point 
 coordinates are taken as local display coordinates.

\item  \verb|vtkUnsignedCharArray = obj.MapScalars (double alpha)| -  Map the scalars (if there are any scalars and ScalarVisibility is on)
 through the lookup table, returning an unsigned char RGBA array. This is
 typically done as part of the rendering process. The alpha parameter 
 allows the blending of the scalars with an additional alpha (typically
 which comes from a vtkActor, etc.)

\item  \verb|obj.ShallowCopy (vtkAbstractMapper m)| -  Make a shallow copy of this mapper.

\end{itemize}
