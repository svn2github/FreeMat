\section{vtkProperty}

\subsection{Usage}

 vtkProperty is an object that represents lighting and other surface
 properties of a geometric object. The primary properties that can be 
 set are colors (overall, ambient, diffuse, specular, and edge color);
 specular power; opacity of the object; the representation of the
 object (points, wireframe, or surface); and the shading method to be 
 used (flat, Gouraud, and Phong). Also, some special graphics features
 like backface properties can be set and manipulated with this object.

To create an instance of class vtkProperty, simply
invoke its constructor as follows
\begin{verbatim}
  obj = vtkProperty
\end{verbatim}
\subsection{Methods}

The class vtkProperty has several methods that can be used.
  They are listed below.
Note that the documentation is translated automatically from the VTK sources,
and may not be completely intelligible.  When in doubt, consult the VTK website.
In the methods listed below, \verb|obj| is an instance of the vtkProperty class.
\begin{itemize}
\item  \verb|string = obj.GetClassName ()|

\item  \verb|int = obj.IsA (string name)|

\item  \verb|vtkProperty = obj.NewInstance ()|

\item  \verb|vtkProperty = obj.SafeDownCast (vtkObject o)|

\item  \verb|obj.DeepCopy (vtkProperty p)| -  Assign one property to another. 

\item  \verb|obj.Render (vtkActor , vtkRenderer )| -  This method causes the property to set up whatever is required for
 its instance variables. This is actually handled by a subclass of
 vtkProperty, which is created automatically. This
 method includes the invoking actor as an argument which can
 be used by property devices that require the actor.

\item  \verb|obj.BackfaceRender (vtkActor , vtkRenderer )| -  This method renders the property as a backface property. TwoSidedLighting
 must be turned off to see any backface properties. Note that only
 colors and opacity are used for backface properties. Other properties
 such as Representation, Culling are specified by the Property.

\item  \verb|bool = obj.GetLighting ()| -  Set/Get lighting flag for an object. Initial value is true.

\item  \verb|obj.SetLighting (bool )| -  Set/Get lighting flag for an object. Initial value is true.

\item  \verb|obj.LightingOn ()| -  Set/Get lighting flag for an object. Initial value is true.

\item  \verb|obj.LightingOff ()| -  Set/Get lighting flag for an object. Initial value is true.

\item  \verb|obj.SetInterpolation (int )| -  Set the shading interpolation method for an object.

\item  \verb|int = obj.GetInterpolationMinValue ()| -  Set the shading interpolation method for an object.

\item  \verb|int = obj.GetInterpolationMaxValue ()| -  Set the shading interpolation method for an object.

\item  \verb|int = obj.GetInterpolation ()| -  Set the shading interpolation method for an object.

\item  \verb|obj.SetInterpolationToFlat ()| -  Set the shading interpolation method for an object.

\item  \verb|obj.SetInterpolationToGouraud ()| -  Set the shading interpolation method for an object.

\item  \verb|obj.SetInterpolationToPhong ()| -  Set the shading interpolation method for an object.

\item  \verb|string = obj.GetInterpolationAsString ()| -  Set the shading interpolation method for an object.

\item  \verb|obj.SetRepresentation (int )| -  Control the surface geometry representation for the object.

\item  \verb|int = obj.GetRepresentationMinValue ()| -  Control the surface geometry representation for the object.

\item  \verb|int = obj.GetRepresentationMaxValue ()| -  Control the surface geometry representation for the object.

\item  \verb|int = obj.GetRepresentation ()| -  Control the surface geometry representation for the object.

\item  \verb|obj.SetRepresentationToPoints ()| -  Control the surface geometry representation for the object.

\item  \verb|obj.SetRepresentationToWireframe ()| -  Control the surface geometry representation for the object.

\item  \verb|obj.SetRepresentationToSurface ()| -  Control the surface geometry representation for the object.

\item  \verb|string = obj.GetRepresentationAsString ()| -  Control the surface geometry representation for the object.

\item  \verb|obj.SetColor (double r, double g, double b)| -  Set the color of the object. Has the side effect of setting the
 ambient diffuse and specular colors as well. This is basically
 a quick overall color setting method.

\item  \verb|obj.SetColor (double a[3])| -  Set the color of the object. Has the side effect of setting the
 ambient diffuse and specular colors as well. This is basically
 a quick overall color setting method.

\item  \verb|double = obj.GetColor ()| -  Set the color of the object. Has the side effect of setting the
 ambient diffuse and specular colors as well. This is basically
 a quick overall color setting method.

\item  \verb|obj.GetColor (double rgb[3])| -  Set the color of the object. Has the side effect of setting the
 ambient diffuse and specular colors as well. This is basically
 a quick overall color setting method.

\item  \verb|obj.SetAmbient (double )| -  Set/Get the ambient lighting coefficient.

\item  \verb|double = obj.GetAmbientMinValue ()| -  Set/Get the ambient lighting coefficient.

\item  \verb|double = obj.GetAmbientMaxValue ()| -  Set/Get the ambient lighting coefficient.

\item  \verb|double = obj.GetAmbient ()| -  Set/Get the ambient lighting coefficient.

\item  \verb|obj.SetDiffuse (double )| -  Set/Get the diffuse lighting coefficient.

\item  \verb|double = obj.GetDiffuseMinValue ()| -  Set/Get the diffuse lighting coefficient.

\item  \verb|double = obj.GetDiffuseMaxValue ()| -  Set/Get the diffuse lighting coefficient.

\item  \verb|double = obj.GetDiffuse ()| -  Set/Get the diffuse lighting coefficient.

\item  \verb|obj.SetSpecular (double )| -  Set/Get the specular lighting coefficient.

\item  \verb|double = obj.GetSpecularMinValue ()| -  Set/Get the specular lighting coefficient.

\item  \verb|double = obj.GetSpecularMaxValue ()| -  Set/Get the specular lighting coefficient.

\item  \verb|double = obj.GetSpecular ()| -  Set/Get the specular lighting coefficient.

\item  \verb|obj.SetSpecularPower (double )| -  Set/Get the specular power.

\item  \verb|double = obj.GetSpecularPowerMinValue ()| -  Set/Get the specular power.

\item  \verb|double = obj.GetSpecularPowerMaxValue ()| -  Set/Get the specular power.

\item  \verb|double = obj.GetSpecularPower ()| -  Set/Get the specular power.

\item  \verb|obj.SetOpacity (double )| -  Set/Get the object's opacity. 1.0 is totally opaque and 0.0 is completely
 transparent.

\item  \verb|double = obj.GetOpacityMinValue ()| -  Set/Get the object's opacity. 1.0 is totally opaque and 0.0 is completely
 transparent.

\item  \verb|double = obj.GetOpacityMaxValue ()| -  Set/Get the object's opacity. 1.0 is totally opaque and 0.0 is completely
 transparent.

\item  \verb|double = obj.GetOpacity ()| -  Set/Get the object's opacity. 1.0 is totally opaque and 0.0 is completely
 transparent.

\item  \verb|obj.SetAmbientColor (double , double , double )| -  Set/Get the ambient surface color. Not all renderers support separate
 ambient and diffuse colors. From a physical standpoint it really
 doesn't make too much sense to have both. For the rendering
 libraries that don't support both, the diffuse color is used.

\item  \verb|obj.SetAmbientColor (double  a[3])| -  Set/Get the ambient surface color. Not all renderers support separate
 ambient and diffuse colors. From a physical standpoint it really
 doesn't make too much sense to have both. For the rendering
 libraries that don't support both, the diffuse color is used.

\item  \verb|double = obj. GetAmbientColor ()| -  Set/Get the ambient surface color. Not all renderers support separate
 ambient and diffuse colors. From a physical standpoint it really
 doesn't make too much sense to have both. For the rendering
 libraries that don't support both, the diffuse color is used.

\item  \verb|obj.SetDiffuseColor (double , double , double )| -  Set/Get the diffuse surface color.

\item  \verb|obj.SetDiffuseColor (double  a[3])| -  Set/Get the diffuse surface color.

\item  \verb|double = obj. GetDiffuseColor ()| -  Set/Get the diffuse surface color.

\item  \verb|obj.SetSpecularColor (double , double , double )| -  Set/Get the specular surface color.

\item  \verb|obj.SetSpecularColor (double  a[3])| -  Set/Get the specular surface color.

\item  \verb|double = obj. GetSpecularColor ()| -  Set/Get the specular surface color.

\item  \verb|int = obj.GetEdgeVisibility ()| -  Turn on/off the visibility of edges. On some renderers it is
 possible to render the edges of geometric primitives separately
 from the interior.

\item  \verb|obj.SetEdgeVisibility (int )| -  Turn on/off the visibility of edges. On some renderers it is
 possible to render the edges of geometric primitives separately
 from the interior.

\item  \verb|obj.EdgeVisibilityOn ()| -  Turn on/off the visibility of edges. On some renderers it is
 possible to render the edges of geometric primitives separately
 from the interior.

\item  \verb|obj.EdgeVisibilityOff ()| -  Turn on/off the visibility of edges. On some renderers it is
 possible to render the edges of geometric primitives separately
 from the interior.

\item  \verb|obj.SetEdgeColor (double , double , double )| -  Set/Get the color of primitive edges (if edge visibility is enabled).

\item  \verb|obj.SetEdgeColor (double  a[3])| -  Set/Get the color of primitive edges (if edge visibility is enabled).

\item  \verb|double = obj. GetEdgeColor ()| -  Set/Get the color of primitive edges (if edge visibility is enabled).

\item  \verb|obj.SetLineWidth (float )| -  Set/Get the width of a Line. The width is expressed in screen units.
 This is only implemented for OpenGL. The default is 1.0.

\item  \verb|float = obj.GetLineWidthMinValue ()| -  Set/Get the width of a Line. The width is expressed in screen units.
 This is only implemented for OpenGL. The default is 1.0.

\item  \verb|float = obj.GetLineWidthMaxValue ()| -  Set/Get the width of a Line. The width is expressed in screen units.
 This is only implemented for OpenGL. The default is 1.0.

\item  \verb|float = obj.GetLineWidth ()| -  Set/Get the width of a Line. The width is expressed in screen units.
 This is only implemented for OpenGL. The default is 1.0.

\item  \verb|obj.SetLineStipplePattern (int )| -  Set/Get the stippling pattern of a Line, as a 16-bit binary pattern 
 (1 = pixel on, 0 = pixel off).
 This is only implemented for OpenGL. The default is 0xFFFF.

\item  \verb|int = obj.GetLineStipplePattern ()| -  Set/Get the stippling pattern of a Line, as a 16-bit binary pattern 
 (1 = pixel on, 0 = pixel off).
 This is only implemented for OpenGL. The default is 0xFFFF.

\item  \verb|obj.SetLineStippleRepeatFactor (int )| -  Set/Get the stippling repeat factor of a Line, which specifies how
 many times each bit in the pattern is to be repeated.
 This is only implemented for OpenGL. The default is 1.

\item  \verb|int = obj.GetLineStippleRepeatFactorMinValue ()| -  Set/Get the stippling repeat factor of a Line, which specifies how
 many times each bit in the pattern is to be repeated.
 This is only implemented for OpenGL. The default is 1.

\item  \verb|int = obj.GetLineStippleRepeatFactorMaxValue ()| -  Set/Get the stippling repeat factor of a Line, which specifies how
 many times each bit in the pattern is to be repeated.
 This is only implemented for OpenGL. The default is 1.

\item  \verb|int = obj.GetLineStippleRepeatFactor ()| -  Set/Get the stippling repeat factor of a Line, which specifies how
 many times each bit in the pattern is to be repeated.
 This is only implemented for OpenGL. The default is 1.

\item  \verb|obj.SetPointSize (float )| -  Set/Get the diameter of a point. The size is expressed in screen units.
 This is only implemented for OpenGL. The default is 1.0.

\item  \verb|float = obj.GetPointSizeMinValue ()| -  Set/Get the diameter of a point. The size is expressed in screen units.
 This is only implemented for OpenGL. The default is 1.0.

\item  \verb|float = obj.GetPointSizeMaxValue ()| -  Set/Get the diameter of a point. The size is expressed in screen units.
 This is only implemented for OpenGL. The default is 1.0.

\item  \verb|float = obj.GetPointSize ()| -  Set/Get the diameter of a point. The size is expressed in screen units.
 This is only implemented for OpenGL. The default is 1.0.

\item  \verb|int = obj.GetBackfaceCulling ()| -  Turn on/off fast culling of polygons based on orientation of normal
 with respect to camera. If backface culling is on, polygons facing
 away from camera are not drawn.

\item  \verb|obj.SetBackfaceCulling (int )| -  Turn on/off fast culling of polygons based on orientation of normal
 with respect to camera. If backface culling is on, polygons facing
 away from camera are not drawn.

\item  \verb|obj.BackfaceCullingOn ()| -  Turn on/off fast culling of polygons based on orientation of normal
 with respect to camera. If backface culling is on, polygons facing
 away from camera are not drawn.

\item  \verb|obj.BackfaceCullingOff ()| -  Turn on/off fast culling of polygons based on orientation of normal
 with respect to camera. If backface culling is on, polygons facing
 away from camera are not drawn.

\item  \verb|int = obj.GetFrontfaceCulling ()| -  Turn on/off fast culling of polygons based on orientation of normal
 with respect to camera. If frontface culling is on, polygons facing
 towards camera are not drawn.

\item  \verb|obj.SetFrontfaceCulling (int )| -  Turn on/off fast culling of polygons based on orientation of normal
 with respect to camera. If frontface culling is on, polygons facing
 towards camera are not drawn.

\item  \verb|obj.FrontfaceCullingOn ()| -  Turn on/off fast culling of polygons based on orientation of normal
 with respect to camera. If frontface culling is on, polygons facing
 towards camera are not drawn.

\item  \verb|obj.FrontfaceCullingOff ()| -  Turn on/off fast culling of polygons based on orientation of normal
 with respect to camera. If frontface culling is on, polygons facing
 towards camera are not drawn.

\item  \verb|vtkXMLMaterial = obj.GetMaterial ()| -  Get the material representation used for shading. The material will be used
 only when shading is enabled.

\item  \verb|string = obj.GetMaterialName ()| -  Returns the name of the material currenly loaded, if any.

\item  \verb|obj.LoadMaterial (string name)| -  Load the material. The material can be the name of a
 built-on material or the filename for a VTK material XML description.

\item  \verb|obj.LoadMaterialFromString (string materialxml)| -  Load the material given the contents of the material file.

\item  \verb|obj.LoadMaterial (vtkXMLMaterial )| -  Load the material given the material representation.

\item  \verb|obj.SetShading (int )| -  Enable/Disable shading. When shading is enabled, the
 Material must be set.

\item  \verb|int = obj.GetShading ()| -  Enable/Disable shading. When shading is enabled, the
 Material must be set.

\item  \verb|obj.ShadingOn ()| -  Enable/Disable shading. When shading is enabled, the
 Material must be set.

\item  \verb|obj.ShadingOff ()| -  Enable/Disable shading. When shading is enabled, the
 Material must be set.

\item  \verb|vtkShaderProgram = obj.GetShaderProgram ()| -  Get the Shader program. If Material is not set/or not loaded properly,
 this will return null.

\item  \verb|obj.AddShaderVariable (string name, int numVars, int x)| -  Provide values to initialize shader variables.
 Useful to initialize shader variables that change over time
 (animation, GUI widgets inputs, etc. )
 -  name - hardware name of the uniform variable
 -  numVars - number of variables being set
 -  x - values

\item  \verb|obj.AddShaderVariable (string name, int numVars, float x)| -  Provide values to initialize shader variables.
 Useful to initialize shader variables that change over time
 (animation, GUI widgets inputs, etc. )
 -  name - hardware name of the uniform variable
 -  numVars - number of variables being set
 -  x - values

\item  \verb|obj.AddShaderVariable (string name, int numVars, double x)| -  Provide values to initialize shader variables.
 Useful to initialize shader variables that change over time
 (animation, GUI widgets inputs, etc. )
 -  name - hardware name of the uniform variable
 -  numVars - number of variables being set
 -  x - values

\item  \verb|obj.AddShaderVariable (string name, int v)| -  Methods to provide to add shader variables from tcl.

\item  \verb|obj.AddShaderVariable (string name, float v)| -  Methods to provide to add shader variables from tcl.

\item  \verb|obj.AddShaderVariable (string name, double v)| -  Methods to provide to add shader variables from tcl.

\item  \verb|obj.AddShaderVariable (string name, int v1, int v2)| -  Methods to provide to add shader variables from tcl.

\item  \verb|obj.AddShaderVariable (string name, float v1, float v2)| -  Methods to provide to add shader variables from tcl.

\item  \verb|obj.AddShaderVariable (string name, double v1, double v2)| -  Methods to provide to add shader variables from tcl.

\item  \verb|obj.AddShaderVariable (string name, int v1, int v2, int v3)| -  Methods to provide to add shader variables from tcl.

\item  \verb|obj.AddShaderVariable (string name, float v1, float v2, float v3)| -  Methods to provide to add shader variables from tcl.

\item  \verb|obj.AddShaderVariable (string name, double v1, double v2, double v3)| -  Set/Get the texture object to control rendering texture maps.  This will
 be a vtkTexture object. A property does not need to have an associated
 texture map and multiple properties can share one texture. Textures
 must be assigned unique names.

\item  \verb|obj.SetTexture (string name, vtkTexture texture)| -  Set/Get the texture object to control rendering texture maps.  This will
 be a vtkTexture object. A property does not need to have an associated
 texture map and multiple properties can share one texture. Textures
 must be assigned unique names.

\item  \verb|vtkTexture = obj.GetTexture (string name)| -  Set/Get the texture object to control rendering texture maps.  This will
 be a vtkTexture object. A property does not need to have an associated
 texture map and multiple properties can share one texture. Textures
 must be assigned unique names.

\item  \verb|obj.SetTexture (int unit, vtkTexture texture)| -  Set/Get the texture object to control rendering texture maps.  This will
 be a vtkTexture object. A property does not need to have an associated
 texture map and multiple properties can share one texture. Textures
 must be assigned unique names.

\item  \verb|vtkTexture = obj.GetTexture (int unit)| -  Set/Get the texture object to control rendering texture maps.  This will
 be a vtkTexture object. A property does not need to have an associated
 texture map and multiple properties can share one texture. Textures
 must be assigned unique names.

\item  \verb|obj.RemoveTexture (int unit)| -  Set/Get the texture object to control rendering texture maps.  This will
 be a vtkTexture object. A property does not need to have an associated
 texture map and multiple properties can share one texture. Textures
 must be assigned unique names.

\item  \verb|obj.RemoveTexture (string name)| -  Remove a texture from the collection. Note that the
 indices of all the subsquent textures, if any, will change.

\item  \verb|obj.RemoveAllTextures ()| -  Remove all the textures.

\item  \verb|int = obj.GetNumberOfTextures ()| -  Returns the number of textures in this property.

\item  \verb|obj.ReleaseGraphicsResources (vtkWindow win)| -  Release any graphics resources that are being consumed by this
 property. The parameter window could be used to determine which graphic
 resources to release.

\end{itemize}
