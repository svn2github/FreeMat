\section{vtkRenderedAreaPicker}

\subsection{Usage}

 Like vtkAreaPicker, this class picks all props within a selection area 
 on the screen. The difference is in implementation. This class uses
 graphics hardware to perform the test where the other uses software 
 bounding box/frustum intersection testing. 

 This picker is more conservative than vtkAreaPicker. It will reject 
 some objects that pass the bounding box test of vtkAreaPicker. This 
 will happen, for instance, when picking through a corner of the bounding
 box when the data set does not have any visible geometry in that corner.

To create an instance of class vtkRenderedAreaPicker, simply
invoke its constructor as follows
\begin{verbatim}
  obj = vtkRenderedAreaPicker
\end{verbatim}
\subsection{Methods}

The class vtkRenderedAreaPicker has several methods that can be used.
  They are listed below.
Note that the documentation is translated automatically from the VTK sources,
and may not be completely intelligible.  When in doubt, consult the VTK website.
In the methods listed below, \verb|obj| is an instance of the vtkRenderedAreaPicker class.
\begin{itemize}
\item  \verb|string = obj.GetClassName ()|

\item  \verb|int = obj.IsA (string name)|

\item  \verb|vtkRenderedAreaPicker = obj.NewInstance ()|

\item  \verb|vtkRenderedAreaPicker = obj.SafeDownCast (vtkObject o)|

\item  \verb|int = obj.AreaPick (double x0, double y0, double x1, double y1, vtkRenderer renderer)| -  Perform pick operation in volume behind the given screen coordinates.
 Props intersecting the selection frustum will be accesible via GetProp3D.
 GetPlanes returns a vtkImplicitFunciton suitable for vtkExtractGeometry.

\end{itemize}
