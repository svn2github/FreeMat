\section{vtkSurfaceLICPainter}

\subsection{Usage}

  vtkSurfaceLICPainter painter performs LIC on the surface of arbitrary
  geometry. Point vectors are used as the vector field for generating the LIC.
  The implementation is based on ''Image Space Based Visualization on Unstread
  Flow on Surfaces'' by Laramee, Jobard and Hauser appered in proceedings of 
  IEEE Visualization '03, pages 131-138.
  

To create an instance of class vtkSurfaceLICPainter, simply
invoke its constructor as follows
\begin{verbatim}
  obj = vtkSurfaceLICPainter
\end{verbatim}
\subsection{Methods}

The class vtkSurfaceLICPainter has several methods that can be used.
  They are listed below.
Note that the documentation is translated automatically from the VTK sources,
and may not be completely intelligible.  When in doubt, consult the VTK website.
In the methods listed below, \verb|obj| is an instance of the vtkSurfaceLICPainter class.
\begin{itemize}
\item  \verb|string = obj.GetClassName ()|

\item  \verb|int = obj.IsA (string name)|

\item  \verb|vtkSurfaceLICPainter = obj.NewInstance ()|

\item  \verb|vtkSurfaceLICPainter = obj.SafeDownCast (vtkObject o)|

\item  \verb|obj.ReleaseGraphicsResources (vtkWindow )| -  Release any graphics resources that are being consumed by this mapper.
 The parameter window could be used to determine which graphic
 resources to release. In this case, releases the display lists.

\item  \verb|vtkDataObject = obj.GetOutput ()| -  Get the output data object from this painter. 
 Overridden to pass the input points (or cells) vectors as the tcoords to
 the deletage painters. This is required by the internal GLSL shader
 programs used for generating LIC.

\item  \verb|obj.SetEnable (int )| -  Enable/Disable this painter.

\item  \verb|int = obj.GetEnable ()| -  Enable/Disable this painter.

\item  \verb|obj.EnableOn ()| -  Enable/Disable this painter.

\item  \verb|obj.EnableOff ()| -  Enable/Disable this painter.

\item  \verb|obj.SetInputArrayToProcess (int fieldAssociation, string name)| -  Set the vectors to used for applying LIC. By default point vectors are
 used. Arguments are same as those passed to
 vtkAlgorithm::SetInputArrayToProcess except the first 3 arguments i.e. idx,
 port, connection.

\item  \verb|obj.SetInputArrayToProcess (int fieldAssociation, int fieldAttributeType)| -  Set the vectors to used for applying LIC. By default point vectors are
 used. Arguments are same as those passed to
 vtkAlgorithm::SetInputArrayToProcess except the first 3 arguments i.e. idx,
 port, connection.

\item  \verb|obj.SetEnhancedLIC (int )| -  Enable/Disable enhanced LIC that improves image quality by increasing
 inter-streamline contrast while suppressing artifacts. Enhanced LIC
 performs two passes of LIC, with a 3x3 Laplacian high-pass filter in
 between that processes the output of pass \#1 LIC and forwards the result
 as the input 'noise' to pass \#2 LIC. This flag is automatically turned
 off during user interaction.

\item  \verb|int = obj.GetEnhancedLIC ()| -  Enable/Disable enhanced LIC that improves image quality by increasing
 inter-streamline contrast while suppressing artifacts. Enhanced LIC
 performs two passes of LIC, with a 3x3 Laplacian high-pass filter in
 between that processes the output of pass \#1 LIC and forwards the result
 as the input 'noise' to pass \#2 LIC. This flag is automatically turned
 off during user interaction.

\item  \verb|obj.EnhancedLICOn ()| -  Enable/Disable enhanced LIC that improves image quality by increasing
 inter-streamline contrast while suppressing artifacts. Enhanced LIC
 performs two passes of LIC, with a 3x3 Laplacian high-pass filter in
 between that processes the output of pass \#1 LIC and forwards the result
 as the input 'noise' to pass \#2 LIC. This flag is automatically turned
 off during user interaction.

\item  \verb|obj.EnhancedLICOff ()| -  Enable/Disable enhanced LIC that improves image quality by increasing
 inter-streamline contrast while suppressing artifacts. Enhanced LIC
 performs two passes of LIC, with a 3x3 Laplacian high-pass filter in
 between that processes the output of pass \#1 LIC and forwards the result
 as the input 'noise' to pass \#2 LIC. This flag is automatically turned
 off during user interaction.

\item  \verb|obj.SetNumberOfSteps (int )| -  Get/Set the number of integration steps in each direction.

\item  \verb|int = obj.GetNumberOfSteps ()| -  Get/Set the number of integration steps in each direction.

\item  \verb|obj.SetStepSize (double )| -  Get/Set the step size (in pixels).

\item  \verb|double = obj.GetStepSize ()| -  Get/Set the step size (in pixels).

\item  \verb|obj.SetLICIntensity (double )| -  Control the contribution of the LIC in the final output image.
 0.0 produces same result as disabling LIC alltogether, while 1.0 implies
 show LIC result alone.

\item  \verb|double = obj.GetLICIntensityMinValue ()| -  Control the contribution of the LIC in the final output image.
 0.0 produces same result as disabling LIC alltogether, while 1.0 implies
 show LIC result alone.

\item  \verb|double = obj.GetLICIntensityMaxValue ()| -  Control the contribution of the LIC in the final output image.
 0.0 produces same result as disabling LIC alltogether, while 1.0 implies
 show LIC result alone.

\item  \verb|double = obj.GetLICIntensity ()| -  Control the contribution of the LIC in the final output image.
 0.0 produces same result as disabling LIC alltogether, while 1.0 implies
 show LIC result alone.

\item  \verb|int = obj.GetRenderingPreparationSuccess ()| -  Check if the LIC process runs properly.

\item  \verb|int = obj.GetLICSuccess ()| -  Returns true is the rendering context supports extensions needed by this
 painter.

\end{itemize}
