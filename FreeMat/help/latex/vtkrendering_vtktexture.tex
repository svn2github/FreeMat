\section{vtkTexture}

\subsection{Usage}

 vtkTexture is an object that handles loading and binding of texture
 maps. It obtains its data from an input image data dataset type.
 Thus you can create visualization pipelines to read, process, and 
 construct textures. Note that textures will only work if texture
 coordinates are also defined, and if the rendering system supports 
 texture.

 Instances of vtkTexture are associated with actors via the actor's
 SetTexture() method. Actors can share texture maps (this is encouraged
 to save memory resources.) 

To create an instance of class vtkTexture, simply
invoke its constructor as follows
\begin{verbatim}
  obj = vtkTexture
\end{verbatim}
\subsection{Methods}

The class vtkTexture has several methods that can be used.
  They are listed below.
Note that the documentation is translated automatically from the VTK sources,
and may not be completely intelligible.  When in doubt, consult the VTK website.
In the methods listed below, \verb|obj| is an instance of the vtkTexture class.
\begin{itemize}
\item  \verb|string = obj.GetClassName ()|

\item  \verb|int = obj.IsA (string name)|

\item  \verb|vtkTexture = obj.NewInstance ()|

\item  \verb|vtkTexture = obj.SafeDownCast (vtkObject o)|

\item  \verb|obj.Render (vtkRenderer ren)| -  Renders a texture map. It first checks the object's modified time
 to make sure the texture maps Input is valid, then it invokes the 
 Load() method.

\item  \verb|obj.PostRender (vtkRenderer )| -  Cleans up after the texture rendering to restore the state of the
 graphics context.

\item  \verb|obj.ReleaseGraphicsResources (vtkWindow )| -  Release any graphics resources that are being consumed by this texture.
 The parameter window could be used to determine which graphic
 resources to release.

\item  \verb|obj.Load (vtkRenderer )| -  Abstract interface to renderer. Each concrete subclass of 
 vtkTexture will load its data into graphics system in response 
 to this method invocation.

\item  \verb|int = obj.GetRepeat ()| -  Turn on/off the repetition of the texture map when the texture
 coords extend beyond the [0,1] range.

\item  \verb|obj.SetRepeat (int )| -  Turn on/off the repetition of the texture map when the texture
 coords extend beyond the [0,1] range.

\item  \verb|obj.RepeatOn ()| -  Turn on/off the repetition of the texture map when the texture
 coords extend beyond the [0,1] range.

\item  \verb|obj.RepeatOff ()| -  Turn on/off the repetition of the texture map when the texture
 coords extend beyond the [0,1] range.

\item  \verb|int = obj.GetEdgeClamp ()| -  Turn on/off the clamping of the texture map when the texture
 coords extend beyond the [0,1] range.
 Only used when Repeat is off, and edge clamping is supported by
 the graphics card.

\item  \verb|obj.SetEdgeClamp (int )| -  Turn on/off the clamping of the texture map when the texture
 coords extend beyond the [0,1] range.
 Only used when Repeat is off, and edge clamping is supported by
 the graphics card.

\item  \verb|obj.EdgeClampOn ()| -  Turn on/off the clamping of the texture map when the texture
 coords extend beyond the [0,1] range.
 Only used when Repeat is off, and edge clamping is supported by
 the graphics card.

\item  \verb|obj.EdgeClampOff ()| -  Turn on/off the clamping of the texture map when the texture
 coords extend beyond the [0,1] range.
 Only used when Repeat is off, and edge clamping is supported by
 the graphics card.

\item  \verb|int = obj.GetInterpolate ()| -  Turn on/off linear interpolation of the texture map when rendering.

\item  \verb|obj.SetInterpolate (int )| -  Turn on/off linear interpolation of the texture map when rendering.

\item  \verb|obj.InterpolateOn ()| -  Turn on/off linear interpolation of the texture map when rendering.

\item  \verb|obj.InterpolateOff ()| -  Turn on/off linear interpolation of the texture map when rendering.

\item  \verb|obj.SetQuality (int )| -  Force texture quality to 16-bit or 32-bit.
 This might not be supported on all machines.

\item  \verb|int = obj.GetQuality ()| -  Force texture quality to 16-bit or 32-bit.
 This might not be supported on all machines.

\item  \verb|obj.SetQualityToDefault ()| -  Force texture quality to 16-bit or 32-bit.
 This might not be supported on all machines.

\item  \verb|obj.SetQualityTo16Bit ()| -  Force texture quality to 16-bit or 32-bit.
 This might not be supported on all machines.

\item  \verb|obj.SetQualityTo32Bit ()| -  Force texture quality to 16-bit or 32-bit.
 This might not be supported on all machines.

\item  \verb|int = obj.GetMapColorScalarsThroughLookupTable ()| -  Turn on/off the mapping of color scalars through the lookup table.
 The default is Off. If Off, unsigned char scalars will be used
 directly as texture. If On, scalars will be mapped through the
 lookup table to generate 4-component unsigned char scalars.
 This ivar does not affect other scalars like unsigned short, float,
 etc. These scalars are always mapped through lookup tables.

\item  \verb|obj.SetMapColorScalarsThroughLookupTable (int )| -  Turn on/off the mapping of color scalars through the lookup table.
 The default is Off. If Off, unsigned char scalars will be used
 directly as texture. If On, scalars will be mapped through the
 lookup table to generate 4-component unsigned char scalars.
 This ivar does not affect other scalars like unsigned short, float,
 etc. These scalars are always mapped through lookup tables.

\item  \verb|obj.MapColorScalarsThroughLookupTableOn ()| -  Turn on/off the mapping of color scalars through the lookup table.
 The default is Off. If Off, unsigned char scalars will be used
 directly as texture. If On, scalars will be mapped through the
 lookup table to generate 4-component unsigned char scalars.
 This ivar does not affect other scalars like unsigned short, float,
 etc. These scalars are always mapped through lookup tables.

\item  \verb|obj.MapColorScalarsThroughLookupTableOff ()| -  Turn on/off the mapping of color scalars through the lookup table.
 The default is Off. If Off, unsigned char scalars will be used
 directly as texture. If On, scalars will be mapped through the
 lookup table to generate 4-component unsigned char scalars.
 This ivar does not affect other scalars like unsigned short, float,
 etc. These scalars are always mapped through lookup tables.

\item  \verb|obj.SetLookupTable (vtkScalarsToColors )| -  Specify the lookup table to convert scalars if necessary

\item  \verb|vtkScalarsToColors = obj.GetLookupTable ()| -  Specify the lookup table to convert scalars if necessary

\item  \verb|vtkUnsignedCharArray = obj.GetMappedScalars ()| -  Get Mapped Scalars

\item  \verb|obj.SetTransform (vtkTransform transform)| -  Set a transform on the texture which allows one to scale,
 rotate and translate the texture.

\item  \verb|vtkTransform = obj.GetTransform ()| -  Set a transform on the texture which allows one to scale,
 rotate and translate the texture.

\item  \verb|int = obj.GetBlendingMode ()| -  Used to specify how the texture will blend its RGB and Alpha values
 with other textures and the fragment the texture is rendered upon.

\item  \verb|obj.SetBlendingMode (int )| -  Used to specify how the texture will blend its RGB and Alpha values
 with other textures and the fragment the texture is rendered upon.

\item  \verb|bool = obj.GetPremultipliedAlpha ()| -  Whether the texture colors are premultiplied by alpha.
 Initial value is false.

\item  \verb|obj.SetPremultipliedAlpha (bool )| -  Whether the texture colors are premultiplied by alpha.
 Initial value is false.

\item  \verb|obj.PremultipliedAlphaOn ()| -  Whether the texture colors are premultiplied by alpha.
 Initial value is false.

\item  \verb|obj.PremultipliedAlphaOff ()| -  Whether the texture colors are premultiplied by alpha.
 Initial value is false.

\item  \verb|int = obj.GetRestrictPowerOf2ImageSmaller ()| -  When the texture is forced to be a power of 2, the default behavior is
 for the ''new'' image's dimensions  to be greater than or equal to with 
 respects to the original.  Setting RestrictPowerOf2ImageSmaller to be
 1 (or ON) with force the new image's dimensions to be less than or equal 
 to with respects to the original.

\item  \verb|obj.SetRestrictPowerOf2ImageSmaller (int )| -  When the texture is forced to be a power of 2, the default behavior is
 for the ''new'' image's dimensions  to be greater than or equal to with 
 respects to the original.  Setting RestrictPowerOf2ImageSmaller to be
 1 (or ON) with force the new image's dimensions to be less than or equal 
 to with respects to the original.

\item  \verb|obj.RestrictPowerOf2ImageSmallerOn ()| -  When the texture is forced to be a power of 2, the default behavior is
 for the ''new'' image's dimensions  to be greater than or equal to with 
 respects to the original.  Setting RestrictPowerOf2ImageSmaller to be
 1 (or ON) with force the new image's dimensions to be less than or equal 
 to with respects to the original.

\item  \verb|obj.RestrictPowerOf2ImageSmallerOff ()| -  When the texture is forced to be a power of 2, the default behavior is
 for the ''new'' image's dimensions  to be greater than or equal to with 
 respects to the original.  Setting RestrictPowerOf2ImageSmaller to be
 1 (or ON) with force the new image's dimensions to be less than or equal 
 to with respects to the original.

\end{itemize}
