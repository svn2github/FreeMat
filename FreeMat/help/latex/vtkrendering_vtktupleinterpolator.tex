\section{vtkTupleInterpolator}

\subsection{Usage}

 This class is used to interpolate a tuple which may have an arbitrary
 number of components (but at least one component). The interpolation may
 be linear in form, or via a subclasses of vtkSpline.

 To use this class, begin by specifying the number of components of the
 tuple and the interpolation function to use. Then specify at least one
 pair of (t,tuple) with the AddTuple() method.  Next interpolate the
 tuples with the InterpolateTuple(t,tuple) method, where ''t'' must be in the
 range of (t\_min,t\_max) parameter values specified by the AddTuple() method
 (if not then t is clamped), and tuple[] is filled in by the method (make
 sure that tuple [] is long enough to hold the interpolated data).

 You can control the type of interpolation to use. By default, the
 interpolation is based on a Kochanek spline. However, other types of
 splines can be specified. You can also set the interpolation method
 to linear, in which case the specified spline has no effect on the
 interpolation.

To create an instance of class vtkTupleInterpolator, simply
invoke its constructor as follows
\begin{verbatim}
  obj = vtkTupleInterpolator
\end{verbatim}
\subsection{Methods}

The class vtkTupleInterpolator has several methods that can be used.
  They are listed below.
Note that the documentation is translated automatically from the VTK sources,
and may not be completely intelligible.  When in doubt, consult the VTK website.
In the methods listed below, \verb|obj| is an instance of the vtkTupleInterpolator class.
\begin{itemize}
\item  \verb|string = obj.GetClassName ()|

\item  \verb|int = obj.IsA (string name)|

\item  \verb|vtkTupleInterpolator = obj.NewInstance ()|

\item  \verb|vtkTupleInterpolator = obj.SafeDownCast (vtkObject o)|

\item  \verb|obj.SetNumberOfComponents (int numComp)| -  Specify the number of tuple components to interpolate. Note that setting
 this value discards any previously inserted data.

\item  \verb|int = obj.GetNumberOfComponents ()| -  Specify the number of tuple components to interpolate. Note that setting
 this value discards any previously inserted data.

\item  \verb|int = obj.GetNumberOfTuples ()| -  Return the number of tuples in the list of tuples to be
 interpolated.

\item  \verb|double = obj.GetMinimumT ()| -  Obtain some information about the interpolation range. The numbers
 returned (corresponding to parameter t, usually thought of as time)
 are undefined if the list of transforms is empty. This is a convenience
 method for interpolation.

\item  \verb|double = obj.GetMaximumT ()| -  Obtain some information about the interpolation range. The numbers
 returned (corresponding to parameter t, usually thought of as time)
 are undefined if the list of transforms is empty. This is a convenience
 method for interpolation.

\item  \verb|obj.Initialize ()| -  Reset the class so that it contains no (t,tuple) information.

\item  \verb|obj.AddTuple (double t, double tuple[])| -  Add another tuple to the list of tuples to be interpolated.  Note that
 using the same time t value more than once replaces the previous tuple
 value at t.  At least two tuples must be added to define an
 interpolation function.

\item  \verb|obj.RemoveTuple (double t)| -  Delete the tuple at a particular parameter t. If there is no
 tuple defined at t, then the method does nothing.

\item  \verb|obj.InterpolateTuple (double t, double tuple[])| -  Interpolate the list of tuples and determine a new tuple (i.e.,
 fill in the tuple provided). If t is outside the range of
 (min,max) values, then t is clamped. Note that each component
 of tuple[] is interpolated independently.

\item  \verb|obj.SetInterpolationType (int type)| -  Specify which type of function to use for interpolation. By default
 spline interpolation (SetInterpolationFunctionToSpline()) is used
 (i.e., a Kochanek spline) and the InterpolatingSpline instance variable
 is used to birth the actual interpolation splines via a combination of
 NewInstance() and DeepCopy(). You may also choose to use linear
 interpolation by invoking SetInterpolationFunctionToLinear(). Note that
 changing the type of interpolation causes previously inserted data
 to be discarded.

\item  \verb|int = obj.GetInterpolationType ()| -  Specify which type of function to use for interpolation. By default
 spline interpolation (SetInterpolationFunctionToSpline()) is used
 (i.e., a Kochanek spline) and the InterpolatingSpline instance variable
 is used to birth the actual interpolation splines via a combination of
 NewInstance() and DeepCopy(). You may also choose to use linear
 interpolation by invoking SetInterpolationFunctionToLinear(). Note that
 changing the type of interpolation causes previously inserted data
 to be discarded.

\item  \verb|obj.SetInterpolationTypeToLinear ()| -  Specify which type of function to use for interpolation. By default
 spline interpolation (SetInterpolationFunctionToSpline()) is used
 (i.e., a Kochanek spline) and the InterpolatingSpline instance variable
 is used to birth the actual interpolation splines via a combination of
 NewInstance() and DeepCopy(). You may also choose to use linear
 interpolation by invoking SetInterpolationFunctionToLinear(). Note that
 changing the type of interpolation causes previously inserted data
 to be discarded.

\item  \verb|obj.SetInterpolationTypeToSpline ()| -  If the InterpolationType is set to spline, then this method applies. By
 default Kochanek interpolation is used, but you can specify any instance
 of vtkSpline to use. Note that the actual interpolating splines are
 created by invoking NewInstance() followed by DeepCopy() on the
 interpolating spline specified here, for each tuple component to
 interpolate.

\item  \verb|obj.SetInterpolatingSpline (vtkSpline )| -  If the InterpolationType is set to spline, then this method applies. By
 default Kochanek interpolation is used, but you can specify any instance
 of vtkSpline to use. Note that the actual interpolating splines are
 created by invoking NewInstance() followed by DeepCopy() on the
 interpolating spline specified here, for each tuple component to
 interpolate.

\item  \verb|vtkSpline = obj.GetInterpolatingSpline ()| -  If the InterpolationType is set to spline, then this method applies. By
 default Kochanek interpolation is used, but you can specify any instance
 of vtkSpline to use. Note that the actual interpolating splines are
 created by invoking NewInstance() followed by DeepCopy() on the
 interpolating spline specified here, for each tuple component to
 interpolate.

\end{itemize}
