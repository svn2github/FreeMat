\section{vtkVolume}

\subsection{Usage}

 vtkVolume is used to represent a volumetric entity in a rendering scene.
 It inherits functions related to the volume's position, orientation and
 origin from vtkProp3D. The volume maintains a reference to the
 volumetric data (i.e., the volume mapper). The volume also contains a
 reference to a volume property which contains all common volume rendering 
 parameters.

To create an instance of class vtkVolume, simply
invoke its constructor as follows
\begin{verbatim}
  obj = vtkVolume
\end{verbatim}
\subsection{Methods}

The class vtkVolume has several methods that can be used.
  They are listed below.
Note that the documentation is translated automatically from the VTK sources,
and may not be completely intelligible.  When in doubt, consult the VTK website.
In the methods listed below, \verb|obj| is an instance of the vtkVolume class.
\begin{itemize}
\item  \verb|string = obj.GetClassName ()|

\item  \verb|int = obj.IsA (string name)|

\item  \verb|vtkVolume = obj.NewInstance ()|

\item  \verb|vtkVolume = obj.SafeDownCast (vtkObject o)|

\item  \verb|obj.SetMapper (vtkAbstractVolumeMapper mapper)| -  Set/Get the volume mapper.

\item  \verb|vtkAbstractVolumeMapper = obj.GetMapper ()| -  Set/Get the volume mapper.

\item  \verb|obj.SetProperty (vtkVolumeProperty property)| -  Set/Get the volume property.

\item  \verb|vtkVolumeProperty = obj.GetProperty ()| -  Set/Get the volume property.

\item  \verb|obj.GetVolumes (vtkPropCollection vc)| -  For some exporters and other other operations we must be
 able to collect all the actors or volumes. This method
 is used in that process.

\item  \verb|obj.Update ()| -  Update the volume rendering pipeline by updating the volume mapper

\item  \verb|double = obj.GetBounds ()| -  Get the bounds - either all six at once 
 (xmin, xmax, ymin, ymax, zmin, zmax) or one at a time.

\item  \verb|obj.GetBounds (double bounds[6])| -  Get the bounds - either all six at once 
 (xmin, xmax, ymin, ymax, zmin, zmax) or one at a time.

\item  \verb|double = obj.GetMinXBound ()| -  Get the bounds - either all six at once 
 (xmin, xmax, ymin, ymax, zmin, zmax) or one at a time.

\item  \verb|double = obj.GetMaxXBound ()| -  Get the bounds - either all six at once 
 (xmin, xmax, ymin, ymax, zmin, zmax) or one at a time.

\item  \verb|double = obj.GetMinYBound ()| -  Get the bounds - either all six at once 
 (xmin, xmax, ymin, ymax, zmin, zmax) or one at a time.

\item  \verb|double = obj.GetMaxYBound ()| -  Get the bounds - either all six at once 
 (xmin, xmax, ymin, ymax, zmin, zmax) or one at a time.

\item  \verb|double = obj.GetMinZBound ()| -  Get the bounds - either all six at once 
 (xmin, xmax, ymin, ymax, zmin, zmax) or one at a time.

\item  \verb|double = obj.GetMaxZBound ()| -  Get the bounds - either all six at once 
 (xmin, xmax, ymin, ymax, zmin, zmax) or one at a time.

\item  \verb|long = obj.GetMTime ()| -  Return the MTime also considering the property etc.

\item  \verb|long = obj.GetRedrawMTime ()| -  Return the mtime of anything that would cause the rendered image to 
 appear differently. Usually this involves checking the mtime of the 
 prop plus anything else it depends on such as properties, mappers,
 etc.

\item  \verb|obj.ShallowCopy (vtkProp prop)| -  Shallow copy of this vtkVolume. Overloads the virtual vtkProp method.

\end{itemize}
