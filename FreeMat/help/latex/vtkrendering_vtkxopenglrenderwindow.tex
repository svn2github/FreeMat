\section{vtkXOpenGLRenderWindow}

\subsection{Usage}

 vtkXOpenGLRenderWindow is a concrete implementation of the abstract class
 vtkRenderWindow. vtkOpenGLRenderer interfaces to the OpenGL graphics
 library. Application programmers should normally use vtkRenderWindow
 instead of the OpenGL specific version.

To create an instance of class vtkXOpenGLRenderWindow, simply
invoke its constructor as follows
\begin{verbatim}
  obj = vtkXOpenGLRenderWindow
\end{verbatim}
\subsection{Methods}

The class vtkXOpenGLRenderWindow has several methods that can be used.
  They are listed below.
Note that the documentation is translated automatically from the VTK sources,
and may not be completely intelligible.  When in doubt, consult the VTK website.
In the methods listed below, \verb|obj| is an instance of the vtkXOpenGLRenderWindow class.
\begin{itemize}
\item  \verb|string = obj.GetClassName ()|

\item  \verb|int = obj.IsA (string name)|

\item  \verb|vtkXOpenGLRenderWindow = obj.NewInstance ()|

\item  \verb|vtkXOpenGLRenderWindow = obj.SafeDownCast (vtkObject o)|

\item  \verb|obj.Start (void )| -  Begin the rendering process.

\item  \verb|obj.Frame (void )| -  End the rendering process and display the image.

\item  \verb|obj.WindowInitialize (void )| -  Initialize the window for rendering.

\item  \verb|obj.Initialize (void )| -  Initialize the rendering window.  This will setup all system-specific
 resources.  This method and Finalize() must be symmetric and it
 should be possible to call them multiple times, even changing WindowId
 in-between.  This is what WindowRemap does.

\item  \verb|obj.Finalize (void )| -  ''Deinitialize'' the rendering window.  This will shutdown all system-specific
 resources.  After having called this, it should be possible to destroy
 a window that was used for a SetWindowId() call without any ill effects.

\item  \verb|obj.SetFullScreen (int )| -  Change the window to fill the entire screen.

\item  \verb|obj.WindowRemap (void )| -  Resize the window.

\item  \verb|obj.PrefFullScreen (void )| -  Set the preferred window size to full screen.

\item  \verb|obj.SetSize (int , int )| -  Specify the size of the rendering window in pixels.

\item  \verb|obj.SetSize (int a[2])| -  Specify the size of the rendering window in pixels.

\item  \verb|int = obj.GetDesiredDepth ()| -  Get the X properties of an ideal rendering window.

\item  \verb|obj.SetStereoCapableWindow (int capable)| -  Prescribe that the window be created in a stereo-capable mode. This
 method must be called before the window is realized. This method
 overrides the superclass method since this class can actually check
 whether the window has been realized yet.

\item  \verb|obj.MakeCurrent ()| -  Make this window the current OpenGL context.

\item  \verb|bool = obj.IsCurrent ()| -  Tells if this window is the current OpenGL context for the calling thread.

\item  \verb|obj.SetForceMakeCurrent ()| -  If called, allow MakeCurrent() to skip cache-check when called.
 MakeCurrent() reverts to original behavior of cache-checking     
 on the next render.     

\item  \verb|string = obj.ReportCapabilities ()| -  Get report of capabilities for the render window

\item  \verb|int = obj.SupportsOpenGL ()| -  Does this render window support OpenGL? 0-false, 1-true

\item  \verb|int = obj.IsDirect ()| -  Is this render window using hardware acceleration? 0-false, 1-true

\item  \verb|obj.SetWindowName (string )|

\item  \verb|obj.SetPosition (int , int )| -  Move the window to a new position on the display.

\item  \verb|obj.SetPosition (int a[2])| -  Move the window to a new position on the display.

\item  \verb|obj.HideCursor ()| -  Hide or Show the mouse cursor, it is nice to be able to hide the
 default cursor if you want VTK to display a 3D cursor instead.

\item  \verb|obj.ShowCursor ()| -  Hide or Show the mouse cursor, it is nice to be able to hide the
 default cursor if you want VTK to display a 3D cursor instead.

\item  \verb|obj.SetCurrentCursor (int )| -  Change the shape of the cursor

\item  \verb|int = obj.GetEventPending ()| -  Check to see if a mouse button has been pressed.
 All other events are ignored by this method.
 This is a useful check to abort a long render.

\item  \verb|obj.SetWindowInfo (string info)| -  Set this RenderWindow's X window id to a pre-existing window.

\item  \verb|obj.SetNextWindowInfo (string info)| -  Set the window info that will be used after WindowRemap()

\item  \verb|obj.SetParentInfo (string info)| -  Sets the X window id of the window that WILL BE created.

\item  \verb|obj.Render ()| -  This computes the size of the render window 
 before calling the supper classes render

\item  \verb|obj.SetOffScreenRendering (int i)| -  Render without displaying the window.

\end{itemize}
