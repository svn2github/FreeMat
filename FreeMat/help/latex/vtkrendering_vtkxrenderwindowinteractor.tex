\section{vtkXRenderWindowInteractor}

\subsection{Usage}

 vtkXRenderWindowInteractor is a convenience object that provides event
 bindings to common graphics functions. For example, camera and actor
 functions such as zoom-in/zoom-out, azimuth, roll, and pan. IT is one of
 the window system specific subclasses of vtkRenderWindowInteractor. Please
 see vtkRenderWindowInteractor documentation for event bindings.


To create an instance of class vtkXRenderWindowInteractor, simply
invoke its constructor as follows
\begin{verbatim}
  obj = vtkXRenderWindowInteractor
\end{verbatim}
\subsection{Methods}

The class vtkXRenderWindowInteractor has several methods that can be used.
  They are listed below.
Note that the documentation is translated automatically from the VTK sources,
and may not be completely intelligible.  When in doubt, consult the VTK website.
In the methods listed below, \verb|obj| is an instance of the vtkXRenderWindowInteractor class.
\begin{itemize}
\item  \verb|string = obj.GetClassName ()|

\item  \verb|int = obj.IsA (string name)|

\item  \verb|vtkXRenderWindowInteractor = obj.NewInstance ()|

\item  \verb|vtkXRenderWindowInteractor = obj.SafeDownCast (vtkObject o)|

\item  \verb|obj.Initialize ()| -  Initializes the event handlers without an XtAppContext.  This is
 good for when you don't have a user interface, but you still
 want to have mouse interaction.

\item  \verb|obj.TerminateApp ()| -  Break the event loop on 'q','e' keypress. Want more ???

\item  \verb|int = obj.GetBreakLoopFlag ()| -  The BreakLoopFlag is checked in the Start() method.
 Setting it to anything other than zero will cause
 the interactor loop to terminate and return to the
 calling function.

\item  \verb|obj.SetBreakLoopFlag (int )| -  The BreakLoopFlag is checked in the Start() method.
 Setting it to anything other than zero will cause
 the interactor loop to terminate and return to the
 calling function.

\item  \verb|obj.BreakLoopFlagOff ()| -  The BreakLoopFlag is checked in the Start() method.
 Setting it to anything other than zero will cause
 the interactor loop to terminate and return to the
 calling function.

\item  \verb|obj.BreakLoopFlagOn ()| -  The BreakLoopFlag is checked in the Start() method.
 Setting it to anything other than zero will cause
 the interactor loop to terminate and return to the
 calling function.

\item  \verb|obj.Enable ()| -  Enable/Disable interactions.  By default interactors are enabled when
 initialized.  Initialize() must be called prior to enabling/disabling
 interaction. These methods are used when a window/widget is being
 shared by multiple renderers and interactors.  This allows a ''modal''
 display where one interactor is active when its data is to be displayed
 and all other interactors associated with the widget are disabled
 when their data is not displayed.

\item  \verb|obj.Disable ()| -  Enable/Disable interactions.  By default interactors are enabled when
 initialized.  Initialize() must be called prior to enabling/disabling
 interaction. These methods are used when a window/widget is being
 shared by multiple renderers and interactors.  This allows a ''modal''
 display where one interactor is active when its data is to be displayed
 and all other interactors associated with the widget are disabled
 when their data is not displayed.

\item  \verb|obj.Start ()| -  This will start up the X event loop and never return. If you
 call this method it will loop processing X events until the
 application is exited.

\item  \verb|obj.UpdateSize (int , int )| -  Update the Size data member and set the associated RenderWindow's
 size.

\item  \verb|obj.GetMousePosition (int x, int y)| -  Re-defines virtual function to get mouse position by querying X-server.

\end{itemize}
