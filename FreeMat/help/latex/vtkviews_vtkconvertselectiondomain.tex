\section{vtkConvertSelectionDomain}

\subsection{Usage}

 vtkConvertSelectionDomain converts a selection from one domain to another
 using known domain mappings. The domain mappings are described by a
 vtkMultiBlockDataSet containing one or more vtkTables.

 The first input port is for the input selection (or collection of annotations
 in a vtkAnnotationLayers object), while the second port
 is for the multi-block of mappings, and the third port is for the
 data that is being selected on.

 If the second or third port is not set, this filter will pass the
 selection/annotation to the output unchanged.

 The second output is the selection associated with the ''current annotation''
 normally representing the current interactive selection.

To create an instance of class vtkConvertSelectionDomain, simply
invoke its constructor as follows
\begin{verbatim}
  obj = vtkConvertSelectionDomain
\end{verbatim}
\subsection{Methods}

The class vtkConvertSelectionDomain has several methods that can be used.
  They are listed below.
Note that the documentation is translated automatically from the VTK sources,
and may not be completely intelligible.  When in doubt, consult the VTK website.
In the methods listed below, \verb|obj| is an instance of the vtkConvertSelectionDomain class.
\begin{itemize}
\item  \verb|string = obj.GetClassName ()|

\item  \verb|int = obj.IsA (string name)|

\item  \verb|vtkConvertSelectionDomain = obj.NewInstance ()|

\item  \verb|vtkConvertSelectionDomain = obj.SafeDownCast (vtkObject o)|

\end{itemize}
