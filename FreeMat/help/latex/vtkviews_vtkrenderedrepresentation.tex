\section{vtkRenderedRepresentation}

\subsection{Usage}


To create an instance of class vtkRenderedRepresentation, simply
invoke its constructor as follows
\begin{verbatim}
  obj = vtkRenderedRepresentation
\end{verbatim}
\subsection{Methods}

The class vtkRenderedRepresentation has several methods that can be used.
  They are listed below.
Note that the documentation is translated automatically from the VTK sources,
and may not be completely intelligible.  When in doubt, consult the VTK website.
In the methods listed below, \verb|obj| is an instance of the vtkRenderedRepresentation class.
\begin{itemize}
\item  \verb|string = obj.GetClassName ()|

\item  \verb|int = obj.IsA (string name)|

\item  \verb|vtkRenderedRepresentation = obj.NewInstance ()|

\item  \verb|vtkRenderedRepresentation = obj.SafeDownCast (vtkObject o)|

\item  \verb|obj.SetLabelRenderMode (int )| -  Set the label render mode.
 vtkRenderView::QT - Use Qt-based labeler with fitted labeling
   and unicode support. Requires VTK\_USE\_QT to be on.
 vtkRenderView::FREETYPE - Use standard freetype text rendering.

\item  \verb|int = obj.GetLabelRenderMode ()| -  Set the label render mode.
 vtkRenderView::QT - Use Qt-based labeler with fitted labeling
   and unicode support. Requires VTK\_USE\_QT to be on.
 vtkRenderView::FREETYPE - Use standard freetype text rendering.

\end{itemize}
