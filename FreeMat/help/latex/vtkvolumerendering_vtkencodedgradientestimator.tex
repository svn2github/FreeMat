\section{vtkEncodedGradientEstimator}

\subsection{Usage}

 vtkEncodedGradientEstimator is an abstract superclass for gradient 
 estimation. It takes a scalar input of vtkImageData, computes
 a gradient value for every point, and encodes this value into a 
 three byte value (2 for direction, 1 for magnitude) using the 
 vtkDirectionEncoder. The direction encoder is defaulted to a
 vtkRecursiveSphereDirectionEncoder, but can be overridden with the
 SetDirectionEncoder method. The scale and the bias values for the gradient
 magnitude are used to convert it into a one byte value according to
 v = m*scale + bias where m is the magnitude and v is the resulting
 one byte value.

To create an instance of class vtkEncodedGradientEstimator, simply
invoke its constructor as follows
\begin{verbatim}
  obj = vtkEncodedGradientEstimator
\end{verbatim}
\subsection{Methods}

The class vtkEncodedGradientEstimator has several methods that can be used.
  They are listed below.
Note that the documentation is translated automatically from the VTK sources,
and may not be completely intelligible.  When in doubt, consult the VTK website.
In the methods listed below, \verb|obj| is an instance of the vtkEncodedGradientEstimator class.
\begin{itemize}
\item  \verb|string = obj.GetClassName ()|

\item  \verb|int = obj.IsA (string name)|

\item  \verb|vtkEncodedGradientEstimator = obj.NewInstance ()|

\item  \verb|vtkEncodedGradientEstimator = obj.SafeDownCast (vtkObject o)|

\item  \verb|obj.SetInput (vtkImageData )| -  Set/Get the scalar input for which the normals will be 
 calculated

\item  \verb|vtkImageData = obj.GetInput ()| -  Set/Get the scalar input for which the normals will be 
 calculated

\item  \verb|obj.SetGradientMagnitudeScale (float )| -  Set/Get the scale and bias for the gradient magnitude

\item  \verb|float = obj.GetGradientMagnitudeScale ()| -  Set/Get the scale and bias for the gradient magnitude

\item  \verb|obj.SetGradientMagnitudeBias (float )| -  Set/Get the scale and bias for the gradient magnitude

\item  \verb|float = obj.GetGradientMagnitudeBias ()| -  Set/Get the scale and bias for the gradient magnitude

\item  \verb|obj.SetBoundsClip (int )| -  Turn on / off the bounding of the normal computation by
 the this->Bounds bounding box

\item  \verb|int = obj.GetBoundsClipMinValue ()| -  Turn on / off the bounding of the normal computation by
 the this->Bounds bounding box

\item  \verb|int = obj.GetBoundsClipMaxValue ()| -  Turn on / off the bounding of the normal computation by
 the this->Bounds bounding box

\item  \verb|int = obj.GetBoundsClip ()| -  Turn on / off the bounding of the normal computation by
 the this->Bounds bounding box

\item  \verb|obj.BoundsClipOn ()| -  Turn on / off the bounding of the normal computation by
 the this->Bounds bounding box

\item  \verb|obj.BoundsClipOff ()| -  Turn on / off the bounding of the normal computation by
 the this->Bounds bounding box

\item  \verb|obj.SetBounds (int , int , int , int , int , int )| -  Set / Get the bounds of the computation (used if 
 this->ComputationBounds is 1.) The bounds are specified
 xmin, xmax, ymin, ymax, zmin, zmax.

\item  \verb|obj.SetBounds (int  a[6])| -  Set / Get the bounds of the computation (used if 
 this->ComputationBounds is 1.) The bounds are specified
 xmin, xmax, ymin, ymax, zmin, zmax.

\item  \verb|int = obj. GetBounds ()| -  Set / Get the bounds of the computation (used if 
 this->ComputationBounds is 1.) The bounds are specified
 xmin, xmax, ymin, ymax, zmin, zmax.

\item  \verb|obj.Update (void )| -  Recompute the encoded normals and gradient magnitudes.

\item  \verb|int = obj.GetEncodedNormalIndex (int xyz\_index)| -  Get the encoded normal at an x,y,z location in the volume

\item  \verb|int = obj.GetEncodedNormalIndex (int x\_index, int y\_index, int z\_index)| -  Get the encoded normal at an x,y,z location in the volume

\item  \verb|obj.SetNumberOfThreads (int )| -  Get/Set the number of threads to create when encoding normals
 This defaults to the number of available processors on the machine

\item  \verb|int = obj.GetNumberOfThreadsMinValue ()| -  Get/Set the number of threads to create when encoding normals
 This defaults to the number of available processors on the machine

\item  \verb|int = obj.GetNumberOfThreadsMaxValue ()| -  Get/Set the number of threads to create when encoding normals
 This defaults to the number of available processors on the machine

\item  \verb|int = obj.GetNumberOfThreads ()| -  Get/Set the number of threads to create when encoding normals
 This defaults to the number of available processors on the machine

\item  \verb|obj.SetDirectionEncoder (vtkDirectionEncoder direnc)| -  Set / Get the direction encoder used to encode normal directions
 to fit within two bytes

\item  \verb|vtkDirectionEncoder = obj.GetDirectionEncoder ()| -  Set / Get the direction encoder used to encode normal directions
 to fit within two bytes

\item  \verb|obj.SetComputeGradientMagnitudes (int )| -  If you don't want to compute gradient magnitudes (but you
 do want normals for shading) this can be used. Be careful - if
 if you a non-constant gradient magnitude transfer function and
 you turn this on, it may crash

\item  \verb|int = obj.GetComputeGradientMagnitudes ()| -  If you don't want to compute gradient magnitudes (but you
 do want normals for shading) this can be used. Be careful - if
 if you a non-constant gradient magnitude transfer function and
 you turn this on, it may crash

\item  \verb|obj.ComputeGradientMagnitudesOn ()| -  If you don't want to compute gradient magnitudes (but you
 do want normals for shading) this can be used. Be careful - if
 if you a non-constant gradient magnitude transfer function and
 you turn this on, it may crash

\item  \verb|obj.ComputeGradientMagnitudesOff ()| -  If you don't want to compute gradient magnitudes (but you
 do want normals for shading) this can be used. Be careful - if
 if you a non-constant gradient magnitude transfer function and
 you turn this on, it may crash

\item  \verb|obj.SetCylinderClip (int )| -  If the data in each slice is only contained within a circle circumscribed
 within the slice, and the slice is square, then don't compute anything
 outside the circle. This circle through the slices forms a cylinder.

\item  \verb|int = obj.GetCylinderClip ()| -  If the data in each slice is only contained within a circle circumscribed
 within the slice, and the slice is square, then don't compute anything
 outside the circle. This circle through the slices forms a cylinder.

\item  \verb|obj.CylinderClipOn ()| -  If the data in each slice is only contained within a circle circumscribed
 within the slice, and the slice is square, then don't compute anything
 outside the circle. This circle through the slices forms a cylinder.

\item  \verb|obj.CylinderClipOff ()| -  If the data in each slice is only contained within a circle circumscribed
 within the slice, and the slice is square, then don't compute anything
 outside the circle. This circle through the slices forms a cylinder.

\item  \verb|float = obj.GetLastUpdateTimeInSeconds ()| -  Get the time required for the last update in seconds or cpu seconds

\item  \verb|float = obj.GetLastUpdateTimeInCPUSeconds ()| -  Get the time required for the last update in seconds or cpu seconds

\item  \verb|int = obj.GetUseCylinderClip ()|

\item  \verb|obj.SetZeroNormalThreshold (float v)| -  Set / Get the ZeroNormalThreshold - this defines the minimum magnitude 
 of a gradient that is considered sufficient to define a 
 direction. Gradients with magnitudes at or less than this value are given
 a ''zero normal'' index. These are handled specially in the shader, 
 and you can set the intensity of light for these zero normals in
 the gradient shader.

\item  \verb|float = obj.GetZeroNormalThreshold ()| -  Set / Get the ZeroNormalThreshold - this defines the minimum magnitude 
 of a gradient that is considered sufficient to define a 
 direction. Gradients with magnitudes at or less than this value are given
 a ''zero normal'' index. These are handled specially in the shader, 
 and you can set the intensity of light for these zero normals in
 the gradient shader.

\item  \verb|obj.SetZeroPad (int )| -  Assume that the data value outside the volume is zero when
 computing normals.

\item  \verb|int = obj.GetZeroPadMinValue ()| -  Assume that the data value outside the volume is zero when
 computing normals.

\item  \verb|int = obj.GetZeroPadMaxValue ()| -  Assume that the data value outside the volume is zero when
 computing normals.

\item  \verb|int = obj.GetZeroPad ()| -  Assume that the data value outside the volume is zero when
 computing normals.

\item  \verb|obj.ZeroPadOn ()| -  Assume that the data value outside the volume is zero when
 computing normals.

\item  \verb|obj.ZeroPadOff ()| -  Assume that the data value outside the volume is zero when
 computing normals.

\end{itemize}
