\section{vtkHAVSVolumeMapper}

\subsection{Usage}


 vtkHAVSVolumeMapper is a class that renders polygonal data
 (represented as an unstructured grid) using the Hardware-Assisted
 Visibility Sorting (HAVS) algorithm.  First the unique triangles are sorted
 in object space, then they are sorted in image space using a fixed size
 A-buffer implemented on the GPU called the k-buffer.  The HAVS algorithm
 excels at rendering large datasets quickly.  The trade-off is that the
 algorithm may produce some rendering artifacts due to an insufficient k
 size (currently 2 or 6 is supported) or read/write race conditions.
 
 A built in level-of-detail (LOD) approach samples the geometry using one of
 two heuristics (field or area).  If LOD is enabled, the amount of geometry
 that is sampled and rendered changes dynamically to stay within the target
 frame rate.  The field sampling method generally works best for datasets
 with cell sizes that don't vary much in size.  On the contrary, the area
 sampling approach gives better approximations when the volume has a lot of
 variation in cell size.

 The HAVS algorithm uses several advanced features on graphics hardware.
 The k-buffer sorting network is implemented using framebuffer objects
 (FBOs) with multiple render targets (MRTs).  Therefore, only cards that
 support these features can run the algorithm (at least an ATI 9500 or an
 NVidia NV40 (6600)).

 .SECTION Notes

 Several issues had to be addressed to get the HAVS algorithm working within
 the vtk framework.  These additions forced the code to forsake speed for
 the sake of compliance and robustness.

 The HAVS algorithm operates on the triangles that compose the mesh.
 Therefore, before rendering, the cells are decomposed into unique triangles
 and stored on the GPU for efficient rendering.  The use of GPU data
 structures is only recommended if the entire geometry can fit in graphics
 memory.  Otherwise this feature should be disabled. 

 Another new feature is the handling of mixed data types (eg., polygonal
 data with volume data).  This is handled by reading the z-buffer from the
 current window and copying it into the framebuffer object for off-screen
 rendering.  The depth test is then enabled so that the volume only appears
 over the opaque geometry.  Finally, the results of the off-screen rendering
 are blended into the framebuffer as a transparent, view-aligned texture. 
 
 Instead of using a preintegrated 3D lookup table for storing the ray
 integral, this implementation uses partial pre-integration.  This improves
 the performance of dynamic transfer function updates by avoiding a costly
 preprocess of the table.

 A final change to the original algorithm is the handling of non-convexities
 in the mesh.  Due to read/write hazards that may create undesired artifacts
 with non-convexities when using a inside/outside toggle in the fragment
 program, another approach was employed.  To handle non-convexities, the
 fragment shader determines if a ray-gap is larger than the max cell size
 and kill the fragment if so.  This approximation performs rather well in
 practice but may miss small non-convexities.
 
 For more information on the HAVS algorithm see:

  ''Hardware-Assisted Visibility Sorting for Unstructured Volume
 Rendering'' by S. P. Callahan, M. Ikits, J. L. D. Comba, and C. T. Silva, 
 IEEE Transactions of Visualization and Computer Graphics; May/June 2005.

 For more information on the Level-of-Detail algorithm, see:

 ''Interactive Rendering of Large Unstructured Grids Using Dynamic
 Level-of-Detail'' by S. P. Callahan, J. L. D. Comba, P. Shirley, and
 C. T. Silva, Proceedings of IEEE Visualization '05, Oct. 2005.

 .SECTION Acknowledgments

 This code was developed by Steven P. Callahan under the supervision
 of Prof. Claudio T. Silva. The code also contains contributions
 from Milan Ikits, Linh Ha, Huy T. Vo, Carlos E. Scheidegger, and 
 Joao L. D. Comba.  

 The work was supported by grants, contracts, and gifts from the
 National Science Foundation, the Department of Energy, the Army
 Research Office, and IBM.

 The port of HAVS to VTK and ParaView has been primarily supported
 by Sandia National Labs.


To create an instance of class vtkHAVSVolumeMapper, simply
invoke its constructor as follows
\begin{verbatim}
  obj = vtkHAVSVolumeMapper
\end{verbatim}
\subsection{Methods}

The class vtkHAVSVolumeMapper has several methods that can be used.
  They are listed below.
Note that the documentation is translated automatically from the VTK sources,
and may not be completely intelligible.  When in doubt, consult the VTK website.
In the methods listed below, \verb|obj| is an instance of the vtkHAVSVolumeMapper class.
\begin{itemize}
\item  \verb|string = obj.GetClassName ()|

\item  \verb|int = obj.IsA (string name)|

\item  \verb|vtkHAVSVolumeMapper = obj.NewInstance ()|

\item  \verb|vtkHAVSVolumeMapper = obj.SafeDownCast (vtkObject o)|

\item  \verb|obj.SetPartiallyRemoveNonConvexities (bool )| -  regions by removing ray segments larger than the max cell size.

\item  \verb|bool = obj.GetPartiallyRemoveNonConvexities ()| -  regions by removing ray segments larger than the max cell size.

\item  \verb|obj.SetLevelOfDetailTargetTime (float )| -  Set/get the desired level of detail target time measured in frames/sec.

\item  \verb|float = obj.GetLevelOfDetailTargetTime ()| -  Set/get the desired level of detail target time measured in frames/sec.

\item  \verb|obj.SetLevelOfDetail (bool )| -  Turn on/off level-of-detail volume rendering

\item  \verb|bool = obj.GetLevelOfDetail ()| -  Turn on/off level-of-detail volume rendering

\item  \verb|obj.SetLevelOfDetailMethod (int )| -  Set/get the current level-of-detail method

\item  \verb|int = obj.GetLevelOfDetailMethod ()| -  Set/get the current level-of-detail method

\item  \verb|obj.SetLevelOfDetailMethodField ()| -  Set/get the current level-of-detail method

\item  \verb|obj.SetLevelOfDetailMethodArea ()| -  Set the kbuffer size

\item  \verb|obj.SetKBufferSize (int )| -  Set the kbuffer size

\item  \verb|int = obj.GetKBufferSize ()| -  Set the kbuffer size

\item  \verb|obj.SetKBufferSizeTo2 ()| -  Set the kbuffer size

\item  \verb|obj.SetKBufferSizeTo6 ()| -  Check hardware support for the HAVS algorithm.  Necessary
 features include off-screen rendering, 32-bit fp textures, multiple
 render targets, and framebuffer objects.
 Subclasses must override this method to indicate if supported by Hardware.

\item  \verb|bool = obj.SupportedByHardware ()| -  Set/get whether or not the data structures should be stored on the GPU 
 for better peformance.

\item  \verb|obj.SetGPUDataStructures (bool )| -  Set/get whether or not the data structures should be stored on the GPU 
 for better peformance.

\item  \verb|bool = obj.GetGPUDataStructures ()| -  Set/get whether or not the data structures should be stored on the GPU 
 for better peformance.

\end{itemize}
