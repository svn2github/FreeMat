\section{vtkUnstructuredGridVolumeRayCastIterator}

\subsection{Usage}


 vtkUnstructuredGridVolumeRayCastIterator is a superclass for iterating
 over the intersections of a viewing ray with a group of unstructured
 cells.  These iterators are created with a
 vtkUnstructuredGridVolumeRayCastFunction.


To create an instance of class vtkUnstructuredGridVolumeRayCastIterator, simply
invoke its constructor as follows
\begin{verbatim}
  obj = vtkUnstructuredGridVolumeRayCastIterator
\end{verbatim}
\subsection{Methods}

The class vtkUnstructuredGridVolumeRayCastIterator has several methods that can be used.
  They are listed below.
Note that the documentation is translated automatically from the VTK sources,
and may not be completely intelligible.  When in doubt, consult the VTK website.
In the methods listed below, \verb|obj| is an instance of the vtkUnstructuredGridVolumeRayCastIterator class.
\begin{itemize}
\item  \verb|string = obj.GetClassName ()|

\item  \verb|int = obj.IsA (string name)|

\item  \verb|vtkUnstructuredGridVolumeRayCastIterator = obj.NewInstance ()|

\item  \verb|vtkUnstructuredGridVolumeRayCastIterator = obj.SafeDownCast (vtkObject o)|

\item  \verb|obj.Initialize (int x, int y)| -  Initializes the iteration to the start of the ray at the given screen
 coordinates.

\item  \verb|vtkIdType = obj.GetNextIntersections (vtkIdList intersectedCells, vtkDoubleArray intersectionLengths, vtkDataArray scalars, vtkDataArray nearIntersections, vtkDataArray farIntersections)| -  Get the intersections of the next several cells.  The cell ids are
 stored in \c intersectedCells and the length of each ray segment
 within the cell is stored in \c intersectionLengths.  The point
 scalars \c scalars are interpolated and stored in \c nearIntersections
 and \c farIntersections.  \c intersectedCells, \c intersectionLengths,
 or \c scalars may be \c NULL to supress passing the associated
 information.  The number of intersections actually encountered is
 returned.  0 is returned if and only if no more intersections are to
 be found.

\item  \verb|obj.SetBounds (double , double )| -  Set/get the bounds of the cast ray (in viewing coordinates).  By
 default the range is [0,1].

\item  \verb|obj.SetBounds (double  a[2])| -  Set/get the bounds of the cast ray (in viewing coordinates).  By
 default the range is [0,1].

\item  \verb|double = obj. GetBounds ()| -  Set/get the bounds of the cast ray (in viewing coordinates).  By
 default the range is [0,1].

\item  \verb|obj.SetMaxNumberOfIntersections (vtkIdType )|

\item  \verb|vtkIdType = obj.GetMaxNumberOfIntersections ()|

\end{itemize}
