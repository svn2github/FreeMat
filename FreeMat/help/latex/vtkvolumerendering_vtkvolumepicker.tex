\section{vtkVolumePicker}

\subsection{Usage}

 vtkVolumePicker is a subclass of vtkCellPicker.  It has one
 advantage over vtkCellPicker for volumes: it will be able to
 correctly perform picking when CroppingPlanes are present.  This
 isn't possible for vtkCellPicker since it doesn't link to
 the VolumeRendering classes and hence cannot access information
 about the CroppingPlanes.


To create an instance of class vtkVolumePicker, simply
invoke its constructor as follows
\begin{verbatim}
  obj = vtkVolumePicker
\end{verbatim}
\subsection{Methods}

The class vtkVolumePicker has several methods that can be used.
  They are listed below.
Note that the documentation is translated automatically from the VTK sources,
and may not be completely intelligible.  When in doubt, consult the VTK website.
In the methods listed below, \verb|obj| is an instance of the vtkVolumePicker class.
\begin{itemize}
\item  \verb|string = obj.GetClassName ()|

\item  \verb|int = obj.IsA (string name)|

\item  \verb|vtkVolumePicker = obj.NewInstance ()|

\item  \verb|vtkVolumePicker = obj.SafeDownCast (vtkObject o)|

\item  \verb|obj.SetPickCroppingPlanes (int )| -  Set whether to pick the cropping planes of props that have them.
 If this is set, then the pick will be done on the cropping planes
 rather than on the data. The GetCroppingPlaneId() method will return
 the index of the cropping plane of the volume that was picked.  This
 setting is only relevant to the picking of volumes.

\item  \verb|obj.PickCroppingPlanesOn ()| -  Set whether to pick the cropping planes of props that have them.
 If this is set, then the pick will be done on the cropping planes
 rather than on the data. The GetCroppingPlaneId() method will return
 the index of the cropping plane of the volume that was picked.  This
 setting is only relevant to the picking of volumes.

\item  \verb|obj.PickCroppingPlanesOff ()| -  Set whether to pick the cropping planes of props that have them.
 If this is set, then the pick will be done on the cropping planes
 rather than on the data. The GetCroppingPlaneId() method will return
 the index of the cropping plane of the volume that was picked.  This
 setting is only relevant to the picking of volumes.

\item  \verb|int = obj.GetPickCroppingPlanes ()| -  Set whether to pick the cropping planes of props that have them.
 If this is set, then the pick will be done on the cropping planes
 rather than on the data. The GetCroppingPlaneId() method will return
 the index of the cropping plane of the volume that was picked.  This
 setting is only relevant to the picking of volumes.

\item  \verb|int = obj.GetCroppingPlaneId ()| -  Get the index of the cropping plane that the pick ray passed
 through on its way to the prop. This will be set regardless
 of whether PickCroppingPlanes is on.  The crop planes are ordered
 as follows: xmin, xmax, ymin, ymax, zmin, zmax.  If the volume is
 not cropped, the value will bet set to -1.

\end{itemize}
