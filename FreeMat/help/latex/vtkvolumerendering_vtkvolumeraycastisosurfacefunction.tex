\section{vtkVolumeRayCastIsosurfaceFunction}

\subsection{Usage}

 vtkVolumeRayCastIsosurfaceFunction is a volume ray cast function that
 intersects a ray with an analytic isosurface in a scalar field. The color
 and shading parameters are defined in the vtkVolumeProperty of the 
 vtkVolume, as well as the interpolation type to use when locating the
 surface (either a nearest neighbor approach or a tri-linear interpolation
 approach)


To create an instance of class vtkVolumeRayCastIsosurfaceFunction, simply
invoke its constructor as follows
\begin{verbatim}
  obj = vtkVolumeRayCastIsosurfaceFunction
\end{verbatim}
\subsection{Methods}

The class vtkVolumeRayCastIsosurfaceFunction has several methods that can be used.
  They are listed below.
Note that the documentation is translated automatically from the VTK sources,
and may not be completely intelligible.  When in doubt, consult the VTK website.
In the methods listed below, \verb|obj| is an instance of the vtkVolumeRayCastIsosurfaceFunction class.
\begin{itemize}
\item  \verb|string = obj.GetClassName ()|

\item  \verb|int = obj.IsA (string name)|

\item  \verb|vtkVolumeRayCastIsosurfaceFunction = obj.NewInstance ()|

\item  \verb|vtkVolumeRayCastIsosurfaceFunction = obj.SafeDownCast (vtkObject o)|

\item  \verb|float = obj.GetZeroOpacityThreshold (vtkVolume vol)| -  Get the scalar value below which all scalar values have 0 opacity

\item  \verb|obj.SetIsoValue (double )| -  Set/Get the value of IsoValue.

\item  \verb|double = obj.GetIsoValue ()| -  Set/Get the value of IsoValue.

\end{itemize}
