\section{vtk3DWidget}

\subsection{Usage}

 vtk3DWidget is an abstract superclass for 3D interactor observers. These
 3D widgets represent themselves in the scene, and have special callbacks
 associated with them that allows interactive manipulation of the widget.
 Inparticular, the difference between a vtk3DWidget and its abstract
 superclass vtkInteractorObserver is that vtk3DWidgets are ''placed'' in 3D
 space.  vtkInteractorObservers have no notion of where they are placed,
 and may not exist in 3D space at all.  3D widgets also provide auxiliary
 functions like producing a transformation, creating polydata (for seeding
 streamlines, probes, etc.) or creating implicit functions. See the
 concrete subclasses for particulars.

 Typically the widget is used by specifying a vtkProp3D or VTK dataset as
 input, and then invoking the ''On'' method to activate it. (You can also
 specify a bounding box to help position the widget.) Prior to invoking the
 On() method, the user may also wish to use the PlaceWidget() to initially
 position it. The 'i' (for ''interactor'') keypresses also can be used to
 turn the widgets on and off (methods exist to change the key value
 and enable keypress activiation).
 
 To support interactive manipulation of objects, this class (and
 subclasses) invoke the events StartInteractionEvent, InteractionEvent, and
 EndInteractionEvent.  These events are invoked when the vtk3DWidget enters
 a state where rapid response is desired: mouse motion, etc. The events can
 be used, for example, to set the desired update frame rate
 (StartInteractionEvent), operate on the vtkProp3D or other object
 (InteractionEvent), and set the desired frame rate back to normal values
 (EndInteractionEvent).

 Note that the Priority attribute inherited from vtkInteractorObserver has
 a new default value which is now 0.5 so that all 3D widgets have a higher
 priority than the usual interactor styles.


To create an instance of class vtk3DWidget, simply
invoke its constructor as follows
\begin{verbatim}
  obj = vtk3DWidget
\end{verbatim}
\subsection{Methods}

The class vtk3DWidget has several methods that can be used.
  They are listed below.
Note that the documentation is translated automatically from the VTK sources,
and may not be completely intelligible.  When in doubt, consult the VTK website.
In the methods listed below, \verb|obj| is an instance of the vtk3DWidget class.
\begin{itemize}
\item  \verb|string = obj.GetClassName ()|

\item  \verb|int = obj.IsA (string name)|

\item  \verb|vtk3DWidget = obj.NewInstance ()|

\item  \verb|vtk3DWidget = obj.SafeDownCast (vtkObject o)|

\item  \verb|obj.PlaceWidget (double bounds[6])| -  This method is used to initially place the widget.  The placement of the
 widget depends on whether a Prop3D or input dataset is provided. If one
 of these two is provided, they will be used to obtain a bounding box,
 around which the widget is placed. Otherwise, you can manually specify a
 bounds with the PlaceWidget(bounds) method. Note: PlaceWidget(bounds)
 is required by all subclasses; the other methods are provided as
 convenience methods.

\item  \verb|obj.PlaceWidget ()| -  This method is used to initially place the widget.  The placement of the
 widget depends on whether a Prop3D or input dataset is provided. If one
 of these two is provided, they will be used to obtain a bounding box,
 around which the widget is placed. Otherwise, you can manually specify a
 bounds with the PlaceWidget(bounds) method. Note: PlaceWidget(bounds)
 is required by all subclasses; the other methods are provided as
 convenience methods.

\item  \verb|obj.PlaceWidget (double xmin, double xmax, double ymin, double ymax, double zmin, double zmax)| -  This method is used to initially place the widget.  The placement of the
 widget depends on whether a Prop3D or input dataset is provided. If one
 of these two is provided, they will be used to obtain a bounding box,
 around which the widget is placed. Otherwise, you can manually specify a
 bounds with the PlaceWidget(bounds) method. Note: PlaceWidget(bounds)
 is required by all subclasses; the other methods are provided as
 convenience methods.

\item  \verb|obj.SetProp3D (vtkProp3D )| -  Specify a vtkProp3D around which to place the widget. This 
 is not required, but if supplied, it is used to initially 
 position the widget.

\item  \verb|vtkProp3D = obj.GetProp3D ()| -  Specify a vtkProp3D around which to place the widget. This 
 is not required, but if supplied, it is used to initially 
 position the widget.

\item  \verb|obj.SetInput (vtkDataSet )| -  Specify the input dataset. This is not required, but if supplied,
 and no vtkProp3D is specified, it is used to initially position 
 the widget.

\item  \verb|vtkDataSet = obj.GetInput ()| -  Specify the input dataset. This is not required, but if supplied,
 and no vtkProp3D is specified, it is used to initially position 
 the widget.

\item  \verb|obj.SetPlaceFactor (double )| -  Set/Get a factor representing the scaling of the widget upon placement
 (via the PlaceWidget() method). Normally the widget is placed so that
 it just fits within the bounding box defined in PlaceWidget(bounds).
 The PlaceFactor will make the widget larger (PlaceFactor > 1) or smaller
 (PlaceFactor < 1). By default, PlaceFactor is set to 0.5.

\item  \verb|double = obj.GetPlaceFactorMinValue ()| -  Set/Get a factor representing the scaling of the widget upon placement
 (via the PlaceWidget() method). Normally the widget is placed so that
 it just fits within the bounding box defined in PlaceWidget(bounds).
 The PlaceFactor will make the widget larger (PlaceFactor > 1) or smaller
 (PlaceFactor < 1). By default, PlaceFactor is set to 0.5.

\item  \verb|double = obj.GetPlaceFactorMaxValue ()| -  Set/Get a factor representing the scaling of the widget upon placement
 (via the PlaceWidget() method). Normally the widget is placed so that
 it just fits within the bounding box defined in PlaceWidget(bounds).
 The PlaceFactor will make the widget larger (PlaceFactor > 1) or smaller
 (PlaceFactor < 1). By default, PlaceFactor is set to 0.5.

\item  \verb|double = obj.GetPlaceFactor ()| -  Set/Get a factor representing the scaling of the widget upon placement
 (via the PlaceWidget() method). Normally the widget is placed so that
 it just fits within the bounding box defined in PlaceWidget(bounds).
 The PlaceFactor will make the widget larger (PlaceFactor > 1) or smaller
 (PlaceFactor < 1). By default, PlaceFactor is set to 0.5.

\item  \verb|obj.SetHandleSize (double )| -  Set/Get the factor that controls the size of the handles that
 appear as part of the widget. These handles (like spheres, etc.)
 are used to manipulate the widget, and are sized as a fraction of
 the screen diagonal.

\item  \verb|double = obj.GetHandleSizeMinValue ()| -  Set/Get the factor that controls the size of the handles that
 appear as part of the widget. These handles (like spheres, etc.)
 are used to manipulate the widget, and are sized as a fraction of
 the screen diagonal.

\item  \verb|double = obj.GetHandleSizeMaxValue ()| -  Set/Get the factor that controls the size of the handles that
 appear as part of the widget. These handles (like spheres, etc.)
 are used to manipulate the widget, and are sized as a fraction of
 the screen diagonal.

\item  \verb|double = obj.GetHandleSize ()| -  Set/Get the factor that controls the size of the handles that
 appear as part of the widget. These handles (like spheres, etc.)
 are used to manipulate the widget, and are sized as a fraction of
 the screen diagonal.

\end{itemize}
