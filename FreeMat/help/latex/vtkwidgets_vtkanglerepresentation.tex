\section{vtkAngleRepresentation}

\subsection{Usage}

 The vtkAngleRepresentation is a superclass for classes representing the
 vtkAngleWidget. This representation consists of two rays and three
 vtkHandleRepresentations to place and manipulate the three points defining
 the angle representation. (Note: the three points are referred to as Point1,
 Center, and Point2, at the two end points (Point1 and Point2) and Center
 (around which the angle is measured).

To create an instance of class vtkAngleRepresentation, simply
invoke its constructor as follows
\begin{verbatim}
  obj = vtkAngleRepresentation
\end{verbatim}
\subsection{Methods}

The class vtkAngleRepresentation has several methods that can be used.
  They are listed below.
Note that the documentation is translated automatically from the VTK sources,
and may not be completely intelligible.  When in doubt, consult the VTK website.
In the methods listed below, \verb|obj| is an instance of the vtkAngleRepresentation class.
\begin{itemize}
\item  \verb|string = obj.GetClassName ()| -  Standard VTK methods.

\item  \verb|int = obj.IsA (string name)| -  Standard VTK methods.

\item  \verb|vtkAngleRepresentation = obj.NewInstance ()| -  Standard VTK methods.

\item  \verb|vtkAngleRepresentation = obj.SafeDownCast (vtkObject o)| -  Standard VTK methods.

\item  \verb|double = obj.GetAngle ()| -  This representation and all subclasses must keep an angle (in degrees)
 consistent with the state of the widget.

\item  \verb|obj.GetPoint1WorldPosition (double pos[3])| -  Methods to Set/Get the coordinates of the three points defining
 this representation. Note that methods are available for both
 display and world coordinates.

\item  \verb|obj.GetCenterWorldPosition (double pos[3])| -  Methods to Set/Get the coordinates of the three points defining
 this representation. Note that methods are available for both
 display and world coordinates.

\item  \verb|obj.GetPoint2WorldPosition (double pos[3])| -  Methods to Set/Get the coordinates of the three points defining
 this representation. Note that methods are available for both
 display and world coordinates.

\item  \verb|obj.SetPoint1DisplayPosition (double pos[3])| -  Methods to Set/Get the coordinates of the three points defining
 this representation. Note that methods are available for both
 display and world coordinates.

\item  \verb|obj.SetCenterDisplayPosition (double pos[3])| -  Methods to Set/Get the coordinates of the three points defining
 this representation. Note that methods are available for both
 display and world coordinates.

\item  \verb|obj.SetPoint2DisplayPosition (double pos[3])| -  Methods to Set/Get the coordinates of the three points defining
 this representation. Note that methods are available for both
 display and world coordinates.

\item  \verb|obj.GetPoint1DisplayPosition (double pos[3])| -  Methods to Set/Get the coordinates of the three points defining
 this representation. Note that methods are available for both
 display and world coordinates.

\item  \verb|obj.GetCenterDisplayPosition (double pos[3])| -  Methods to Set/Get the coordinates of the three points defining
 this representation. Note that methods are available for both
 display and world coordinates.

\item  \verb|obj.GetPoint2DisplayPosition (double pos[3])| -  Methods to Set/Get the coordinates of the three points defining
 this representation. Note that methods are available for both
 display and world coordinates.

\item  \verb|obj.SetHandleRepresentation (vtkHandleRepresentation handle)| -  This method is used to specify the type of handle representation to use
 for the three internal vtkHandleWidgets within vtkAngleRepresentation.
 To use this method, create a dummy vtkHandleRepresentation (or
 subclass), and then invoke this method with this dummy. Then the
 vtkAngleRepresentation uses this dummy to clone three
 vtkHandleRepresentations of the same type. Make sure you set the handle
 representation before the widget is enabled. (The method
 InstantiateHandleRepresentation() is invoked by the vtkAngle widget.)

\item  \verb|obj.InstantiateHandleRepresentation ()| -  This method is used to specify the type of handle representation to use
 for the three internal vtkHandleWidgets within vtkAngleRepresentation.
 To use this method, create a dummy vtkHandleRepresentation (or
 subclass), and then invoke this method with this dummy. Then the
 vtkAngleRepresentation uses this dummy to clone three
 vtkHandleRepresentations of the same type. Make sure you set the handle
 representation before the widget is enabled. (The method
 InstantiateHandleRepresentation() is invoked by the vtkAngle widget.)

\item  \verb|vtkHandleRepresentation = obj.GetPoint1Representation ()| -  Set/Get the handle representations used for the vtkAngleRepresentation.

\item  \verb|vtkHandleRepresentation = obj.GetCenterRepresentation ()| -  Set/Get the handle representations used for the vtkAngleRepresentation.

\item  \verb|vtkHandleRepresentation = obj.GetPoint2Representation ()| -  Set/Get the handle representations used for the vtkAngleRepresentation.

\item  \verb|obj.SetTolerance (int )| -  The tolerance representing the distance to the representation (in
 pixels) in which the cursor is considered near enough to the end points
 of the representation to be active.

\item  \verb|int = obj.GetToleranceMinValue ()| -  The tolerance representing the distance to the representation (in
 pixels) in which the cursor is considered near enough to the end points
 of the representation to be active.

\item  \verb|int = obj.GetToleranceMaxValue ()| -  The tolerance representing the distance to the representation (in
 pixels) in which the cursor is considered near enough to the end points
 of the representation to be active.

\item  \verb|int = obj.GetTolerance ()| -  The tolerance representing the distance to the representation (in
 pixels) in which the cursor is considered near enough to the end points
 of the representation to be active.

\item  \verb|obj.SetLabelFormat (string )| -  Specify the format to use for labelling the angle. Note that an empty
 string results in no label, or a format string without a ''%'' character
 will not print the angle value.

\item  \verb|string = obj.GetLabelFormat ()| -  Specify the format to use for labelling the angle. Note that an empty
 string results in no label, or a format string without a ''%'' character
 will not print the angle value.

\item  \verb|obj.SetRay1Visibility (int )| -  Special methods for turning off the rays and arc that define the cone
 and arc of the angle.

\item  \verb|int = obj.GetRay1Visibility ()| -  Special methods for turning off the rays and arc that define the cone
 and arc of the angle.

\item  \verb|obj.Ray1VisibilityOn ()| -  Special methods for turning off the rays and arc that define the cone
 and arc of the angle.

\item  \verb|obj.Ray1VisibilityOff ()| -  Special methods for turning off the rays and arc that define the cone
 and arc of the angle.

\item  \verb|obj.SetRay2Visibility (int )| -  Special methods for turning off the rays and arc that define the cone
 and arc of the angle.

\item  \verb|int = obj.GetRay2Visibility ()| -  Special methods for turning off the rays and arc that define the cone
 and arc of the angle.

\item  \verb|obj.Ray2VisibilityOn ()| -  Special methods for turning off the rays and arc that define the cone
 and arc of the angle.

\item  \verb|obj.Ray2VisibilityOff ()| -  Special methods for turning off the rays and arc that define the cone
 and arc of the angle.

\item  \verb|obj.SetArcVisibility (int )| -  Special methods for turning off the rays and arc that define the cone
 and arc of the angle.

\item  \verb|int = obj.GetArcVisibility ()| -  Special methods for turning off the rays and arc that define the cone
 and arc of the angle.

\item  \verb|obj.ArcVisibilityOn ()| -  Special methods for turning off the rays and arc that define the cone
 and arc of the angle.

\item  \verb|obj.ArcVisibilityOff ()| -  Special methods for turning off the rays and arc that define the cone
 and arc of the angle.

\item  \verb|obj.BuildRepresentation ()| -  These are methods that satisfy vtkWidgetRepresentation's API.

\item  \verb|int = obj.ComputeInteractionState (int X, int Y, int modify)| -  These are methods that satisfy vtkWidgetRepresentation's API.

\item  \verb|obj.StartWidgetInteraction (double e[2])| -  These are methods that satisfy vtkWidgetRepresentation's API.

\item  \verb|obj.CenterWidgetInteraction (double e[2])| -  These are methods that satisfy vtkWidgetRepresentation's API.

\item  \verb|obj.WidgetInteraction (double e[2])| -  These are methods that satisfy vtkWidgetRepresentation's API.

\end{itemize}
