\section{vtkBiDimensionalRepresentation2D}

\subsection{Usage}

 The vtkBiDimensionalRepresentation2D is used to represent the
 bi-dimensional measure in a 2D (overlay) context. This representation
 consists of two perpendicular lines defined by four
 vtkHandleRepresentations. The four handles can be independently
 manipulated consistent with the orthogonal constraint on the lines. (Note:
 the four points are referred to as Point1, Point2, Point3 and
 Point4. Point1 and Point2 define the first line; and Point3 and Point4
 define the second orthogonal line.)

 To create this widget, you click to place the first two points. The third
 point is mirrored with the fourth point; when you place the third point
 (which is orthogonal to the lined defined by the first two points), the
 fourth point is dropped as well. After definition, the four points can be
 moved (in constrained fashion, preserving orthogonality). Further, the
 entire widget can be translated by grabbing the center point of the widget;
 each line can be moved along the other line; and the entire widget can be
 rotated around its center point.

To create an instance of class vtkBiDimensionalRepresentation2D, simply
invoke its constructor as follows
\begin{verbatim}
  obj = vtkBiDimensionalRepresentation2D
\end{verbatim}
\subsection{Methods}

The class vtkBiDimensionalRepresentation2D has several methods that can be used.
  They are listed below.
Note that the documentation is translated automatically from the VTK sources,
and may not be completely intelligible.  When in doubt, consult the VTK website.
In the methods listed below, \verb|obj| is an instance of the vtkBiDimensionalRepresentation2D class.
\begin{itemize}
\item  \verb|string = obj.GetClassName ()| -  Standard VTK methods.

\item  \verb|int = obj.IsA (string name)| -  Standard VTK methods.

\item  \verb|vtkBiDimensionalRepresentation2D = obj.NewInstance ()| -  Standard VTK methods.

\item  \verb|vtkBiDimensionalRepresentation2D = obj.SafeDownCast (vtkObject o)| -  Standard VTK methods.

\item  \verb|obj.SetPoint1WorldPosition (double pos[3])| -  Methods to Set/Get the coordinates of the four points defining
 this representation. Note that methods are available for both
 display and world coordinates.

\item  \verb|obj.SetPoint2WorldPosition (double pos[3])| -  Methods to Set/Get the coordinates of the four points defining
 this representation. Note that methods are available for both
 display and world coordinates.

\item  \verb|obj.SetPoint3WorldPosition (double pos[3])| -  Methods to Set/Get the coordinates of the four points defining
 this representation. Note that methods are available for both
 display and world coordinates.

\item  \verb|obj.SetPoint4WorldPosition (double pos[3])| -  Methods to Set/Get the coordinates of the four points defining
 this representation. Note that methods are available for both
 display and world coordinates.

\item  \verb|obj.GetPoint1WorldPosition (double pos[3])| -  Methods to Set/Get the coordinates of the four points defining
 this representation. Note that methods are available for both
 display and world coordinates.

\item  \verb|obj.GetPoint2WorldPosition (double pos[3])| -  Methods to Set/Get the coordinates of the four points defining
 this representation. Note that methods are available for both
 display and world coordinates.

\item  \verb|obj.GetPoint3WorldPosition (double pos[3])| -  Methods to Set/Get the coordinates of the four points defining
 this representation. Note that methods are available for both
 display and world coordinates.

\item  \verb|obj.GetPoint4WorldPosition (double pos[3])| -  Methods to Set/Get the coordinates of the four points defining
 this representation. Note that methods are available for both
 display and world coordinates.

\item  \verb|obj.SetPoint1DisplayPosition (double pos[3])| -  Methods to Set/Get the coordinates of the four points defining
 this representation. Note that methods are available for both
 display and world coordinates.

\item  \verb|obj.SetPoint2DisplayPosition (double pos[3])| -  Methods to Set/Get the coordinates of the four points defining
 this representation. Note that methods are available for both
 display and world coordinates.

\item  \verb|obj.SetPoint3DisplayPosition (double pos[3])| -  Methods to Set/Get the coordinates of the four points defining
 this representation. Note that methods are available for both
 display and world coordinates.

\item  \verb|obj.SetPoint4DisplayPosition (double pos[3])| -  Methods to Set/Get the coordinates of the four points defining
 this representation. Note that methods are available for both
 display and world coordinates.

\item  \verb|obj.GetPoint1DisplayPosition (double pos[3])| -  Methods to Set/Get the coordinates of the four points defining
 this representation. Note that methods are available for both
 display and world coordinates.

\item  \verb|obj.GetPoint2DisplayPosition (double pos[3])| -  Methods to Set/Get the coordinates of the four points defining
 this representation. Note that methods are available for both
 display and world coordinates.

\item  \verb|obj.GetPoint3DisplayPosition (double pos[3])| -  Methods to Set/Get the coordinates of the four points defining
 this representation. Note that methods are available for both
 display and world coordinates.

\item  \verb|obj.GetPoint4DisplayPosition (double pos[3])| -  Methods to Set/Get the coordinates of the four points defining
 this representation. Note that methods are available for both
 display and world coordinates.

\item  \verb|obj.SetLine1Visibility (int )| -  Special methods for turning off the lines that define the bi-dimensional
 measure. Generally these methods are used by the vtkBiDimensionalWidget to
 control the appearance of the widget. Note: turning off Line1 actually turns
 off Line1 and Line2.

\item  \verb|int = obj.GetLine1Visibility ()| -  Special methods for turning off the lines that define the bi-dimensional
 measure. Generally these methods are used by the vtkBiDimensionalWidget to
 control the appearance of the widget. Note: turning off Line1 actually turns
 off Line1 and Line2.

\item  \verb|obj.Line1VisibilityOn ()| -  Special methods for turning off the lines that define the bi-dimensional
 measure. Generally these methods are used by the vtkBiDimensionalWidget to
 control the appearance of the widget. Note: turning off Line1 actually turns
 off Line1 and Line2.

\item  \verb|obj.Line1VisibilityOff ()| -  Special methods for turning off the lines that define the bi-dimensional
 measure. Generally these methods are used by the vtkBiDimensionalWidget to
 control the appearance of the widget. Note: turning off Line1 actually turns
 off Line1 and Line2.

\item  \verb|obj.SetLine2Visibility (int )| -  Special methods for turning off the lines that define the bi-dimensional
 measure. Generally these methods are used by the vtkBiDimensionalWidget to
 control the appearance of the widget. Note: turning off Line1 actually turns
 off Line1 and Line2.

\item  \verb|int = obj.GetLine2Visibility ()| -  Special methods for turning off the lines that define the bi-dimensional
 measure. Generally these methods are used by the vtkBiDimensionalWidget to
 control the appearance of the widget. Note: turning off Line1 actually turns
 off Line1 and Line2.

\item  \verb|obj.Line2VisibilityOn ()| -  Special methods for turning off the lines that define the bi-dimensional
 measure. Generally these methods are used by the vtkBiDimensionalWidget to
 control the appearance of the widget. Note: turning off Line1 actually turns
 off Line1 and Line2.

\item  \verb|obj.Line2VisibilityOff ()| -  Special methods for turning off the lines that define the bi-dimensional
 measure. Generally these methods are used by the vtkBiDimensionalWidget to
 control the appearance of the widget. Note: turning off Line1 actually turns
 off Line1 and Line2.

\item  \verb|obj.SetHandleRepresentation (vtkHandleRepresentation handle)| -  This method is used to specify the type of handle representation to use
 for the four internal vtkHandleRepresentations within
 vtkBiDimensionalRepresentation2D.  To use this method, create a dummy
 vtkHandleRepresentation (or subclass), and then invoke this method with
 this dummy. Then the vtkBiDimensionalRepresentation2D uses this dummy to
 clone four vtkHandleRepresentations of the same type. Make sure you set the
 handle representation before the widget is enabled. (The method
 InstantiateHandleRepresentation() is invoked by the vtkBiDimensionalWidget
 for the purposes of cloning.)

\item  \verb|obj.InstantiateHandleRepresentation ()| -  This method is used to specify the type of handle representation to use
 for the four internal vtkHandleRepresentations within
 vtkBiDimensionalRepresentation2D.  To use this method, create a dummy
 vtkHandleRepresentation (or subclass), and then invoke this method with
 this dummy. Then the vtkBiDimensionalRepresentation2D uses this dummy to
 clone four vtkHandleRepresentations of the same type. Make sure you set the
 handle representation before the widget is enabled. (The method
 InstantiateHandleRepresentation() is invoked by the vtkBiDimensionalWidget
 for the purposes of cloning.)

\item  \verb|vtkHandleRepresentation = obj.GetPoint1Representation ()| -  Set/Get the handle representations used within the
 vtkBiDimensionalRepresentation2D. (Note: properties can be set by
 grabbing these representations and setting the properties
 appropriately.)

\item  \verb|vtkHandleRepresentation = obj.GetPoint2Representation ()| -  Set/Get the handle representations used within the
 vtkBiDimensionalRepresentation2D. (Note: properties can be set by
 grabbing these representations and setting the properties
 appropriately.)

\item  \verb|vtkHandleRepresentation = obj.GetPoint3Representation ()| -  Set/Get the handle representations used within the
 vtkBiDimensionalRepresentation2D. (Note: properties can be set by
 grabbing these representations and setting the properties
 appropriately.)

\item  \verb|vtkHandleRepresentation = obj.GetPoint4Representation ()| -  Set/Get the handle representations used within the
 vtkBiDimensionalRepresentation2D. (Note: properties can be set by
 grabbing these representations and setting the properties
 appropriately.)

\item  \verb|vtkProperty2D = obj.GetLineProperty ()| -  Retrieve the property used to control the appearance of the two
 orthogonal lines.

\item  \verb|vtkProperty2D = obj.GetSelectedLineProperty ()| -  Retrieve the property used to control the appearance of the two
 orthogonal lines.

\item  \verb|vtkTextProperty = obj.GetTextProperty ()| -  Retrieve the property used to control the appearance of the text
 labels.

\item  \verb|obj.SetTolerance (int )| -  The tolerance representing the distance to the representation (in
 pixels) in which the cursor is considered near enough to the
 representation to be active.

\item  \verb|int = obj.GetToleranceMinValue ()| -  The tolerance representing the distance to the representation (in
 pixels) in which the cursor is considered near enough to the
 representation to be active.

\item  \verb|int = obj.GetToleranceMaxValue ()| -  The tolerance representing the distance to the representation (in
 pixels) in which the cursor is considered near enough to the
 representation to be active.

\item  \verb|int = obj.GetTolerance ()| -  The tolerance representing the distance to the representation (in
 pixels) in which the cursor is considered near enough to the
 representation to be active.

\item  \verb|double = obj.GetLength1 ()| -  Return the length of the line defined by (Point1,Point2). This is the
 distance in the world coordinate system.

\item  \verb|double = obj.GetLength2 ()| -  Return the length of the line defined by (Point3,Point4). This is the
 distance in the world coordinate system.

\item  \verb|obj.SetLabelFormat (string )| -  Specify the format to use for labelling the distance. Note that an empty
 string results in no label, or a format string without a ''%'' character
 will not print the distance value.

\item  \verb|string = obj.GetLabelFormat ()| -  Specify the format to use for labelling the distance. Note that an empty
 string results in no label, or a format string without a ''%'' character
 will not print the distance value.

\item  \verb|obj.BuildRepresentation ()| -  These are methods that satisfy vtkWidgetRepresentation's API.

\item  \verb|int = obj.ComputeInteractionState (int X, int Y, int modify)| -  These are methods that satisfy vtkWidgetRepresentation's API.

\item  \verb|obj.StartWidgetDefinition (double e[2])| -  These are methods that satisfy vtkWidgetRepresentation's API.

\item  \verb|obj.Point2WidgetInteraction (double e[2])| -  These are methods that satisfy vtkWidgetRepresentation's API.

\item  \verb|obj.Point3WidgetInteraction (double e[2])| -  These are methods that satisfy vtkWidgetRepresentation's API.

\item  \verb|obj.StartWidgetManipulation (double e[2])| -  These are methods that satisfy vtkWidgetRepresentation's API.

\item  \verb|obj.WidgetInteraction (double e[2])| -  These are methods that satisfy vtkWidgetRepresentation's API.

\item  \verb|obj.Highlight (int highlightOn)| -  These are methods that satisfy vtkWidgetRepresentation's API.

\item  \verb|obj.ReleaseGraphicsResources (vtkWindow w)| -  Methods required by vtkProp superclass.

\item  \verb|int = obj.RenderOverlay (vtkViewport viewport)| -  Methods required by vtkProp superclass.

\item  \verb|obj.SetShowLabelAboveWidget (int )| -  Toggle whether to display the label above or below the widget.
 Defaults to 1.

\item  \verb|int = obj.GetShowLabelAboveWidget ()| -  Toggle whether to display the label above or below the widget.
 Defaults to 1.

\item  \verb|obj.ShowLabelAboveWidgetOn ()| -  Toggle whether to display the label above or below the widget.
 Defaults to 1.

\item  \verb|obj.ShowLabelAboveWidgetOff ()| -  Toggle whether to display the label above or below the widget.
 Defaults to 1.

\item  \verb|obj.SetID (long id)| -  Set/get the id to display in the label.

\item  \verb|long = obj.GetID ()| -  Set/get the id to display in the label.

\item  \verb|string = obj.GetLabelText ()| -  Get the text shown in the widget's label.

\item  \verb|obj.GetLabelPosition (double pos[3])| -  Get the position of the widget's label in display coordinates.

\item  \verb|obj.GetWorldLabelPosition (double pos[3])| -  Get the position of the widget's label in display coordinates.

\end{itemize}
