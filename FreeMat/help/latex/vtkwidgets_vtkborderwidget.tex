\section{vtkBorderWidget}

\subsection{Usage}

 This class is a superclass for 2D widgets that may require a rectangular
 border. Besides drawing a border, the widget provides methods for resizing
 and moving the rectangular region (and associated border). The widget
 provides methods and internal data members so that subclasses can take
 advantage of this widgets capabilities, requiring only that the subclass
 defines a ''representation'', i.e., some combination of props or actors
 that can be managed in the 2D rectangular region. 

 The class defines basic positioning functionality, including the ability
 to size the widget with locked x/y proportions. The area within the border
 may be made ''selectable'' as well, meaning that a selection event interior
 to the widget invokes a virtual SelectRegion() method, which can be used
 to pick objects or otherwise manipulate data interior to the widget.

 .SECTION Event Bindings
 By default, the widget responds to the following VTK events (i.e., it
 watches the vtkRenderWindowInteractor for these events):
 \begin{verbatim}
 On the boundary of the widget:
   LeftButtonPressEvent - select boundary
   LeftButtonReleaseEvent - deselect boundary
   MouseMoveEvent - move/resize widget depending on which portion of the
                    boundary was selected.
 On the interior of the widget:
   LeftButtonPressEvent - invoke SelectButton() callback (if the ivar
                          Selectable is on)
 Anywhere on the widget:
   MiddleButtonPressEvent - move the widget
 \end{verbatim}

 Note that the event bindings described above can be changed using this
 class's vtkWidgetEventTranslator. This class translates VTK events 
 into the vtkBorderWidget's widget events:
 \begin{verbatim}
   vtkWidgetEvent::Select -- some part of the widget has been selected
   vtkWidgetEvent::EndSelect -- the selection process has completed
   vtkWidgetEvent::Translate -- the widget is to be translated
   vtkWidgetEvent::Move -- a request for slider motion has been invoked
 \end{verbatim}

 In turn, when these widget events are processed, this widget invokes the
 following VTK events on itself (which observers can listen for):
 \begin{verbatim}
   vtkCommand::StartInteractionEvent (on vtkWidgetEvent::Select)
   vtkCommand::EndInteractionEvent (on vtkWidgetEvent::EndSelect)
   vtkCommand::InteractionEvent (on vtkWidgetEvent::Move)
 \end{verbatim}

To create an instance of class vtkBorderWidget, simply
invoke its constructor as follows
\begin{verbatim}
  obj = vtkBorderWidget
\end{verbatim}
\subsection{Methods}

The class vtkBorderWidget has several methods that can be used.
  They are listed below.
Note that the documentation is translated automatically from the VTK sources,
and may not be completely intelligible.  When in doubt, consult the VTK website.
In the methods listed below, \verb|obj| is an instance of the vtkBorderWidget class.
\begin{itemize}
\item  \verb|string = obj.GetClassName ()|

\item  \verb|int = obj.IsA (string name)|

\item  \verb|vtkBorderWidget = obj.NewInstance ()|

\item  \verb|vtkBorderWidget = obj.SafeDownCast (vtkObject o)|

\item  \verb|obj.SetSelectable (int )| -  Indicate whether the interior region of the widget can be selected or
 not. If not, then events (such as left mouse down) allow the user to
 ''move'' the widget, and no selection is possible. Otherwise the
 SelectRegion() method is invoked.

\item  \verb|int = obj.GetSelectable ()| -  Indicate whether the interior region of the widget can be selected or
 not. If not, then events (such as left mouse down) allow the user to
 ''move'' the widget, and no selection is possible. Otherwise the
 SelectRegion() method is invoked.

\item  \verb|obj.SelectableOn ()| -  Indicate whether the interior region of the widget can be selected or
 not. If not, then events (such as left mouse down) allow the user to
 ''move'' the widget, and no selection is possible. Otherwise the
 SelectRegion() method is invoked.

\item  \verb|obj.SelectableOff ()| -  Indicate whether the interior region of the widget can be selected or
 not. If not, then events (such as left mouse down) allow the user to
 ''move'' the widget, and no selection is possible. Otherwise the
 SelectRegion() method is invoked.

\item  \verb|obj.SetResizable (int )| -  Indicate whether the boundary of the widget can be resized.
 If not, the cursor will not change to ''resize'' type when mouse
 over the boundary.

\item  \verb|int = obj.GetResizable ()| -  Indicate whether the boundary of the widget can be resized.
 If not, the cursor will not change to ''resize'' type when mouse
 over the boundary.

\item  \verb|obj.ResizableOn ()| -  Indicate whether the boundary of the widget can be resized.
 If not, the cursor will not change to ''resize'' type when mouse
 over the boundary.

\item  \verb|obj.ResizableOff ()| -  Indicate whether the boundary of the widget can be resized.
 If not, the cursor will not change to ''resize'' type when mouse
 over the boundary.

\item  \verb|obj.SetRepresentation (vtkBorderRepresentation r)| -  Create the default widget representation if one is not set. 

\item  \verb|obj.CreateDefaultRepresentation ()| -  Create the default widget representation if one is not set. 

\end{itemize}
