\section{vtkCameraWidget}

\subsection{Usage}

 This class provides support for interactively saving a series of camera
 views into an interpolated path (using vtkCameraInterpolator). To use the
 class start by specifying a camera to interpolate, and then simply start
 recording by hitting the ''record'' button, manipulate the camera (by using
 an interactor, direct scripting, or any other means), and then save the
 camera view. Repeat this process to record a series of views.  The user
 can then play back interpolated camera views using the 
 vtkCameraInterpolator.

To create an instance of class vtkCameraWidget, simply
invoke its constructor as follows
\begin{verbatim}
  obj = vtkCameraWidget
\end{verbatim}
\subsection{Methods}

The class vtkCameraWidget has several methods that can be used.
  They are listed below.
Note that the documentation is translated automatically from the VTK sources,
and may not be completely intelligible.  When in doubt, consult the VTK website.
In the methods listed below, \verb|obj| is an instance of the vtkCameraWidget class.
\begin{itemize}
\item  \verb|string = obj.GetClassName ()| -  Standar VTK class methods.

\item  \verb|int = obj.IsA (string name)| -  Standar VTK class methods.

\item  \verb|vtkCameraWidget = obj.NewInstance ()| -  Standar VTK class methods.

\item  \verb|vtkCameraWidget = obj.SafeDownCast (vtkObject o)| -  Standar VTK class methods.

\item  \verb|obj.SetRepresentation (vtkCameraRepresentation r)| -  Create the default widget representation if one is not set. 

\item  \verb|obj.CreateDefaultRepresentation ()| -  Create the default widget representation if one is not set. 

\end{itemize}
