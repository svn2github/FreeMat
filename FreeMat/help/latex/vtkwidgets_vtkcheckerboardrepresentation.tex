\section{vtkCheckerboardRepresentation}

\subsection{Usage}

 The vtkCheckerboardRepresentation is used to implement the representation of
 the vtkCheckerboardWidget. The user can adjust the number of divisions in
 each of the i-j directions in a 2D image. A frame appears around the
 vtkImageActor with sliders along each side of the frame. The user can
 interactively adjust the sliders to the desired number of checkerboard
 subdivisions. The representation uses four instances of 
 vtkSliderRepresentation3D to implement itself.

To create an instance of class vtkCheckerboardRepresentation, simply
invoke its constructor as follows
\begin{verbatim}
  obj = vtkCheckerboardRepresentation
\end{verbatim}
\subsection{Methods}

The class vtkCheckerboardRepresentation has several methods that can be used.
  They are listed below.
Note that the documentation is translated automatically from the VTK sources,
and may not be completely intelligible.  When in doubt, consult the VTK website.
In the methods listed below, \verb|obj| is an instance of the vtkCheckerboardRepresentation class.
\begin{itemize}
\item  \verb|string = obj.GetClassName ()| -  Standard VTK methods.

\item  \verb|int = obj.IsA (string name)| -  Standard VTK methods.

\item  \verb|vtkCheckerboardRepresentation = obj.NewInstance ()| -  Standard VTK methods.

\item  \verb|vtkCheckerboardRepresentation = obj.SafeDownCast (vtkObject o)| -  Standard VTK methods.

\item  \verb|obj.SetCheckerboard (vtkImageCheckerboard chkrbrd)| -  Specify an instance of vtkImageCheckerboard to manipulate.

\item  \verb|vtkImageCheckerboard = obj.GetCheckerboard ()| -  Specify an instance of vtkImageCheckerboard to manipulate.

\item  \verb|obj.SetImageActor (vtkImageActor imageActor)| -  Specify an instance of vtkImageActor to decorate.

\item  \verb|vtkImageActor = obj.GetImageActor ()| -  Specify an instance of vtkImageActor to decorate.

\item  \verb|obj.SetCornerOffset (double )| -  Specify the offset of the ends of the sliders (on the boundary edges of
 the image) from the corner of the image. The offset is expressed as a
 normalized fraction of the border edges.

\item  \verb|double = obj.GetCornerOffsetMinValue ()| -  Specify the offset of the ends of the sliders (on the boundary edges of
 the image) from the corner of the image. The offset is expressed as a
 normalized fraction of the border edges.

\item  \verb|double = obj.GetCornerOffsetMaxValue ()| -  Specify the offset of the ends of the sliders (on the boundary edges of
 the image) from the corner of the image. The offset is expressed as a
 normalized fraction of the border edges.

\item  \verb|double = obj.GetCornerOffset ()| -  Specify the offset of the ends of the sliders (on the boundary edges of
 the image) from the corner of the image. The offset is expressed as a
 normalized fraction of the border edges.

\item  \verb|obj.SliderValueChanged (int sliderNum)| -  This method is invoked by the vtkCheckerboardWidget() when a value of some
 slider has changed.

\item  \verb|obj.SetTopRepresentation (vtkSliderRepresentation3D )| -  Set and get the instances of vtkSliderRepresention used to implement this
 representation. Normally default representations are created, but you can
 specify the ones you want to use.

\item  \verb|obj.SetRightRepresentation (vtkSliderRepresentation3D )| -  Set and get the instances of vtkSliderRepresention used to implement this
 representation. Normally default representations are created, but you can
 specify the ones you want to use.

\item  \verb|obj.SetBottomRepresentation (vtkSliderRepresentation3D )| -  Set and get the instances of vtkSliderRepresention used to implement this
 representation. Normally default representations are created, but you can
 specify the ones you want to use.

\item  \verb|obj.SetLeftRepresentation (vtkSliderRepresentation3D )| -  Set and get the instances of vtkSliderRepresention used to implement this
 representation. Normally default representations are created, but you can
 specify the ones you want to use.

\item  \verb|vtkSliderRepresentation3D = obj.GetTopRepresentation ()| -  Set and get the instances of vtkSliderRepresention used to implement this
 representation. Normally default representations are created, but you can
 specify the ones you want to use.

\item  \verb|vtkSliderRepresentation3D = obj.GetRightRepresentation ()| -  Set and get the instances of vtkSliderRepresention used to implement this
 representation. Normally default representations are created, but you can
 specify the ones you want to use.

\item  \verb|vtkSliderRepresentation3D = obj.GetBottomRepresentation ()| -  Set and get the instances of vtkSliderRepresention used to implement this
 representation. Normally default representations are created, but you can
 specify the ones you want to use.

\item  \verb|vtkSliderRepresentation3D = obj.GetLeftRepresentation ()| -  Set and get the instances of vtkSliderRepresention used to implement this
 representation. Normally default representations are created, but you can
 specify the ones you want to use.

\item  \verb|obj.BuildRepresentation ()| -  Methods required by superclass.

\item  \verb|obj.GetActors (vtkPropCollection )| -  Methods required by superclass.

\item  \verb|obj.ReleaseGraphicsResources (vtkWindow w)| -  Methods required by superclass.

\item  \verb|int = obj.RenderOverlay (vtkViewport viewport)| -  Methods required by superclass.

\item  \verb|int = obj.RenderOpaqueGeometry (vtkViewport viewport)| -  Methods required by superclass.

\item  \verb|int = obj.RenderTranslucentPolygonalGeometry (vtkViewport viewport)| -  Methods required by superclass.

\item  \verb|int = obj.HasTranslucentPolygonalGeometry ()| -  Methods required by superclass.

\end{itemize}
