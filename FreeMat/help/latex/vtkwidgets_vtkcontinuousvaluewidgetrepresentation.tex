\section{vtkContinuousValueWidgetRepresentation}

\subsection{Usage}

 This class is used mainly as a superclass for continuous value widgets


To create an instance of class vtkContinuousValueWidgetRepresentation, simply
invoke its constructor as follows
\begin{verbatim}
  obj = vtkContinuousValueWidgetRepresentation
\end{verbatim}
\subsection{Methods}

The class vtkContinuousValueWidgetRepresentation has several methods that can be used.
  They are listed below.
Note that the documentation is translated automatically from the VTK sources,
and may not be completely intelligible.  When in doubt, consult the VTK website.
In the methods listed below, \verb|obj| is an instance of the vtkContinuousValueWidgetRepresentation class.
\begin{itemize}
\item  \verb|string = obj.GetClassName ()| -  Standard methods for the class.

\item  \verb|int = obj.IsA (string name)| -  Standard methods for the class.

\item  \verb|vtkContinuousValueWidgetRepresentation = obj.NewInstance ()| -  Standard methods for the class.

\item  \verb|vtkContinuousValueWidgetRepresentation = obj.SafeDownCast (vtkObject o)| -  Standard methods for the class.

\item  \verb|obj.PlaceWidget (double bounds[6])| -  Methods to interface with the vtkSliderWidget. The PlaceWidget() method
 assumes that the parameter bounds[6] specifies the location in display
 space where the widget should be placed.

\item  \verb|obj.BuildRepresentation ()| -  Methods to interface with the vtkSliderWidget. The PlaceWidget() method
 assumes that the parameter bounds[6] specifies the location in display
 space where the widget should be placed.

\item  \verb|obj.StartWidgetInteraction (double eventPos[2])| -  Methods to interface with the vtkSliderWidget. The PlaceWidget() method
 assumes that the parameter bounds[6] specifies the location in display
 space where the widget should be placed.

\item  \verb|obj.WidgetInteraction (double eventPos[2])| -  Methods to interface with the vtkSliderWidget. The PlaceWidget() method
 assumes that the parameter bounds[6] specifies the location in display
 space where the widget should be placed.

\item  \verb|obj.SetValue (double value)|

\item  \verb|double = obj.GetValue ()|

\end{itemize}
