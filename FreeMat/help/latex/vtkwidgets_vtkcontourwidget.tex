\section{vtkContourWidget}

\subsection{Usage}

 The vtkContourWidget is used to select a set of points, and draw lines
 between these points. The contour may be opened or closed, depending on
 how the last point is added. The widget handles all processing of widget
 events (that are triggered by VTK events). The vtkContourRepresentation is
 responsible for all placement of the points, calculation of the lines, and
 contour manipulation. This is done through two main helper classes:
 vtkPointPlacer and vtkContourLineInterpolator. The representation is also
 responsible for drawing the points and lines.

 .SECTION Event Bindings
 By default, the widget responds to the following VTK events (i.e., it
 watches the vtkRenderWindowInteractor for these events):
 \begin{verbatim}
   LeftButtonPressEvent - triggers a Select event
   RightButtonPressEvent - triggers a AddFinalPoint event
   MouseMoveEvent - triggers a Move event
   LeftButtonReleaseEvent - triggers an EndSelect event
   Delete key event - triggers a Delete event
   Shift + Delete key event - triggers a Reset event
 \end{verbatim}

 Note that the event bindings described above can be changed using this
 class's vtkWidgetEventTranslator. This class translates VTK events 
 into the vtkContourWidget's widget events:
 \begin{verbatim}
   vtkWidgetEvent::Select 
        widget state is: 
            Start or
            Define: If we already have at least 2 nodes, test
                 whether the current (X,Y) location is near an existing
                 node. If so, close the contour and change to Manipulate
                 state. Otherwise, attempt to add a node at this (X,Y)
                 location.
            Manipulate: If this (X,Y) location activates a node, then
                 set the current operation to Translate. Otherwise, if
                 this location is near the contour, attempt to add a 
                 new node on the contour at this (X,Y) location.

   vtkWidgetEvent::AddFinalPoint
        widget state is: 
            Start: Do nothing.
            Define: If we already have at least 2 nodes, test
                 whether the current (X,Y) location is near an existing
                 node. If so, close the contour and change to Manipulate
                 state. Otherwise, attempt to add a node at this (X,Y)
                 location. If we do, then leave the contour open and
                 change to Manipulate state.
            Manipulate: Do nothing.

   vtkWidgetEvent::Move
        widget state is: 
            Start or
            Define: Do nothing.
            Manipulate: If our operation is Translate, then invoke
                  WidgetInteraction() on the representation. If our 
                  operation is Inactive, then just attempt to activate
                  a node at this (X,Y) location.

   vtkWidgetEvent::EndSelect
        widget state is: 
            Start or
            Define: Do nothing.
            Manipulate: If our operation is not Inactive, set it to
                  Inactive.

   vtkWidgetEvent::Delete
        widget state is: 
            Start: Do nothing.
            Define: Remove the last point on the contour.
            Manipulate: Attempt to activate a node at (X,Y). If
                   we do activate a node, delete it. If we now
                   have less than 3 nodes, go back to Define state.

   vtkWidgetEvent::Reset
        widget state is: 
            Start: Do nothing.
            Define: Remove all points and line segments of the contour.
                 Essentially calls Intialize(NULL) 
            Manipulate: Do nothing.
 \end{verbatim}

 This widget invokes the following VTK events on itself (which observers
 can listen for):
 \begin{verbatim}
   vtkCommand::StartInteractionEvent (beginning to interact)
   vtkCommand::EndInteractionEvent (completing interaction)
   vtkCommand::InteractionEvent (moving after selecting something)
   vtkCommand::PlacePointEvent (after point is positioned; 
                                call data includes handle id (0,1))
   vtkCommand::WidgetValueChangedEvent (Invoked when the contour is closed
                                        for the first time. )
 \end{verbatim}

To create an instance of class vtkContourWidget, simply
invoke its constructor as follows
\begin{verbatim}
  obj = vtkContourWidget
\end{verbatim}
\subsection{Methods}

The class vtkContourWidget has several methods that can be used.
  They are listed below.
Note that the documentation is translated automatically from the VTK sources,
and may not be completely intelligible.  When in doubt, consult the VTK website.
In the methods listed below, \verb|obj| is an instance of the vtkContourWidget class.
\begin{itemize}
\item  \verb|string = obj.GetClassName ()| -  Standard methods for a VTK class.

\item  \verb|int = obj.IsA (string name)| -  Standard methods for a VTK class.

\item  \verb|vtkContourWidget = obj.NewInstance ()| -  Standard methods for a VTK class.

\item  \verb|vtkContourWidget = obj.SafeDownCast (vtkObject o)| -  Standard methods for a VTK class.

\item  \verb|obj.SetEnabled (int )| -  The method for activiating and deactiviating this widget. This method
 must be overridden because it is a composite widget and does more than
 its superclasses' vtkAbstractWidget::SetEnabled() method.

\item  \verb|obj.SetRepresentation (vtkContourRepresentation r)| -  Create the default widget representation if one is not set. 

\item  \verb|obj.CreateDefaultRepresentation ()| -  Create the default widget representation if one is not set. 

\item  \verb|obj.CloseLoop ()| -  Convenient method to close the contour loop.

\item  \verb|obj.SetAllowNodePicking (int )| -  Set / Get the AllowNodePicking value. This ivar indicates whether the nodes
 and points between nodes can be picked/un-picked by Ctrl+Click on the node.

\item  \verb|int = obj.GetAllowNodePicking ()| -  Set / Get the AllowNodePicking value. This ivar indicates whether the nodes
 and points between nodes can be picked/un-picked by Ctrl+Click on the node.

\item  \verb|obj.AllowNodePickingOn ()| -  Set / Get the AllowNodePicking value. This ivar indicates whether the nodes
 and points between nodes can be picked/un-picked by Ctrl+Click on the node.

\item  \verb|obj.AllowNodePickingOff ()| -  Set / Get the AllowNodePicking value. This ivar indicates whether the nodes
 and points between nodes can be picked/un-picked by Ctrl+Click on the node.

\item  \verb|obj.SetFollowCursor (int )| -  Follow the cursor ? If this is ON, during definition, the last node of the
 contour will automatically follow the cursor, without waiting for the the
 point to be dropped. This may be useful for some interpolators, such as the
 live-wire interpolator to see the shape of the contour that will be placed
 as you move the mouse cursor.

\item  \verb|int = obj.GetFollowCursor ()| -  Follow the cursor ? If this is ON, during definition, the last node of the
 contour will automatically follow the cursor, without waiting for the the
 point to be dropped. This may be useful for some interpolators, such as the
 live-wire interpolator to see the shape of the contour that will be placed
 as you move the mouse cursor.

\item  \verb|obj.FollowCursorOn ()| -  Follow the cursor ? If this is ON, during definition, the last node of the
 contour will automatically follow the cursor, without waiting for the the
 point to be dropped. This may be useful for some interpolators, such as the
 live-wire interpolator to see the shape of the contour that will be placed
 as you move the mouse cursor.

\item  \verb|obj.FollowCursorOff ()| -  Follow the cursor ? If this is ON, during definition, the last node of the
 contour will automatically follow the cursor, without waiting for the the
 point to be dropped. This may be useful for some interpolators, such as the
 live-wire interpolator to see the shape of the contour that will be placed
 as you move the mouse cursor.

\item  \verb|obj.SetContinuousDraw (int )| -  Define a contour by continuously drawing with the mouse cursor.
 Press and hold the left mouse button down to continuously draw.
 Releasing the left mouse button switches into a snap drawing mode.
 Terminate the contour by pressing the right mouse button.  If you
 do not want to see the nodes as they are added to the contour, set the
 opacity to 0 of the representation's property.  If you do not want to 
 see the last active node as it is being added, set the opacity to 0 
 of the representation's active property.

\item  \verb|int = obj.GetContinuousDraw ()| -  Define a contour by continuously drawing with the mouse cursor.
 Press and hold the left mouse button down to continuously draw.
 Releasing the left mouse button switches into a snap drawing mode.
 Terminate the contour by pressing the right mouse button.  If you
 do not want to see the nodes as they are added to the contour, set the
 opacity to 0 of the representation's property.  If you do not want to 
 see the last active node as it is being added, set the opacity to 0 
 of the representation's active property.

\item  \verb|obj.ContinuousDrawOn ()| -  Define a contour by continuously drawing with the mouse cursor.
 Press and hold the left mouse button down to continuously draw.
 Releasing the left mouse button switches into a snap drawing mode.
 Terminate the contour by pressing the right mouse button.  If you
 do not want to see the nodes as they are added to the contour, set the
 opacity to 0 of the representation's property.  If you do not want to 
 see the last active node as it is being added, set the opacity to 0 
 of the representation's active property.

\item  \verb|obj.ContinuousDrawOff ()| -  Define a contour by continuously drawing with the mouse cursor.
 Press and hold the left mouse button down to continuously draw.
 Releasing the left mouse button switches into a snap drawing mode.
 Terminate the contour by pressing the right mouse button.  If you
 do not want to see the nodes as they are added to the contour, set the
 opacity to 0 of the representation's property.  If you do not want to 
 see the last active node as it is being added, set the opacity to 0 
 of the representation's active property.

\item  \verb|obj.Initialize (vtkPolyData poly, int state)| -  Initialize the contour widget from a user supplied set of points. The
 state of the widget decides if you are still defining the widget, or
 if you've finished defining (added the last point) are manipulating
 it. Note that if the polydata supplied is closed, the state will be
 set to manipulate.
  State: Define = 0, Manipulate = 1.

\item  \verb|obj.Initialize ()|

\end{itemize}
