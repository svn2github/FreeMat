\section{vtkDistanceWidget}

\subsection{Usage}

 The vtkDistanceWidget is used to measure the distance between two points.
 The two end points can be positioned independently, and when they are
 released, a special PlacePointEvent is invoked so that special operations
 may be take to reposition the point (snap to grid, etc.) The widget has
 two different modes of interaction: when initially defined (i.e., placing
 the two points) and then a manipulate mode (adjusting the position of
 the two points).

 To use this widget, specify an instance of vtkDistanceWidget and a
 representation (a subclass of vtkDistanceRepresentation). The widget is
 implemented using two instances of vtkHandleWidget which are used to
 position the end points of the line. The representations for these two
 handle widgets are provided by the vtkDistanceRepresentation.

 .SECTION Event Bindings
 By default, the widget responds to the following VTK events (i.e., it
 watches the vtkRenderWindowInteractor for these events):
 \begin{verbatim}
   LeftButtonPressEvent - add a point or select a handle
   MouseMoveEvent - position the second point or move a handle
   LeftButtonReleaseEvent - release the handle
 \end{verbatim}

 Note that the event bindings described above can be changed using this
 class's vtkWidgetEventTranslator. This class translates VTK events
 into the vtkDistanceWidget's widget events:
 \begin{verbatim}
   vtkWidgetEvent::AddPoint -- add one point; depending on the state
                               it may the first or second point added. Or,
                               if near a handle, select the handle.
   vtkWidgetEvent::Move -- move the second point or handle depending on the state.
   vtkWidgetEvent::EndSelect -- the handle manipulation process has completed.
 \end{verbatim}

 This widget invokes the following VTK events on itself (which observers
 can listen for):
 \begin{verbatim}
   vtkCommand::StartInteractionEvent (beginning to interact)
   vtkCommand::EndInteractionEvent (completing interaction)
   vtkCommand::InteractionEvent (moving after selecting something)
   vtkCommand::PlacePointEvent (after point is positioned;
                                call data includes handle id (0,1))
 \end{verbatim}

To create an instance of class vtkDistanceWidget, simply
invoke its constructor as follows
\begin{verbatim}
  obj = vtkDistanceWidget
\end{verbatim}
\subsection{Methods}

The class vtkDistanceWidget has several methods that can be used.
  They are listed below.
Note that the documentation is translated automatically from the VTK sources,
and may not be completely intelligible.  When in doubt, consult the VTK website.
In the methods listed below, \verb|obj| is an instance of the vtkDistanceWidget class.
\begin{itemize}
\item  \verb|string = obj.GetClassName ()| -  Standard methods for a VTK class.

\item  \verb|int = obj.IsA (string name)| -  Standard methods for a VTK class.

\item  \verb|vtkDistanceWidget = obj.NewInstance ()| -  Standard methods for a VTK class.

\item  \verb|vtkDistanceWidget = obj.SafeDownCast (vtkObject o)| -  Standard methods for a VTK class.

\item  \verb|obj.SetEnabled (int )| -  The method for activiating and deactiviating this widget. This method
 must be overridden because it is a composite widget and does more than
 its superclasses' vtkAbstractWidget::SetEnabled() method.

\item  \verb|obj.SetRepresentation (vtkDistanceRepresentation r)| -  Create the default widget representation if one is not set.

\item  \verb|obj.CreateDefaultRepresentation ()| -  Create the default widget representation if one is not set.

\item  \verb|obj.SetProcessEvents (int )| -  Methods to change the whether the widget responds to interaction.
 Overridden to pass the state to component widgets.

\end{itemize}
