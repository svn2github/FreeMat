\section{vtkHandleRepresentation}

\subsection{Usage}

 This class defines an API for widget handle representations. These
 representations interact with vtkHandleWidget. Various representations
 can be used depending on the nature of the handle. The basic functionality 
 of the handle representation is to maintain a position. The position is
 represented via a vtkCoordinate, meaning that the position can be easily
 obtained in a variety of coordinate systems.

 Optional features for this representation include an active mode (the widget
 appears only when the mouse pointer is close to it). The active distance is
 expressed in pixels and represents a circle in display space.

 The class may be subclassed so that alternative representations can
 be created.  The class defines an API and a default implementation that
 the vtkHandleWidget interacts with to render itself in the scene.

To create an instance of class vtkHandleRepresentation, simply
invoke its constructor as follows
\begin{verbatim}
  obj = vtkHandleRepresentation
\end{verbatim}
\subsection{Methods}

The class vtkHandleRepresentation has several methods that can be used.
  They are listed below.
Note that the documentation is translated automatically from the VTK sources,
and may not be completely intelligible.  When in doubt, consult the VTK website.
In the methods listed below, \verb|obj| is an instance of the vtkHandleRepresentation class.
\begin{itemize}
\item  \verb|string = obj.GetClassName ()| -  Standard methods for instances of this class.

\item  \verb|int = obj.IsA (string name)| -  Standard methods for instances of this class.

\item  \verb|vtkHandleRepresentation = obj.NewInstance ()| -  Standard methods for instances of this class.

\item  \verb|vtkHandleRepresentation = obj.SafeDownCast (vtkObject o)| -  Standard methods for instances of this class.

\item  \verb|obj.SetDisplayPosition (double pos[3])| -  Handles usually have their coordinates set in display coordinates
 (generally by an associated widget) and internally maintain the position
 in world coordinates. (Using world coordinates insures that handles are
 rendered in the right position when the camera view changes.) These
 methods are often subclassed because special constraint operations can
 be used to control the actual positioning.

\item  \verb|obj.GetDisplayPosition (double pos[3])| -  Handles usually have their coordinates set in display coordinates
 (generally by an associated widget) and internally maintain the position
 in world coordinates. (Using world coordinates insures that handles are
 rendered in the right position when the camera view changes.) These
 methods are often subclassed because special constraint operations can
 be used to control the actual positioning.

\item  \verb|double = obj.GetDisplayPosition ()| -  Handles usually have their coordinates set in display coordinates
 (generally by an associated widget) and internally maintain the position
 in world coordinates. (Using world coordinates insures that handles are
 rendered in the right position when the camera view changes.) These
 methods are often subclassed because special constraint operations can
 be used to control the actual positioning.

\item  \verb|obj.SetWorldPosition (double pos[3])| -  Handles usually have their coordinates set in display coordinates
 (generally by an associated widget) and internally maintain the position
 in world coordinates. (Using world coordinates insures that handles are
 rendered in the right position when the camera view changes.) These
 methods are often subclassed because special constraint operations can
 be used to control the actual positioning.

\item  \verb|obj.GetWorldPosition (double pos[3])| -  Handles usually have their coordinates set in display coordinates
 (generally by an associated widget) and internally maintain the position
 in world coordinates. (Using world coordinates insures that handles are
 rendered in the right position when the camera view changes.) These
 methods are often subclassed because special constraint operations can
 be used to control the actual positioning.

\item  \verb|double = obj.GetWorldPosition ()| -  Handles usually have their coordinates set in display coordinates
 (generally by an associated widget) and internally maintain the position
 in world coordinates. (Using world coordinates insures that handles are
 rendered in the right position when the camera view changes.) These
 methods are often subclassed because special constraint operations can
 be used to control the actual positioning.

\item  \verb|obj.SetTolerance (int )| -  The tolerance representing the distance to the widget (in pixels)
 in which the cursor is considered near enough to the widget to
 be active.

\item  \verb|int = obj.GetToleranceMinValue ()| -  The tolerance representing the distance to the widget (in pixels)
 in which the cursor is considered near enough to the widget to
 be active.

\item  \verb|int = obj.GetToleranceMaxValue ()| -  The tolerance representing the distance to the widget (in pixels)
 in which the cursor is considered near enough to the widget to
 be active.

\item  \verb|int = obj.GetTolerance ()| -  The tolerance representing the distance to the widget (in pixels)
 in which the cursor is considered near enough to the widget to
 be active.

\item  \verb|obj.SetActiveRepresentation (int )| -  Flag controls whether the widget becomes visible when the mouse pointer
 moves close to it (i.e., the widget becomes active). By default,
 ActiveRepresentation is off and the representation is always visible.

\item  \verb|int = obj.GetActiveRepresentation ()| -  Flag controls whether the widget becomes visible when the mouse pointer
 moves close to it (i.e., the widget becomes active). By default,
 ActiveRepresentation is off and the representation is always visible.

\item  \verb|obj.ActiveRepresentationOn ()| -  Flag controls whether the widget becomes visible when the mouse pointer
 moves close to it (i.e., the widget becomes active). By default,
 ActiveRepresentation is off and the representation is always visible.

\item  \verb|obj.ActiveRepresentationOff ()| -  Flag controls whether the widget becomes visible when the mouse pointer
 moves close to it (i.e., the widget becomes active). By default,
 ActiveRepresentation is off and the representation is always visible.

\item  \verb|obj.SetInteractionState (int )| -  The interaction state may be set from a widget (e.g., HandleWidget) or
 other object. This controls how the interaction with the widget
 proceeds. Normally this method is used as part of a handshaking
 processwith the widget: First ComputeInteractionState() is invoked that
 returns a state based on geometric considerations (i.e., cursor near a
 widget feature), then based on events, the widget may modify this
 further.

\item  \verb|int = obj.GetInteractionStateMinValue ()| -  The interaction state may be set from a widget (e.g., HandleWidget) or
 other object. This controls how the interaction with the widget
 proceeds. Normally this method is used as part of a handshaking
 processwith the widget: First ComputeInteractionState() is invoked that
 returns a state based on geometric considerations (i.e., cursor near a
 widget feature), then based on events, the widget may modify this
 further.

\item  \verb|int = obj.GetInteractionStateMaxValue ()| -  The interaction state may be set from a widget (e.g., HandleWidget) or
 other object. This controls how the interaction with the widget
 proceeds. Normally this method is used as part of a handshaking
 processwith the widget: First ComputeInteractionState() is invoked that
 returns a state based on geometric considerations (i.e., cursor near a
 widget feature), then based on events, the widget may modify this
 further.

\item  \verb|obj.SetConstrained (int )| -  Specify whether any motions (such as scale, translate, etc.) are
 constrained in some way (along an axis, etc.) Widgets can use this
 to control the resulting motion.

\item  \verb|int = obj.GetConstrained ()| -  Specify whether any motions (such as scale, translate, etc.) are
 constrained in some way (along an axis, etc.) Widgets can use this
 to control the resulting motion.

\item  \verb|obj.ConstrainedOn ()| -  Specify whether any motions (such as scale, translate, etc.) are
 constrained in some way (along an axis, etc.) Widgets can use this
 to control the resulting motion.

\item  \verb|obj.ConstrainedOff ()| -  Specify whether any motions (such as scale, translate, etc.) are
 constrained in some way (along an axis, etc.) Widgets can use this
 to control the resulting motion.

\item  \verb|int = obj.CheckConstraint (vtkRenderer renderer, double pos[2])| -  Method has to be overridden in the subclasses which has
 constraints on placing the handle
 (Ex. vtkConstrainedPointHandleRepresentation). It should return 1
 if the position is within the constraint, else it should return
 0. By default it returns 1.

\item  \verb|obj.ShallowCopy (vtkProp prop)| -  Methods to make this class properly act like a vtkWidgetRepresentation.

\item  \verb|obj.DeepCopy (vtkProp prop)| -  Methods to make this class properly act like a vtkWidgetRepresentation.

\item  \verb|obj.SetRenderer (vtkRenderer ren)| -  Methods to make this class properly act like a vtkWidgetRepresentation.

\item  \verb|long = obj.GetMTime ()| -  Overload the superclasses' GetMTime() because the internal vtkCoordinates
 are used to keep the state of the representation.

\item  \verb|obj.SetPointPlacer (vtkPointPlacer )| -  Set/Get the point placer. Point placers can be used to dictate constraints 
 on the placement of handles. As an example, see vtkBoundedPlanePointPlacer
 (constrains the placement of handles to a set of bounded planes)
 vtkFocalPlanePointPlacer (constrains placement on the focal plane) etc.
 The default point placer is vtkPointPlacer (which does not apply any 
 constraints, so the handles are free to move anywhere).

\item  \verb|vtkPointPlacer = obj.GetPointPlacer ()| -  Set/Get the point placer. Point placers can be used to dictate constraints 
 on the placement of handles. As an example, see vtkBoundedPlanePointPlacer
 (constrains the placement of handles to a set of bounded planes)
 vtkFocalPlanePointPlacer (constrains placement on the focal plane) etc.
 The default point placer is vtkPointPlacer (which does not apply any 
 constraints, so the handles are free to move anywhere).

\end{itemize}
