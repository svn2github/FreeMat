\section{vtkImageTracerWidget}

\subsection{Usage}

 vtkImageTracerWidget is different from other widgets in three distinct ways:
 1) any sub-class of vtkProp can be input rather than just vtkProp3D, so that
 vtkImageActor can be set as the prop and then traced over, 2) the widget fires
 pick events at the input prop to decide where to move its handles, 3) the
 widget has 2D glyphs for handles instead of 3D spheres as is done in other
 sub-classes of vtk3DWidget. This widget is primarily designed for manually
 tracing over image data.
 The button actions and key modifiers are as follows for controlling the
 widget:
 1) left button click over the image, hold and drag draws a free hand line.
 2) left button click and release erases the widget line,
 if it exists, and repositions the first handle.
 3) middle button click starts a snap drawn line.  The line is terminated by 
 clicking the middle button while depressing the ctrl key.
 4) when tracing a continuous or snap drawn line, if the last cursor position
 is within a specified tolerance to the first handle, the widget line will form
 a closed loop.
 5) right button clicking and holding on any handle that is part of a snap
 drawn line allows handle dragging: existing line segments are updated
 accordingly.  If the path is open and AutoClose is set to On, the path can
 be closed by repositioning the first and last points over one another.
 6) ctrl key + right button down on any handle will erase it: existing
 snap drawn line segments are updated accordingly.  If the line was formed by
 continous tracing, the line is deleted leaving one handle.
 7) shift key + right button down on any snap drawn line segment will insert
 a handle at the cursor position.  The line segment is split accordingly.

To create an instance of class vtkImageTracerWidget, simply
invoke its constructor as follows
\begin{verbatim}
  obj = vtkImageTracerWidget
\end{verbatim}
\subsection{Methods}

The class vtkImageTracerWidget has several methods that can be used.
  They are listed below.
Note that the documentation is translated automatically from the VTK sources,
and may not be completely intelligible.  When in doubt, consult the VTK website.
In the methods listed below, \verb|obj| is an instance of the vtkImageTracerWidget class.
\begin{itemize}
\item  \verb|string = obj.GetClassName ()|

\item  \verb|int = obj.IsA (string name)|

\item  \verb|vtkImageTracerWidget = obj.NewInstance ()|

\item  \verb|vtkImageTracerWidget = obj.SafeDownCast (vtkObject o)|

\item  \verb|obj.SetEnabled (int )| -  Methods that satisfy the superclass' API.

\item  \verb|obj.PlaceWidget (double bounds[6])| -  Methods that satisfy the superclass' API.

\item  \verb|obj.PlaceWidget ()| -  Methods that satisfy the superclass' API.

\item  \verb|obj.PlaceWidget (double xmin, double xmax, double ymin, double ymax, double zmin, double zmax)| -  Set/Get the handle properties (the 2D glyphs are the handles). The
 properties of the handles when selected and normal can be manipulated.

\item  \verb|obj.SetHandleProperty (vtkProperty )| -  Set/Get the handle properties (the 2D glyphs are the handles). The
 properties of the handles when selected and normal can be manipulated.

\item  \verb|vtkProperty = obj.GetHandleProperty ()| -  Set/Get the handle properties (the 2D glyphs are the handles). The
 properties of the handles when selected and normal can be manipulated.

\item  \verb|obj.SetSelectedHandleProperty (vtkProperty )| -  Set/Get the handle properties (the 2D glyphs are the handles). The
 properties of the handles when selected and normal can be manipulated.

\item  \verb|vtkProperty = obj.GetSelectedHandleProperty ()| -  Set/Get the handle properties (the 2D glyphs are the handles). The
 properties of the handles when selected and normal can be manipulated.

\item  \verb|obj.SetLineProperty (vtkProperty )| -  Set/Get the line properties. The properties of the line when selected
 and unselected can be manipulated.

\item  \verb|vtkProperty = obj.GetLineProperty ()| -  Set/Get the line properties. The properties of the line when selected
 and unselected can be manipulated.

\item  \verb|obj.SetSelectedLineProperty (vtkProperty )| -  Set/Get the line properties. The properties of the line when selected
 and unselected can be manipulated.

\item  \verb|vtkProperty = obj.GetSelectedLineProperty ()| -  Set/Get the line properties. The properties of the line when selected
 and unselected can be manipulated.

\item  \verb|obj.SetViewProp (vtkProp prop)| -  Set the prop, usually a vtkImageActor, to trace over.

\item  \verb|obj.SetProjectToPlane (int )| -  Force handles to be on a specific ortho plane. Default is Off.

\item  \verb|int = obj.GetProjectToPlane ()| -  Force handles to be on a specific ortho plane. Default is Off.

\item  \verb|obj.ProjectToPlaneOn ()| -  Force handles to be on a specific ortho plane. Default is Off.

\item  \verb|obj.ProjectToPlaneOff ()| -  Force handles to be on a specific ortho plane. Default is Off.

\item  \verb|obj.SetProjectionNormal (int )| -  Set the projection normal.  The normal in SetProjectionNormal is 0,1,2
 for YZ,XZ,XY planes respectively.  Since the handles are 2D glyphs, it is
 necessary to specify a plane on which to generate them, even though
 ProjectToPlane may be turned off.

\item  \verb|int = obj.GetProjectionNormalMinValue ()| -  Set the projection normal.  The normal in SetProjectionNormal is 0,1,2
 for YZ,XZ,XY planes respectively.  Since the handles are 2D glyphs, it is
 necessary to specify a plane on which to generate them, even though
 ProjectToPlane may be turned off.

\item  \verb|int = obj.GetProjectionNormalMaxValue ()| -  Set the projection normal.  The normal in SetProjectionNormal is 0,1,2
 for YZ,XZ,XY planes respectively.  Since the handles are 2D glyphs, it is
 necessary to specify a plane on which to generate them, even though
 ProjectToPlane may be turned off.

\item  \verb|int = obj.GetProjectionNormal ()| -  Set the projection normal.  The normal in SetProjectionNormal is 0,1,2
 for YZ,XZ,XY planes respectively.  Since the handles are 2D glyphs, it is
 necessary to specify a plane on which to generate them, even though
 ProjectToPlane may be turned off.

\item  \verb|obj.SetProjectionNormalToXAxes ()| -  Set the projection normal.  The normal in SetProjectionNormal is 0,1,2
 for YZ,XZ,XY planes respectively.  Since the handles are 2D glyphs, it is
 necessary to specify a plane on which to generate them, even though
 ProjectToPlane may be turned off.

\item  \verb|obj.SetProjectionNormalToYAxes ()| -  Set the projection normal.  The normal in SetProjectionNormal is 0,1,2
 for YZ,XZ,XY planes respectively.  Since the handles are 2D glyphs, it is
 necessary to specify a plane on which to generate them, even though
 ProjectToPlane may be turned off.

\item  \verb|obj.SetProjectionNormalToZAxes ()| -  Set the position of the widgets' handles in terms of a plane's position.
 e.g., if ProjectionNormal is 0, all of the x-coordinate values of the
 handles are set to ProjectionPosition.  No attempt is made to ensure that
 the position is within the bounds of either the underlying image data or
 the prop on which tracing is performed.

\item  \verb|obj.SetProjectionPosition (double position)| -  Set the position of the widgets' handles in terms of a plane's position.
 e.g., if ProjectionNormal is 0, all of the x-coordinate values of the
 handles are set to ProjectionPosition.  No attempt is made to ensure that
 the position is within the bounds of either the underlying image data or
 the prop on which tracing is performed.

\item  \verb|double = obj.GetProjectionPosition ()| -  Set the position of the widgets' handles in terms of a plane's position.
 e.g., if ProjectionNormal is 0, all of the x-coordinate values of the
 handles are set to ProjectionPosition.  No attempt is made to ensure that
 the position is within the bounds of either the underlying image data or
 the prop on which tracing is performed.

\item  \verb|obj.SetSnapToImage (int snap)| -  Force snapping to image data while tracing. Default is Off.

\item  \verb|int = obj.GetSnapToImage ()| -  Force snapping to image data while tracing. Default is Off.

\item  \verb|obj.SnapToImageOn ()| -  Force snapping to image data while tracing. Default is Off.

\item  \verb|obj.SnapToImageOff ()| -  Force snapping to image data while tracing. Default is Off.

\item  \verb|obj.SetAutoClose (int )| -  In concert with a CaptureRadius value, automatically
 form a closed path by connecting first to last path points.
 Default is Off.

\item  \verb|int = obj.GetAutoClose ()| -  In concert with a CaptureRadius value, automatically
 form a closed path by connecting first to last path points.
 Default is Off.

\item  \verb|obj.AutoCloseOn ()| -  In concert with a CaptureRadius value, automatically
 form a closed path by connecting first to last path points.
 Default is Off.

\item  \verb|obj.AutoCloseOff ()| -  In concert with a CaptureRadius value, automatically
 form a closed path by connecting first to last path points.
 Default is Off.

\item  \verb|obj.SetCaptureRadius (double )| -  Set/Get the capture radius for automatic path closing.  For image
 data, capture radius should be half the distance between voxel/pixel
 centers.
 Default is 1.0

\item  \verb|double = obj.GetCaptureRadius ()| -  Set/Get the capture radius for automatic path closing.  For image
 data, capture radius should be half the distance between voxel/pixel
 centers.
 Default is 1.0

\item  \verb|obj.GetPath (vtkPolyData pd)| -  Grab the points and lines that define the traced path. These point values
 are guaranteed to be up-to-date when either the InteractionEvent or
 EndInteraction events are invoked. The user provides the vtkPolyData and
 the points and cells representing the line are added to it.

\item  \verb|vtkGlyphSource2D = obj.GetGlyphSource ()| -  Set/Get the type of snapping to image data: center of a pixel/voxel or
 nearest point defining a pixel/voxel.

\item  \verb|obj.SetImageSnapType (int )| -  Set/Get the type of snapping to image data: center of a pixel/voxel or
 nearest point defining a pixel/voxel.

\item  \verb|int = obj.GetImageSnapTypeMinValue ()| -  Set/Get the type of snapping to image data: center of a pixel/voxel or
 nearest point defining a pixel/voxel.

\item  \verb|int = obj.GetImageSnapTypeMaxValue ()| -  Set/Get the type of snapping to image data: center of a pixel/voxel or
 nearest point defining a pixel/voxel.

\item  \verb|int = obj.GetImageSnapType ()| -  Set/Get the type of snapping to image data: center of a pixel/voxel or
 nearest point defining a pixel/voxel.

\item  \verb|obj.SetHandlePosition (int handle, double xyz[3])| -  Set/Get the handle position in terms of a zero-based array of handles.

\item  \verb|obj.SetHandlePosition (int handle, double x, double y, double z)| -  Set/Get the handle position in terms of a zero-based array of handles.

\item  \verb|obj.GetHandlePosition (int handle, double xyz[3])| -  Set/Get the handle position in terms of a zero-based array of handles.

\item  \verb|double = obj.GetHandlePosition (int handle)| -  Set/Get the handle position in terms of a zero-based array of handles.

\item  \verb|int = obj.GetNumberOfHandles ()| -  Get the number of handles.

\item  \verb|obj.SetInteraction (int interact)| -  Enable/disable mouse interaction when the widget is visible.

\item  \verb|int = obj.GetInteraction ()| -  Enable/disable mouse interaction when the widget is visible.

\item  \verb|obj.InteractionOn ()| -  Enable/disable mouse interaction when the widget is visible.

\item  \verb|obj.InteractionOff ()| -  Enable/disable mouse interaction when the widget is visible.

\item  \verb|obj.InitializeHandles (vtkPoints )| -  Initialize the widget with a set of points and generate
 lines between them.  If AutoClose is on it will handle the
 case wherein the first and last points are congruent.

\item  \verb|int = obj.IsClosed ()| -  Is the path closed or open?

\item  \verb|obj.SetProp (vtkProp prop)| -  @deprecated Replaced by vtkImageTracerWidget::SetViewProp() as of VTK 5.0.

\end{itemize}
