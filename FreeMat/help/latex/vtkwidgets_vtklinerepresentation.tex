\section{vtkLineRepresentation}

\subsection{Usage}

 This class is a concrete representation for the vtkLineWidget2. It
 represents a straight line with three handles: one at the beginning and
 ending of the line, and one used to translate the line. Through
 interaction with the widget, the line representation can be arbitrarily
 placed in the 3D space.

 To use this representation, you normally specify the position of the two
 end points (either in world or display coordinates). The PlaceWidget()
 method is also used to intially position the representation.

To create an instance of class vtkLineRepresentation, simply
invoke its constructor as follows
\begin{verbatim}
  obj = vtkLineRepresentation
\end{verbatim}
\subsection{Methods}

The class vtkLineRepresentation has several methods that can be used.
  They are listed below.
Note that the documentation is translated automatically from the VTK sources,
and may not be completely intelligible.  When in doubt, consult the VTK website.
In the methods listed below, \verb|obj| is an instance of the vtkLineRepresentation class.
\begin{itemize}
\item  \verb|string = obj.GetClassName ()| -  Standard methods for the class.

\item  \verb|int = obj.IsA (string name)| -  Standard methods for the class.

\item  \verb|vtkLineRepresentation = obj.NewInstance ()| -  Standard methods for the class.

\item  \verb|vtkLineRepresentation = obj.SafeDownCast (vtkObject o)| -  Standard methods for the class.

\item  \verb|obj.GetPoint1WorldPosition (double pos[3])| -  Methods to Set/Get the coordinates of the two points defining
 this representation. Note that methods are available for both
 display and world coordinates.

\item  \verb|double = obj.GetPoint1WorldPosition ()| -  Methods to Set/Get the coordinates of the two points defining
 this representation. Note that methods are available for both
 display and world coordinates.

\item  \verb|obj.GetPoint1DisplayPosition (double pos[3])| -  Methods to Set/Get the coordinates of the two points defining
 this representation. Note that methods are available for both
 display and world coordinates.

\item  \verb|double = obj.GetPoint1DisplayPosition ()| -  Methods to Set/Get the coordinates of the two points defining
 this representation. Note that methods are available for both
 display and world coordinates.

\item  \verb|obj.SetPoint1WorldPosition (double pos[3])| -  Methods to Set/Get the coordinates of the two points defining
 this representation. Note that methods are available for both
 display and world coordinates.

\item  \verb|obj.SetPoint1DisplayPosition (double pos[3])| -  Methods to Set/Get the coordinates of the two points defining
 this representation. Note that methods are available for both
 display and world coordinates.

\item  \verb|obj.GetPoint2DisplayPosition (double pos[3])| -  Methods to Set/Get the coordinates of the two points defining
 this representation. Note that methods are available for both
 display and world coordinates.

\item  \verb|double = obj.GetPoint2DisplayPosition ()| -  Methods to Set/Get the coordinates of the two points defining
 this representation. Note that methods are available for both
 display and world coordinates.

\item  \verb|obj.GetPoint2WorldPosition (double pos[3])| -  Methods to Set/Get the coordinates of the two points defining
 this representation. Note that methods are available for both
 display and world coordinates.

\item  \verb|double = obj.GetPoint2WorldPosition ()| -  Methods to Set/Get the coordinates of the two points defining
 this representation. Note that methods are available for both
 display and world coordinates.

\item  \verb|obj.SetPoint2WorldPosition (double pos[3])| -  Methods to Set/Get the coordinates of the two points defining
 this representation. Note that methods are available for both
 display and world coordinates.

\item  \verb|obj.SetPoint2DisplayPosition (double pos[3])| -  Methods to Set/Get the coordinates of the two points defining
 this representation. Note that methods are available for both
 display and world coordinates.

\item  \verb|obj.SetHandleRepresentation (vtkPointHandleRepresentation3D handle)| -  This method is used to specify the type of handle representation to
 use for the three internal vtkHandleWidgets within vtkLineWidget2.
 To use this method, create a dummy vtkHandleWidget (or subclass),
 and then invoke this method with this dummy. Then the 
 vtkLineRepresentation uses this dummy to clone three vtkHandleWidgets
 of the same type. Make sure you set the handle representation before
 the widget is enabled. (The method InstantiateHandleRepresentation()
 is invoked by the vtkLineWidget2.)

\item  \verb|obj.InstantiateHandleRepresentation ()| -  This method is used to specify the type of handle representation to
 use for the three internal vtkHandleWidgets within vtkLineWidget2.
 To use this method, create a dummy vtkHandleWidget (or subclass),
 and then invoke this method with this dummy. Then the 
 vtkLineRepresentation uses this dummy to clone three vtkHandleWidgets
 of the same type. Make sure you set the handle representation before
 the widget is enabled. (The method InstantiateHandleRepresentation()
 is invoked by the vtkLineWidget2.)

\item  \verb|vtkPointHandleRepresentation3D = obj.GetPoint1Representation ()| -  Get the three handle representations used for the vtkLineWidget2. 

\item  \verb|vtkPointHandleRepresentation3D = obj.GetPoint2Representation ()| -  Get the three handle representations used for the vtkLineWidget2. 

\item  \verb|vtkPointHandleRepresentation3D = obj.GetLineHandleRepresentation ()| -  Get the three handle representations used for the vtkLineWidget2. 

\item  \verb|vtkProperty = obj.GetEndPointProperty ()| -  Get the end-point (sphere) properties. The properties of the end-points 
 when selected and unselected can be manipulated.

\item  \verb|vtkProperty = obj.GetSelectedEndPointProperty ()| -  Get the end-point (sphere) properties. The properties of the end-points 
 when selected and unselected can be manipulated.

\item  \verb|vtkProperty = obj.GetEndPoint2Property ()| -  Get the end-point (sphere) properties. The properties of the end-points 
 when selected and unselected can be manipulated.

\item  \verb|vtkProperty = obj.GetSelectedEndPoint2Property ()| -  Get the end-point (sphere) properties. The properties of the end-points 
 when selected and unselected can be manipulated.

\item  \verb|vtkProperty = obj.GetLineProperty ()| -  Get the line properties. The properties of the line when selected
 and unselected can be manipulated.

\item  \verb|vtkProperty = obj.GetSelectedLineProperty ()| -  Get the line properties. The properties of the line when selected
 and unselected can be manipulated.

\item  \verb|obj.SetTolerance (int )| -  The tolerance representing the distance to the widget (in pixels) in
 which the cursor is considered near enough to the line or end point 
 to be active.

\item  \verb|int = obj.GetToleranceMinValue ()| -  The tolerance representing the distance to the widget (in pixels) in
 which the cursor is considered near enough to the line or end point 
 to be active.

\item  \verb|int = obj.GetToleranceMaxValue ()| -  The tolerance representing the distance to the widget (in pixels) in
 which the cursor is considered near enough to the line or end point 
 to be active.

\item  \verb|int = obj.GetTolerance ()| -  The tolerance representing the distance to the widget (in pixels) in
 which the cursor is considered near enough to the line or end point 
 to be active.

\item  \verb|obj.SetResolution (int res)| -  Set/Get the resolution (number of subdivisions) of the line. A line with
 resolution greater than one is useful when points along the line are
 desired; e.g., generating a rake of streamlines.

\item  \verb|int = obj.GetResolution ()| -  Set/Get the resolution (number of subdivisions) of the line. A line with
 resolution greater than one is useful when points along the line are
 desired; e.g., generating a rake of streamlines.

\item  \verb|obj.GetPolyData (vtkPolyData pd)| -  Retrieve the polydata (including points) that defines the line.  The
 polydata consists of n+1 points, where n is the resolution of the
 line. These point values are guaranteed to be up-to-date whenever any
 one of the three handles are moved. To use this method, the user
 provides the vtkPolyData as an input argument, and the points and
 polyline are copied into it.

\item  \verb|obj.PlaceWidget (double bounds[6])| -  These are methods that satisfy vtkWidgetRepresentation's API.

\item  \verb|obj.BuildRepresentation ()| -  These are methods that satisfy vtkWidgetRepresentation's API.

\item  \verb|int = obj.ComputeInteractionState (int X, int Y, int modify)| -  These are methods that satisfy vtkWidgetRepresentation's API.

\item  \verb|obj.StartWidgetInteraction (double e[2])| -  These are methods that satisfy vtkWidgetRepresentation's API.

\item  \verb|obj.WidgetInteraction (double e[2])| -  These are methods that satisfy vtkWidgetRepresentation's API.

\item  \verb|double = obj.GetBounds ()| -  These are methods that satisfy vtkWidgetRepresentation's API.

\item  \verb|obj.GetActors (vtkPropCollection pc)| -  Methods supporting the rendering process.

\item  \verb|obj.ReleaseGraphicsResources (vtkWindow )| -  Methods supporting the rendering process.

\item  \verb|int = obj.RenderOpaqueGeometry (vtkViewport )| -  Methods supporting the rendering process.

\item  \verb|int = obj.RenderTranslucentPolygonalGeometry (vtkViewport )| -  Methods supporting the rendering process.

\item  \verb|int = obj.HasTranslucentPolygonalGeometry ()| -  Methods supporting the rendering process.

\item  \verb|obj.SetInteractionState (int )| -  The interaction state may be set from a widget (e.g., vtkLineWidget2) or
 other object. This controls how the interaction with the widget
 proceeds. Normally this method is used as part of a handshaking
 process with the widget: First ComputeInteractionState() is invoked that
 returns a state based on geometric considerations (i.e., cursor near a
 widget feature), then based on events, the widget may modify this
 further.

\item  \verb|int = obj.GetInteractionStateMinValue ()| -  The interaction state may be set from a widget (e.g., vtkLineWidget2) or
 other object. This controls how the interaction with the widget
 proceeds. Normally this method is used as part of a handshaking
 process with the widget: First ComputeInteractionState() is invoked that
 returns a state based on geometric considerations (i.e., cursor near a
 widget feature), then based on events, the widget may modify this
 further.

\item  \verb|int = obj.GetInteractionStateMaxValue ()| -  The interaction state may be set from a widget (e.g., vtkLineWidget2) or
 other object. This controls how the interaction with the widget
 proceeds. Normally this method is used as part of a handshaking
 process with the widget: First ComputeInteractionState() is invoked that
 returns a state based on geometric considerations (i.e., cursor near a
 widget feature), then based on events, the widget may modify this
 further.

\item  \verb|obj.SetRepresentationState (int )| -  Sets the visual appearance of the representation based on the
 state it is in. This state is usually the same as InteractionState.

\item  \verb|int = obj.GetRepresentationState ()| -  Sets the visual appearance of the representation based on the
 state it is in. This state is usually the same as InteractionState.

\item  \verb|long = obj.GetMTime ()| -  Overload the superclasses' GetMTime() because internal classes
 are used to keep the state of the representation.

\item  \verb|obj.SetRenderer (vtkRenderer ren)| -  Overridden to set the rendererer on the internal representations.

\item  \verb|obj.SetDistanceAnnotationVisibility (int )| -  Show the distance between the points

\item  \verb|int = obj.GetDistanceAnnotationVisibility ()| -  Show the distance between the points

\item  \verb|obj.DistanceAnnotationVisibilityOn ()| -  Show the distance between the points

\item  \verb|obj.DistanceAnnotationVisibilityOff ()| -  Show the distance between the points

\item  \verb|obj.SetDistanceAnnotationFormat (string )| -  Specify the format to use for labelling the angle. Note that an empty
 string results in no label, or a format string without a ''%'' character
 will not print the angle value.

\item  \verb|string = obj.GetDistanceAnnotationFormat ()| -  Specify the format to use for labelling the angle. Note that an empty
 string results in no label, or a format string without a ''%'' character
 will not print the angle value.

\item  \verb|obj.SetDistanceAnnotationScale (double scale[3])| -  Scale text (font size along each dimension).

\item  \verb|double = obj.GetDistance ()| -  Get the distance between the points.

\item  \verb|obj.SetLineColor (double r, double g, double b)| -  Convenience method to set the line color.
 Ideally one should use GetLineProperty()->SetColor().

\item  \verb|vtkProperty = obj.GetDistanceAnnotationProperty ()| -  Get the distance annotation property

\item  \verb|vtkFollower = obj.GetTextActor ()| -  Get the text actor

\end{itemize}
