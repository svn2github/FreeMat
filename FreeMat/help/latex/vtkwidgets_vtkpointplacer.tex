\section{vtkPointPlacer}

\subsection{Usage}

 Most widgets in VTK have a need to translate of 2D display coordinates (as
 reported by the RenderWindowInteractor) to 3D world coordinates. This class
 is an abstraction of this functionality. A few subclasses are listed below:
 <p>1) vtkFocalPlanePointPlacer: This class converts 2D display positions to 
 world positions such that they lie on the focal plane.
 <p>2) vtkPolygonalSurfacePointPlacer: Converts 2D display positions to 
 world positions such that they lie on the surface of one or more specified 
 polydatas.
 <p>3) vtkImageActorPointPlacer: Converts 2D display positions to world 
 positions such that they lie on an ImageActor
 <p>4) vtkBoundedPlanePointPlacer: Converts 2D display positions to world 
 positions such that they lie within a set of specified bounding planes.
 <p>5) vtkTerrainDataPointPlacer: Converts 2D display positions to world 
 positions such that they lie on a height field.
 <p> Point placers provide an extensible framework to specify constraints on 
 points. The methods ComputeWorldPosition, ValidateDisplayPosition and
 ValidateWorldPosition may be overridden to dictate whether a world or
 display position is allowed. These classes are currently used by the
 HandleWidget and the ContourWidget to allow various constraints to be 
 enforced on the placement of their handles.

To create an instance of class vtkPointPlacer, simply
invoke its constructor as follows
\begin{verbatim}
  obj = vtkPointPlacer
\end{verbatim}
\subsection{Methods}

The class vtkPointPlacer has several methods that can be used.
  They are listed below.
Note that the documentation is translated automatically from the VTK sources,
and may not be completely intelligible.  When in doubt, consult the VTK website.
In the methods listed below, \verb|obj| is an instance of the vtkPointPlacer class.
\begin{itemize}
\item  \verb|string = obj.GetClassName ()| -  Standard methods for instances of this class.

\item  \verb|int = obj.IsA (string name)| -  Standard methods for instances of this class.

\item  \verb|vtkPointPlacer = obj.NewInstance ()| -  Standard methods for instances of this class.

\item  \verb|vtkPointPlacer = obj.SafeDownCast (vtkObject o)| -  Standard methods for instances of this class.

\item  \verb|int = obj.ComputeWorldPosition (vtkRenderer ren, double displayPos[2], double worldPos[3], double worldOrient[9])| -  Given a renderer and a display position in pixel coordinates,
 compute the world position and orientation where this point
 will be placed. This method is typically used by the
 representation to place the point initially. A return value of 1
 indicates that constraints of the placer are met.

\item  \verb|int = obj.ComputeWorldPosition (vtkRenderer ren, double displayPos[2], double refWorldPos[3], double worldPos[3], double worldOrient[9])| -  Given a renderer, a display position, and a reference world
 position, compute the new world position and orientation 
 of this point. This method is typically used by the 
 representation to move the point. A return value of 1 indicates that 
 constraints of the placer are met.

\item  \verb|int = obj.ValidateWorldPosition (double worldPos[3])| -  Given a world position check the validity of this 
 position according to the constraints of the placer.

\item  \verb|int = obj.ValidateDisplayPosition (vtkRenderer , double displayPos[2])| -  Given a display position, check the validity of this position.

\item  \verb|int = obj.ValidateWorldPosition (double worldPos[3], double worldOrient[9])| -  Given a world position and a world orientation,
 validate it according to the constraints of the placer.

\item  \verb|int = obj.UpdateWorldPosition (vtkRenderer ren, double worldPos[3], double worldOrient[9])| -  Given a current renderer, world position and orientation,
 update them according to the constraints of the placer.
 This method is typically used when UpdateContour is called
 on the representation, which must be called after changes
 are made to the constraints in the placer. A return 
 value of 1 indicates that the point has been updated. A
 return value of 0 indicates that the point could not
 be updated and was left alone. By default this is a no-op - 
 leaving the point as is.

\item  \verb|int = obj.UpdateInternalState ()| -  Set/get the tolerance used when performing computations
 in display coordinates.

\item  \verb|obj.SetPixelTolerance (int )| -  Set/get the tolerance used when performing computations
 in display coordinates.

\item  \verb|int = obj.GetPixelToleranceMinValue ()| -  Set/get the tolerance used when performing computations
 in display coordinates.

\item  \verb|int = obj.GetPixelToleranceMaxValue ()| -  Set/get the tolerance used when performing computations
 in display coordinates.

\item  \verb|int = obj.GetPixelTolerance ()| -  Set/get the tolerance used when performing computations
 in display coordinates.

\item  \verb|obj.SetWorldTolerance (double )| -  Set/get the tolerance used when performing computations
 in world coordinates.

\item  \verb|double = obj.GetWorldToleranceMinValue ()| -  Set/get the tolerance used when performing computations
 in world coordinates.

\item  \verb|double = obj.GetWorldToleranceMaxValue ()| -  Set/get the tolerance used when performing computations
 in world coordinates.

\item  \verb|double = obj.GetWorldTolerance ()| -  Set/get the tolerance used when performing computations
 in world coordinates.

\end{itemize}
