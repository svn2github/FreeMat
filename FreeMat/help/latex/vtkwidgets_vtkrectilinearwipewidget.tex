\section{vtkRectilinearWipeWidget}

\subsection{Usage}

 The vtkRectilinearWipeWidget is used to interactively control an instance
 of vtkImageRectilinearWipe (and an associated vtkImageActor used to
 display the rectilinear wipe). A rectilinear wipe is a 2x2 checkerboard
 pattern created by combining two separate images, where various
 combinations of the checker squares are possible. Using this widget, the
 user can adjust the layout of the checker pattern, such as moving the
 center point, moving the horizontal separator, or moving the vertical
 separator. These capabilities are particularly useful for comparing two
 images.
 
 To use this widget, specify its representation (by default the
 representation is an instance of vtkRectilinearWipeProp). The
 representation generally requires that you specify an instance of
 vtkImageRectilinearWipe and an instance of vtkImageActor. Other instance
 variables may also be required to be set -- see the documentation for
 vtkRectilinearWipeProp (or appropriate subclass).

 By default, the widget responds to the following events:
 \begin{verbatim}
 Selecting the center point, horizontal separator, and verticel separator:
   LeftButtonPressEvent - move the separators
   LeftButtonReleaseEvent - release the separators 
   MouseMoveEvent - move the separators
 \end{verbatim}
 Selecting the center point allows you to move the horizontal and vertical
 separators simultaneously. Otherwise only horizontal or vertical motion
 is possible/

 Note that the event bindings described above can be changed using this
 class's vtkWidgetEventTranslator. This class translates VTK events into
 the vtkRectilinearWipeWidget's widget events:
 \begin{verbatim}
   vtkWidgetEvent::Select -- some part of the widget has been selected
   vtkWidgetEvent::EndSelect -- the selection process has completed
   vtkWidgetEvent::Move -- a request for motion has been invoked
 \end{verbatim}

 In turn, when these widget events are processed, the
 vtkRectilinearWipeWidget invokes the following VTK events (which
 observers can listen for):
 \begin{verbatim}
   vtkCommand::StartInteractionEvent (on vtkWidgetEvent::Select)
   vtkCommand::EndInteractionEvent (on vtkWidgetEvent::EndSelect)
   vtkCommand::InteractionEvent (on vtkWidgetEvent::Move)
 \end{verbatim}


To create an instance of class vtkRectilinearWipeWidget, simply
invoke its constructor as follows
\begin{verbatim}
  obj = vtkRectilinearWipeWidget
\end{verbatim}
\subsection{Methods}

The class vtkRectilinearWipeWidget has several methods that can be used.
  They are listed below.
Note that the documentation is translated automatically from the VTK sources,
and may not be completely intelligible.  When in doubt, consult the VTK website.
In the methods listed below, \verb|obj| is an instance of the vtkRectilinearWipeWidget class.
\begin{itemize}
\item  \verb|string = obj.GetClassName ()| -  Standard macros.

\item  \verb|int = obj.IsA (string name)| -  Standard macros.

\item  \verb|vtkRectilinearWipeWidget = obj.NewInstance ()| -  Standard macros.

\item  \verb|vtkRectilinearWipeWidget = obj.SafeDownCast (vtkObject o)| -  Standard macros.

\item  \verb|obj.SetRepresentation (vtkRectilinearWipeRepresentation r)| -  Create the default widget representation if one is not set. 

\item  \verb|obj.CreateDefaultRepresentation ()| -  Create the default widget representation if one is not set. 

\end{itemize}
