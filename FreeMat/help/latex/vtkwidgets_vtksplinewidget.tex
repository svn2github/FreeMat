\section{vtkSplineWidget}

\subsection{Usage}

 This 3D widget defines a spline that can be interactively placed in a
 scene. The spline has handles, the number of which can be changed, plus it
 can be picked on the spline itself to translate or rotate it in the scene.
 A nice feature of the object is that the vtkSplineWidget, like any 3D
 widget, will work with the current interactor style. That is, if
 vtkSplineWidget does not handle an event, then all other registered
 observers (including the interactor style) have an opportunity to process
 the event. Otherwise, the vtkSplineWidget will terminate the processing of
 the event that it handles.

 To use this object, just invoke SetInteractor() with the argument of the
 method a vtkRenderWindowInteractor.  You may also wish to invoke
 ''PlaceWidget()'' to initially position the widget. The interactor will act
 normally until the ''i'' key (for ''interactor'') is pressed, at which point the
 vtkSplineWidget will appear. (See superclass documentation for information
 about changing this behavior.) Events that occur outside of the widget
 (i.e., no part of the widget is picked) are propagated to any other
 registered obsevers (such as the interaction style).  Turn off the widget
 by pressing the ''i'' key again (or invoke the Off() method).

 The button actions and key modifiers are as follows for controlling the
 widget:
 1) left button down on and drag one of the spherical handles to change the
 shape of the spline: the handles act as ''control points''.
 2) left button or middle button down on a line segment forming the spline
 allows uniform translation of the widget.
 3) ctrl + middle button down on the widget enables spinning of the widget
 about its center.
 4) right button down on the widget enables scaling of the widget. By moving
 the mouse ''up'' the render window the spline will be made bigger; by moving
 ''down'' the render window the widget will be made smaller.
 5) ctrl key + right button down on any handle will erase it providing there
 will be two or more points remaining to form a spline.
 6) shift key + right button down on any line segment will insert a handle
 onto the spline at the cursor position.

 The vtkSplineWidget has several methods that can be used in conjunction with
 other VTK objects. The Set/GetResolution() methods control the number of
 subdivisions of the spline; the GetPolyData() method can be used to get the
 polygonal representation and can be used for things like seeding
 streamlines or probing other data sets. Typical usage of the widget is to
 make use of the StartInteractionEvent, InteractionEvent, and
 EndInteractionEvent events. The InteractionEvent is called on mouse motion;
 the other two events are called on button down and button up (either left or
 right button).

 Some additional features of this class include the ability to control the
 properties of the widget. You can set the properties of the selected and
 unselected representations of the spline. For example, you can set the
 property for the handles and spline. In addition there are methods to
 constrain the spline so that it is aligned with a plane.  Note that a simple
 ruler widget can be derived by setting the resolution to 1, the number of
 handles to 2, and calling the GetSummedLength method!

To create an instance of class vtkSplineWidget, simply
invoke its constructor as follows
\begin{verbatim}
  obj = vtkSplineWidget
\end{verbatim}
\subsection{Methods}

The class vtkSplineWidget has several methods that can be used.
  They are listed below.
Note that the documentation is translated automatically from the VTK sources,
and may not be completely intelligible.  When in doubt, consult the VTK website.
In the methods listed below, \verb|obj| is an instance of the vtkSplineWidget class.
\begin{itemize}
\item  \verb|string = obj.GetClassName ()|

\item  \verb|int = obj.IsA (string name)|

\item  \verb|vtkSplineWidget = obj.NewInstance ()|

\item  \verb|vtkSplineWidget = obj.SafeDownCast (vtkObject o)|

\item  \verb|obj.SetEnabled (int )| -  Methods that satisfy the superclass' API.

\item  \verb|obj.PlaceWidget (double bounds[6])| -  Methods that satisfy the superclass' API.

\item  \verb|obj.PlaceWidget ()| -  Methods that satisfy the superclass' API.

\item  \verb|obj.PlaceWidget (double xmin, double xmax, double ymin, double ymax, double zmin, double zmax)| -  Force the spline widget to be projected onto one of the orthogonal planes.
 Remember that when the state changes, a ModifiedEvent is invoked.
 This can be used to snap the spline to the plane if it is orginally
 not aligned.  The normal in SetProjectionNormal is 0,1,2 for YZ,XZ,XY
 planes respectively and 3 for arbitrary oblique planes when the widget
 is tied to a vtkPlaneSource.

\item  \verb|obj.SetProjectToPlane (int )| -  Force the spline widget to be projected onto one of the orthogonal planes.
 Remember that when the state changes, a ModifiedEvent is invoked.
 This can be used to snap the spline to the plane if it is orginally
 not aligned.  The normal in SetProjectionNormal is 0,1,2 for YZ,XZ,XY
 planes respectively and 3 for arbitrary oblique planes when the widget
 is tied to a vtkPlaneSource.

\item  \verb|int = obj.GetProjectToPlane ()| -  Force the spline widget to be projected onto one of the orthogonal planes.
 Remember that when the state changes, a ModifiedEvent is invoked.
 This can be used to snap the spline to the plane if it is orginally
 not aligned.  The normal in SetProjectionNormal is 0,1,2 for YZ,XZ,XY
 planes respectively and 3 for arbitrary oblique planes when the widget
 is tied to a vtkPlaneSource.

\item  \verb|obj.ProjectToPlaneOn ()| -  Force the spline widget to be projected onto one of the orthogonal planes.
 Remember that when the state changes, a ModifiedEvent is invoked.
 This can be used to snap the spline to the plane if it is orginally
 not aligned.  The normal in SetProjectionNormal is 0,1,2 for YZ,XZ,XY
 planes respectively and 3 for arbitrary oblique planes when the widget
 is tied to a vtkPlaneSource.

\item  \verb|obj.ProjectToPlaneOff ()| -  Force the spline widget to be projected onto one of the orthogonal planes.
 Remember that when the state changes, a ModifiedEvent is invoked.
 This can be used to snap the spline to the plane if it is orginally
 not aligned.  The normal in SetProjectionNormal is 0,1,2 for YZ,XZ,XY
 planes respectively and 3 for arbitrary oblique planes when the widget
 is tied to a vtkPlaneSource.

\item  \verb|obj.SetPlaneSource (vtkPlaneSource plane)| -  Set up a reference to a vtkPlaneSource that could be from another widget
 object, e.g. a vtkPolyDataSourceWidget.

\item  \verb|obj.SetProjectionNormal (int )|

\item  \verb|int = obj.GetProjectionNormalMinValue ()|

\item  \verb|int = obj.GetProjectionNormalMaxValue ()|

\item  \verb|int = obj.GetProjectionNormal ()|

\item  \verb|obj.SetProjectionNormalToXAxes ()|

\item  \verb|obj.SetProjectionNormalToYAxes ()|

\item  \verb|obj.SetProjectionNormalToZAxes ()|

\item  \verb|obj.SetProjectionNormalToOblique ()| -  Set the position of spline handles and points in terms of a plane's
 position. i.e., if ProjectionNormal is 0, all of the x-coordinate
 values of the points are set to position. Any value can be passed (and is
 ignored) to update the spline points when Projection normal is set to 3
 for arbritrary plane orientations.

\item  \verb|obj.SetProjectionPosition (double position)| -  Set the position of spline handles and points in terms of a plane's
 position. i.e., if ProjectionNormal is 0, all of the x-coordinate
 values of the points are set to position. Any value can be passed (and is
 ignored) to update the spline points when Projection normal is set to 3
 for arbritrary plane orientations.

\item  \verb|double = obj.GetProjectionPosition ()| -  Set the position of spline handles and points in terms of a plane's
 position. i.e., if ProjectionNormal is 0, all of the x-coordinate
 values of the points are set to position. Any value can be passed (and is
 ignored) to update the spline points when Projection normal is set to 3
 for arbritrary plane orientations.

\item  \verb|obj.GetPolyData (vtkPolyData pd)| -  Grab the polydata (including points) that defines the spline.  The
 polydata consists of points and line segments numbering Resolution + 1
 and Resoltuion, respectively. Points are guaranteed to be up-to-date when
 either the InteractionEvent or  EndInteraction events are invoked. The
 user provides the vtkPolyData and the points and polyline are added to it.

\item  \verb|obj.SetHandleProperty (vtkProperty )| -  Set/Get the handle properties (the spheres are the handles). The
 properties of the handles when selected and unselected can be manipulated.

\item  \verb|vtkProperty = obj.GetHandleProperty ()| -  Set/Get the handle properties (the spheres are the handles). The
 properties of the handles when selected and unselected can be manipulated.

\item  \verb|obj.SetSelectedHandleProperty (vtkProperty )| -  Set/Get the handle properties (the spheres are the handles). The
 properties of the handles when selected and unselected can be manipulated.

\item  \verb|vtkProperty = obj.GetSelectedHandleProperty ()| -  Set/Get the handle properties (the spheres are the handles). The
 properties of the handles when selected and unselected can be manipulated.

\item  \verb|obj.SetLineProperty (vtkProperty )| -  Set/Get the line properties. The properties of the line when selected
 and unselected can be manipulated.

\item  \verb|vtkProperty = obj.GetLineProperty ()| -  Set/Get the line properties. The properties of the line when selected
 and unselected can be manipulated.

\item  \verb|obj.SetSelectedLineProperty (vtkProperty )| -  Set/Get the line properties. The properties of the line when selected
 and unselected can be manipulated.

\item  \verb|vtkProperty = obj.GetSelectedLineProperty ()| -  Set/Get the line properties. The properties of the line when selected
 and unselected can be manipulated.

\item  \verb|obj.SetNumberOfHandles (int npts)| -  Set/Get the number of handles for this widget.

\item  \verb|int = obj.GetNumberOfHandles ()| -  Set/Get the number of handles for this widget.

\item  \verb|obj.SetResolution (int resolution)| -  Set/Get the number of line segments representing the spline for
 this widget.

\item  \verb|int = obj.GetResolution ()| -  Set/Get the number of line segments representing the spline for
 this widget.

\item  \verb|obj.SetParametricSpline (vtkParametricSpline )| -  Set the parametric spline object. Through vtkParametricSpline's API, the
 user can supply and configure one of currently two types of spline:
 vtkCardinalSpline, vtkKochanekSpline. The widget controls the open
 or closed configuration of the spline.
 WARNING: The widget does not enforce internal consistency so that all
 three are of the same type.

\item  \verb|vtkParametricSpline = obj.GetParametricSpline ()| -  Set the parametric spline object. Through vtkParametricSpline's API, the
 user can supply and configure one of currently two types of spline:
 vtkCardinalSpline, vtkKochanekSpline. The widget controls the open
 or closed configuration of the spline.
 WARNING: The widget does not enforce internal consistency so that all
 three are of the same type.

\item  \verb|obj.SetHandlePosition (int handle, double x, double y, double z)| -  Set/Get the position of the spline handles. Call GetNumberOfHandles
 to determine the valid range of handle indices.

\item  \verb|obj.SetHandlePosition (int handle, double xyz[3])| -  Set/Get the position of the spline handles. Call GetNumberOfHandles
 to determine the valid range of handle indices.

\item  \verb|obj.GetHandlePosition (int handle, double xyz[3])| -  Set/Get the position of the spline handles. Call GetNumberOfHandles
 to determine the valid range of handle indices.

\item  \verb|double = obj.GetHandlePosition (int handle)| -  Set/Get the position of the spline handles. Call GetNumberOfHandles
 to determine the valid range of handle indices.

\item  \verb|obj.SetClosed (int closed)| -  Control whether the spline is open or closed. A closed spline forms
 a continuous loop: the first and last points are the same, and
 derivatives are continuous.  A minimum of 3 handles are required to
 form a closed loop.  This method enforces consistency with
 user supplied subclasses of vtkSpline.

\item  \verb|int = obj.GetClosed ()| -  Control whether the spline is open or closed. A closed spline forms
 a continuous loop: the first and last points are the same, and
 derivatives are continuous.  A minimum of 3 handles are required to
 form a closed loop.  This method enforces consistency with
 user supplied subclasses of vtkSpline.

\item  \verb|obj.ClosedOn ()| -  Control whether the spline is open or closed. A closed spline forms
 a continuous loop: the first and last points are the same, and
 derivatives are continuous.  A minimum of 3 handles are required to
 form a closed loop.  This method enforces consistency with
 user supplied subclasses of vtkSpline.

\item  \verb|obj.ClosedOff ()| -  Control whether the spline is open or closed. A closed spline forms
 a continuous loop: the first and last points are the same, and
 derivatives are continuous.  A minimum of 3 handles are required to
 form a closed loop.  This method enforces consistency with
 user supplied subclasses of vtkSpline.

\item  \verb|int = obj.IsClosed ()| -  Convenience method to determine whether the spline is
 closed in a geometric sense.  The widget may be set ''closed'' but still
 be geometrically open (e.g., a straight line).

\item  \verb|double = obj.GetSummedLength ()| -  Get the approximate vs. the true arc length of the spline. Calculated as
 the summed lengths of the individual straight line segments. Use
 SetResolution to control the accuracy.

\item  \verb|obj.InitializeHandles (vtkPoints points)| -  Convenience method to allocate and set the handles from a vtkPoints
 instance.  If the first and last points are the same, the spline sets
 Closed to the on state and disregards the last point, otherwise Closed
 remains unchanged.

\end{itemize}
