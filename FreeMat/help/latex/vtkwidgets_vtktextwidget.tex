\section{vtkTextWidget}

\subsection{Usage}

 This class provides support for interactively placing text on the 2D
 overlay plane. The text is defined by an instance of vtkTextActor. It uses
 the event bindings of its superclass (vtkBorderWidget). In addition, when
 the text is selected, the widget emits a WidgetActivateEvent that
 observers can watch for. This is useful for opening GUI dialogues to
 adjust font characteristics, etc. (Please see the superclass for a
 description of event bindings.)

To create an instance of class vtkTextWidget, simply
invoke its constructor as follows
\begin{verbatim}
  obj = vtkTextWidget
\end{verbatim}
\subsection{Methods}

The class vtkTextWidget has several methods that can be used.
  They are listed below.
Note that the documentation is translated automatically from the VTK sources,
and may not be completely intelligible.  When in doubt, consult the VTK website.
In the methods listed below, \verb|obj| is an instance of the vtkTextWidget class.
\begin{itemize}
\item  \verb|string = obj.GetClassName ()| -  Standard VTK methods.

\item  \verb|int = obj.IsA (string name)| -  Standard VTK methods.

\item  \verb|vtkTextWidget = obj.NewInstance ()| -  Standard VTK methods.

\item  \verb|vtkTextWidget = obj.SafeDownCast (vtkObject o)| -  Standard VTK methods.

\item  \verb|obj.SetRepresentation (vtkTextRepresentation r)| -  Specify a vtkTextActor to manage. This is a convenient, alternative
 method to specify the representation for the widget (i.e., used instead
 of SetRepresentation()). It internally creates a vtkTextRepresentation
 and then invokes vtkTextRepresentation::SetTextActor().

\item  \verb|obj.SetTextActor (vtkTextActor textActor)| -  Specify a vtkTextActor to manage. This is a convenient, alternative
 method to specify the representation for the widget (i.e., used instead
 of SetRepresentation()). It internally creates a vtkTextRepresentation
 and then invokes vtkTextRepresentation::SetTextActor().

\item  \verb|vtkTextActor = obj.GetTextActor ()| -  Specify a vtkTextActor to manage. This is a convenient, alternative
 method to specify the representation for the widget (i.e., used instead
 of SetRepresentation()). It internally creates a vtkTextRepresentation
 and then invokes vtkTextRepresentation::SetTextActor().

\item  \verb|obj.CreateDefaultRepresentation ()| -  Create the default widget representation if one is not set. 

\end{itemize}
