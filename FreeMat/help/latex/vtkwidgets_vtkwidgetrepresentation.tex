\section{vtkWidgetRepresentation}

\subsection{Usage}

 This class is used to define the API for, and partially implement, a
 representation for different types of widgets. Note that the widget
 representation (i.e., subclasses of vtkWidgetRepresentation) are a type of
 vtkProp; meaning that they can be associated with a vtkRenderer end
 embedded in a scene like any other vtkActor. However,
 vtkWidgetRepresentation also defines an API that enables it to be paired
 with a subclass vtkAbstractWidget, meaning that it can be driven by a
 widget, serving to represent the widget as the widget responds to
 registered events. 

 The API defined here should be regarded as a guideline for implementing
 widgets and widget representations. Widget behavior is complex, as is the
 way the representation responds to the registered widget events, so the API
 may vary from widget to widget to reflect this complexity.

To create an instance of class vtkWidgetRepresentation, simply
invoke its constructor as follows
\begin{verbatim}
  obj = vtkWidgetRepresentation
\end{verbatim}
\subsection{Methods}

The class vtkWidgetRepresentation has several methods that can be used.
  They are listed below.
Note that the documentation is translated automatically from the VTK sources,
and may not be completely intelligible.  When in doubt, consult the VTK website.
In the methods listed below, \verb|obj| is an instance of the vtkWidgetRepresentation class.
\begin{itemize}
\item  \verb|string = obj.GetClassName ()| -  Standard methods for instances of this class.

\item  \verb|int = obj.IsA (string name)| -  Standard methods for instances of this class.

\item  \verb|vtkWidgetRepresentation = obj.NewInstance ()| -  Standard methods for instances of this class.

\item  \verb|vtkWidgetRepresentation = obj.SafeDownCast (vtkObject o)| -  Standard methods for instances of this class.

\item  \verb|obj.SetRenderer (vtkRenderer ren)| -  Subclasses of vtkWidgetRepresentation must implement these methods. This is
 considered the minimum API for a widget representation.
 \begin{verbatim}
 SetRenderer() - the renderer in which the widget is to appear must be set.
 BuildRepresentation() - update the geometry of the widget based on its
                         current state.
 \end{verbatim}
 WARNING: The renderer is NOT reference counted by the representation,
 in order to avoid reference loops.  Be sure that the representation
 lifetime does not extend beyond the renderer lifetime.

\item  \verb|vtkRenderer = obj.GetRenderer ()| -  Subclasses of vtkWidgetRepresentation must implement these methods. This is
 considered the minimum API for a widget representation.
 \begin{verbatim}
 SetRenderer() - the renderer in which the widget is to appear must be set.
 BuildRepresentation() - update the geometry of the widget based on its
                         current state.
 \end{verbatim}
 WARNING: The renderer is NOT reference counted by the representation,
 in order to avoid reference loops.  Be sure that the representation
 lifetime does not extend beyond the renderer lifetime.

\item  \verb|obj.BuildRepresentation ()| -  Subclasses of vtkWidgetRepresentation must implement these methods. This is
 considered the minimum API for a widget representation.
 \begin{verbatim}
 SetRenderer() - the renderer in which the widget is to appear must be set.
 BuildRepresentation() - update the geometry of the widget based on its
                         current state.
 \end{verbatim}
 WARNING: The renderer is NOT reference counted by the representation,
 in order to avoid reference loops.  Be sure that the representation
 lifetime does not extend beyond the renderer lifetime.

\item  \verb|obj.PlaceWidget (double )| -  The following is a suggested API for widget representations. These methods
 define the communication between the widget and its representation. These
 methods are only suggestions because widgets take on so many different
 forms that a universal API is not deemed practical. However, these methods
 should be implemented when possible to insure that the VTK widget hierarchy
 remains self-consistent.
 \begin{verbatim}
 PlaceWidget() - given a bounding box (xmin,xmax,ymin,ymax,zmin,zmax), place 
                 the widget inside of it. The current orientation of the widget 
                 is preserved, only scaling and translation is performed.
 StartWidgetInteraction() - generally corresponds to a initial event (e.g.,
                            mouse down) that starts the interaction process
                            with the widget.
 WidgetInteraction() - invoked when an event causes the widget to change 
                       appearance.
 EndWidgetInteraction() - generally corresponds to a final event (e.g., mouse up)
                          and completes the interaction sequence.
 ComputeInteractionState() - given (X,Y) display coordinates in a renderer, with a
                             possible flag that modifies the computation,
                             what is the state of the widget?
 GetInteractionState() - return the current state of the widget. Note that the
                         value of ''0'' typically refers to ''outside''. The 
                         interaction state is strictly a function of the
                         representation, and the widget/represent must agree
                         on what they mean.
 Highlight() - turn on or off any highlights associated with the widget.
               Highlights are generally turned on when the widget is selected.
 \end{verbatim}
 Note that subclasses may ignore some of these methods and implement their own
 depending on the specifics of the widget.

\item  \verb|obj.StartWidgetInteraction (double eventPos[2])| -  The following is a suggested API for widget representations. These methods
 define the communication between the widget and its representation. These
 methods are only suggestions because widgets take on so many different
 forms that a universal API is not deemed practical. However, these methods
 should be implemented when possible to insure that the VTK widget hierarchy
 remains self-consistent.
 \begin{verbatim}
 PlaceWidget() - given a bounding box (xmin,xmax,ymin,ymax,zmin,zmax), place 
                 the widget inside of it. The current orientation of the widget 
                 is preserved, only scaling and translation is performed.
 StartWidgetInteraction() - generally corresponds to a initial event (e.g.,
                            mouse down) that starts the interaction process
                            with the widget.
 WidgetInteraction() - invoked when an event causes the widget to change 
                       appearance.
 EndWidgetInteraction() - generally corresponds to a final event (e.g., mouse up)
                          and completes the interaction sequence.
 ComputeInteractionState() - given (X,Y) display coordinates in a renderer, with a
                             possible flag that modifies the computation,
                             what is the state of the widget?
 GetInteractionState() - return the current state of the widget. Note that the
                         value of ''0'' typically refers to ''outside''. The 
                         interaction state is strictly a function of the
                         representation, and the widget/represent must agree
                         on what they mean.
 Highlight() - turn on or off any highlights associated with the widget.
               Highlights are generally turned on when the widget is selected.
 \end{verbatim}
 Note that subclasses may ignore some of these methods and implement their own
 depending on the specifics of the widget.

\item  \verb|obj.WidgetInteraction (double newEventPos[2])| -  The following is a suggested API for widget representations. These methods
 define the communication between the widget and its representation. These
 methods are only suggestions because widgets take on so many different
 forms that a universal API is not deemed practical. However, these methods
 should be implemented when possible to insure that the VTK widget hierarchy
 remains self-consistent.
 \begin{verbatim}
 PlaceWidget() - given a bounding box (xmin,xmax,ymin,ymax,zmin,zmax), place 
                 the widget inside of it. The current orientation of the widget 
                 is preserved, only scaling and translation is performed.
 StartWidgetInteraction() - generally corresponds to a initial event (e.g.,
                            mouse down) that starts the interaction process
                            with the widget.
 WidgetInteraction() - invoked when an event causes the widget to change 
                       appearance.
 EndWidgetInteraction() - generally corresponds to a final event (e.g., mouse up)
                          and completes the interaction sequence.
 ComputeInteractionState() - given (X,Y) display coordinates in a renderer, with a
                             possible flag that modifies the computation,
                             what is the state of the widget?
 GetInteractionState() - return the current state of the widget. Note that the
                         value of ''0'' typically refers to ''outside''. The 
                         interaction state is strictly a function of the
                         representation, and the widget/represent must agree
                         on what they mean.
 Highlight() - turn on or off any highlights associated with the widget.
               Highlights are generally turned on when the widget is selected.
 \end{verbatim}
 Note that subclasses may ignore some of these methods and implement their own
 depending on the specifics of the widget.

\item  \verb|obj.EndWidgetInteraction (double newEventPos[2])| -  The following is a suggested API for widget representations. These methods
 define the communication between the widget and its representation. These
 methods are only suggestions because widgets take on so many different
 forms that a universal API is not deemed practical. However, these methods
 should be implemented when possible to insure that the VTK widget hierarchy
 remains self-consistent.
 \begin{verbatim}
 PlaceWidget() - given a bounding box (xmin,xmax,ymin,ymax,zmin,zmax), place 
                 the widget inside of it. The current orientation of the widget 
                 is preserved, only scaling and translation is performed.
 StartWidgetInteraction() - generally corresponds to a initial event (e.g.,
                            mouse down) that starts the interaction process
                            with the widget.
 WidgetInteraction() - invoked when an event causes the widget to change 
                       appearance.
 EndWidgetInteraction() - generally corresponds to a final event (e.g., mouse up)
                          and completes the interaction sequence.
 ComputeInteractionState() - given (X,Y) display coordinates in a renderer, with a
                             possible flag that modifies the computation,
                             what is the state of the widget?
 GetInteractionState() - return the current state of the widget. Note that the
                         value of ''0'' typically refers to ''outside''. The 
                         interaction state is strictly a function of the
                         representation, and the widget/represent must agree
                         on what they mean.
 Highlight() - turn on or off any highlights associated with the widget.
               Highlights are generally turned on when the widget is selected.
 \end{verbatim}
 Note that subclasses may ignore some of these methods and implement their own
 depending on the specifics of the widget.

\item  \verb|int = obj.ComputeInteractionState (int X, int Y, int modify)| -  The following is a suggested API for widget representations. These methods
 define the communication between the widget and its representation. These
 methods are only suggestions because widgets take on so many different
 forms that a universal API is not deemed practical. However, these methods
 should be implemented when possible to insure that the VTK widget hierarchy
 remains self-consistent.
 \begin{verbatim}
 PlaceWidget() - given a bounding box (xmin,xmax,ymin,ymax,zmin,zmax), place 
                 the widget inside of it. The current orientation of the widget 
                 is preserved, only scaling and translation is performed.
 StartWidgetInteraction() - generally corresponds to a initial event (e.g.,
                            mouse down) that starts the interaction process
                            with the widget.
 WidgetInteraction() - invoked when an event causes the widget to change 
                       appearance.
 EndWidgetInteraction() - generally corresponds to a final event (e.g., mouse up)
                          and completes the interaction sequence.
 ComputeInteractionState() - given (X,Y) display coordinates in a renderer, with a
                             possible flag that modifies the computation,
                             what is the state of the widget?
 GetInteractionState() - return the current state of the widget. Note that the
                         value of ''0'' typically refers to ''outside''. The 
                         interaction state is strictly a function of the
                         representation, and the widget/represent must agree
                         on what they mean.
 Highlight() - turn on or off any highlights associated with the widget.
               Highlights are generally turned on when the widget is selected.
 \end{verbatim}
 Note that subclasses may ignore some of these methods and implement their own
 depending on the specifics of the widget.

\item  \verb|int = obj.GetInteractionState ()| -  The following is a suggested API for widget representations. These methods
 define the communication between the widget and its representation. These
 methods are only suggestions because widgets take on so many different
 forms that a universal API is not deemed practical. However, these methods
 should be implemented when possible to insure that the VTK widget hierarchy
 remains self-consistent.
 \begin{verbatim}
 PlaceWidget() - given a bounding box (xmin,xmax,ymin,ymax,zmin,zmax), place 
                 the widget inside of it. The current orientation of the widget 
                 is preserved, only scaling and translation is performed.
 StartWidgetInteraction() - generally corresponds to a initial event (e.g.,
                            mouse down) that starts the interaction process
                            with the widget.
 WidgetInteraction() - invoked when an event causes the widget to change 
                       appearance.
 EndWidgetInteraction() - generally corresponds to a final event (e.g., mouse up)
                          and completes the interaction sequence.
 ComputeInteractionState() - given (X,Y) display coordinates in a renderer, with a
                             possible flag that modifies the computation,
                             what is the state of the widget?
 GetInteractionState() - return the current state of the widget. Note that the
                         value of ''0'' typically refers to ''outside''. The 
                         interaction state is strictly a function of the
                         representation, and the widget/represent must agree
                         on what they mean.
 Highlight() - turn on or off any highlights associated with the widget.
               Highlights are generally turned on when the widget is selected.
 \end{verbatim}
 Note that subclasses may ignore some of these methods and implement their own
 depending on the specifics of the widget.

\item  \verb|obj.Highlight (int )| -  Set/Get a factor representing the scaling of the widget upon placement
 (via the PlaceWidget() method). Normally the widget is placed so that
 it just fits within the bounding box defined in PlaceWidget(bounds).
 The PlaceFactor will make the widget larger (PlaceFactor > 1) or smaller
 (PlaceFactor < 1). By default, PlaceFactor is set to 0.5.

\item  \verb|obj.SetPlaceFactor (double )| -  Set/Get a factor representing the scaling of the widget upon placement
 (via the PlaceWidget() method). Normally the widget is placed so that
 it just fits within the bounding box defined in PlaceWidget(bounds).
 The PlaceFactor will make the widget larger (PlaceFactor > 1) or smaller
 (PlaceFactor < 1). By default, PlaceFactor is set to 0.5.

\item  \verb|double = obj.GetPlaceFactorMinValue ()| -  Set/Get a factor representing the scaling of the widget upon placement
 (via the PlaceWidget() method). Normally the widget is placed so that
 it just fits within the bounding box defined in PlaceWidget(bounds).
 The PlaceFactor will make the widget larger (PlaceFactor > 1) or smaller
 (PlaceFactor < 1). By default, PlaceFactor is set to 0.5.

\item  \verb|double = obj.GetPlaceFactorMaxValue ()| -  Set/Get a factor representing the scaling of the widget upon placement
 (via the PlaceWidget() method). Normally the widget is placed so that
 it just fits within the bounding box defined in PlaceWidget(bounds).
 The PlaceFactor will make the widget larger (PlaceFactor > 1) or smaller
 (PlaceFactor < 1). By default, PlaceFactor is set to 0.5.

\item  \verb|double = obj.GetPlaceFactor ()| -  Set/Get a factor representing the scaling of the widget upon placement
 (via the PlaceWidget() method). Normally the widget is placed so that
 it just fits within the bounding box defined in PlaceWidget(bounds).
 The PlaceFactor will make the widget larger (PlaceFactor > 1) or smaller
 (PlaceFactor < 1). By default, PlaceFactor is set to 0.5.

\item  \verb|obj.SetHandleSize (double )| -  Set/Get the factor that controls the size of the handles that appear as
 part of the widget (if any). These handles (like spheres, etc.)  are
 used to manipulate the widget. The HandleSize data member allows you
 to change the relative size of the handles. Note that while the handle
 size is typically expressed in pixels, some subclasses may use a relative size
 with respect to the viewport. (As a corollary, the value of this ivar is often
 set by subclasses of this class during instance instantiation.)

\item  \verb|double = obj.GetHandleSizeMinValue ()| -  Set/Get the factor that controls the size of the handles that appear as
 part of the widget (if any). These handles (like spheres, etc.)  are
 used to manipulate the widget. The HandleSize data member allows you
 to change the relative size of the handles. Note that while the handle
 size is typically expressed in pixels, some subclasses may use a relative size
 with respect to the viewport. (As a corollary, the value of this ivar is often
 set by subclasses of this class during instance instantiation.)

\item  \verb|double = obj.GetHandleSizeMaxValue ()| -  Set/Get the factor that controls the size of the handles that appear as
 part of the widget (if any). These handles (like spheres, etc.)  are
 used to manipulate the widget. The HandleSize data member allows you
 to change the relative size of the handles. Note that while the handle
 size is typically expressed in pixels, some subclasses may use a relative size
 with respect to the viewport. (As a corollary, the value of this ivar is often
 set by subclasses of this class during instance instantiation.)

\item  \verb|double = obj.GetHandleSize ()| -  Set/Get the factor that controls the size of the handles that appear as
 part of the widget (if any). These handles (like spheres, etc.)  are
 used to manipulate the widget. The HandleSize data member allows you
 to change the relative size of the handles. Note that while the handle
 size is typically expressed in pixels, some subclasses may use a relative size
 with respect to the viewport. (As a corollary, the value of this ivar is often
 set by subclasses of this class during instance instantiation.)

\item  \verb|int = obj.GetNeedToRender ()| -  Some subclasses use this data member to keep track of whether to render
 or not (i.e., to minimize the total number of renders).

\item  \verb|obj.SetNeedToRender (int )| -  Some subclasses use this data member to keep track of whether to render
 or not (i.e., to minimize the total number of renders).

\item  \verb|int = obj.GetNeedToRenderMinValue ()| -  Some subclasses use this data member to keep track of whether to render
 or not (i.e., to minimize the total number of renders).

\item  \verb|int = obj.GetNeedToRenderMaxValue ()| -  Some subclasses use this data member to keep track of whether to render
 or not (i.e., to minimize the total number of renders).

\item  \verb|obj.NeedToRenderOn ()| -  Some subclasses use this data member to keep track of whether to render
 or not (i.e., to minimize the total number of renders).

\item  \verb|obj.NeedToRenderOff ()| -  Some subclasses use this data member to keep track of whether to render
 or not (i.e., to minimize the total number of renders).

\item  \verb|double = obj.GetBounds ()| -  Methods to make this class behave as a vtkProp. They are repeated here (from the
 vtkProp superclass) as a reminder to the widget implementor. Failure to implement
 these methods properly may result in the representation not appearing in the scene
 (i.e., not implementing the Render() methods properly) or leaking graphics resources
 (i.e., not implementing ReleaseGraphicsResources() properly).

\item  \verb|obj.ShallowCopy (vtkProp prop)| -  Methods to make this class behave as a vtkProp. They are repeated here (from the
 vtkProp superclass) as a reminder to the widget implementor. Failure to implement
 these methods properly may result in the representation not appearing in the scene
 (i.e., not implementing the Render() methods properly) or leaking graphics resources
 (i.e., not implementing ReleaseGraphicsResources() properly).

\item  \verb|obj.GetActors (vtkPropCollection )| -  Methods to make this class behave as a vtkProp. They are repeated here (from the
 vtkProp superclass) as a reminder to the widget implementor. Failure to implement
 these methods properly may result in the representation not appearing in the scene
 (i.e., not implementing the Render() methods properly) or leaking graphics resources
 (i.e., not implementing ReleaseGraphicsResources() properly).

\item  \verb|obj.GetActors2D (vtkPropCollection )| -  Methods to make this class behave as a vtkProp. They are repeated here (from the
 vtkProp superclass) as a reminder to the widget implementor. Failure to implement
 these methods properly may result in the representation not appearing in the scene
 (i.e., not implementing the Render() methods properly) or leaking graphics resources
 (i.e., not implementing ReleaseGraphicsResources() properly).

\item  \verb|obj.GetVolumes (vtkPropCollection )| -  Methods to make this class behave as a vtkProp. They are repeated here (from the
 vtkProp superclass) as a reminder to the widget implementor. Failure to implement
 these methods properly may result in the representation not appearing in the scene
 (i.e., not implementing the Render() methods properly) or leaking graphics resources
 (i.e., not implementing ReleaseGraphicsResources() properly).

\item  \verb|obj.ReleaseGraphicsResources (vtkWindow )| -  Methods to make this class behave as a vtkProp. They are repeated here (from the
 vtkProp superclass) as a reminder to the widget implementor. Failure to implement
 these methods properly may result in the representation not appearing in the scene
 (i.e., not implementing the Render() methods properly) or leaking graphics resources
 (i.e., not implementing ReleaseGraphicsResources() properly).

\item  \verb|int = obj.RenderOverlay (vtkViewport )| -  Methods to make this class behave as a vtkProp. They are repeated here (from the
 vtkProp superclass) as a reminder to the widget implementor. Failure to implement
 these methods properly may result in the representation not appearing in the scene
 (i.e., not implementing the Render() methods properly) or leaking graphics resources
 (i.e., not implementing ReleaseGraphicsResources() properly).

\item  \verb|int = obj.RenderOpaqueGeometry (vtkViewport )| -  Methods to make this class behave as a vtkProp. They are repeated here (from the
 vtkProp superclass) as a reminder to the widget implementor. Failure to implement
 these methods properly may result in the representation not appearing in the scene
 (i.e., not implementing the Render() methods properly) or leaking graphics resources
 (i.e., not implementing ReleaseGraphicsResources() properly).

\item  \verb|int = obj.RenderTranslucentPolygonalGeometry (vtkViewport )| -  Methods to make this class behave as a vtkProp. They are repeated here (from the
 vtkProp superclass) as a reminder to the widget implementor. Failure to implement
 these methods properly may result in the representation not appearing in the scene
 (i.e., not implementing the Render() methods properly) or leaking graphics resources
 (i.e., not implementing ReleaseGraphicsResources() properly).

\item  \verb|int = obj.RenderVolumetricGeometry (vtkViewport )| -  Methods to make this class behave as a vtkProp. They are repeated here (from the
 vtkProp superclass) as a reminder to the widget implementor. Failure to implement
 these methods properly may result in the representation not appearing in the scene
 (i.e., not implementing the Render() methods properly) or leaking graphics resources
 (i.e., not implementing ReleaseGraphicsResources() properly).

\item  \verb|int = obj.HasTranslucentPolygonalGeometry ()|

\end{itemize}
