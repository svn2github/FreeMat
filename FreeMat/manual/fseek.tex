% Copyright (c) 2002, 2003 Samit Basu
%
% Permission is hereby granted, free of charge, to any person obtaining a 
% copy of this software and associated documentation files (the "Software"), 
% to deal in the Software without restriction, including without limitation 
% the rights to use, copy, modify, merge, publish, distribute, sublicense, 
% and/or sell copies of the Software, and to permit persons to whom the 
% Software is furnished to do so, subject to the following conditions:
%
% The above copyright notice and this permission notice shall be included 
% in all copies or substantial portions of the Software.
%
% THE SOFTWARE IS PROVIDED "AS IS", WITHOUT WARRANTY OF ANY KIND, EXPRESS 
% OR IMPLIED, INCLUDING BUT NOT LIMITED TO THE WARRANTIES OF MERCHANTABILITY, 
% FITNESS FOR A PARTICULAR PURPOSE AND NONINFRINGEMENT. IN NO EVENT SHALL 
% THE AUTHORS OR COPYRIGHT HOLDERS BE LIABLE FOR ANY CLAIM, DAMAGES OR OTHER 
% LIABILITY, WHETHER IN AN ACTION OF CONTRACT, TORT OR OTHERWISE, ARISING
% FROM, OUT OF OR IN CONNECTION WITH THE SOFTWARE OR THE USE OR OTHER 
% DEALINGS IN THE SOFTWARE.
\subsection{FSEEK Seek File To A Given Position}
\subsubsection{Usage}
Moves the file pointer associated with the given file handle to 
the specified offset (in bytes).  The usage is
\begin{verbatim}
  fseek(handle,offset,style)
\end{verbatim}
The \verb|handle| argument must be a value and active file handle.  The
\verb|offset| parameter indicates the desired seek offset (how much the
file pointer is moved in bytes).  The \verb|style| parameter determines
how the offset is treated.  Three values for the \verb|style| parameter
are understood:
\begin{itemize}
\item string \verb|'bof'| or the value -1, which indicate the seek is relative
to the beginning of the file.  This is equivalent to \verb|SEEK_SET| in
ANSI C.
\item string \verb|'cof'| or the value 0, which indicates the seek is relative
to the current position of the file.  This is equivalent to 
\verb|SEEK_CUR| in ANSI C.
\item string \verb|'eof'| or the value 1, which indicates the seek is relative
to the end of the file.  This is equivalent to \verb|SEEK_END| in ANSI
C.
\end{itemize}
The offset can be positive or negative.
\subsubsection{Example}
The first example reads a file and then ``rewinds'' the file pointer by seeking to the beginning.
\begin{verbatim}
--> fp = fopen('test.dat','rb');
--> x = fread(fp,[1,inf],'float');
--> fseek(fp,0,'bof');
--> y = fread(fp,[1,inf],'float');
--> who x y
  Variable Name      Type   Flags   Size
              x     float           [1 262144]
              y     float           [1 262144]
\end{verbatim}
The next example seeks forward by $2048$ bytes from the files current position, and then reads a line of $512$ floats.
\begin{verbatim}
--> fseek(fp,2048,'cof');
--> x = fread(fp,[512,1],'float');
\end{verbatim}
