% Copyright (c) 2002, 2003 Samit Basu
%
% Permission is hereby granted, free of charge, to any person obtaining a 
% copy of this software and associated documentation files (the "Software"), 
% to deal in the Software without restriction, including without limitation 
% the rights to use, copy, modify, merge, publish, distribute, sublicense, 
% and/or sell copies of the Software, and to permit persons to whom the 
% Software is furnished to do so, subject to the following conditions:
%
% The above copyright notice and this permission notice shall be included 
% in all copies or substantial portions of the Software.
%
% THE SOFTWARE IS PROVIDED "AS IS", WITHOUT WARRANTY OF ANY KIND, EXPRESS 
% OR IMPLIED, INCLUDING BUT NOT LIMITED TO THE WARRANTIES OF MERCHANTABILITY, 
% FITNESS FOR A PARTICULAR PURPOSE AND NONINFRINGEMENT. IN NO EVENT SHALL 
% THE AUTHORS OR COPYRIGHT HOLDERS BE LIABLE FOR ANY CLAIM, DAMAGES OR OTHER 
% LIABILITY, WHETHER IN AN ACTION OF CONTRACT, TORT OR OTHERWISE, ARISING
% FROM, OUT OF OR IN CONNECTION WITH THE SOFTWARE OR THE USE OR OTHER 
% DEALINGS IN THE SOFTWARE.
\subsection{FTELL File Position Function}
\subsubsection{Usage}
Returns the current file position for a valid file handle.
The general use of this function is
\begin{verbatim}
  n = ftell(handle)
\end{verbatim}
The \verb|handle| argument must be a valid and active file handle.  The
return is the offset into the file relative to the start of the
file (in bytes).
\subsubsection{Example}
Here is an example of using \verb|ftell| to determine the current file position.  We read $512$ 4-byte floats, which results in the file pointer being at position $512*4 = 2048$.
\begin{verbatim}
--> fp = fopen('test.dat','rb');
--> x = fread(fp,[512,1],'float');
--> ftell(fp)
ans =
  <uint32>  - size: [1 1]
         2048
\end{verbatim}
