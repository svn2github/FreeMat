% Copyright (c) 2002, 2003 Samit Basu
%
% Permission is hereby granted, free of charge, to any person obtaining a 
% copy of this software and associated documentation files (the "Software"), 
% to deal in the Software without restriction, including without limitation 
% the rights to use, copy, modify, merge, publish, distribute, sublicense, 
% and/or sell copies of the Software, and to permit persons to whom the 
% Software is furnished to do so, subject to the following conditions:
%
% The above copyright notice and this permission notice shall be included 
% in all copies or substantial portions of the Software.
%
% THE SOFTWARE IS PROVIDED "AS IS", WITHOUT WARRANTY OF ANY KIND, EXPRESS 
% OR IMPLIED, INCLUDING BUT NOT LIMITED TO THE WARRANTIES OF MERCHANTABILITY, 
% FITNESS FOR A PARTICULAR PURPOSE AND NONINFRINGEMENT. IN NO EVENT SHALL 
% THE AUTHORS OR COPYRIGHT HOLDERS BE LIABLE FOR ANY CLAIM, DAMAGES OR OTHER 
% LIABILITY, WHETHER IN AN ACTION OF CONTRACT, TORT OR OTHERWISE, ARISING
% FROM, OUT OF OR IN CONNECTION WITH THE SOFTWARE OR THE USE OR OTHER 
% DEALINGS IN THE SOFTWARE.
\subsection{GETPRINTLIMIT Get Limit For Printing Of Arrays}
\subsubsection{Usage}
Returns the limit on how many elements of an array are printed
using either the \verb|disp| function or using expressions on the
command line without a semi-colon.  The default is set to 
one thousand elements.  You can increase or decrease this
limit by calling \verb|setprintlimit|.  This function is provided
primarily so that you can temporarily change the output truncation
and then restore it to the previous value (see the examples).
\begin{verbatim}
   n=getprintlimit
\end{verbatim}
where $n$ is the current limit in use.
\subsubsection{Example}
Here is an example of using \verb|getprintlimit| along with \verb|setprintlimit| to temporarily change the output behavior of FreeMat.
\begin{verbatim}
--> n = getprintlimit
n =
  <uint32>  - size: [1 1]
         1000
--> setprintlimit(5);
--> A
ans =
  <double>  - size: [512 512]
  
Columns 1 to 2
   -0.496297986481338        -0.692245304946869
   -0.276042423147714        -1.692222693378866
    0.453899406407093
 
... Output truncated - use setprintlimit function to see more of the output ...
--> setprintlimit(n)
\end{verbatim}
