% Copyright (c) 2002, 2003 Samit Basu
%
% Permission is hereby granted, free of charge, to any person obtaining a 
% copy of this software and associated documentation files (the "Software"), 
% to deal in the Software without restriction, including without limitation 
% the rights to use, copy, modify, merge, publish, distribute, sublicense, 
% and/or sell copies of the Software, and to permit persons to whom the 
% Software is furnished to do so, subject to the following conditions:
%
% The above copyright notice and this permission notice shall be included 
% in all copies or substantial portions of the Software.
%
% THE SOFTWARE IS PROVIDED "AS IS", WITHOUT WARRANTY OF ANY KIND, EXPRESS 
% OR IMPLIED, INCLUDING BUT NOT LIMITED TO THE WARRANTIES OF MERCHANTABILITY, 
% FITNESS FOR A PARTICULAR PURPOSE AND NONINFRINGEMENT. IN NO EVENT SHALL 
% THE AUTHORS OR COPYRIGHT HOLDERS BE LIABLE FOR ANY CLAIM, DAMAGES OR OTHER 
% LIABILITY, WHETHER IN AN ACTION OF CONTRACT, TORT OR OTHERWISE, ARISING
% FROM, OUT OF OR IN CONNECTION WITH THE SOFTWARE OR THE USE OR OTHER 
% DEALINGS IN THE SOFTWARE.
\subsection{HOLD Plot Hold Toggle Function}
\subsubsection{Usage}
Toggles the hold state on the currently active plot.  The
general syntax for its use is
\begin{verbatim}
   grid(state)
\end{verbatim}
where \verb|state| is either
\begin{verbatim}
   hold('on')
\end{verbatim}
to turn hold on, or
\begin{verbatim}
   hold('off')
\end{verbatim}
to turn hold off.
\subsubsection{Function Internals}
The \verb|hold| function allows one to construct a plot sequence
incrementally, instead of issuing all of the plots simultaneously
using the \verb|plot| command.
\subsubsection{Example}
Here is an example of using both the \verb|hold| command and the
multiple-argument \verb|plot| command to construct a plot composed
of three sets of data.  The first is a plot of a modulated Gaussian.
\begin{verbatim}
--> x = linspace(-5,5,500);
--> t = exp(-x.^2);
--> y = t.*cos(2*pi*x*3);
--> plot(x,y);
\end{verbatim}

\doplot{width=8cm}{hold1}

We now turn the hold state to \verb|'on'|, and add another plot
sequence, this time composed of the top and bottom envelopes of
the modulated Gaussian.  We add the two envelopes simultaneously
using a single \verb|plot| command.  The fact that \verb|hold| is
\verb|'on'| means that these two envelopes are add to (instead of
replace) the current contents of the plot.
\begin{verbatim}
--> plot(x,t,'g-',x,-t,'b-')
\end{verbatim}

\doplot{width=8cm}{hold2}
