% Copyright (c) 2002, 2003 Samit Basu
%
% Permission is hereby granted, free of charge, to any person obtaining a 
% copy of this software and associated documentation files (the "Software"), 
% to deal in the Software without restriction, including without limitation 
% the rights to use, copy, modify, merge, publish, distribute, sublicense, 
% and/or sell copies of the Software, and to permit persons to whom the 
% Software is furnished to do so, subject to the following conditions:
%
% The above copyright notice and this permission notice shall be included 
% in all copies or substantial portions of the Software.
%
% THE SOFTWARE IS PROVIDED "AS IS", WITHOUT WARRANTY OF ANY KIND, EXPRESS 
% OR IMPLIED, INCLUDING BUT NOT LIMITED TO THE WARRANTIES OF MERCHANTABILITY, 
% FITNESS FOR A PARTICULAR PURPOSE AND NONINFRINGEMENT. IN NO EVENT SHALL 
% THE AUTHORS OR COPYRIGHT HOLDERS BE LIABLE FOR ANY CLAIM, DAMAGES OR OTHER 
% LIABILITY, WHETHER IN AN ACTION OF CONTRACT, TORT OR OTHERWISE, ARISING
% FROM, OUT OF OR IN CONNECTION WITH THE SOFTWARE OR THE USE OR OTHER 
% DEALINGS IN THE SOFTWARE.
\subsection{INF Infinity Constant}
\subsubsection{Usage}
Returns a value that represents positive infinity 
for both 32 and 64-bit floating point values.
\begin{verbatim}
   y = inf
\end{verbatim}
The returned type is a 32-bit float, but promotion to
64 bits preserves the infinity.  The infinity constant has
several interesting properties.  In particular:
\begin{itemize}
\item $\infty \times 0 = \mathrm{NaN}$
\item $\infty \times a = \infty$ for all $a > 0$
\item $\infty \times a = -\infty$ for all $a < 0$
\item $\infty / \infty = \mathrm{NaN}$
\item $\infty / 0 = \infty$
\end{itemize}
Note that infinities are not preserved under type conversion to integer types (see the examples below).
\subsubsection{Example}
The following examples demonstrate the various properties of the infinity constant.
\begin{verbatim}
--> inf*0
ans =
  <float>  - size: [1 1]
    nan
--> inf*2
ans =
  <float>  - size: [1 1]
    inf
--> inf*-2
ans =
  <float>  - size: [1 1]
   -inf
--> inf/inf
ans =
  <float>  - size: [1 1]
    nan
--> inf/0
ans =
  <float>  - size: [1 1]
    inf
--> inf/nan
ans =
  <float>  - size: [1 1]
    nan
\end{verbatim}
Note that infinities are preserved under type conversion to floating point types (i.e., \verb|float|, \verb|double|, \verb|complex| and \verb|dcomplex| types), but not integer  types.
\begin{verbatim}
--> uint32(inf)
ans =
  <uint32>  - size: [1 1]
            0
--> complex(inf)
ans =
  <complex>  - size: [1 1]
    inf               0.00000000    i
\end{verbatim}

