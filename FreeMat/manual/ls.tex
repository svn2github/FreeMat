% Copyright (c) 2002, 2003 Samit Basu
%
% Permission is hereby granted, free of charge, to any person obtaining a 
% copy of this software and associated documentation files (the "Software"), 
% to deal in the Software without restriction, including without limitation 
% the rights to use, copy, modify, merge, publish, distribute, sublicense, 
% and/or sell copies of the Software, and to permit persons to whom the 
% Software is furnished to do so, subject to the following conditions:
%
% The above copyright notice and this permission notice shall be included 
% in all copies or substantial portions of the Software.
%
% THE SOFTWARE IS PROVIDED "AS IS", WITHOUT WARRANTY OF ANY KIND, EXPRESS 
% OR IMPLIED, INCLUDING BUT NOT LIMITED TO THE WARRANTIES OF MERCHANTABILITY, 
% FITNESS FOR A PARTICULAR PURPOSE AND NONINFRINGEMENT. IN NO EVENT SHALL 
% THE AUTHORS OR COPYRIGHT HOLDERS BE LIABLE FOR ANY CLAIM, DAMAGES OR OTHER 
% LIABILITY, WHETHER IN AN ACTION OF CONTRACT, TORT OR OTHERWISE, ARISING
% FROM, OUT OF OR IN CONNECTION WITH THE SOFTWARE OR THE USE OR OTHER 
% DEALINGS IN THE SOFTWARE.
\subsection{LS/DIR List Files Function}
\subsubsection{Usage}
Lists the files in a directory or directories.  The general syntax for its use is
\begin{verbatim}
  ls('dirname1','dirname2',...,'dirnameN')
\end{verbatim}
but this can also be expressed as
\begin{verbatim}
  ls 'dirname1' 'dirname2' ... 'dirnameN'
\end{verbatim}
or 
\begin{verbatim}
  ls dirname1 dirname2 ... dirnameN
\end{verbatim}
For compatibility with some environments, the function \verb|dir| can also be used instead of \verb|ls|.  Generally speaking, \verb|dirname| is any string that would be accepted by the underlying OS as a valid directory name.  For example, on most systems, \verb|'.'| refers to the current directory, and \verb|'..'| refers to the parent directory.  Also, depending on the OS, it may be necessary to ``escape'' the directory seperators.  In particular, if directories are seperated with the backwards-slash character \verb|'\\'|, then the path specification must use double-slashes \verb|'\\\\'|. Two points worth mentioning about the \verb|ls| function:
\begin{itemize}
  \item To get file-name completion to work at this time, you must use one of the first two forms of the command.
  \item If you want to capture the output of the \verb|ls| command, use the \verb|system| function instead.
\end{itemize}

\subsubsection{Example}
First, we use the simplest form of the \verb|ls| command, in which the directory name argument is given unquoted.
\begin{verbatim}
--> ls . ..
.:
App.o             Exception.o            LoadCore.o        Serialize.o
Array.o           File.o                 Makefile          ServerSocket.o
AST.o             foo.dat                Makefile~         Shell.o
<<truncated...>>

..:
config.log     doc      libFFTPack  Makefile  src           wxWindows-2.4.0
config.status  libCore  libtecla    manual    wxBase-2.4.0
\end{verbatim}
Next, we use the ``traditional'' form of the function call, using both the parenthesis and the quoted string.
\begin{verbatim}
--> ls('..')
config.log     doc      libFFTPack  Makefile  src           wxWindows-2.4.0
config.status  libCore  libtecla    manual    wxBase-2.4.0
\end{verbatim}
In the third version, we use only the quoted string argument without parenthesis.  We also use the filename-completion feature (usually by pressing the \verb|TAB| key) to get a list of possible completions.
\begin{verbatim}
--> ls '../../M
MFiles/      Makefile.am  Makefile.in
--> ls '../../MFiles/'
copper.m   flipud.m    linspace.m~  reshape.m  std.m~   tst.m~
copper.m~  flipud.m~   multcall.m   sqrt.m     tort.m   winlev.m
fliplr.m   gray.m      popdir.m     sqrt.m~    tort.m~  winlev.m~
fliplr.m~  linspace.m  pushdir.m    std.m      tst.m
\end{verbatim}
