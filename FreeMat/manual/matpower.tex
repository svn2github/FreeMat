% Copyright (c) 2002, 2003 Samit Basu
%
% Permission is hereby granted, free of charge, to any person obtaining a 
% copy of this software and associated documentation files (the "Software"), 
% to deal in the Software without restriction, including without limitation 
% the rights to use, copy, modify, merge, publish, distribute, sublicense, 
% and/or sell copies of the Software, and to permit persons to whom the 
% Software is furnished to do so, subject to the following conditions:
%
% The above copyright notice and this permission notice shall be included 
% in all copies or substantial portions of the Software.
%
% THE SOFTWARE IS PROVIDED "AS IS", WITHOUT WARRANTY OF ANY KIND, EXPRESS 
% OR IMPLIED, INCLUDING BUT NOT LIMITED TO THE WARRANTIES OF MERCHANTABILITY, 
% FITNESS FOR A PARTICULAR PURPOSE AND NONINFRINGEMENT. IN NO EVENT SHALL 
% THE AUTHORS OR COPYRIGHT HOLDERS BE LIABLE FOR ANY CLAIM, DAMAGES OR OTHER 
% LIABILITY, WHETHER IN AN ACTION OF CONTRACT, TORT OR OTHERWISE, ARISING
% FROM, OUT OF OR IN CONNECTION WITH THE SOFTWARE OR THE USE OR OTHER 
% DEALINGS IN THE SOFTWARE.
\subsection{POWER Matrix Power Operator}
\subsubsection{Usage}
The power operator for scalars and square matrices.  This operator is really a 
combination of two operators, both of which have the same general syntax:
\begin{verbatim}
  y = a ^ b
\end{verbatim}
The exact action taken by this operator, and the size and type of the output, 
depends on which of the two configurations of $a$ and $b$ is present:
\begin{enumerate}
  \item $a$ is a scalar, $b$ is a square matrix
  \item $a$ is a square matrix, $b$ is a scalar
\end{enumerate}
\subsubsection{Function Internals}
In the first case that $a$ is a scalar, and $b$ is a square matrix, the matrix power is defined in terms of the eigenvalue decomposition of $b$.  Let $b$ have the following eigen-decomposition (problems arise with non-symmetric matrices $b$, so let us assume that $b$ is symmetric):
\[
  b = E \begin{bmatrix} \lambda_1 & 0          & \cdots  & 0 \\
                              0   & \lambda_2  &  \ddots & \vdots \\
                              \vdots & \ddots & \ddots & 0 \\
                              0   & \hdots & 0 & \lambda_n \end{bmatrix}
      E^{-1}
\]
Then $a$ raised to the power $b$ is defined as
\[
  a^{b} = E \begin{bmatrix} a^{\lambda_1} & 0          & \cdots  & 0 \\
                              0   & a^{\lambda_2}  &  \ddots & \vdots \\
                              \vdots & \ddots & \ddots & 0 \\
                              0   & \hdots & 0 & a^{\lambda_n} \end{bmatrix}
      E^{-1}
\]
Similarly, if $a$ is a square matrix, then $a$ has the following eigen-decomposition (again, suppose $a$ is symmetric):
\[
  a = E \begin{bmatrix} \lambda_1 & 0          & \cdots  & 0 \\
                              0   & \lambda_2  &  \ddots & \vdots \\
                              \vdots & \ddots & \ddots & 0 \\
                              0   & \hdots & 0 & \lambda_n \end{bmatrix}
      E^{-1}
\]
Then $a$ raised to the power $b$ is defined as
\[
  a^{b} = E \begin{bmatrix} \lambda_1^b & 0          & \cdots  & 0 \\
                              0   & \lambda_2^b  &  \ddots & \vdots \\
                              \vdots & \ddots & \ddots & 0 \\
                              0   & \hdots & 0 & \lambda_n^b \end{bmatrix}
      E^{-1}
\]
\subsubsection{Examples}
We first define a simple $2 \times 2$ symmetric matrix
\begin{verbatim}
--> A = 1.5
A =
  <double>  - size: [1 1]
    1.500000000000000
--> B = [1,.2;.2,1]
B =
  <double>  - size: [2 2]
  
Columns 1 to 2
    1.000000000000000         0.200000000000000
    0.200000000000000         1.000000000000000
\end{verbatim}
First, we raise $B$ to the (scalar power) $A$:
\begin{verbatim}
--> C = B^A
C =
  <double>  - size: [2 2]
  
Columns 1 to 2
    1.015037945406166         0.299496192606233
    0.299496192606233         1.015037945406166
\end{verbatim}
Next, we raise $A$ to the matrix power $B$:
\begin{verbatim}
--> C = A^B
C =
  <double>  - size: [2 2]
  
Columns 1 to 2
    1.504934762009570         0.121772894786978
    0.121772894786978         1.504934762009570
\end{verbatim}
