% Copyright (c) 2002, 2003 Samit Basu
%
% Permission is hereby granted, free of charge, to any person obtaining a 
% copy of this software and associated documentation files (the "Software"), 
% to deal in the Software without restriction, including without limitation 
% the rights to use, copy, modify, merge, publish, distribute, sublicense, 
% and/or sell copies of the Software, and to permit persons to whom the 
% Software is furnished to do so, subject to the following conditions:
%
% The above copyright notice and this permission notice shall be included 
% in all copies or substantial portions of the Software.
%
% THE SOFTWARE IS PROVIDED "AS IS", WITHOUT WARRANTY OF ANY KIND, EXPRESS 
% OR IMPLIED, INCLUDING BUT NOT LIMITED TO THE WARRANTIES OF MERCHANTABILITY, 
% FITNESS FOR A PARTICULAR PURPOSE AND NONINFRINGEMENT. IN NO EVENT SHALL 
% THE AUTHORS OR COPYRIGHT HOLDERS BE LIABLE FOR ANY CLAIM, DAMAGES OR OTHER 
% LIABILITY, WHETHER IN AN ACTION OF CONTRACT, TORT OR OTHERWISE, ARISING
% FROM, OUT OF OR IN CONNECTION WITH THE SOFTWARE OR THE USE OR OTHER 
% DEALINGS IN THE SOFTWARE.
\subsection{PLOT Plot Function}
\subsubsection{Usage}
This is the basic plot command for FreeMat.  The general syntax for its
use is
\begin{verbatim}
  plot(<data 1>,{linespec 1},<data 2>,{linespec 2}...)
\end{verbatim}
where the \verb|<data>| arguments can have various forms, and the
\verb|linespec| arguments are optional.  We start with the
\verb|<data>| term, which can take on one of multiple forms:
\begin{itemize}
  \item \emph{Vector Matrix Case} -- In this case the argument data is a pair
    of variables.  A set of \verb|x| coordinates in a numeric vector, and a 
    set of \verb|y| coordinates in the columns of the second, numeric matrix.
    \verb|x| must have as many elements as \verb|y| has columns (unless \verb|y|
    is a vector, in which case only the number of elements must match).  Each
    column of \verb|y| is plotted sequentially against the common vector \verb|x|.
  \item \emph{Unpaired Matrix Case} -- In this case the argument data is a 
    single numeric matrix \verb|y| that constitutes the \verb|y|-values
    of the plot.  An \verb|x| vector is synthesized as $x = 1:\mathrm{length}(y)$,
    and each column of $y$ is plotted sequentially against this common \verb|x|
    axis.
  \item \emph{Complex Matrix Case} -- Here the argument data is a complex
    matrix, in which case, the real part of each column is plotted against
    the imaginary part of each column.  All columns receive the same line
    styles.
\end{itemize}
Multiple data arguments in a single plot command are treated as a \emph{sequence}, meaning
that all of the plots are overlapped on the same set of axes.
The \verb|linespec| is a string used to change the characteristics of the line.  In general,
the \verb|linespec| is composed of three optional parts, the \verb|colorspec|, the 
\verb|symbolspec| and the \verb|linestylespec| in any order.  Each of these specifications
is a single character that determines the corresponding characteristic.  First, the 
\verb|colorspec|:
\begin{itemize}
  \item \verb|'r'| - Color Red
  \item \verb|'g'| - Color Green
  \item \verb|'b'| - Color Blue
  \item \verb|'k'| - Color Black
  \item \verb|'c'| - Color Cyan
  \item \verb|'m'| - Color Magenta
  \item \verb|'y'| - Color Yellow
\end{itemize}
The \verb|symbolspec| specifies the (optional) symbol to be drawn at each data point:
\begin{itemize}
  \item \verb|'.'| - Dot symbol
  \item \verb|'o'| - Circle symbol
  \item \verb|'x'| - $\times$ (times or x) symbol
  \item \verb|'+'| - $+$ (plus) symbol
  \item \verb|'*'| - $*$ (times) symbol
  \item \verb|'s'| - Square symbol
  \item \verb|'d'| - Diamond symbol
  \item \verb|'v'| - Downward-pointing triangle symbol
  \item \verb|'^'| - Upward-pointing triangle symbol
  \item \verb|'<'| - Left-pointing triangle symbol
  \item \verb|'>'| - Right-pointing triangle symbol
\end{itemize}
The \verb|linestylespec| specifies the (optional) line style to use for each data series:
\begin{itemize}
  \item \verb|'-'| - Solid line style
  \item \verb|':'| - Dotted line style
  \item \verb|';'| - Dot-Dash-Dot-Dash line style
  \item \verb|'||'| - Dashed line style
\end{itemize}
For sequences of plots, the \verb|linespec| is recycled with colors taken sequentially from
the palette.
\subsubsection{Example}
The most common use of the \verb|plot| command probably involves the vector-matrix
paired case.  Here, we generate a simple cosine, and plot it using a red line, with
no symbols (i.e., a \verb|linespec| of \verb|'r-'|).
\begin{verbatim}
--> x = linspace(-pi,pi);
--> y = cos(x);
--> plot(x,y,'r-');
\end{verbatim}
which results in the following plot.

\doplot{width=8cm}{plot1}

Next, we plot multiple sinusoids (at different frequencies).  First, we construct
a matrix, in which each column corresponds to a different sinusoid, and then plot
them all at once.
\begin{verbatim}
--> x = linspace(-pi,pi);
--> y = [cos(x(:)),cos(3*x(:)),cos(5*x(:))];
--> plot(x,y);
\end{verbatim}
In this case, we do not specify a \verb|linespec|, so that we cycle through the
colors automatically (in the order listed in the previous section).

\doplot{width=8cm}{plot2}

This time, we produce the same plot, but as we want to assign individual
\verb|linespec|s to each line, we use a sequence of arguments in a single plot
command, which has the effect of plotting all of the data sets on a common 
axis, but which allows us to control the \verb|linespec| of each plot. In 
the following example, the first line (harmonic) has red, solid lines with $\times$ symbols
marking the data points, the second line (third harmonic) has blue, solid lines
with right-pointing triangle symbols, and the third line (fifth harmonic) has
green, dotted lines with $*$ symbols.
\begin{verbatim}
--> plot(x,y(:,1),'rx-',x,y(:,2),'b>-',x,y(:,3),'g*:');
\end{verbatim}

\doplot{width=8cm}{plot3}

The second most frequently used case is the unpaired matrix case.  Here, we need
to provide only one data component, which will be automatically plotted against
a vector of natural number of the appropriate length.  Here, we use a plot sequence
to change the style of each line to be dotted, dot-dashed, and dashed.
\begin{verbatim}
--> plot(y(:,1),'r:',y(:,2),'b;',y(:,3),'g|');
\end{verbatim}
Note in the resulting plot that the $x$-axis no longer runs from $[-\pi,\pi]$, but 
instead runs from $[1,100]$.

\doplot{width=8cm}{plot4}

The final case is for complex matrices.  For complex arguments, the real part is
plotted against the imaginary part.  Hence, we can generate a 2-dimensional plot
from a vector as follows.
\begin{verbatim}
--> y = cos(2*x) + i * cos(3*x);
--> plot(y);
\end{verbatim}

\doplot{width=8cm}{plot5}

