% Copyright (c) 2002, 2003 Samit Basu
%
% Permission is hereby granted, free of charge, to any person obtaining a 
% copy of this software and associated documentation files (the "Software"), 
% to deal in the Software without restriction, including without limitation 
% the rights to use, copy, modify, merge, publish, distribute, sublicense, 
% and/or sell copies of the Software, and to permit persons to whom the 
% Software is furnished to do so, subject to the following conditions:
%
% The above copyright notice and this permission notice shall be included 
% in all copies or substantial portions of the Software.
%
% THE SOFTWARE IS PROVIDED "AS IS", WITHOUT WARRANTY OF ANY KIND, EXPRESS 
% OR IMPLIED, INCLUDING BUT NOT LIMITED TO THE WARRANTIES OF MERCHANTABILITY, 
% FITNESS FOR A PARTICULAR PURPOSE AND NONINFRINGEMENT. IN NO EVENT SHALL 
% THE AUTHORS OR COPYRIGHT HOLDERS BE LIABLE FOR ANY CLAIM, DAMAGES OR OTHER 
% LIABILITY, WHETHER IN AN ACTION OF CONTRACT, TORT OR OTHERWISE, ARISING
% FROM, OUT OF OR IN CONNECTION WITH THE SOFTWARE OR THE USE OR OTHER 
% DEALINGS IN THE SOFTWARE.
\subsection{POINT Image Point Information Function}
\subsubsection{Usage}
Returns information about the currently displayed image based on a use
supplied mouse-click.  The general syntax for its use is
\begin{verbatim}
   y = point
\end{verbatim}
The returned vector $y$ has three elements: 
\[
  y= [r,c,v]
\]
where $r,c$ are the row and column coordinates of the scalar image selected
by the user, and $v$ is the value of the scalar image at that point.  Image
zoom is automatically compensated for, so that $r,c$ are the corrdinates into
the original matrix.  They will generally be fractional to account for the
exact location of the mouse click.
\subsubsection{Example}
Let us suppose that the following image represents some function of interest
\begin{verbatim}
--> t = linspace(-1,1,400); x = cos(pi*t)'*cos(pi*t);
--> image(x)
--> zoom(1)
\end{verbatim}
 Which results in the following window appearing.

\doplot{width=5cm}{point1}

We then issue the \verb|point| command, and click on a point, which results
in the following output:
\begin{verbatim}
--> point
ans = 
  <double>  - size: [1 3]
 
Columns 1 to 2
  187.000000000000          190.000000000000         
 
Columns 3 to 3
    0.964156278324225
\end{verbatim}
