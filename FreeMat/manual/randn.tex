% Copyright (c) 2002, 2003 Samit Basu
%
% Permission is hereby granted, free of charge, to any person obtaining a 
% copy of this software and associated documentation files (the "Software"), 
% to deal in the Software without restriction, including without limitation 
% the rights to use, copy, modify, merge, publish, distribute, sublicense, 
% and/or sell copies of the Software, and to permit persons to whom the 
% Software is furnished to do so, subject to the following conditions:
%
% The above copyright notice and this permission notice shall be included 
% in all copies or substantial portions of the Software.
%
% THE SOFTWARE IS PROVIDED "AS IS", WITHOUT WARRANTY OF ANY KIND, EXPRESS 
% OR IMPLIED, INCLUDING BUT NOT LIMITED TO THE WARRANTIES OF MERCHANTABILITY, 
% FITNESS FOR A PARTICULAR PURPOSE AND NONINFRINGEMENT. IN NO EVENT SHALL 
% THE AUTHORS OR COPYRIGHT HOLDERS BE LIABLE FOR ANY CLAIM, DAMAGES OR OTHER 
% LIABILITY, WHETHER IN AN ACTION OF CONTRACT, TORT OR OTHERWISE, ARISING
% FROM, OUT OF OR IN CONNECTION WITH THE SOFTWARE OR THE USE OR OTHER 
% DEALINGS IN THE SOFTWARE.
\subsection{RANDN Gaussian (Normal) Random Number Generator}
\subsubsection{Usage}
Creates an array of pseudo-random numbers of the specified size.
The numbers are normally distributed with zero mean and a unit
standard deviation (i.e., $\mu = 0, \sigma = 1$). 
 Two seperate syntaxes are possible.  The first syntax specifies the array 
dimensions as a sequence of scalar dimensions:
\begin{verbatim}
  y = randn(d1,d2,...,dn).
\end{verbatim}
The resulting array has the given dimensions, and is filled with
random numbers.  The type of $y$ is \verb|double|, a 64-bit floating
point array.  To get arrays of other types, use the typecast 
functions.
    
The second syntax specifies the array dimensions as a vector,
where each element in the vector specifies a dimension length:
\begin{verbatim}
  y = randn([d1,d2,...,dn]).
\end{verbatim}
This syntax is more convenient for calling \verb|randn| using a 
variable for the argument.
\subsubsection{Function Internals}
Recall that the
probability density function (PDF) of a normal random variable is
\[
f(x) = \frac{1}{\sqrt{2\pi \sigma^2}} e^{\frac{-(x-\mu)^2}{2\sigma^2}},
\]
a plot of which (for $\sigma = 1$, $\mu = 0$) is:

\doplot{width=8cm}{gaus1}

The Gaussian random numbers are generated from pairs of uniform random numbers using a transformation technique. 
\subsubsection{Example}
The following example demonstrates an example of using the first form of the \verb|randn| function.
\begin{verbatim}
--> randn(2,2,2)
ans =
  <double>  - size: [2 2 2]
(:,:,1) =
  
Columns 1 to 2
   -1.824714799177492         0.499477291970000
   -2.242661857028426        -0.809619782690456
(:,:,2) =
  
Columns 1 to 2
    0.131523288781253        -0.0878669456790549
   -1.257297608727736        -0.208643747229610
\end{verbatim}
The second example demonstrates the second form of the \verb|randn| function.
\begin{verbatim}
--> randn([2,2,2])
ans =
  <double>  - size: [2 2 2]
(:,:,1) =
  
Columns 1 to 2
   -1.131985151699417         1.008877169831612
    0.757046741279659         0.697402631506596
(:,:,2) =
  
Columns 1 to 2
   -2.069556696507877         0.225805985082327
   -0.0722111216427055       -1.885904751186595
\end{verbatim}
In the next example, we create a large array of $10000$  normally distributed pseudo-random numbers.  We then shift the mean to $10$, and the variance to $5$.  We then numerically calculate the mean and variance using \verb|mean| and \verb|var|, respectively.
\begin{verbatim}
--> x = 10+sqrt(5)*randn(1,10000);
--> mean(x)
ans =
  <double>  - size: [1 1]
   10.0156526856828
--> var(x)
ans =
  <double>  - size: [1 1]
    4.969843340685114
\end{verbatim}
