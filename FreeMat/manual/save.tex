% Copyright (c) 2002, 2003 Samit Basu
%
% Permission is hereby granted, free of charge, to any person obtaining a 
% copy of this software and associated documentation files (the "Software"), 
% to deal in the Software without restriction, including without limitation 
% the rights to use, copy, modify, merge, publish, distribute, sublicense, 
% and/or sell copies of the Software, and to permit persons to whom the 
% Software is furnished to do so, subject to the following conditions:
%
% The above copyright notice and this permission notice shall be included 
% in all copies or substantial portions of the Software.
%
% THE SOFTWARE IS PROVIDED "AS IS", WITHOUT WARRANTY OF ANY KIND, EXPRESS 
% OR IMPLIED, INCLUDING BUT NOT LIMITED TO THE WARRANTIES OF MERCHANTABILITY, 
% FITNESS FOR A PARTICULAR PURPOSE AND NONINFRINGEMENT. IN NO EVENT SHALL 
% THE AUTHORS OR COPYRIGHT HOLDERS BE LIABLE FOR ANY CLAIM, DAMAGES OR OTHER 
% LIABILITY, WHETHER IN AN ACTION OF CONTRACT, TORT OR OTHERWISE, ARISING
% FROM, OUT OF OR IN CONNECTION WITH THE SOFTWARE OR THE USE OR OTHER 
% DEALINGS IN THE SOFTWARE.
\subsection{SAVE Save Variables To A File}
\subsubsection{Usage}
Saves a set of variables to a file in a machine independent format.
There are two formats for the function call.  The first is the explicit
form, in which a list of variables are provided to write to the file:
\begin{verbatim}
  save filename a1 a2 ...
\end{verbatim}
In the second form,
\begin{verbatim}
  save filename
\end{verbatim}
all variables in the current context are written to the file.  The 
format of the file is a simple binary encoding (raw) of the data
with enough information to restore the variables with the \verb|load|
command.  The endianness of the machine is encoded in the file, and
the resulting file should be portable between machines of similar
types (in particular, machines that support IEEE floating point).
\subsubsection{Example}
Here is a simple example of \verb|save|/\verb|load|.  First, we save some variables to a file.
\begin{verbatim}
--> who
  Variable Name      Type   Flags   Size
              D      cell           [1 4]
              s    string           [1 5]
              x     float           [512 1]
              y     float           [1 262144]
             fp    uint32           [1 1]
            ans    double           []
--> save loadsave.dat
\end{verbatim}
Next, we clear all of the variables, and then load them back from the file.
\begin{verbatim}
--> clear all
--> who
  Variable Name      Type   Flags   Size
            ans    double           []
--> load loadsave.dat
--> who
  Variable Name      Type   Flags   Size
              D      cell           [1 4]
              s    string           [1 5]
              x     float           [512 1]
              y     float           [1 262144]
             fp    uint32           [1 1]
            ans    double           []
\end{verbatim}


