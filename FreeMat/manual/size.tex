% Copyright (c) 2002, 2003 Samit Basu
%
% Permission is hereby granted, free of charge, to any person obtaining a 
% copy of this software and associated documentation files (the "Software"), 
% to deal in the Software without restriction, including without limitation 
% the rights to use, copy, modify, merge, publish, distribute, sublicense, 
% and/or sell copies of the Software, and to permit persons to whom the 
% Software is furnished to do so, subject to the following conditions:
%
% The above copyright notice and this permission notice shall be included 
% in all copies or substantial portions of the Software.
%
% THE SOFTWARE IS PROVIDED "AS IS", WITHOUT WARRANTY OF ANY KIND, EXPRESS 
% OR IMPLIED, INCLUDING BUT NOT LIMITED TO THE WARRANTIES OF MERCHANTABILITY, 
% FITNESS FOR A PARTICULAR PURPOSE AND NONINFRINGEMENT. IN NO EVENT SHALL 
% THE AUTHORS OR COPYRIGHT HOLDERS BE LIABLE FOR ANY CLAIM, DAMAGES OR OTHER 
% LIABILITY, WHETHER IN AN ACTION OF CONTRACT, TORT OR OTHERWISE, ARISING
% FROM, OUT OF OR IN CONNECTION WITH THE SOFTWARE OR THE USE OR OTHER 
% DEALINGS IN THE SOFTWARE.
\subsection{SIZE Size of a Variable}
\subsubsection{Usage}
Returns the size of a variable.  There are two syntaxes for its
use.  The first syntax returns the size of the array as a vector
of integers, one integer for each dimension
\begin{verbatim}
  [d1,d2,...,dn] = size(x)
\end{verbatim}
The other format returns the size of $x$ along a particular
dimension:
\begin{verbatim}
  d = size(x,n)
\end{verbatim}
where $n$ is the dimension along which to return the size.
\subsubsection{Example}
\begin{verbatim}
--> a = randn(23,12,5);
--> size(a)
ans =
  <uint32>  - size: [1 3]
  
Columns 1 to 3
           23            12             5
\end{verbatim}
Here is an example of the second form of \verb|size|.
\begin{verbatim}
--> size(a,2)
ans =
  <uint32>  - size: [1 1]
           12
\end{verbatim}
