% Copyright (c) 2002, 2003 Samit Basu
%
% Permission is hereby granted, free of charge, to any person obtaining a 
% copy of this software and associated documentation files (the "Software"), 
% to deal in the Software without restriction, including without limitation 
% the rights to use, copy, modify, merge, publish, distribute, sublicense, 
% and/or sell copies of the Software, and to permit persons to whom the 
% Software is furnished to do so, subject to the following conditions:
%
% The above copyright notice and this permission notice shall be included 
% in all copies or substantial portions of the Software.
%
% THE SOFTWARE IS PROVIDED "AS IS", WITHOUT WARRANTY OF ANY KIND, EXPRESS 
% OR IMPLIED, INCLUDING BUT NOT LIMITED TO THE WARRANTIES OF MERCHANTABILITY, 
% FITNESS FOR A PARTICULAR PURPOSE AND NONINFRINGEMENT. IN NO EVENT SHALL 
% THE AUTHORS OR COPYRIGHT HOLDERS BE LIABLE FOR ANY CLAIM, DAMAGES OR OTHER 
% LIABILITY, WHETHER IN AN ACTION OF CONTRACT, TORT OR OTHERWISE, ARISING
% FROM, OUT OF OR IN CONNECTION WITH THE SOFTWARE OR THE USE OR OTHER 
% DEALINGS IN THE SOFTWARE.
\subsection{TRANSPOSE Matrix Transpose Operator}
\subsubsection{Usage}
Performs a transpose of the argument (a 2D matrix).  The syntax for its use is
\begin{verbatim}
  y = a.';
\end{verbatim}
where $a$ is a $M \times N$ numerical matrix.  The output $y$ is a numerical matrix
of the same type of size $N \times M$.  This operator is the non-conjugating transpose,
which is different from the Hermitian operator \verb|'| (which conjugates complex values).
\subsubsection{Function Internals}
The transpose operator is defined simply as
\[
  y_{i,j} = a_{j,i}
\]
where $y_{i,j}$ is the element in the $i$th row and $j$th column of the output matrix $y$.
\subsubsection{Examples}
A simple transpose example:
\begin{verbatim}
--> A = [1,2,0;4,1,-1]
A =
  <int32>  - size: [2 3]
  
Columns 1 to 3
             1              2              0
             4              1             -1
--> A.'
ans =
  <int32>  - size: [3 2]
  
Columns 1 to 2
             1              4
             2              1
             0             -1
\end{verbatim}
Here, we use a complex matrix to demonstrate how the transpose does \emph{not} conjugate the entries.
\begin{verbatim}
--> A = [1+i,2-i]
A =
  <complex>  - size: [1 2]
  
Columns 1 to 2
    1.0000000         1.0000000     i     2.0000000        -1.0000000     i
--> A.'
ans =
  <complex>  - size: [2 1]
  
Columns 1 to 1
    1.0000000         1.0000000     i
    2.0000000        -1.0000000     i
\end{verbatim}
