% Copyright (c) 2002, 2003 Samit Basu
%
% Permission is hereby granted, free of charge, to any person obtaining a 
% copy of this software and associated documentation files (the "Software"), 
% to deal in the Software without restriction, including without limitation 
% the rights to use, copy, modify, merge, publish, distribute, sublicense, 
% and/or sell copies of the Software, and to permit persons to whom the 
% Software is furnished to do so, subject to the following conditions:
%
% The above copyright notice and this permission notice shall be included 
% in all copies or substantial portions of the Software.
%
% THE SOFTWARE IS PROVIDED "AS IS", WITHOUT WARRANTY OF ANY KIND, EXPRESS 
% OR IMPLIED, INCLUDING BUT NOT LIMITED TO THE WARRANTIES OF MERCHANTABILITY, 
% FITNESS FOR A PARTICULAR PURPOSE AND NONINFRINGEMENT. IN NO EVENT SHALL 
% THE AUTHORS OR COPYRIGHT HOLDERS BE LIABLE FOR ANY CLAIM, DAMAGES OR OTHER 
% LIABILITY, WHETHER IN AN ACTION OF CONTRACT, TORT OR OTHERWISE, ARISING
% FROM, OUT OF OR IN CONNECTION WITH THE SOFTWARE OR THE USE OR OTHER 
% DEALINGS IN THE SOFTWARE.
\subsection{ZEROS Array of Zeros}
\subsubsection{Usage}
Creates an array of zeros of the specified size.  Two seperate 
syntaxes are possible.  The first syntax specifies the array 
dimensions as a sequence of scalar dimensions:
\begin{verbatim}
   y = zeros(d1,d2,...,dn).
\end{verbatim}
The resulting array has the given dimensions, and is filled with
all zeros.  The type of $y$ is \verb|float|, a 32-bit floating
point array.  To get arrays of other types, use the typecast 
functions (e.g., \verb|uint8|, \verb|int8|, etc.).
    
The second syntax specifies the array dimensions as a vector,
where each element in the vector specifies a dimension length:
\begin{verbatim}
   y = zeros([d1,d2,...,dn]).
\end{verbatim}
This syntax is more convenient for calling \verb|zeros| using a 
variable for the argument.  In both cases, specifying only one
dimension results in a square matrix output.
\subsubsection{Example}
The following examples demonstrate generation of some zero arrays 
using the first form.
\begin{verbatim}
--> zeros(2,3,2)
ans =
  <float>  - size: [2 3 2]
(:,:,1) =
  
Columns 1 to 3
    0.00000000         0.00000000         0.00000000
    0.00000000         0.00000000         0.00000000
(:,:,2) =
  
Columns 1 to 3
    0.00000000         0.00000000         0.00000000
    0.00000000         0.00000000         0.00000000
--> zeros(1,3)
ans =
  <float>  - size: [1 3]
  
Columns 1 to 3
    0.00000000         0.00000000         0.00000000
\end{verbatim}
The same expressions, using the second form.
\begin{verbatim}
--> zeros([2,6])
ans =
  <float>  - size: [2 6]
  
Columns 1 to 3
    0.00000000         0.00000000         0.00000000
    0.00000000         0.00000000         0.00000000
  
Columns 4 to 6
    0.00000000         0.00000000         0.00000000
    0.00000000         0.00000000         0.00000000
--> zeros([1,3])
ans =
  <float>  - size: [1 3]
  
Columns 1 to 3
    0.00000000         0.00000000         0.00000000
\end{verbatim}
Finally, an example of using the type casting function \verb|uint16| to generate an array of 16-bit unsigned integers with zero values.
\begin{verbatim}
--> uint16(zeros(3))
ans =
  <uint16>  - size: [3 3]
  
Columns 1 to 3
     0      0      0
     0      0      0
     0      0      0  
\end{verbatim}
