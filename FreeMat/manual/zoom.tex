% Copyright (c) 2002, 2003 Samit Basu
%
% Permission is hereby granted, free of charge, to any person obtaining a 
% copy of this software and associated documentation files (the "Software"), 
% to deal in the Software without restriction, including without limitation 
% the rights to use, copy, modify, merge, publish, distribute, sublicense, 
% and/or sell copies of the Software, and to permit persons to whom the 
% Software is furnished to do so, subject to the following conditions:
%
% The above copyright notice and this permission notice shall be included 
% in all copies or substantial portions of the Software.
%
% THE SOFTWARE IS PROVIDED "AS IS", WITHOUT WARRANTY OF ANY KIND, EXPRESS 
% OR IMPLIED, INCLUDING BUT NOT LIMITED TO THE WARRANTIES OF MERCHANTABILITY, 
% FITNESS FOR A PARTICULAR PURPOSE AND NONINFRINGEMENT. IN NO EVENT SHALL 
% THE AUTHORS OR COPYRIGHT HOLDERS BE LIABLE FOR ANY CLAIM, DAMAGES OR OTHER 
% LIABILITY, WHETHER IN AN ACTION OF CONTRACT, TORT OR OTHERWISE, ARISING
% FROM, OUT OF OR IN CONNECTION WITH THE SOFTWARE OR THE USE OR OTHER 
% DEALINGS IN THE SOFTWARE.
\subsection{ZOOM Image Zoom Function}
\subsubsection{Usage}
This function changes the zoom factor associated with the currently active
image.  The generic syntax for its use is
\begin{verbatim}
  zoom(x)
\end{verbatim}
where \verb|x| is the zoom factor to be used.  The exact behavior of the zoom
factor is as follows:
\begin{itemize}
\item $x>0$ The image is zoomed by a factor $x$ in both directions.
\item $x=0$ The image on display is zoomed to fit the size of the image window, but
  the aspect ratio of the image is not changed.  (see the Examples section for
more details).  This is the default zoom level for images displayed with the
\verb|image| command.
\item $x<0$ The image on display is zoomed to fit the size of the image window, with
  the zoom factor in the row and column directions chosen to fill the entire window.
  The aspect ratio of the image is \emph{not} preserved.  The exact value of $x$ is
  irrelevant.
\end{itemize}
\subsubsection{Example}
To demonstrate the use of the \verb|zoom| function, we create a rectangular image 
of a Gaussian pulse.  We start with a display of the image using the \verb|image|
command, and a zoom of 1.
\begin{verbatim}
--> x = linspace(-1,1,300)'*ones(1,600);
--> y = ones(300,1)*linspace(-1,1,600);
--> Z = exp(-(x.^2+y.^2)/0.3);
--> image(Z);
--> zoom(1.0);
\end{verbatim}

\doplot{width=6cm}{zoom1}

At this point, resizing the window accomplishes nothing, as with a zoom factor 
greater than zero, the size of the image is fixed (as seen in the following image).

\doplot{width=6cm}{zoom2}

If we change the zoom to another factor larger than 1, we enlarge the image by
the specified factor (or shrink it, for zoom factors $0 < x < 1$.  Here is the
same image zoomed out to 60\% of its true size.
\begin{verbatim}
--> zoom(0.6);
\end{verbatim}

\doplot{width=3.6cm}{zoom3}

Similarly, we can enlarge it to 130\% of its true size.
\begin{verbatim}
--> zoom(1.3);
\end{verbatim}

\doplot{width=7.8cm}{zoom4}

The ``free'' zoom of $x = 0$ results in the image being zoomed to fit the window
without changing the aspect ratio.  The image is zoomed as much as possible in
one direction.
\begin{verbatim}
--> zoom(0);
\end{verbatim}

\doplot{width=6cm}{zoom5}

The case of a negative zoom $x < 0$ results in the image being scaled arbitrarily.
This allows the image aspect ratio to be changed, as in the following example.
\begin{verbatim}
--> zoom(-1);
\end{verbatim}

\doplot{width=6cm}{zoom6}
