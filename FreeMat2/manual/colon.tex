% Copyright (c) 2002, 2003 Samit Basu
%
% Permission is hereby granted, free of charge, to any person obtaining a 
% copy of this software and associated documentation files (the "Software"), 
% to deal in the Software without restriction, including without limitation 
% the rights to use, copy, modify, merge, publish, distribute, sublicense, 
% and/or sell copies of the Software, and to permit persons to whom the 
% Software is furnished to do so, subject to the following conditions:
%
% The above copyright notice and this permission notice shall be included 
% in all copies or substantial portions of the Software.
%
% THE SOFTWARE IS PROVIDED "AS IS", WITHOUT WARRANTY OF ANY KIND, EXPRESS 
% OR IMPLIED, INCLUDING BUT NOT LIMITED TO THE WARRANTIES OF MERCHANTABILITY, 
% FITNESS FOR A PARTICULAR PURPOSE AND NONINFRINGEMENT. IN NO EVENT SHALL 
% THE AUTHORS OR COPYRIGHT HOLDERS BE LIABLE FOR ANY CLAIM, DAMAGES OR OTHER 
% LIABILITY, WHETHER IN AN ACTION OF CONTRACT, TORT OR OTHERWISE, ARISING
% FROM, OUT OF OR IN CONNECTION WITH THE SOFTWARE OR THE USE OR OTHER 
% DEALINGS IN THE SOFTWARE.
\subsection{COLON Index Generation Operator}
\subsubsection{Usage}
There are two distinct syntaxes for the colon \verb|:| operator - the two argument form
\begin{verbatim}
  y = a : c
\end{verbatim}
and the three argument form
\begin{verbatim}
  y = a : b : c
\end{verbatim}
The two argument form is exactly equivalent to \verb|a:1:c|.  The output $y$ is the vector
\[
  y = [a,a+b,a+2b,\ldots,a+nb]
\]
where $a+nb <= c$.
\subsubsection{Examples}
Some simple examples of index generation.
\begin{verbatim}
--> y = 1:4
y =
  <int32>  - size: [1 4]
  
Columns 1 to 4
             1              2              3              4
\end{verbatim}
Now by half-steps:
\begin{verbatim}
--> y = 1:.5:4
y =
  <double>  - size: [1 7]
  
Columns 1 to 2
    1.000000000000000         1.500000000000000
  
Columns 3 to 4
    2.000000000000000         2.500000000000000
  
Columns 5 to 6
    3.000000000000000         3.500000000000000
  
Columns 7 to 7
    4.000000000000000
\end{verbatim}
Now going backwards (negative steps)
\begin{verbatim}
--> y = 4:-.5:1
y =
  <double>  - size: [1 7]
  
Columns 1 to 2
    4.000000000000000         3.500000000000000
  
Columns 3 to 4
    3.000000000000000         2.500000000000000
  
Columns 5 to 6
    2.000000000000000         1.500000000000000
  
Columns 7 to 7
    1.000000000000000
\end{verbatim}
If the endpoints are the same, one point is generated, regardless of the step size (middle argument)
\begin{verbatim}
--> y = 4:1:4
y =
  <int32>  - size: [1 1]
             4
\end{verbatim}
If the endpoints define an empty interval, the output is an empty matrix:
\begin{verbatim}
--> y = 5:4
y =
  <int32>  - size: []
  []
\end{verbatim}
