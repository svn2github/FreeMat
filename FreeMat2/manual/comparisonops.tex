% Copyright (c) 2002, 2003 Samit Basu
%
% Permission is hereby granted, free of charge, to any person obtaining a 
% copy of this software and associated documentation files (the "Software"), 
% to deal in the Software without restriction, including without limitation 
% the rights to use, copy, modify, merge, publish, distribute, sublicense, 
% and/or sell copies of the Software, and to permit persons to whom the 
% Software is furnished to do so, subject to the following conditions:
%
% The above copyright notice and this permission notice shall be included 
% in all copies or substantial portions of the Software.
%
% THE SOFTWARE IS PROVIDED "AS IS", WITHOUT WARRANTY OF ANY KIND, EXPRESS 
% OR IMPLIED, INCLUDING BUT NOT LIMITED TO THE WARRANTIES OF MERCHANTABILITY, 
% FITNESS FOR A PARTICULAR PURPOSE AND NONINFRINGEMENT. IN NO EVENT SHALL 
% THE AUTHORS OR COPYRIGHT HOLDERS BE LIABLE FOR ANY CLAIM, DAMAGES OR OTHER 
% LIABILITY, WHETHER IN AN ACTION OF CONTRACT, TORT OR OTHERWISE, ARISING
% FROM, OUT OF OR IN CONNECTION WITH THE SOFTWARE OR THE USE OR OTHER 
% DEALINGS IN THE SOFTWARE.
\subsection{COMPARISON OPERATORS Array Comparison Operators}
\subsubsection{Usage}
There are a total of six comparison operators available in FreeMat, all of which are binary operators with the following syntax
\begin{verbatim}
  y = a < b
  y = a <= b
  y = a > b
  y = a >= b
  y = a ~= b
  y = a == b
\end{verbatim}
where $a$ and $b$ are numerical arrays or scalars, and $y$ is a \verb|logical| array of the appropriate size.  Each of the operators has three modes of operation, summarized in the following list:
\begin{enumerate}
  \item $a$ is a scalar, $b$ is an n-dimensional array - the output is then the same size as $b$, and contains the result of comparing each element in $b$ to the scalar $a$.
  \item $a$ is an n-dimensional array, $b$ is a scalar - the output is the same size as $a$, and contains the result of comparing each element in $a$ to the scalar $b$.
  \item $a$ and $b$ are both n-dimensional arrays of the same size - the output is then the same size as both $a$ and $b$, and contains the result of an element-wise comparison between $a$ and $b$.
\end{enumerate}
The operators behave the same way as in \verb|C|, with unequal types meing promoted using the standard type promotion rules prior to comparisons.  The only difference is that in FreeMat, the not-equals operator is \verb|~=| instead of \verb|!=|.
\subsubsection{Examples}
Some simple examples of comparison operations.  First a comparison with a scalar:
\begin{verbatim}
--> a = randn(1,5)
a =
  <double>  - size: [1 5]
  
Columns 1 to 2
    0.0101167556598398        0.160105749424486
  
Columns 3 to 4
   -1.465385481298682        -0.0395884566172688
  
Columns 5 to 5
    1.182465366442761
--> a>0
ans =
  <logical>  - size: [1 5]
  
Columns 1 to 5
 1  1  0  0  1
\end{verbatim}
Next, we construct two vectors, and test for equality:
\begin{verbatim}
--> a = [1,2,5,7,3]
a =
  <int32>  - size: [1 5]
  
Columns 1 to 5
             1              2              5              7              3
--> b = [2,2,5,9,4]
b =
  <int32>  - size: [1 5]
  
Columns 1 to 5
             2              2              5              9              4
--> c = a == b
c =
  <logical>  - size: [1 5]
  
Columns 1 to 5
 0  1  1  0  0
\end{verbatim}
