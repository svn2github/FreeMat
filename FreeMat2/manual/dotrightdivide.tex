% Copyright (c) 2002, 2003 Samit Basu
%
% Permission is hereby granted, free of charge, to any person obtaining a 
% copy of this software and associated documentation files (the "Software"), 
% to deal in the Software without restriction, including without limitation 
% the rights to use, copy, modify, merge, publish, distribute, sublicense, 
% and/or sell copies of the Software, and to permit persons to whom the 
% Software is furnished to do so, subject to the following conditions:
%
% The above copyright notice and this permission notice shall be included 
% in all copies or substantial portions of the Software.
%
% THE SOFTWARE IS PROVIDED "AS IS", WITHOUT WARRANTY OF ANY KIND, EXPRESS 
% OR IMPLIED, INCLUDING BUT NOT LIMITED TO THE WARRANTIES OF MERCHANTABILITY, 
% FITNESS FOR A PARTICULAR PURPOSE AND NONINFRINGEMENT. IN NO EVENT SHALL 
% THE AUTHORS OR COPYRIGHT HOLDERS BE LIABLE FOR ANY CLAIM, DAMAGES OR OTHER 
% LIABILITY, WHETHER IN AN ACTION OF CONTRACT, TORT OR OTHERWISE, ARISING
% FROM, OUT OF OR IN CONNECTION WITH THE SOFTWARE OR THE USE OR OTHER 
% DEALINGS IN THE SOFTWARE.
\subsection{DOTRIGHTDIVIDE Element-wise Right-Division Operator}
\subsubsection{Usage}
Divides two numerical arrays (elementwise).  There are two forms
for its use, both with the same general syntax:
\begin{verbatim}
  y = a ./ b
\end{verbatim}
where $a$ and $b$ are $n$-dimensional arrays of numerical type.  In the
first case, the two arguments are the same size, in which case, the 
output $y$ is the same size as the inputs, and is the element-wise
division of $b$ by $a$.  In the second case, either $a$ or $b$ is a scalar, 
in which case $y$ is the same size as the larger argument,
and is the division of the scalar with each element of the other argument.

The type of $y$ depends on the types of $a$ and $b$ using type 
promotion rules, with one important exception: unlike \verb|C|, integer
types are promoted to \verb|double| prior to division.
\subsubsection{Function Internals}
There are three formulae for the dot-right-divide operator, depending on the
sizes of the three arguments.  In the most general case, in which 
the two arguments are the same size, the output is computed via:
\[
y(m_1,\ldots,m_d) = \frac{a(m_1,\ldots,m_d)}{b(m_1,\ldots,m_d)}
\]
If $a$ is a scalar, then the output is computed via
\[
y(m_1,\ldots,m_d) = \frac{a}{b(m_1,\ldots,m_d)}
\]
On the other hand, if $b$ is a scalar, then the output is computed via
\[
y(m_1,\ldots,m_d) = \frac{a(m_1,\ldots,m_d)}{b}.
\]
\subsubsection{Examples}
Here are some examples of using the dot-right-divide operator.  First, a 
straight-forward usage of the \verb|./| operator.  The first example
is straightforward:
\begin{verbatim}
--> 3 ./ 8
ans =
  <double>  - size: [1 1]
    0.375000000000000
\end{verbatim}
Note that this is not the same as evaluating $3/8$ in \verb|C| - there,
the output would be $0$, the result of the integer division.

We can also divide complex arguments:
\begin{verbatim}
--> a = 3 + 4*i
a =
  <complex>  - size: [1 1]
    3.0000000         4.0000000     i
--> b = 5 + 8*i
b =
  <complex>  - size: [1 1]
    5.0000000         8.0000000     i
--> c = a ./ b
c =
  <complex>  - size: [1 1]
    0.52808988       -0.044943821   i
\end{verbatim}
If a \verb|complex| value is divided by a \verb|double|, the result is 
promoted to \verb|dcomplex|.
\begin{verbatim}
--> b = a ./ 2.0
b =
  <dcomplex>  - size: [1 1]
    1.500000000000000        2.000000000000000    i
\end{verbatim}
We can also demonstrate the three forms of the dot-right-divide operator.  First
the element-wise version:
\begin{verbatim}
--> a = [1,2;3,4]
a =
  <int32>  - size: [2 2]
  
Columns 1 to 2
             1              2
             3              4
--> b = [2,3;6,7]
b =
  <int32>  - size: [2 2]
  
Columns 1 to 2
             2              3
             6              7
--> c = a ./ b
c =
  <double>  - size: [2 2]
  
Columns 1 to 2
    0.500000000000000         0.666666666666667
    0.500000000000000         0.571428571428571
\end{verbatim}
Then the scalar versions
\begin{verbatim}
--> c = a ./ 3
c =
  <double>  - size: [2 2]
  
Columns 1 to 2
    0.333333333333333         0.666666666666667
    1.000000000000000         1.333333333333333
--> c = 3 ./ a
c =
  <double>  - size: [2 2]
  
Columns 1 to 2
    3.000000000000000         1.500000000000000
    1.000000000000000         0.750000000000000
\end{verbatim}
