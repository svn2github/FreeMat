% Copyright (c) 2002, 2003 Samit Basu
%
% Permission is hereby granted, free of charge, to any person obtaining a 
% copy of this software and associated documentation files (the "Software"), 
% to deal in the Software without restriction, including without limitation 
% the rights to use, copy, modify, merge, publish, distribute, sublicense, 
% and/or sell copies of the Software, and to permit persons to whom the 
% Software is furnished to do so, subject to the following conditions:
%
% The above copyright notice and this permission notice shall be included 
% in all copies or substantial portions of the Software.
%
% THE SOFTWARE IS PROVIDED "AS IS", WITHOUT WARRANTY OF ANY KIND, EXPRESS 
% OR IMPLIED, INCLUDING BUT NOT LIMITED TO THE WARRANTIES OF MERCHANTABILITY, 
% FITNESS FOR A PARTICULAR PURPOSE AND NONINFRINGEMENT. IN NO EVENT SHALL 
% THE AUTHORS OR COPYRIGHT HOLDERS BE LIABLE FOR ANY CLAIM, DAMAGES OR OTHER 
% LIABILITY, WHETHER IN AN ACTION OF CONTRACT, TORT OR OTHERWISE, ARISING
% FROM, OUT OF OR IN CONNECTION WITH THE SOFTWARE OR THE USE OR OTHER 
% DEALINGS IN THE SOFTWARE.
\subsection{DOUBLE Convert to 64-bit Floating Point}
\subsubsection{Usage}
Converts the argument to a 64-bit floating point number.  The syntax
for its use is
\begin{verbatim}
   y = double(x)
\end{verbatim}
where $x$ is an $n$-dimensional numerical array.  Conversion follows the general C rules.  Note that both \verb|NaN| and \verb|Inf| are both preserved under type conversion.
\subsubsection{Example}
The following piece of code demonstrates several uses of \verb|double|.  First, we convert from an integer (the argument is an integer because no decimal is present):
\begin{verbatim}
--> double(200)
ans =
  <double>  - size: [1 1]
  200.00000
\end{verbatim}
In the next example, a single precision argument is passed in (the presence of  the \verb|f| suffix implies single precision).
\begin{verbatim}
--> double(400.0f)
ans =
  <double>  - size: [1 1]
  400.000000000000
\end{verbatim}
In the next example, a dcomplex argument is passed in.  The result is the real part of the argument, and in this context, \verb|double| is equivalent to the function \verb|real|.
\begin{verbatim}
--> double(3.0+4.0*i)
ans =
  <double>  - size: [1 1]
    3.000000000000000
\end{verbatim}
In the next example, a string argument is passed in.  The string argument is converted into an integer array corresponding to the ASCII values of each character.
\begin{verbatim}
--> double('helo')
ans =
  <double>  - size: [1 4]
  
Columns 1 to 3
  104.000000000000          101.000000000000          108.000000000000
  
Columns 4 to 4
  111.000000000000
\end{verbatim}
In the last example, a cell-array is passed in.  For cell-arrays and structure arrays, the result is an error.
\begin{verbatim}
--> double({4})
Error: Cannot convert cell-arrays to any other type.
\end{verbatim}
