% Copyright (c) 2002, 2003 Samit Basu
%
% Permission is hereby granted, free of charge, to any person obtaining a 
% copy of this software and associated documentation files (the "Software"), 
% to deal in the Software without restriction, including without limitation 
% the rights to use, copy, modify, merge, publish, distribute, sublicense, 
% and/or sell copies of the Software, and to permit persons to whom the 
% Software is furnished to do so, subject to the following conditions:
%
% The above copyright notice and this permission notice shall be included 
% in all copies or substantial portions of the Software.
%
% THE SOFTWARE IS PROVIDED "AS IS", WITHOUT WARRANTY OF ANY KIND, EXPRESS 
% OR IMPLIED, INCLUDING BUT NOT LIMITED TO THE WARRANTIES OF MERCHANTABILITY, 
% FITNESS FOR A PARTICULAR PURPOSE AND NONINFRINGEMENT. IN NO EVENT SHALL 
% THE AUTHORS OR COPYRIGHT HOLDERS BE LIABLE FOR ANY CLAIM, DAMAGES OR OTHER 
% LIABILITY, WHETHER IN AN ACTION OF CONTRACT, TORT OR OTHERWISE, ARISING
% FROM, OUT OF OR IN CONNECTION WITH THE SOFTWARE OR THE USE OR OTHER 
% DEALINGS IN THE SOFTWARE.
\subsection{FIND Find Non-zero Elements of An Array}
\subsubsection{Usage}
Returns a vector that contains the indicies of all non-zero elements 
in an array.  The usage is
\begin{verbatim}
   y = find(x)
\end{verbatim}
The indices returned are generalized column indices, meaning that if 
the array $x$ is of size $[d_1,d_2,\ldots,d_n]$, and the
element $x(i_1,i_2,\ldots,i_n)$ is nonzero, then $y$
will contain the integer
\[
   i_1 + (i_2-1) d_1 + (i_3-1) d_1 d_2 + \dots
\]
\subsubsection{Example}
Some simple examples of its usage, and some common uses of \verb|find| in FreeMat programs.
\begin{verbatim}
--> a = [1,2,5,2,4];
--> find(a==2)
ans =
  <uint32>  - size: [2 1]
  
Columns 1 to 1
            2
            4
\end{verbatim}
Here is an example of using find to replace elements of \verb|A| that are $0$ with the number $5$.
\begin{verbatim}
--> A = [1,0,3;0,2,1;3,0,0]
A =
  <int32>  - size: [3 3]
  
Columns 1 to 3
             1              0              3
             0              2              1
             3              0              0
--> n = find(A==0)
n =
  <uint32>  - size: [4 1]
  
Columns 1 to 1
            2
            4
            6
            9
--> A(n) = 5
A =
  <int32>  - size: [3 3]
  
Columns 1 to 3
             1              5              3
             5              2              1
             3              5              5
\end{verbatim}
Incidentally, a better way to achieve the same concept is:
\begin{verbatim}
--> A = [1,0,3;0,2,1;3,0,0]
A =
  <int32>  - size: [3 3]
                                                                                
Columns 1 to 3
             1              0              3
             0              2              1
             3              0              0
--> A(A==0) = 5
A =
  <int32>  - size: [3 3]
  
Columns 1 to 3
             1              5              3
             5              2              1
             3              5              5
\end{verbatim}
