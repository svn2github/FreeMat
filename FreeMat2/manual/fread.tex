% Copyright (c) 2002, 2003 Samit Basu
%
% Permission is hereby granted, free of charge, to any person obtaining a 
% copy of this software and associated documentation files (the "Software"), 
% to deal in the Software without restriction, including without limitation 
% the rights to use, copy, modify, merge, publish, distribute, sublicense, 
% and/or sell copies of the Software, and to permit persons to whom the 
% Software is furnished to do so, subject to the following conditions:
%
% The above copyright notice and this permission notice shall be included 
% in all copies or substantial portions of the Software.
%
% THE SOFTWARE IS PROVIDED "AS IS", WITHOUT WARRANTY OF ANY KIND, EXPRESS 
% OR IMPLIED, INCLUDING BUT NOT LIMITED TO THE WARRANTIES OF MERCHANTABILITY, 
% FITNESS FOR A PARTICULAR PURPOSE AND NONINFRINGEMENT. IN NO EVENT SHALL 
% THE AUTHORS OR COPYRIGHT HOLDERS BE LIABLE FOR ANY CLAIM, DAMAGES OR OTHER 
% LIABILITY, WHETHER IN AN ACTION OF CONTRACT, TORT OR OTHERWISE, ARISING
% FROM, OUT OF OR IN CONNECTION WITH THE SOFTWARE OR THE USE OR OTHER 
% DEALINGS IN THE SOFTWARE.
\subsection{FREAD File Read Function}
\subsubsection{Usage}
Reads a block of binary data from the given file handle into a variable
of a given shape and precision.  The general use of the function is
\begin{verbatim}
  A = fread(handle,size,precision)
\end{verbatim}
The \verb|handle| argument must be a valid value returned by the fopen 
function, and accessable for reading.  The \verb|size| argument determines
the number of values read from the file.  The \verb|size| argument is simply
a vector indicating the size of the array \verb|A|.  The \verb|size| argument
can also contain a single \verb|inf| dimension, indicating that FreeMat should
calculate the size of the array along that dimension so as to read as
much data as possible from the file (see the examples listed below for
more details).  The data is stored as \emph{columns} in the file, not 
\emph{rows}.
    
The third argument determines the type of the data.  Legal values for this
argument are listed below:
\begin{itemize}
\item 'uint8','uchar','unsigned char' for an unsigned, 8-bit integer.
\item 'int8','char','integer*1' for a signed, 8-bit integer.
\item 'uint16','unsigned short' for an unsigned, 16-bit  integer.
\item 'int16','short','integer*2' for a signed, 16-bit integer.
\item 'uint32','unsigned int' for an unsigned, 32-bit integer.
\item 'int32','int','integer*4' for a signed, 32-bit integer.
\item 'single','float32','float','real*4' for a 32-bit floating point.
\item 'double','float64','real*8' for a 64-bit floating point.
\item 'complex','complex*8' for a 64-bit complex floating point (32 bits 
for the real and imaginary part).
\item 'dcomplex','complex*16' for a 128-bit complex floating point (64
bits for the real and imaginary part).
\end{itemize}
\subsubsection{Example}
First, we create an array of $512 \times 512$ Gaussian-distributed \verb|float| random variables, and then writing them to a file called \verb|test.dat|.
\begin{verbatim}
--> A = float(randn(512));
--> fp = fopen('test.dat','wb');
--> fwrite(fp,A);
--> fclose(fp);
\end{verbatim}
Read as many floats as possible into a row vector
\begin{verbatim}
--> fp = fopen('test.dat','rb');
--> x = fread(fp,[1,inf],'float');
--> who x
  Variable Name      Type   Flags   Size
              x     float           [1 262144]
\end{verbatim}
Read the same floats into a float array (we use \verb|fseek| to reset the file pointer).
\begin{verbatim}
--> fseek(fp,0,-1);
--> x = fread(fp,[512,inf],'float');
--> who x
  Variable Name      Type   Flags   Size
              x     float           [512 512]
\end{verbatim}
