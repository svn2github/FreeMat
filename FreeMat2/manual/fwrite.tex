% Copyright (c) 2002, 2003 Samit Basu
%
% Permission is hereby granted, free of charge, to any person obtaining a 
% copy of this software and associated documentation files (the "Software"), 
% to deal in the Software without restriction, including without limitation 
% the rights to use, copy, modify, merge, publish, distribute, sublicense, 
% and/or sell copies of the Software, and to permit persons to whom the 
% Software is furnished to do so, subject to the following conditions:
%
% The above copyright notice and this permission notice shall be included 
% in all copies or substantial portions of the Software.
%
% THE SOFTWARE IS PROVIDED "AS IS", WITHOUT WARRANTY OF ANY KIND, EXPRESS 
% OR IMPLIED, INCLUDING BUT NOT LIMITED TO THE WARRANTIES OF MERCHANTABILITY, 
% FITNESS FOR A PARTICULAR PURPOSE AND NONINFRINGEMENT. IN NO EVENT SHALL 
% THE AUTHORS OR COPYRIGHT HOLDERS BE LIABLE FOR ANY CLAIM, DAMAGES OR OTHER 
% LIABILITY, WHETHER IN AN ACTION OF CONTRACT, TORT OR OTHERWISE, ARISING
% FROM, OUT OF OR IN CONNECTION WITH THE SOFTWARE OR THE USE OR OTHER 
% DEALINGS IN THE SOFTWARE.
\subsection{FWRITE File Write Function}
\subsubsection{Usage}
Writes an array to a given file handle as a block of binary (raw) data.
The general use of the function is
\begin{verbatim}
  n = fwrite(handle,A)
\end{verbatim}
The \verb|handle| argument must be a valid value returned by the fopen 
function, and accessable for writing. The array \verb|A| is written to
the file a \emph{column} at a time.  The form of the output data depends
on (and is inferred from) the precision of the array \verb|A|.  If the 
write fails (because we ran out of disk space, etc.) then an error
is returned.  The output \verb|n| indicates the number of elements
successfully written.
\subsubsection{Example}
Here's an example of writing an array of $512 \times 512$ Gaussian-distributed \verb|float| random variables, and then writing them to a file called \verb|test.dat|.
\begin{verbatim}
--> A = float(randn(512));
--> fp = fopen('test.dat','wb');
--> fwrite(fp,A);
--> fclose(fp);
\end{verbatim}
