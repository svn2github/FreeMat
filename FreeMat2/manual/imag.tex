% Copyright (c) 2002, 2003 Samit Basu
%
% Permission is hereby granted, free of charge, to any person obtaining a 
% copy of this software and associated documentation files (the "Software"), 
% to deal in the Software without restriction, including without limitation 
% the rights to use, copy, modify, merge, publish, distribute, sublicense, 
% and/or sell copies of the Software, and to permit persons to whom the 
% Software is furnished to do so, subject to the following conditions:
%
% The above copyright notice and this permission notice shall be included 
% in all copies or substantial portions of the Software.
%
% THE SOFTWARE IS PROVIDED "AS IS", WITHOUT WARRANTY OF ANY KIND, EXPRESS 
% OR IMPLIED, INCLUDING BUT NOT LIMITED TO THE WARRANTIES OF MERCHANTABILITY, 
% FITNESS FOR A PARTICULAR PURPOSE AND NONINFRINGEMENT. IN NO EVENT SHALL 
% THE AUTHORS OR COPYRIGHT HOLDERS BE LIABLE FOR ANY CLAIM, DAMAGES OR OTHER 
% LIABILITY, WHETHER IN AN ACTION OF CONTRACT, TORT OR OTHERWISE, ARISING
% FROM, OUT OF OR IN CONNECTION WITH THE SOFTWARE OR THE USE OR OTHER 
% DEALINGS IN THE SOFTWARE.
\subsection{IMAG Imaginary Function}
\subsubsection{Usage}
Returns the imaginary part of the input array for all elements.  The 
general syntax for its use is
\begin{verbatim}
   y = imag(x)
\end{verbatim}
where $x$ is an $n$-dimensional array of numerical type.  The output 
is the same numerical type as the input, unless the input is \verb|complex|
or \verb|dcomplex|.  For \verb|complex| inputs, the imaginary part is a floating
point array, so that the return type is \verb|float|.  For \verb|dcomplex|
inputs, the imaginary part is a double precision floating point array, so that
the return type is \verb|double|.  The \verb|imag| function does
nothing to real and integer types.
\subsubsection{Example}
The following demonstrates \verb|imag| applied to a complex scalar.
\begin{verbatim}
--> imag(3+4*i)
ans =
  <float>  - size: [1 1]
    4.0000000
\end{verbatim}
The imaginary part of real and integer arguments is a vector of zeros, the
same type and size of the argument.
\begin{verbatim}
--> imag([2,4,5,6])
ans =
  <int32>  - size: [1 4]
  
Columns 1 to 4
             0              0              0              0
\end{verbatim}
For a double-precision complex array,
\begin{verbatim}
--> imag([2.0+3.0*i,i])
ans =
  <double>  - size: [1 2]
  
Columns 1 to 2
    3.000000000000000         1.000000000000000
\end{verbatim}
