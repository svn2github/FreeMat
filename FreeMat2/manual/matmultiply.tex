% Copyright (c) 2002, 2003 Samit Basu
%
% Permission is hereby granted, free of charge, to any person obtaining a 
% copy of this software and associated documentation files (the "Software"), 
% to deal in the Software without restriction, including without limitation 
% the rights to use, copy, modify, merge, publish, distribute, sublicense, 
% and/or sell copies of the Software, and to permit persons to whom the 
% Software is furnished to do so, subject to the following conditions:
%
% The above copyright notice and this permission notice shall be included 
% in all copies or substantial portions of the Software.
%
% THE SOFTWARE IS PROVIDED "AS IS", WITHOUT WARRANTY OF ANY KIND, EXPRESS 
% OR IMPLIED, INCLUDING BUT NOT LIMITED TO THE WARRANTIES OF MERCHANTABILITY, 
% FITNESS FOR A PARTICULAR PURPOSE AND NONINFRINGEMENT. IN NO EVENT SHALL 
% THE AUTHORS OR COPYRIGHT HOLDERS BE LIABLE FOR ANY CLAIM, DAMAGES OR OTHER 
% LIABILITY, WHETHER IN AN ACTION OF CONTRACT, TORT OR OTHERWISE, ARISING
% FROM, OUT OF OR IN CONNECTION WITH THE SOFTWARE OR THE USE OR OTHER 
% DEALINGS IN THE SOFTWARE.
\subsection{TIMES Matrix Multiply Operator}
\subsubsection{Usage}
Multiplies two numerical arrays.  This operator is really a combination
of three operators, all of which have the same general syntax:
\begin{verbatim}
  y = a * b
\end{verbatim}
where $a$ and $b$ are arrays of numerical type.  The result $y$ depends
on which of the following three situations applies to the arguments
$a$ and $b$:
\begin{enumerate}
  \item $a$ is a scalar, $b$ is an arbitrary $n$-dimensional numerical array, in which case the output is the element-wise product of $b$ with the scalar $a$.
  \item $b$ is a scalar, $a$ is an arbitrary $n$-dimensional numerical array, in which case the output is the element-wise product of $a$ with the scalar $b$.
  \item $a,b$ are conformant matrices, i.e., $a$ is of size $M \times K$, and $b$ is of size $K \times N$, in which case the output is of size $M \times N$ and is the matrix product of $a$, and $b$.
\end{enumerate}
The output follows the standard type promotion rules, although in the first two cases, if $a$ and $b$ are integers, the output is an integer also, while in the third case if $a$ and $b$ are integers, ,the output is of type \verb|double|.
\subsubsection{Function Internals}
There are three formulae for the times operator.  For the first form
\[
y(m_1,\ldots,m_d) = a \times b(m_1,\ldots,m_d),
\]
and the second form
\[
y(m_1,\ldots,m_d) = a(m_1,\ldots,m_d) \times b.
\]
In the third form, the output is the matrix product of the arguments
\[
y(m,n) = \sum_{k=1}^{K} a(m,k) b(k,n)
\]
\subsubsection{Examples}
Here are some examples of using the matrix multiplication operator.  First,
the scalar examples (types 1 and 2 from the list above):
\begin{verbatim}
--> a = [1,3,4;0,2,1]
a =
  <int32>  - size: [2 3]
  
Columns 1 to 3
             1              3              4
             0              2              1
--> b = a * 2
b =
  <int32>  - size: [2 3]
  
Columns 1 to 3
             2              6              8
             0              4              2
\end{verbatim}
The matrix form, where the first argument is \verb|2 x 3|, and the
second argument is \verb|3 x 1|, so that the product is size 
\verb|2 x 1|.
\begin{verbatim}
--> a = [1,2,0;4,2,3]
a =
  <int32>  - size: [2 3]
  
Columns 1 to 3
             1              2              0
             4              2              3
--> b = [5;3;1]
b =
  <int32>  - size: [3 1]
  
Columns 1 to 1
             5
             3
             1
--> c = a*b
c =
  <double>  - size: [2 1]
  
Columns 1 to 1
   11.0000000000000
   29.0000000000000
\end{verbatim}
Note that the output is double precision.
