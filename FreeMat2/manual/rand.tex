% Copyright (c) 2002, 2003 Samit Basu
%
% Permission is hereby granted, free of charge, to any person obtaining a 
% copy of this software and associated documentation files (the "Software"), 
% to deal in the Software without restriction, including without limitation 
% the rights to use, copy, modify, merge, publish, distribute, sublicense, 
% and/or sell copies of the Software, and to permit persons to whom the 
% Software is furnished to do so, subject to the following conditions:
%
% The above copyright notice and this permission notice shall be included 
% in all copies or substantial portions of the Software.
%
% THE SOFTWARE IS PROVIDED "AS IS", WITHOUT WARRANTY OF ANY KIND, EXPRESS 
% OR IMPLIED, INCLUDING BUT NOT LIMITED TO THE WARRANTIES OF MERCHANTABILITY, 
% FITNESS FOR A PARTICULAR PURPOSE AND NONINFRINGEMENT. IN NO EVENT SHALL 
% THE AUTHORS OR COPYRIGHT HOLDERS BE LIABLE FOR ANY CLAIM, DAMAGES OR OTHER 
% LIABILITY, WHETHER IN AN ACTION OF CONTRACT, TORT OR OTHERWISE, ARISING
% FROM, OUT OF OR IN CONNECTION WITH THE SOFTWARE OR THE USE OR OTHER 
% DEALINGS IN THE SOFTWARE.
\subsection{RAND Uniform Random Number Generator}
\subsubsection{Usage}
Creates an array of pseudo-random numbers of the specified size.
The numbers are uniformly distributed on $[0,1)$.  
Two seperate syntaxes are possible.  The first syntax specifies the array 
dimensions as a sequence of scalar dimensions:
\begin{verbatim}
  y = rand(d1,d2,...,dn).
\end{verbatim}
The resulting array has the given dimensions, and is filled with
random numbers.  The type of $y$ is \verb|double|, a 64-bit floating
point array.  To get arrays of other types, use the typecast 
functions.
    
The second syntax specifies the array dimensions as a vector,
where each element in the vector specifies a dimension length:
\begin{verbatim}
  y = rand([d1,d2,...,dn]).
\end{verbatim}
This syntax is more convenient for calling \verb|rand| using a 
variable for the argument.
\subsubsection{Example}
The following example demonstrates an example of using the first form of the \verb|rand| function.
\begin{verbatim}
--> rand(2,2,2)
ans =
  <double>  - size: [2 2 2]
(:,:,1) =
  
Columns 1 to 2
    0.816205139112481         0.928488070407937
    0.101086560308625         0.609109168991309
(:,:,2) =
  
Columns 1 to 2
    0.596553436123275         0.345186243305358
    0.0917841347962021        0.662752523185588
\end{verbatim}
The second example demonstrates the second form of the \verb|rand| function.
\begin{verbatim}
--> rand([2,2,2])
ans =
  <double>  - size: [2 2 2]
(:,:,1) =
  
Columns 1 to 2
    0.441713488461236         0.703712491427084
    0.551487786393961         0.589401231301516
(:,:,2) =
  
Columns 1 to 2
    0.0499327594155466        0.766358472374211
    0.561791839934797         0.910908329667760
\end{verbatim}
The third example computes the mean and variance of a large number of uniform random numbers.  Recall that the mean should be $1/2$, and the variance should be $1/12 \sim 0.083$.
\begin{verbatim}
--> x = rand(1,10000);
--> mean(x)
ans =
  <double>  - size: [1 1]
    0.501955631704274
--> var(x)
ans =
  <double>  - size: [1 1]
    0.0816622288902852
\end{verbatim}

