% Copyright (c) 2002, 2003 Samit Basu
%
% Permission is hereby granted, free of charge, to any person obtaining a 
% copy of this software and associated documentation files (the "Software"), 
% to deal in the Software without restriction, including without limitation 
% the rights to use, copy, modify, merge, publish, distribute, sublicense, 
% and/or sell copies of the Software, and to permit persons to whom the 
% Software is furnished to do so, subject to the following conditions:
%
% The above copyright notice and this permission notice shall be included 
% in all copies or substantial portions of the Software.
%
% THE SOFTWARE IS PROVIDED "AS IS", WITHOUT WARRANTY OF ANY KIND, EXPRESS 
% OR IMPLIED, INCLUDING BUT NOT LIMITED TO THE WARRANTIES OF MERCHANTABILITY, 
% FITNESS FOR A PARTICULAR PURPOSE AND NONINFRINGEMENT. IN NO EVENT SHALL 
% THE AUTHORS OR COPYRIGHT HOLDERS BE LIABLE FOR ANY CLAIM, DAMAGES OR OTHER 
% LIABILITY, WHETHER IN AN ACTION OF CONTRACT, TORT OR OTHERWISE, ARISING
% FROM, OUT OF OR IN CONNECTION WITH THE SOFTWARE OR THE USE OR OTHER 
% DEALINGS IN THE SOFTWARE.
\subsection{SUM Sum Function}
\subsubsection{Usage}
Computes the summation of an array along a given dimension.  The general
syntax for its use is
\begin{verbatim}
  y = sum(x,{d})
\end{verbatim}
where $x$ is an $n$-dimensions array of numerical type.
The output is of the same numerical type as the input.  The argument
$d$ is optional, and denotes the dimension along which to take
the summation.  The output is computed via
\[
y(m_1,\ldots,m_{d-1},1,m_{d+1},\ldots,m_{p}) = 
\sum_{k} x(m_1,\ldots,m_{d-1},k,m_{d+1},\ldots,m_{p})
\]
If $d$ is omitted, then the summation is taken along the 
first non-singleton dimension of $x$. 
\subsubsection{Example}
The following piece of code demonstrates various uses of the summation
function
\begin{verbatim}
--> A = [5,1,3;3,2,1;0,3,1]
A =
  <int32>  - size: [3 3]
  
Columns 1 to 3
             5              1              3
             3              2              1
             0              3              1
\end{verbatim}
We start by calling \verb|sum| without a dimension argument, in which case it defaults to the first nonsingular dimension (in this case, along the columns or $d = 1$).
\begin{verbatim}
--> sum(A)
ans =
  <int32>  - size: [1 3]
  
Columns 1 to 3
             8              6              5
\end{verbatim}
Next, we take the sum along the rows.
\begin{verbatim}
--> sum(A,2)
ans =
  <int32>  - size: [3 1]
  
Columns 1 to 1
             9
             6
             4
\end{verbatim}
